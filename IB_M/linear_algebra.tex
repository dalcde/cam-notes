\documentclass[a4paper]{article}

\def\npart {IB}
\def\nterm {Michaelmas}
\def\nyear {2015}
\def\nlecturer {S. J. Wadsley}
\def\ncourse {Linear Algebra}
\def\nofficial{https://www.dpmms.cam.ac.uk/~sjw47/LinearAlgebra.pdf}
\def\nnotready {}

% Imports
\ifx \nextra \undefined
  \usepackage[pdftex,
    hidelinks,
    pdfauthor={Dexter Chua},
    pdfsubject={Cambridge Maths Notes: Part \npart\ - \ncourse},
    pdftitle={Part \npart\ - \ncourse},
  pdfkeywords={Cambridge Mathematics Maths Math \npart\ \nterm\ \nyear\ \ncourse}]{hyperref}
  \title{Part \npart\ - \ncourse}
\else
  \usepackage[pdftex,
    hidelinks,
    pdfauthor={Dexter Chua},
    pdfsubject={Cambridge Maths Notes: Part \npart\ - \ncourse\ (\nextra)},
    pdftitle={Part \npart\ - \ncourse\ (\nextra)},
  pdfkeywords={Cambridge Mathematics Maths Math \npart\ \nterm\ \nyear\ \ncourse\ \nextra}]{hyperref}

  \title{Part \npart\ - \ncourse \\ {\Large \nextra}}
\fi

\author{Lectured by \nlecturer \\\small Notes taken by Dexter Chua}
\date{\nterm\ \nyear}

\usepackage{alltt}
\usepackage{amsfonts}
\usepackage{amsmath}
\usepackage{amssymb}
\usepackage{amsthm}
\usepackage{booktabs}
\usepackage{caption}
\usepackage{enumitem}
\usepackage{fancyhdr}
\usepackage{graphicx}
\usepackage{mathtools}
\usepackage{microtype}
\usepackage{multirow}
\usepackage{pdflscape}
\usepackage{pgfplots}
\usepackage{siunitx}
\usepackage{tabularx}
\usepackage{tikz}
\usepackage{tkz-euclide}
\usepackage[normalem]{ulem}
\usepackage[all]{xy}

\pgfplotsset{compat=1.12}

\pagestyle{fancyplain}
\lhead{\emph{\nouppercase{\leftmark}}}
\ifx \nextra \undefined
  \rhead{
    \ifnum\thepage=1
    \else
      \npart\ \ncourse
    \fi}
\else
  \rhead{
    \ifnum\thepage=1
    \else
      \npart\ \ncourse\ (\nextra)
    \fi}
\fi
\usetikzlibrary{arrows}
\usetikzlibrary{decorations.markings}
\usetikzlibrary{decorations.pathmorphing}
\usetikzlibrary{positioning}
\usetikzlibrary{fadings}
\usetikzlibrary{intersections}
\usetikzlibrary{cd}

\newcommand*{\Cdot}{\raisebox{-0.25ex}{\scalebox{1.5}{$\cdot$}}}
\newcommand {\pd}[2][ ]{
  \ifx #1 { }
    \frac{\partial}{\partial #2}
  \else
    \frac{\partial^{#1}}{\partial #2^{#1}}
  \fi
}

% Theorems
\theoremstyle{definition}
\newtheorem*{aim}{Aim}
\newtheorem*{axiom}{Axiom}
\newtheorem*{claim}{Claim}
\newtheorem*{cor}{Corollary}
\newtheorem*{defi}{Definition}
\newtheorem*{eg}{Example}
\newtheorem*{fact}{Fact}
\newtheorem*{law}{Law}
\newtheorem*{lemma}{Lemma}
\newtheorem*{notation}{Notation}
\newtheorem*{prop}{Proposition}
\newtheorem*{thm}{Theorem}

\renewcommand{\labelitemi}{--}
\renewcommand{\labelitemii}{$\circ$}
\renewcommand{\labelenumi}{(\roman{*})}

\let\stdsection\section
\renewcommand\section{\newpage\stdsection}

% Strike through
\def\st{\bgroup \ULdepth=-.55ex \ULset}

% Maths symbols
\newcommand{\bra}{\langle}
\newcommand{\ket}{\rangle}

\newcommand{\N}{\mathbb{N}}
\newcommand{\Z}{\mathbb{Z}}
\newcommand{\Q}{\mathbb{Q}}
\renewcommand{\H}{\mathbb{H}}
\newcommand{\R}{\mathbb{R}}
\newcommand{\C}{\mathbb{C}}
\newcommand{\Prob}{\mathbb{P}}
\renewcommand{\P}{\mathbb{P}}
\newcommand{\E}{\mathbb{E}}
\newcommand{\F}{\mathbb{F}}
\newcommand{\cU}{\mathcal{U}}
\newcommand{\RP}{\mathbb{RP}}
\newcommand{\CP}{\mathbb{CP}}

\newcommand{\ph}{\,\cdot\,}

\DeclareMathOperator{\sech}{sech}
\DeclareMathOperator{\cosech}{cosech}
\DeclareMathOperator{\cosec}{cosec}

\DeclareMathOperator{\covol}{covol}
\DeclareMathOperator{\vol}{vol}

\let\Im\relax
\let\Re\relax
\DeclareMathOperator{\Im}{Im}
\DeclareMathOperator{\Re}{Re}
\DeclareMathOperator{\im}{im}
\DeclareMathOperator{\image}{image}
\DeclareMathOperator{\Ann}{Ann}

\DeclareMathOperator*{\res}{res}
\DeclareMathOperator{\Res}{Res}
\DeclareMathOperator{\Ind}{Ind}

\DeclareMathOperator{\tr}{tr}
\DeclareMathOperator{\diag}{diag}
\DeclareMathOperator{\rank}{rank}
\DeclareMathOperator{\card}{card}
\DeclareMathOperator{\spn}{span}
\DeclareMathOperator{\adj}{adj}

\DeclareMathOperator{\erf}{erf}
\DeclareMathOperator{\erfc}{erfc}

\DeclareMathOperator{\ord}{ord}
\DeclareMathOperator{\Sym}{Sym}

\DeclareMathOperator{\sgn}{sgn}
\DeclareMathOperator{\orb}{orb}
\DeclareMathOperator{\stab}{stab}
\DeclareMathOperator{\ccl}{ccl}

\DeclareMathOperator{\lcm}{lcm}
\DeclareMathOperator{\hcf}{hcf}

\DeclareMathOperator{\Int}{Int}
\DeclareMathOperator{\id}{id}

\DeclareMathOperator{\betaD}{beta}
\DeclareMathOperator{\gammaD}{gamma}
\DeclareMathOperator{\Poisson}{Poisson}
\DeclareMathOperator{\binomial}{binomial}
\DeclareMathOperator{\multinomial}{multinomial}
\DeclareMathOperator{\Bernoulli}{Bernoulli}
\DeclareMathOperator{\like}{like}

\DeclareMathOperator{\var}{var}
\DeclareMathOperator{\cov}{cov}
\DeclareMathOperator{\bias}{bias}
\DeclareMathOperator{\mse}{mse}
\DeclareMathOperator{\corr}{corr}

\DeclareMathOperator{\otp}{otp}
\DeclareMathOperator{\dom}{dom}

\DeclareMathOperator{\Root}{Root}
\DeclareMathOperator{\supp}{supp}
\DeclareMathOperator{\rel}{rel}
\DeclareMathOperator{\Hom}{Hom}
\DeclareMathOperator{\Aut}{Aut}
\DeclareMathOperator{\Gal}{Gal}
\DeclareMathOperator{\Mat}{Mat}
\DeclareMathOperator{\End}{End}
\DeclareMathOperator{\Char}{char}
\DeclareMathOperator{\ev}{ev}
\DeclareMathOperator{\St}{St}
\DeclareMathOperator{\Lk}{Lk}
\DeclareMathOperator{\disc}{disc}
\DeclareMathOperator{\Isom}{Isom}
\DeclareMathOperator{\length}{length}
\DeclareMathOperator{\energy}{energy}
\DeclareMathOperator{\area}{area}
\DeclareMathOperator{\Syl}{Syl}
\DeclareMathOperator{\cl}{cl}
\DeclareMathOperator{\fix}{fix}

\newcommand{\GL}{\mathrm{GL}}
\newcommand{\SL}{\mathrm{SL}}
\newcommand{\PGL}{\mathrm{PGL}}
\newcommand{\PSL}{\mathrm{PSL}}
\newcommand{\PSU}{\mathrm{PSU}}
\newcommand{\Or}{\mathrm{O}}
\newcommand{\SO}{\mathrm{SO}}
\newcommand{\U}{\mathrm{U}}
\newcommand{\SU}{\mathrm{SU}}

\renewcommand{\d}{\mathrm{d}}
\newcommand{\D}{\mathrm{D}}

\tikzset{->/.style = {decoration={markings,
                                  mark=at position 1 with {\arrow[scale=2]{latex'}}},
                      postaction={decorate}}}
\tikzset{<-/.style = {decoration={markings,
                                  mark=at position 0 with {\arrowreversed[scale=2]{latex'}}},
                      postaction={decorate}}}
\tikzset{<->/.style = {decoration={markings,
                                   mark=at position 0 with {\arrowreversed[scale=2]{latex'}},
                                   mark=at position 1 with {\arrow[scale=2]{latex'}}},
                       postaction={decorate}}}
\tikzset{->-/.style = {decoration={markings,
                                   mark=at position #1 with {\arrow[scale=2]{latex'}}},
                       postaction={decorate}}}
\tikzset{-<-/.style = {decoration={markings,
                                   mark=at position #1 with {\arrowreversed[scale=2]{latex'}}},
                       postaction={decorate}}}

\tikzset{circ/.style = {fill, circle, inner sep = 0, minimum size = 3}}
\tikzset{mstate/.style={circle, draw, blue, text=black, minimum width=0.7cm}}

\definecolor{mblue}{rgb}{0.2, 0.3, 0.8}
\definecolor{morange}{rgb}{1, 0.5, 0}
\definecolor{mgreen}{rgb}{0.1, 0.4, 0.2}
\definecolor{mred}{rgb}{0.5, 0, 0}

\def\drawcirculararc(#1,#2)(#3,#4)(#5,#6){%
    \pgfmathsetmacro\cA{(#1*#1+#2*#2-#3*#3-#4*#4)/2}%
    \pgfmathsetmacro\cB{(#1*#1+#2*#2-#5*#5-#6*#6)/2}%
    \pgfmathsetmacro\cy{(\cB*(#1-#3)-\cA*(#1-#5))/%
                        ((#2-#6)*(#1-#3)-(#2-#4)*(#1-#5))}%
    \pgfmathsetmacro\cx{(\cA-\cy*(#2-#4))/(#1-#3)}%
    \pgfmathsetmacro\cr{sqrt((#1-\cx)*(#1-\cx)+(#2-\cy)*(#2-\cy))}%
    \pgfmathsetmacro\cA{atan2(#2-\cy,#1-\cx)}%
    \pgfmathsetmacro\cB{atan2(#6-\cy,#5-\cx)}%
    \pgfmathparse{\cB<\cA}%
    \ifnum\pgfmathresult=1
        \pgfmathsetmacro\cB{\cB+360}%
    \fi
    \draw (#1,#2) arc (\cA:\cB:\cr);%
}
\newcommand\getCoord[3]{\newdimen{#1}\newdimen{#2}\pgfextractx{#1}{\pgfpointanchor{#3}{center}}\pgfextracty{#2}{\pgfpointanchor{#3}{center}}}

\def\Xint#1{\mathchoice
   {\XXint\displaystyle\textstyle{#1}}%
   {\XXint\textstyle\scriptstyle{#1}}%
   {\XXint\scriptstyle\scriptscriptstyle{#1}}%
   {\XXint\scriptscriptstyle\scriptscriptstyle{#1}}%
   \!\int}
\def\XXint#1#2#3{{\setbox0=\hbox{$#1{#2#3}{\int}$}
     \vcenter{\hbox{$#2#3$}}\kern-.5\wd0}}
\def\ddashint{\Xint=}
\def\dashint{\Xint-}


\begin{document}
\maketitle
{\small
\noindent Definition of a vector space (over $\R$ or $\C$), subspaces, the space spanned by a subset. Linear independence, bases, dimension. Direct sums and complementary subspaces. \hspace*{\fill} [3]

\vspace{5pt}
\noindent Linear maps, isomorphisms. Relation between rank and nullity. The space of linear maps from $U$ to $V$, representation by matrices. Change of basis. Row rank and column rank.\hspace*{\fill} [4]

\vspace{5pt}
\noindent Determinant and trace of a square matrix. Determinant of a product of two matrices and of the inverse matrix. Determinant of an endomorphism. The adjugate matrix.\hspace*{\fill} [3]

\vspace{5pt}
\noindent Eigenvalues and eigenvectors. Diagonal and triangular forms. Characteristic and minimal polynomials. Cayley-Hamilton Theorem over $\C$. Algebraic and geometric multiplicity of eigenvalues. Statement and illustration of Jordan normal form.\hspace*{\fill} [4]

\vspace{5pt}
\noindent Dual of a finite-dimensional vector space, dual bases and maps. Matrix representation, rank and determinant of dual map.\hspace*{\fill} [2]

\vspace{5pt}
\noindent Bilinear forms. Matrix representation, change of basis. Symmetric forms and their link with quadratic forms. Diagonalisation of quadratic forms. Law of inertia, classification by rank and signature. Complex Hermitian forms.\hspace*{\fill} [4]

\vspace{5pt}
\noindent Inner product spaces, orthonormal sets, orthogonal projection, $V = W \oplus W^\bot$. Gram-Schmidt orthogonalisation. Adjoints. Diagonalisation of Hermitian matrices. Orthogonality of eigenvectors and properties of eigenvalues.\hspace*{\fill} [4]}

\tableofcontents
\setcounter{section}{-1}
\section{Introduction}
Linear algebra is the study of vector spaces and linear maps between vector spaces. Unlike IA vector and matrices, most of the time, we will not assume that we already have a fixed coordinate system. In particular, we no longer treat vectors as a ``list of numbers''. Instead, we will provide an axiomatic treatment of vector spaces and the maps.

\section{Definitions}
\subsection{Definitions and examples}
We will start with a few examples of vector spaces. Intuitively, vector spaces are spaces where we can add and do scalar multiplication. After these examples, we will then come to a formal axiomatic definition afterwards.

\begin{eg}\leavevmode
  \begin{enumerate}
    \item $\R^n = \{\text{column vectors of length }n\text{ with coefficients in }\R\}$ with the usual addition and scalar multiplication is a vector space.

      An $m\times n$ matrix $A$ with coefficients in $\R$ can be viewed as a linear map from $\R^m$ to $\R^n$ via $\mathbf{v} \mapsto A\mathbf{v}$.

      This is a motivational example for vector spaces. When confused about definitions, we can often think what the definition means in terms of $\R^n$ and matrices to get some intuition.

    \item Let $X$ be a set and define $\R^X = \{f: X\to \R\}$ with addition $(f + g)(x) = f(x) + g(x)$ and scalar multiplication $(\lambda f)(x) = \lambda f(x)$.

      More generally, if $V$ is a vector space, $X$ is a set, we can define $V^X = \{f: X \to V\}$ with addition and scalar multiplication as above.
    \item Let $[a, b]\subseteq \R$ be a closed interval, then
      \[
        C([a, b], \R) = \{f\in \R^{[a,b]}: f\text{ is continuous}\}
      \]
      is a vector space, with operations as above. We also have
      \[
        C^{\infty}([a, b], \R) = \{f\in \R^{[a,b]}: f\text{ is infinitely differentiable}\}
      \]
    \item The set of $m\times n$ matrices with coefficients in $\R$ is a vector space, using the obvious operations, is a vector space.
  \end{enumerate}
\end{eg}

\begin{notation}
  We will use $\F$ to denote an arbitrary field, usually $\R$ or $\C$.
\end{notation}
Our examples have the common feature that they have notions of addition and scalar multiplication. This will be used as the definition of a vector space.

\begin{defi}[Vector space]
  An \emph{$\F$-vector space} is an abelian group $V$ together with a function $\F \times V \to V$, written $(\lambda, \mathbf{v}) \mapsto \lambda \mathbf{v}$, such that
  \begin{enumerate}
    \item $\lambda(\mu \mathbf{v}) = \lambda \mu \mathbf{v}$ for all $\lambda, \mu \in \F$, $\mathbf{v}\in V$
    \item $\lambda(\mathbf{u} + \mathbf{v}) = \lambda \mathbf{u} + \lambda \mathbf{v}$ for all $\lambda\in \F$, $\mathbf{u}, \mathbf{v}\in V$
    \item $(\lambda + \mu) \mathbf{v} = \lambda \mathbf{v} + \mu \mathbf{v}$ for all $\lambda, \mu \in \F$, $\mathbf{v}\in V$.
    \item $1\mathbf{v} = \mathbf{v}$ for all $\mathbf{v}\in V$.
  \end{enumerate}

  We always write $\mathbf{0}$ for the identity in $V$, and call this the identity. By abuse of notation, we also write $0$ for the trivial vector space $\{0\}$.
\end{defi}
As mentioned, there is in general no natural coordinate system, and no notion of length, angle or distance.

\begin{prop}
  In any vector space $V$, $0\mathbf{v} = \mathbf{0}$ for all $v\in V$, and $(-1)\mathbf{v} = -\mathbf{v}$, where $-\mathbf{v}$ is the additive inverse of $\mathbf{v}$.
\end{prop}
Proof is left as an exercise.

Similar to groups and subgroups, we can have a subspace of a vector space.
\begin{defi}[Subspace]
  If $V$ is an $\F$-vector space, then $U\subseteq V$ is an ($\F$-linear) \emph{subspace} if
  \begin{enumerate}
    \item $\mathbf{u}, \mathbf{v}\in U$ implies $\mathbf{u} + \mathbf{v} \in U$.
    \item $\mathbf{u}\in U, \lambda \in \F$ implies $\lambda u\in U$.
    \item $\mathbf{0}\in U$.
  \end{enumerate}
  These conditions can be expressed more concisely as ``$U$ is non-empty and if $\lambda, \mu\in \F, \mathbf{u}, \mathbf{v}\in U$, then $\lambda \mathbf{u} + \mu \mathbf{v}\in U$''.

  Alternatively, we can write the requirements as $U$ is also a vector space (inheriting the operations from $V$).
\end{defi}

\begin{eg}\leavevmode
  \begin{enumerate}
    \item $\{(x_1, x_2, x_3) \in \R^3: x_1 + x_2 + x_3 = t\}$ is a subspace of $\R^3$ iff $t = 0$.
    \item Let $X$ be a set. We define the \emph{support} of $f$ in $\F^X$ to by $\supp(f) = \{x\in X: f(x) \not= 0\}$. Then the set of functions with finite support forms a vector subspace. This is since $\supp (f + g) \subseteq \supp(f) \cup \supp(g)$, $\supp (\lambda f) = \supp (f)$ (for $\lambda \not= 0$) and $\supp (0) = \emptyset$.
  \end{enumerate}
\end{eg}

If we have to subspaces $U$ and $V$, there are several things we can do with them. For example, we can take the intersection $U\cap V$. However, taking the union will in general not produce a vector space. Instead, we need the sum:
\begin{defi}[Sum of subspaces]
  Suppose $U, W$ are subspaces of an $\F$ vector space $V$. The \emph{sum} of $U$ and $V$ is
  \[
    U + W = \{u + w: u\in U, w\in W\}.
  \]
\end{defi}

\begin{prop}
  Let $U, W$ be subspaces of $V$. Then $U + W$ and $U\cap W$ are subspaces.
\end{prop}

\begin{proof}
  Let $\mathbf{u}_i + \mathbf{w}_i \in U + W$, $\lambda, \mu\in \F$. Then
  \[
    \lambda(\mathbf{u}_1 + \mathbf{w}_1) + \mu(\mathbf{u}_2 + \mathbf{w}_2) = (\lambda\mathbf{u}_1 + \mu\mathbf{u}_2) + (\lambda\mathbf{w}_1 + \mu\mathbf{w}_2) \in U + W.
  \]
  Similarly, if $\mathbf{v}_i \in U\cap W$, then $\lambda \mathbf{v}_1 + \mu \mathbf{v}_2\in U$ and $\lambda \mathbf{v}_1 + \mu \mathbf{v}_2\in W$. So $\lambda \mathbf{v}_1 + \mu \mathbf{v}_2\in U\cap W$.

  Both $U\cap W$ and $U + W$ contain $\mathbf{0}$, and are non-empty. So done.
\end{proof}

\begin{defi}[Quotient spaces*]
  Suppose that $V$ is a vector space, and $U\subseteq V$ is a subspace. Then the quotient group $V/U$ can be made into a vector space called the \emph{quotient space}, where scalar multiplication is given by $(\lambda, v + U) = (\lambda v) + U$.

  This is well defined since if $v + U = w + U\in V/U$, then $v - w \in U$. Hence for $\lambda \in \F$, we have $\lambda v - \lambda w \in U$. So $\lambda v + U = \lambda w + U$.
\end{defi}
\end{document}
