\documentclass[a4paper]{article}

\def\npart {IB}
\def\nterm {Michaelmas}
\def\nyear {2015}
\def\nlecturer {N. Wickramasekera}
\def\ncourse {Analysis II}
\def\nlectures {MWF.12}
\def\nnotready {}

% Imports
\ifx \nextra \undefined
  \usepackage[pdftex,
    hidelinks,
    pdfauthor={Dexter Chua},
    pdfsubject={Cambridge Maths Notes: Part \npart\ - \ncourse},
    pdftitle={Part \npart\ - \ncourse},
  pdfkeywords={Cambridge Mathematics Maths Math \npart\ \nterm\ \nyear\ \ncourse}]{hyperref}
  \title{Part \npart\ - \ncourse}
\else
  \usepackage[pdftex,
    hidelinks,
    pdfauthor={Dexter Chua},
    pdfsubject={Cambridge Maths Notes: Part \npart\ - \ncourse\ (\nextra)},
    pdftitle={Part \npart\ - \ncourse\ (\nextra)},
  pdfkeywords={Cambridge Mathematics Maths Math \npart\ \nterm\ \nyear\ \ncourse\ \nextra}]{hyperref}

  \title{Part \npart\ - \ncourse \\ {\Large \nextra}}
\fi

\author{Lectured by \nlecturer \\\small Notes taken by Dexter Chua}
\date{\nterm\ \nyear}

\usepackage{alltt}
\usepackage{amsfonts}
\usepackage{amsmath}
\usepackage{amssymb}
\usepackage{amsthm}
\usepackage{booktabs}
\usepackage{caption}
\usepackage{enumitem}
\usepackage{fancyhdr}
\usepackage{graphicx}
\usepackage{mathtools}
\usepackage{microtype}
\usepackage{multirow}
\usepackage{pdflscape}
\usepackage{pgfplots}
\usepackage{siunitx}
\usepackage{tabularx}
\usepackage{tikz}
\usepackage{tkz-euclide}
\usepackage[normalem]{ulem}
\usepackage[all]{xy}

\pgfplotsset{compat=1.12}

\pagestyle{fancyplain}
\lhead{\emph{\nouppercase{\leftmark}}}
\ifx \nextra \undefined
  \rhead{
    \ifnum\thepage=1
    \else
      \npart\ \ncourse
    \fi}
\else
  \rhead{
    \ifnum\thepage=1
    \else
      \npart\ \ncourse\ (\nextra)
    \fi}
\fi
\usetikzlibrary{arrows}
\usetikzlibrary{decorations.markings}
\usetikzlibrary{decorations.pathmorphing}
\usetikzlibrary{positioning}
\usetikzlibrary{fadings}
\usetikzlibrary{intersections}
\usetikzlibrary{cd}

\newcommand*{\Cdot}{\raisebox{-0.25ex}{\scalebox{1.5}{$\cdot$}}}
\newcommand {\pd}[2][ ]{
  \ifx #1 { }
    \frac{\partial}{\partial #2}
  \else
    \frac{\partial^{#1}}{\partial #2^{#1}}
  \fi
}

% Theorems
\theoremstyle{definition}
\newtheorem*{aim}{Aim}
\newtheorem*{axiom}{Axiom}
\newtheorem*{claim}{Claim}
\newtheorem*{cor}{Corollary}
\newtheorem*{defi}{Definition}
\newtheorem*{eg}{Example}
\newtheorem*{fact}{Fact}
\newtheorem*{law}{Law}
\newtheorem*{lemma}{Lemma}
\newtheorem*{notation}{Notation}
\newtheorem*{prop}{Proposition}
\newtheorem*{thm}{Theorem}

\renewcommand{\labelitemi}{--}
\renewcommand{\labelitemii}{$\circ$}
\renewcommand{\labelenumi}{(\roman{*})}

\let\stdsection\section
\renewcommand\section{\newpage\stdsection}

% Strike through
\def\st{\bgroup \ULdepth=-.55ex \ULset}

% Maths symbols
\newcommand{\bra}{\langle}
\newcommand{\ket}{\rangle}

\newcommand{\N}{\mathbb{N}}
\newcommand{\Z}{\mathbb{Z}}
\newcommand{\Q}{\mathbb{Q}}
\renewcommand{\H}{\mathbb{H}}
\newcommand{\R}{\mathbb{R}}
\newcommand{\C}{\mathbb{C}}
\newcommand{\Prob}{\mathbb{P}}
\renewcommand{\P}{\mathbb{P}}
\newcommand{\E}{\mathbb{E}}
\newcommand{\F}{\mathbb{F}}
\newcommand{\cU}{\mathcal{U}}
\newcommand{\RP}{\mathbb{RP}}
\newcommand{\CP}{\mathbb{CP}}

\newcommand{\ph}{\,\cdot\,}

\DeclareMathOperator{\sech}{sech}
\DeclareMathOperator{\cosech}{cosech}
\DeclareMathOperator{\cosec}{cosec}

\DeclareMathOperator{\covol}{covol}
\DeclareMathOperator{\vol}{vol}

\let\Im\relax
\let\Re\relax
\DeclareMathOperator{\Im}{Im}
\DeclareMathOperator{\Re}{Re}
\DeclareMathOperator{\im}{im}
\DeclareMathOperator{\image}{image}
\DeclareMathOperator{\Ann}{Ann}

\DeclareMathOperator*{\res}{res}
\DeclareMathOperator{\Res}{Res}
\DeclareMathOperator{\Ind}{Ind}

\DeclareMathOperator{\tr}{tr}
\DeclareMathOperator{\diag}{diag}
\DeclareMathOperator{\rank}{rank}
\DeclareMathOperator{\card}{card}
\DeclareMathOperator{\spn}{span}
\DeclareMathOperator{\adj}{adj}

\DeclareMathOperator{\erf}{erf}
\DeclareMathOperator{\erfc}{erfc}

\DeclareMathOperator{\ord}{ord}
\DeclareMathOperator{\Sym}{Sym}

\DeclareMathOperator{\sgn}{sgn}
\DeclareMathOperator{\orb}{orb}
\DeclareMathOperator{\stab}{stab}
\DeclareMathOperator{\ccl}{ccl}

\DeclareMathOperator{\lcm}{lcm}
\DeclareMathOperator{\hcf}{hcf}

\DeclareMathOperator{\Int}{Int}
\DeclareMathOperator{\id}{id}

\DeclareMathOperator{\betaD}{beta}
\DeclareMathOperator{\gammaD}{gamma}
\DeclareMathOperator{\Poisson}{Poisson}
\DeclareMathOperator{\binomial}{binomial}
\DeclareMathOperator{\multinomial}{multinomial}
\DeclareMathOperator{\Bernoulli}{Bernoulli}
\DeclareMathOperator{\like}{like}

\DeclareMathOperator{\var}{var}
\DeclareMathOperator{\cov}{cov}
\DeclareMathOperator{\bias}{bias}
\DeclareMathOperator{\mse}{mse}
\DeclareMathOperator{\corr}{corr}

\DeclareMathOperator{\otp}{otp}
\DeclareMathOperator{\dom}{dom}

\DeclareMathOperator{\Root}{Root}
\DeclareMathOperator{\supp}{supp}
\DeclareMathOperator{\rel}{rel}
\DeclareMathOperator{\Hom}{Hom}
\DeclareMathOperator{\Aut}{Aut}
\DeclareMathOperator{\Gal}{Gal}
\DeclareMathOperator{\Mat}{Mat}
\DeclareMathOperator{\End}{End}
\DeclareMathOperator{\Char}{char}
\DeclareMathOperator{\ev}{ev}
\DeclareMathOperator{\St}{St}
\DeclareMathOperator{\Lk}{Lk}
\DeclareMathOperator{\disc}{disc}
\DeclareMathOperator{\Isom}{Isom}
\DeclareMathOperator{\length}{length}
\DeclareMathOperator{\energy}{energy}
\DeclareMathOperator{\area}{area}
\DeclareMathOperator{\Syl}{Syl}
\DeclareMathOperator{\cl}{cl}
\DeclareMathOperator{\fix}{fix}

\newcommand{\GL}{\mathrm{GL}}
\newcommand{\SL}{\mathrm{SL}}
\newcommand{\PGL}{\mathrm{PGL}}
\newcommand{\PSL}{\mathrm{PSL}}
\newcommand{\PSU}{\mathrm{PSU}}
\newcommand{\Or}{\mathrm{O}}
\newcommand{\SO}{\mathrm{SO}}
\newcommand{\U}{\mathrm{U}}
\newcommand{\SU}{\mathrm{SU}}

\renewcommand{\d}{\mathrm{d}}
\newcommand{\D}{\mathrm{D}}

\tikzset{->/.style = {decoration={markings,
                                  mark=at position 1 with {\arrow[scale=2]{latex'}}},
                      postaction={decorate}}}
\tikzset{<-/.style = {decoration={markings,
                                  mark=at position 0 with {\arrowreversed[scale=2]{latex'}}},
                      postaction={decorate}}}
\tikzset{<->/.style = {decoration={markings,
                                   mark=at position 0 with {\arrowreversed[scale=2]{latex'}},
                                   mark=at position 1 with {\arrow[scale=2]{latex'}}},
                       postaction={decorate}}}
\tikzset{->-/.style = {decoration={markings,
                                   mark=at position #1 with {\arrow[scale=2]{latex'}}},
                       postaction={decorate}}}
\tikzset{-<-/.style = {decoration={markings,
                                   mark=at position #1 with {\arrowreversed[scale=2]{latex'}}},
                       postaction={decorate}}}

\tikzset{circ/.style = {fill, circle, inner sep = 0, minimum size = 3}}
\tikzset{mstate/.style={circle, draw, blue, text=black, minimum width=0.7cm}}

\definecolor{mblue}{rgb}{0.2, 0.3, 0.8}
\definecolor{morange}{rgb}{1, 0.5, 0}
\definecolor{mgreen}{rgb}{0.1, 0.4, 0.2}
\definecolor{mred}{rgb}{0.5, 0, 0}

\def\drawcirculararc(#1,#2)(#3,#4)(#5,#6){%
    \pgfmathsetmacro\cA{(#1*#1+#2*#2-#3*#3-#4*#4)/2}%
    \pgfmathsetmacro\cB{(#1*#1+#2*#2-#5*#5-#6*#6)/2}%
    \pgfmathsetmacro\cy{(\cB*(#1-#3)-\cA*(#1-#5))/%
                        ((#2-#6)*(#1-#3)-(#2-#4)*(#1-#5))}%
    \pgfmathsetmacro\cx{(\cA-\cy*(#2-#4))/(#1-#3)}%
    \pgfmathsetmacro\cr{sqrt((#1-\cx)*(#1-\cx)+(#2-\cy)*(#2-\cy))}%
    \pgfmathsetmacro\cA{atan2(#2-\cy,#1-\cx)}%
    \pgfmathsetmacro\cB{atan2(#6-\cy,#5-\cx)}%
    \pgfmathparse{\cB<\cA}%
    \ifnum\pgfmathresult=1
        \pgfmathsetmacro\cB{\cB+360}%
    \fi
    \draw (#1,#2) arc (\cA:\cB:\cr);%
}
\newcommand\getCoord[3]{\newdimen{#1}\newdimen{#2}\pgfextractx{#1}{\pgfpointanchor{#3}{center}}\pgfextracty{#2}{\pgfpointanchor{#3}{center}}}

\def\Xint#1{\mathchoice
   {\XXint\displaystyle\textstyle{#1}}%
   {\XXint\textstyle\scriptstyle{#1}}%
   {\XXint\scriptstyle\scriptscriptstyle{#1}}%
   {\XXint\scriptscriptstyle\scriptscriptstyle{#1}}%
   \!\int}
\def\XXint#1#2#3{{\setbox0=\hbox{$#1{#2#3}{\int}$}
     \vcenter{\hbox{$#2#3$}}\kern-.5\wd0}}
\def\ddashint{\Xint=}
\def\dashint{\Xint-}


\begin{document}
\maketitle
{\small
\noindent\textbf{Uniform convergence}\\
The general principle of uniform convergence. A uniform limit of continuous functions is continuous.  Uniform convergence and termwise integration and differentiation of series of real-valued functions. Local uniform convergence of power series.\hspace*{\fill} [3]

\vspace{10pt}
\noindent\textbf{Uniform continuity and integration}\\
Continuous functions on closed bounded intervals are uniformly continuous. Review of basic facts on Riemann integration (from Analysis I). Informal discussion of integration of complex-valued and $\R^n$-valued functions of one variable; proof that $\|\int_a^b f(x) \;\d x\| \leq \int_a^b \|f(x)\|\;\d x$.\hspace*{\fill} [2]

\vspace{10pt}
\noindent\textbf{$\R^n$ as a normed space}\\
Definition of a normed space. Examples, including the Euclidean norm on $\R^n$ and the uniform norm on $\mathcal{C}[a, b]$. Lipschitz mappings and Lipschitz equivalence of norms. The Bolzano-Weierstrass theorem in $\R^n$. Completeness. Open and closed sets. Continuity for functions between normed spaces. A continuous function on a closed bounded set in $\R^n$ is uniformly continuous and has closed bounded image. All norms on a finite-dimensional space are Lipschitz equivalent.\hspace*{\fill} [5]

\vspace{10pt}
\noindent\textbf{Differentiation from $\R^m$ to $\R^n$}\\
Definition of derivative as a linear map; elementary properties, the chain rule. Partial derivatives; continuous partial derivatives imply differentiability. Higher-order derivatives; symmetry of mixed partial derivatives (assumed continuous). Taylor's theorem. The mean value inequality. Path-connectedness for subsets of $\R^n$; a function having zero derivative on a path-connected open subset is constant.\hspace*{\fill} [6]

\vspace{10pt}
\noindent\textbf{Metric spaces}\\
Definition and examples. *Metrics used in Geometry*. Limits, continuity, balls, neighbourhoods, open and closed sets.\hspace*{\fill} [4]

\vspace{10pt}
\noindent\textbf{The Contraction Mapping Theorem}\\
The contraction mapping theorem. Applications including the inverse function theorem (proof of continuity of inverse function, statement of differentiability). Picard's solution of differential equations.\hspace*{\fill} [4]}

\tableofcontents

\section{Uniform convergence}
Here we are concerned with sequences of functions. In general, let $E$ be any set, and $f_n: E\to \R$ for $n = 1, 2, \cdots$. Suppose for all $x\in E$, the sequence $(f_n(x))$ of real numbers converges (to some element in the real line).

We can define $f: E \to \R$ by $f(x) = \lim\limits_{n \to \infty} f_n(x)$. We say that $f_n$ converges \emph{pointwise} to $f$.

\begin{defi}[Pointwise convergence]
  The sequence $f_n$ converges \emph{pointwise} to $f$ if
  \[
    f(x) = \lim_{n\to \infty} f(x)
  \]
  for all $x$.
\end{defi}

Ideally, We want to deduce properties of $f$ from properties of $f_n$. For example, it would be great that continuity of all $f_n$ implies continuity of $f$, and similarly for integrability and values of derivatives and integrals. However, it turns out we cannot. The notion of pointwise convergence is too weak. We will look at many examples where $f$ fails to preserve the properties of $f_n$.

\begin{eg}
  Let $f_n: [-1, 1] \to \R$ be defined by $f_n(x) = x^{1/(2n + 1)}$. These are all continuous, but the limit function is
  \[
    f_n(x) \to f(x) =
    \begin{cases}
      1 & 0 < x \leq 1\\
      0 & x = 1\\
      -1 & -1 \leq x < 0
    \end{cases},
  \]
  which is not continuous.
\end{eg}

\begin{eg}
  Let $f_n: [0, 1] \to \R$ be the piecewise linear function formed by joining $(0, 0), (\frac{1}{n}, n), (\frac{2}{n}, 0)$ and $(1, 0)$.
  \begin{center}
    \begin{tikzpicture}
      \draw [->] (0, 0) -- (4, 0) node [right] {$x$};
      \draw [->] (0, 0) -- (0, 3) node [above] {$y$};
      \draw [thick, mred] (0, 0) -- (0.7, 2) -- (1.4, 0) -- (4, 0);
      \node [anchor = north east] {$0$};
      \node at (1.4, 0) [below] {$\frac{2}{n}$};
      \draw [dashed] (0.7, 0) node [below] {$\frac{1}{n}$} -- (0.7, 2) -- (0, 2) node [left] {$n$};
    \end{tikzpicture}
  \end{center}
  The pointwise limit of this function is $f_n(x) \to f(x) = 0$. However, we have
  \[
    \int_0^a f_n(x)\;\d x = 1\text{ for all }n;\quad \int_0^1 f(x) \;\d x = 0.
  \]
  So the limit of the integral is not the integral of the limit.
\end{eg}

\begin{eg}
  Let $f_n: [0, 1] \to \R$ be defined as
  \[
    f_n (x) =
    \begin{cases}
      1 & n!x \in \Z\\
      0 & \text{otherwise}
    \end{cases}
  \]
  Since $f_n$ has finitely many discontinuities, it is Riemann integrable. However, the limit is
  \[
    f_n(x) \to f(x) =
    \begin{cases}
      1 & x\in \Q\\
      0 & x\not\in \Q
    \end{cases}
  \]
  which is not integrable. So integrability of a function is not preserved by pointwise limits.
\end{eg}
This suggests that we need a stronger notion of convergence. Of course, we don't want this notion to be too strong. For example, we could define $f_n \to f$ to mean ``$f_n = f$ for all sufficiently large $n$'', then any property common to $f_n$ is obviously inherited by the limit. However, this is clearly silly since only the most trivial sequences would converge.

Hence we want to find a middle ground between the two cases - a notion of convergence that is sufficiently strong to preserve most interesting properties, without being too trivial. This is uniform convergence.

\begin{defi}[Uniform convergence]
  A sequence of functions $f_n: E\to \R$ converges \emph{uniformly} to $f$ if
  \[
    (\forall \varepsilon)(\exists N)(\forall x)(\forall n > N)\; |f_n(x) - f(x)| < \varepsilon.
  \]
  Alternatively, we can say
  \[
    (\forall \varepsilon)(\exists N)(\forall n > N)\; \sup_{x\in E} |f_n(x) - f(x)| < \varepsilon.
  \]
\end{defi}
We can compare this with the definition of pointwise convergence:
\[
  (\forall \varepsilon)(\forall x)(\exists N)(\forall n > N)\;  |f_n(x) - f(x)| < \varepsilon.
\]
The only difference is in where there $(\forall x)$ sits, and this is what makes all the difference. Uniform convergence requires that there is an $N$ that works for \emph{every} $x$, while pointwise convergence just requires that for each $x$, we can find an $N$ that works.

It should now be clear from definition that if $f_n \to f$ uniformly, then $f_n \to f$ pointwise. We will show that the converse is false:
\begin{eg}
  Again consider our first example, where $f_n: [-1, 1] \to \R$ is defined by $f_n(x) = x^{1/(2n + 1)}$. If the uniform limit existed, then it must be given by
  \[
    f_n(x) \to f(x) =
    \begin{cases}
      1 & 0 < x \leq 1\\
      0 & x = 1\\
      -1 & -1 \leq x < 0
    \end{cases},
  \]
  since uniform convergence implies pointwise convergence.

  We will show that we don't have uniform convergence. Pick $\varepsilon = \frac{1}{4}$. Then for each $n$, $x = 2^{-(2n + 1)}$ will have $f_n(x) = \frac{1}{2}$, $f(x) = 1$. So there is some $x$ such that $|f_n(x) - f(x)| > \varepsilon$. So $f_n \not\to f$ uniformly.
\end{eg}

\begin{eg}
  Let $f_n: \R \to \R$ be defined by $f_n (x) = \frac{x}{n}$. Then $f_n(x) \to f(x) = 0$ pointwise. However, this convergence is not uniform in $\R$ since $|f_n(x) - f(x)| = \frac{|x|}{n}$, and this can be arbitrarily large for any $n$.

  However, if we restrict $f_n$ to a bounded domain, then the convergence is uniform. Let the domain be $[-a, a]$ for some positive, finite $a$. Then
  \[
    \sup |f_n(x) - f(x)| = \frac{|x|}{n} \leq \frac{a}{n}.
  \]
  So given $\varepsilon$, pick $N$ such that $N > \frac{a}{\varepsilon}$, and we are done.
\end{eg}

Recall that for sequences of normal numbers, we have normal convergence and Cauchy convergence, which we proved to be the same. Then clearly pointwise convergence and pointwise Cauchy convergence of functions are equivalent. We will now look into the case of uniform convergence.

\begin{defi}[Uniformly Cauchy sequence]
  A sequence $f_n: E\to \R$ of functions is \emph{uniformly Cauchy} if
  \[
    (\forall \varepsilon > 0)(\exists N)(\forall m,n > N)\;\sup_{x\in E}|f_n(x) - f_m(x)| < \varepsilon.
  \]
\end{defi}

Our first theorem will be that uniform Cauchy convergence and uniform convergence are equivalent.

\begin{thm}[]
  Let $f_n: E\to \R$ be a sequence of functions. Then $(f_n)$ converges uniformly if and only if $(f_n)$ is uniformly Cauchy.
\end{thm}

\begin{proof}
  First suppose that $f_n \to f$ uniformly. Given $\varepsilon$, we know that there is some $N$ such that
  \[
    (\forall n > N)\; \sup_{x\in E} |f_n(x) - f(x)| < \frac{\varepsilon}{2}.
  \]
  Then if $n, m > N$, $x\in E$ we have
  \[
    |f_n(x) - f_m(x)| \leq |f_n(x) - f(x)| + |f_m(x) - f(x)| < \varepsilon.
  \]
  So done.

  Now suppose $(f_n)$ is uniformly Cauchy. Then $(f_n(x))$ is Cauchy for all $x$. So it converges. Let
  \[
    f(x) = \lim_{n\to \infty}f_n(x).
  \]
  We want to show that $f_n \to f$ uniformly. Given $\varepsilon > 0$, choose $N$ such that whenever $n, m > N$, $x\in E$, we have $|f_n(x) - f_m(x)| < \frac{\varepsilon}{2}$. Letting $m\to \infty$, $f_m(x) \to f(x)$. So we have $|f_n(x) - f(x)| \leq \frac{\varepsilon}{2} < \varepsilon$. So done.
\end{proof}

The important theorem here is that continuity is conserved by uniform convergence.
\begin{thm}[Uniform convergence and continuity]
  Let $f_n, f: E\to \R$, where $E\subseteq \R$. If $f_n \to f$ uniformly and $f_n$ are all continuous at some point $x\in E$ for all $n$, then $f$ is also continuous at $x$.

  In particular, if $f_n$ are continuous everywhere, then $f$ is continuous everywhere.
\end{thm}
This can be concisely phrased as ``the uniform limit of continuous functions is continuous''.

\begin{proof}
  Let $\varepsilon > 0$. Choose $N$ such that for all $n \geq N$, we have
  \[
    \sup_{y\in E}|f_n(y) - f(y)| < \varepsilon.
  \]
  Since $f_N$ is continuous at $x$, there is some $\delta$ such that
  \[
    |x - y| < \delta \Rightarrow |f_N(x) - f_N(y)| < \varepsilon.
  \]
  Then for each $y$ such that $|x - y| < \delta$, we have
  \[
    |f(x) - f(y)| \leq |f(x) - f_N(x)| + |f_N(x) - f_N(y)| + |f_N(y) - f(y)| < 3\varepsilon.
  \]
\end{proof}

\begin{thm}[Uniform convergnece and integrals]
  Let $f_n, f: [a, b]\to \R$ be Riemann integrable, with $f_n \to f$ uniformly. Then
  \[
    \int_a^b f_n(t)\;\d t \to \int_a^b f(t)\;\d t.
  \]
\end{thm}

\begin{proof}
  We have
  \begin{align*}
    \left|\int_a^b f_n(t) \;\d t - \int_a^b f(t)\;\d t\right| &= \left|\int_a^b f_n(t) - f(t))\;\d t\right|\\
    &\leq \int_a^b |f_n(t) - f(t)|\;\d t\\
    &\leq \sup_{t\in [a, b]}|f_n(t) - f(t)| (b - a)\\
    &\to 0\text{ as }n\to \infty
  \end{align*}
\end{proof}
This is really the easy part. What we would also want to prove is that if $f_n$ is integrable, $f_n \to f$ uniformly, then $f$ is integrable. We will come to this later on.

So far so good. However, the relationship between uniform convergence and differentiability, things are more subtle. The uniform limit of differential functions need not be differentiable. Even if it were, the limit of the derivative is not necessarily the same as the derivative of the limit, even if we just want pointwise convergence of the derivative.

\begin{eg}
  Let $f_n = |x|^{1 + 1/n}$, where $x \in [-1, 1]$. Then $f_n \to f$ uniformly (exercise).

  Each $f_n$ is differentiable - this is obvious at $x \not= 0$, and at $x = 0$, the derivative is
  \[
    f_n'(0) = \lim_{x \to 0}\frac{f_n (x) - f_n(0)}{x} = \lim_{x \to 0} \sgn(x) |x|^{1/n} = 0
  \]
  However, the limit $f$ is not differentiable at $x = 0$.
\end{eg}

\begin{eg}
  Let
  \[
    f_n(x) = \frac{\sin nx}{\sqrt{n}}
  \]
  for all $x\in \R$. Then
  \[
    \sup_{x\in \R}|f_n (x)| \leq \frac{1}{\sqrt{n}} \to 0.
  \]
  So $f_n \to f = 0$ uniformly in $\R$. However, the derivative is
  \[
    f_n'(x) = \sqrt{n} \cos nx,
  \]
  which does not converge to $0$, eg. at $x = 0$.
\end{eg}

Hence, for differentiability to play nice, we need a condition even stronger than uniform convergence.
\begin{thm}
  Let $f_n: [a, b]\to \R$ be a sequence of functions differentiable on $[a, b]$ (at the end points $a$, $b$, this means that the one-sided derivatives exist). Suppose the following holds:
  \begin{enumerate}
    \item For some $c\in [a, b]$, $f_n(c)$ converges.
    \item The sequence of derivatives $(f_n')$ converges uniformly on $[a, b]$.
  \end{enumerate}
  Then $(f_n)$ converges uniformly on $[a, b]$, and if $f = \lim f_n$, then $f$ is differentiable with derivative $f'(x) = \lim f_n'(x)$.
\end{thm}
Note that we do \emph{not} assume that $f_n'$ are continuous or even Riemann integrable. If they are, then the proof is much easier!

\begin{proof}
  If we are given a specific sequence of functions and are asked to prove that they converge uniformly, we usually take the pointwise limit and show that the convergence is uniform. However, given a general function, this is usually not helpful. Instead, we can use the Cauchy criterion by showing that the sequence is uniformly Cauchy.

  We want to find an $N$ such that $n, m> N$ implies $\sup|f_n - f_m| < \varepsilon$. We want to relate this to the derivatives. We might want to use the fundamental theorem of algebra for this. However, we don't know that the derivative is integrable! So instead, we go for the mean value theorem.

  Fix $x\in [a, b]$. We apply the mean value theorem to $f_n - f_m$ to get
  \[
    (f_n - f_m)(x) - (f_n - f_m)(c) = (x - c)(f'_n - f'_m)(t)
  \]
  for some $t\in (x, c)$.

  Taking the supremum and rearranging terms, we obtain
  \[
    \sup_{x\in [a, b]}|f_n(x) - f(x)| \leq |f_n(c) - f_m(c)| + (b - a)\sup_{t\in [a, b]}|f_n'(t) - f_m'(t)|.
  \]
  So given any $\varepsilon$, since $f_n'$ and $f_n(c)$ converge and are hence Cauchy, there is some $N$ such that for any $n, m \geq N$,
  \[
    \sup_{t \in [a, b]} |f_n'(t) - f_m'(t)| < \varepsilon, \quad |f_n(c) - f_m(c)| < \varepsilon.
  \]
  Hence we obtain
  \[
    n, m \geq N \Rightarrow  \sup_{x\in [a, b]}|f_n(x) - f_m(x)| < (1 + b - a) \varepsilon.
  \]
  So by the Cauchy criterion, we know that $f_n$ converges uniformly. Let $f = \lim f_n$.

  Now we have to check differentiability. Let $f_n' \to h$. For any fixed $y\in [a, b]$, define
  \[
    g_n(x) =
    \begin{cases}
      \frac{f_n(x) - f_n(y)}{x - y} & x \not= y\\
      f_n'(y) & x = y
    \end{cases}
  \]
  Then by definition, $f_n$ is differentiable at $y$ iff $g_n$ is continuous at $y$. Also, define
  \[
    g(x) =
    \begin{cases}
      \frac{f(x) - f(y)}{x - y} & x \not= y\\
      h'(y) & x = y
    \end{cases}
  \]
  Then $f$ is differentiable with derivative $h$ at $y$ iff $g$ is continuous at $y$. However, we know that $g_x \to g$ pointwise on $[a, b]$, and we know that $g_x$ are all continuous. To conclude that $g$ is continuous, we have to show that the convergence is uniform. To show that $g_x$ congerges uniformly, we rely on the Cauchy criterion and the mean value theorem.

  For $x \not = y$, we know that
  \[
    g_n(x) - g_m(x) = \frac{(f_n - f_m)(x) - (f_n - f_m)(y)}{x - y} = (f'_n - f'_m)(t)
  \]
  for some $t \in [x, y]$.

  Let $\varepsilon > 0$. Since $f'$ converges uniformly, there is some $N$ such that for all $x\not= y$, $n, m > N$, we have
  \[
    |g_n(x), g_m(x)| \leq \sup |(f'_n - f'_m)(t)| < \varepsilon.
  \]
  For $x = y$, this is still true since $|g_n(y) - g_m(y)| = |f_n'(y) - f_m'(y)|$ by definition. So
  \[
    n, m\geq N \Rightarrow \sup_{[a, b]}|g_n - g_m| < \varepsilon,
  \]
  ie. $g_n$ converges uniformly. Hence the limit function $g$ is continuous, in particular at $x = y$. So $f$ is differentiable at $y$ and $f'(y) = h(y) = \lim f_n'(y)$.
\end{proof}

If we assume additionally that $f_n'$ are continuous, then there is an easy proof of this theorem. By the fundamental theorem of calculus, we have
\[
  f_n(x) = f_n(c) + \int_c^x f_n'(t)\;\d t.\tag{*}
\]
Then we get that
\begin{align*}
  \sup_{[a, b]}|f_n(x) - f_m(x)| &\leq |f_n(c) - f_m(c)| + \sup_{x\in [a, b]}\left|\int_c^x (f_n'(t) - f_m'(t))\;\d t\right|\\
  &\leq |f_n(c) - f_m(c)| + (b - a)\sup_{t\in [a, b]}|f_n'(t) - f_m'(t)|\\
  &< \varepsilon
\end{align*}
for sufficienlty large $n, m> N$.

So by the Cauchy criterion, $f_n \to f$ uniformly for some function $f: [a, b] \to \R$.

Since the $f_n'$ are continuous, $h = \lim\limits_{n\to \infty} f_n'$ is continuous and hence integrable. Taking the limit of $(*)$, we get
\[
  f(x) = f(c) + \int_c^x h(t)\;\d t.
\]
Then the fundamental theorem of calculus says that $f$ is differentiable and $f'(x) = h(x) = \lim f_n'(x)$. So done.

Finally, we have a small proposition that can come handy.
\begin{prop}\leavevmode
  \begin{enumerate}
    \item Let $f_n, g_n: E\to \R$, be sequences, and $f_n \to f$, $g_n \to g$ uniformly on $E$. Then for any $a, b\in \R$, $af_n + bg_n \to af + bg$ uniformly.
    \item Let $f_n \to f$ uniformly, and let $g: E\to \R$ is bounded. then $gf_n: E\to \R$ converges uniformly to $gf$.
  \end{enumerate}
\end{prop}

\begin{proof}\leavevmode
  \begin{enumerate}
    \item Easy exercise.
    \item Say $|g(x)| < M$ for all $x\in E$. Then
    \[
      |(gf_n)(x) - (gf)(x)| \leq M |f_n(x) - f(x)|.
    \]
    So
    \[
      \sup_E |gf_n - gf| \leq M \sup_E |f_n - f| \to 0.
    \]
  \end{enumerate}
\end{proof}
Note that (ii) is false without assuming boundedness. An easy example is to take $f_n = \frac{1}{n}$, $x\in \R$, and $g(x) = x$. Then $x_n \to 0$ uniformly, but $(gf_n)(x) = \frac{x}{n}$ does not converge.

\section{Series of functions}
\subsection{Convergence of series}
Recall that in Analysis I, we studied the convergence of a series of numbers. Here we will look at a series of \emph{functions}.
\begin{defi}[Convergence of series]
  Let $g_n; E \to \R$ be a sequence of functions. Then we say the series $\sum_{n = 1}^\infty g_n$ converges at a point $x \in E$ if the sequence of partial sums
  \[
    f_n = \sum_{j = 1}^ng_j
  \]
  converges at $x$. The series converges uniformly if $f_n$ converges uniformly.
\end{defi}

\begin{defi}[Absolute convergence]
  $\sum g_n$ converges \emph{absolutely} at a point $x\in E$ if $\sum |g_n|$ converges at $x$.

  $\sum g_n$ converges \emph{absolutely uniformly} if $\sum |g_n|$ converges uniformly.
\end{defi}

\begin{prop}
  Let $g_n: E\to \R$. If $\sum g_n$ converges absolutely uniformly, then $\sum g_n$ converges uniformly.
\end{prop}

\begin{proof}
  Again, we don't have a candidate for the limit. So we use the Cauchy criterion.

  Let $f_n = \sum\limits_{j = 1}^n g_j$ and $h_n(x) = \sum\limits_{j = 1}^n |g_j|$ be the partial sums. Then for $n > m$, we have
  \[
    |f_n(x) - f_m(x)| = \left|\sum_{j = m + 1}^n g_j(x)\right| \leq \sum_{j = m + 1}^n |g_j(x)| = |h_n(x) - h_m(x)|.
  \]
  By hypothesis, we have
  \[
    \sup_{x\in E}|h_n(x) - h_m(x)| \to 0\text{ as }n, m\to \infty.
  \]
  So we get
  \[
    \sup_{x\in E}|f_n(x) - f_m(x)| \to 0\text{ as }n, m\to \infty.
  \]
  So the result follows from the Cauchy criteria.
\end{proof}
It is important to remember that uniform convergence plus absolute pointwise convergence does not imply absolute uniform convergence.

\begin{eg}
  Consider the series
  \[
    \sum_{n = 1}^\infty \frac{(-1)^n}{n}x^n.
  \]
  This converges absolutely for every $x\in [0, 1)$ since it is bounded by the geometric series. In fact, it converges \emph{uniformly} on $[0, 1)$ (see example sheet). However, this does \emph{not} converge absolutely uniformly on $[0, 1)$.

  We can consider the difference in partial sums
  \[
    \sum_{j = m}^n \left|\frac{(-1)^j}{j}x^j\right| = \sum_{j = m}^n \frac{1}{j}|x|^j \geq \left(\frac{1}{m} + \frac{1}{m + 1} + \cdots + \frac{1}{n}\right)|x|^n.
  \]
  For each $N$, we can make this difference large enough by picking a really large $n$, and then making $x$ close enough to $1$. So the supremum is unbounded.
\end{eg}

\begin{thm}[Weierstrass M-test]
  Let $g_n: E \to \R$ be a sequence of functions. Suppose there is some sequence $M_n$ such that for all $n$, we have
  \[
    \sup_{x\in E}|g_n (x)| \leq M_n.
  \]
  If $\sum M_n$ converges, then $\sum g_n$ converges absolutely uniformly.
\end{thm}

\begin{proof}
  Let $f_n = \sum\limits_{j = 1}^n |g_j|$ be the partial sums. Then for $n > m$, we have
  \[
    |f_n(x) - f_m(x)| = \sum_{j = m + 1}^n |g_j(x)| \leq \sum_{j = m + 1}^n M_j.
  \]
  Taking supremum, we have
  \[
    \sup|f_n(x) - f_m(x)| \leq  \sum_{j = m + 1}^n M_j \to 0\text{ as }n, m\to \infty.
  \]
  So done by the Cauchy criterion.
\end{proof}

\subsection{Power series}
\begin{thm}[]
  Let $\sum\limits_{n = 0}^\infty c_n (x - a)^n$ be a real power series. Then there exists a unique number $R\in [0, +\infty]$ (called the radius of convergence) such that
  \begin{enumerate}
    \item If $|x - a| < R$, then $\sum c_n (x - a)^n$ converges absolutely.
    \item If $|x - a| > R$, then $\sum c_n (x - a)^n$ diverges.
    \item If $R > 0$ and $0 < r < R$, then $\sum c_n (x - a)^n$ converges absolutely uniformly on $[a - r, a + r]$. We say that the sum converges locally absolutely uniformly inside circle of convergence, ie. for every point $y\in (a - R, a + R)$, there is some open interval around $y$ on which the sum converges absolutely uniformly.
  \end{enumerate}
  These results hold for complex power series as well, but for concreteness we will just do it for real series.
\end{thm}
Note that the first two statements are things we already know from IA Analysis I, and we are not going to prove them.
\begin{proof}
  See Analysis I for (i) and (ii).

  For (iii), note that from (i), taking $x = a - r$, we know that $\sum |c_n| r^n$ is convergent. But we know that if $x\in [a - r, a + r]$, then
  \[
    |c_n (x - a)^n| \leq |c_n| r^n.
  \]
  So the result follows from the Weierstrass M-test by taking $M_n = |c_n| r^n$.
\end{proof}
Note that uniform convergence need not hold on the entire interval of convergence.
\begin{eg}
  Consider $\sum x^n$. This converges for $x\in (-1, 1)$, but uniform convergence fails on $(-1, 1)$ since the tail
  \[
    \sum_{j = m}^n x^j = x^n \sum_{j = 0}^{n - m} x^j \geq \frac{x^m}{1 - x}.
  \]
  This is not uniformly small since we can make this large by picking $x$ close to $1$.
\end{eg}

\begin{thm}[Termwise differentiation of power series]
  Suppose $\sum c_n (x - n)^n$ is a real power series with radius of convergence $R > 0$. Then
  \begin{enumerate}
    \item The ``derived series''
      \[
        \sum_{n = 1}^\infty n c_n (x - a)^{n - 1}
      \]
      has radius of convergence $R$.
    \item The function defined by $f(x) = \sum c_n (x - a)^n$, $x\in (a - R, a + R)$ is differentiable with derivative $f'(x) = \sum n c_n (x - a)^{n - 1}$ within the (open) circle of convergence.
  \end{enumerate}
\end{thm}

\begin{proof}\leavevmode
  \begin{enumerate}
    \item Let $R_1$ be the radius of convergence of the derived series. Since
      \[
        |c_n(x - a)^n| = |c_n||x - a|^{n - 1}|x - a| \leq |n c_n (x - a)^{n - 1}| |x - a|.
      \]
      Hence if the derived series $\sum n c_n(x - a)^{n - 1}$ converges absolutely for some $x$, then so does $\sum c_n (x - a)^n$. So $R_1 \leq R$.

      Suppose that the inequality is strict, ie. $R_1 < R$, then there are $r_1, r$ such that $R_1 < r_1 < r < R$, where $\sum n|c_n| r_1^{n - 1}$ diverges while $\sum |c_n| r^n$ diverges. But this cannot be true since $n|c_n| r_1^{n - 1} \leq |c_n| r^n$ for sufficiently large $n$.

    \item Let $f_n(x) = \sum\limits_{j = 0}^n c_j (x - a)^j$. Then $f_n '(x) = \sum\limits_{j = 1}^n j c_j (x - a)^{j - 1}$. We want to use the result that the derivative of limit is limit of derivative. This requires that $f_n$ converges at a point, and that $f_n'$ converges uniformly. The first is obviously true, and we know that $f_n'$ converges uniformly on $[a - r, a + r]$ for any $r < R$. So for each $x_0$, there is some interval containing $x_0$ on which $f_n'$ is convergent. So on this interval, we know that
      \[
        f(x) = \lim_{n \to \infty}f_n (x)
      \]
      is differentiable with
      \[
        f'(x) = \lim_{n \to \infty}f_n'(x) = \sum_{j = 1}^\infty jc_j (x - a)^j.
      \]
      In particular,
      \[
        f'(x_0) =  \sum_{j = 1}^\infty jc_j (x_0 - a)^j.
      \]
      Since this is true for all $x_0$, the result follows.
  \end{enumerate}
\end{proof}

\section{Uniform continuity and integration}
\subsection{Uniform continuity}
Recall that we had a rather weak notion of convergence, known as pointwise convergence, and then promoted it to \emph{uniform convergence}. The process of this promotion is to replace the condition ``for each $x$, we can find an $\varepsilon$'' to ``we can find an $\varepsilon$ that works for each $x$''. We are going to do the same for continuity to obtain uniform continuity.

\begin{defi}[Uniform continuity]
  Let $E \subseteq \R$ and $f: E\to \R$. We say that $f$ is \emph{uniformly continuous} on $E$ if
  \[
    (\forall \varepsilon)(\exists \delta > 0)(\forall x)(\forall y)\; |x - y| < \delta \Rightarrow |f(x) - f(y)| < \varepsilon.
  \]
\end{defi}
Compare this to the definition of continuity:
\[
  (\forall \varepsilon)(\forall x)(\exists \delta > 0)(\forall y)\; |x - y| < \delta \Rightarrow |f(x) - f(y)| < \varepsilon.
\]
Again, we have shifted the $(\forall x)$ out of the $(\exists \delta)$ quantifier. The difference is that in regular continuity, $\delta$ can depend on our choice of $x$, but in uniform continuity, it only depends on $y$. Again, clearly a uniformly continuous function is continuous.

A very important theorem is
\begin{thm}[]
  Any continuous function on a closed, bounded interval is uniformly continuous.
\end{thm}

\begin{proof}
  We are going to prove by contradiction. Suppose $f: [a, b] \to \R$ not uniformly continuous. Since $f$ is not uniformly continuous, there is some $\varepsilon > 0$ such that for all $\delta = \frac{1}{n}$, there is some $x_n, y_n$ such that $|x_n - y_n| < \frac{1}{n}$ but $|f(x_n) - f(y_n)| > \varepsilon$.

  Since we are on a closed, bounded interval, by Bolazno-Weierstrass, $(x_n)$ has a convergent subsequence $(x_{n_i}) \to x$. Then we also have $y_{n_i}\to x$. So by continuity, we must have $f(x_{n_i}) \to f(x)$ and $f(y_{n_i}) \to f(x)$. But $|f(x_{n_i}) - f(y_{n_i})| > \varepsilon$ for all $n_i$. This is a contradiction.
\end{proof}
\end{document}
