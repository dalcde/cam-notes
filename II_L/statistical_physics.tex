\documentclass[a4paper]{article}

\def\npart {II}
\def\nterm {Lent}
\def\nyear {2017}
\def\nlecturer {H. S. Reall}
\def\ncourse {Statistical Physics}

% Imports
\ifx \nextra \undefined
  \usepackage[pdftex,
    hidelinks,
    pdfauthor={Dexter Chua},
    pdfsubject={Cambridge Maths Notes: Part \npart\ - \ncourse},
    pdftitle={Part \npart\ - \ncourse},
  pdfkeywords={Cambridge Mathematics Maths Math \npart\ \nterm\ \nyear\ \ncourse}]{hyperref}
  \title{Part \npart\ - \ncourse}
\else
  \usepackage[pdftex,
    hidelinks,
    pdfauthor={Dexter Chua},
    pdfsubject={Cambridge Maths Notes: Part \npart\ - \ncourse\ (\nextra)},
    pdftitle={Part \npart\ - \ncourse\ (\nextra)},
  pdfkeywords={Cambridge Mathematics Maths Math \npart\ \nterm\ \nyear\ \ncourse\ \nextra}]{hyperref}

  \title{Part \npart\ - \ncourse \\ {\Large \nextra}}
\fi

\author{Lectured by \nlecturer \\\small Notes taken by Dexter Chua}
\date{\nterm\ \nyear}

\usepackage{alltt}
\usepackage{amsfonts}
\usepackage{amsmath}
\usepackage{amssymb}
\usepackage{amsthm}
\usepackage{booktabs}
\usepackage{caption}
\usepackage{enumitem}
\usepackage{fancyhdr}
\usepackage{graphicx}
\usepackage{mathtools}
\usepackage{microtype}
\usepackage{multirow}
\usepackage{pdflscape}
\usepackage{pgfplots}
\usepackage{siunitx}
\usepackage{tabularx}
\usepackage{tikz}
\usepackage{tkz-euclide}
\usepackage[normalem]{ulem}
\usepackage[all]{xy}

\pgfplotsset{compat=1.12}

\pagestyle{fancyplain}
\lhead{\emph{\nouppercase{\leftmark}}}
\ifx \nextra \undefined
  \rhead{
    \ifnum\thepage=1
    \else
      \npart\ \ncourse
    \fi}
\else
  \rhead{
    \ifnum\thepage=1
    \else
      \npart\ \ncourse\ (\nextra)
    \fi}
\fi
\usetikzlibrary{arrows}
\usetikzlibrary{decorations.markings}
\usetikzlibrary{decorations.pathmorphing}
\usetikzlibrary{positioning}
\usetikzlibrary{fadings}
\usetikzlibrary{intersections}
\usetikzlibrary{cd}

\newcommand*{\Cdot}{\raisebox{-0.25ex}{\scalebox{1.5}{$\cdot$}}}
\newcommand {\pd}[2][ ]{
  \ifx #1 { }
    \frac{\partial}{\partial #2}
  \else
    \frac{\partial^{#1}}{\partial #2^{#1}}
  \fi
}

% Theorems
\theoremstyle{definition}
\newtheorem*{aim}{Aim}
\newtheorem*{axiom}{Axiom}
\newtheorem*{claim}{Claim}
\newtheorem*{cor}{Corollary}
\newtheorem*{defi}{Definition}
\newtheorem*{eg}{Example}
\newtheorem*{fact}{Fact}
\newtheorem*{law}{Law}
\newtheorem*{lemma}{Lemma}
\newtheorem*{notation}{Notation}
\newtheorem*{prop}{Proposition}
\newtheorem*{thm}{Theorem}

\renewcommand{\labelitemi}{--}
\renewcommand{\labelitemii}{$\circ$}
\renewcommand{\labelenumi}{(\roman{*})}

\let\stdsection\section
\renewcommand\section{\newpage\stdsection}

% Strike through
\def\st{\bgroup \ULdepth=-.55ex \ULset}

% Maths symbols
\newcommand{\bra}{\langle}
\newcommand{\ket}{\rangle}

\newcommand{\N}{\mathbb{N}}
\newcommand{\Z}{\mathbb{Z}}
\newcommand{\Q}{\mathbb{Q}}
\renewcommand{\H}{\mathbb{H}}
\newcommand{\R}{\mathbb{R}}
\newcommand{\C}{\mathbb{C}}
\newcommand{\Prob}{\mathbb{P}}
\renewcommand{\P}{\mathbb{P}}
\newcommand{\E}{\mathbb{E}}
\newcommand{\F}{\mathbb{F}}
\newcommand{\cU}{\mathcal{U}}
\newcommand{\RP}{\mathbb{RP}}
\newcommand{\CP}{\mathbb{CP}}

\newcommand{\ph}{\,\cdot\,}

\DeclareMathOperator{\sech}{sech}
\DeclareMathOperator{\cosech}{cosech}
\DeclareMathOperator{\cosec}{cosec}

\DeclareMathOperator{\covol}{covol}
\DeclareMathOperator{\vol}{vol}

\let\Im\relax
\let\Re\relax
\DeclareMathOperator{\Im}{Im}
\DeclareMathOperator{\Re}{Re}
\DeclareMathOperator{\im}{im}
\DeclareMathOperator{\image}{image}
\DeclareMathOperator{\Ann}{Ann}

\DeclareMathOperator*{\res}{res}
\DeclareMathOperator{\Res}{Res}
\DeclareMathOperator{\Ind}{Ind}

\DeclareMathOperator{\tr}{tr}
\DeclareMathOperator{\diag}{diag}
\DeclareMathOperator{\rank}{rank}
\DeclareMathOperator{\card}{card}
\DeclareMathOperator{\spn}{span}
\DeclareMathOperator{\adj}{adj}

\DeclareMathOperator{\erf}{erf}
\DeclareMathOperator{\erfc}{erfc}

\DeclareMathOperator{\ord}{ord}
\DeclareMathOperator{\Sym}{Sym}

\DeclareMathOperator{\sgn}{sgn}
\DeclareMathOperator{\orb}{orb}
\DeclareMathOperator{\stab}{stab}
\DeclareMathOperator{\ccl}{ccl}

\DeclareMathOperator{\lcm}{lcm}
\DeclareMathOperator{\hcf}{hcf}

\DeclareMathOperator{\Int}{Int}
\DeclareMathOperator{\id}{id}

\DeclareMathOperator{\betaD}{beta}
\DeclareMathOperator{\gammaD}{gamma}
\DeclareMathOperator{\Poisson}{Poisson}
\DeclareMathOperator{\binomial}{binomial}
\DeclareMathOperator{\multinomial}{multinomial}
\DeclareMathOperator{\Bernoulli}{Bernoulli}
\DeclareMathOperator{\like}{like}

\DeclareMathOperator{\var}{var}
\DeclareMathOperator{\cov}{cov}
\DeclareMathOperator{\bias}{bias}
\DeclareMathOperator{\mse}{mse}
\DeclareMathOperator{\corr}{corr}

\DeclareMathOperator{\otp}{otp}
\DeclareMathOperator{\dom}{dom}

\DeclareMathOperator{\Root}{Root}
\DeclareMathOperator{\supp}{supp}
\DeclareMathOperator{\rel}{rel}
\DeclareMathOperator{\Hom}{Hom}
\DeclareMathOperator{\Aut}{Aut}
\DeclareMathOperator{\Gal}{Gal}
\DeclareMathOperator{\Mat}{Mat}
\DeclareMathOperator{\End}{End}
\DeclareMathOperator{\Char}{char}
\DeclareMathOperator{\ev}{ev}
\DeclareMathOperator{\St}{St}
\DeclareMathOperator{\Lk}{Lk}
\DeclareMathOperator{\disc}{disc}
\DeclareMathOperator{\Isom}{Isom}
\DeclareMathOperator{\length}{length}
\DeclareMathOperator{\energy}{energy}
\DeclareMathOperator{\area}{area}
\DeclareMathOperator{\Syl}{Syl}
\DeclareMathOperator{\cl}{cl}
\DeclareMathOperator{\fix}{fix}

\newcommand{\GL}{\mathrm{GL}}
\newcommand{\SL}{\mathrm{SL}}
\newcommand{\PGL}{\mathrm{PGL}}
\newcommand{\PSL}{\mathrm{PSL}}
\newcommand{\PSU}{\mathrm{PSU}}
\newcommand{\Or}{\mathrm{O}}
\newcommand{\SO}{\mathrm{SO}}
\newcommand{\U}{\mathrm{U}}
\newcommand{\SU}{\mathrm{SU}}

\renewcommand{\d}{\mathrm{d}}
\newcommand{\D}{\mathrm{D}}

\tikzset{->/.style = {decoration={markings,
                                  mark=at position 1 with {\arrow[scale=2]{latex'}}},
                      postaction={decorate}}}
\tikzset{<-/.style = {decoration={markings,
                                  mark=at position 0 with {\arrowreversed[scale=2]{latex'}}},
                      postaction={decorate}}}
\tikzset{<->/.style = {decoration={markings,
                                   mark=at position 0 with {\arrowreversed[scale=2]{latex'}},
                                   mark=at position 1 with {\arrow[scale=2]{latex'}}},
                       postaction={decorate}}}
\tikzset{->-/.style = {decoration={markings,
                                   mark=at position #1 with {\arrow[scale=2]{latex'}}},
                       postaction={decorate}}}
\tikzset{-<-/.style = {decoration={markings,
                                   mark=at position #1 with {\arrowreversed[scale=2]{latex'}}},
                       postaction={decorate}}}

\tikzset{circ/.style = {fill, circle, inner sep = 0, minimum size = 3}}
\tikzset{mstate/.style={circle, draw, blue, text=black, minimum width=0.7cm}}

\definecolor{mblue}{rgb}{0.2, 0.3, 0.8}
\definecolor{morange}{rgb}{1, 0.5, 0}
\definecolor{mgreen}{rgb}{0.1, 0.4, 0.2}
\definecolor{mred}{rgb}{0.5, 0, 0}

\def\drawcirculararc(#1,#2)(#3,#4)(#5,#6){%
    \pgfmathsetmacro\cA{(#1*#1+#2*#2-#3*#3-#4*#4)/2}%
    \pgfmathsetmacro\cB{(#1*#1+#2*#2-#5*#5-#6*#6)/2}%
    \pgfmathsetmacro\cy{(\cB*(#1-#3)-\cA*(#1-#5))/%
                        ((#2-#6)*(#1-#3)-(#2-#4)*(#1-#5))}%
    \pgfmathsetmacro\cx{(\cA-\cy*(#2-#4))/(#1-#3)}%
    \pgfmathsetmacro\cr{sqrt((#1-\cx)*(#1-\cx)+(#2-\cy)*(#2-\cy))}%
    \pgfmathsetmacro\cA{atan2(#2-\cy,#1-\cx)}%
    \pgfmathsetmacro\cB{atan2(#6-\cy,#5-\cx)}%
    \pgfmathparse{\cB<\cA}%
    \ifnum\pgfmathresult=1
        \pgfmathsetmacro\cB{\cB+360}%
    \fi
    \draw (#1,#2) arc (\cA:\cB:\cr);%
}
\newcommand\getCoord[3]{\newdimen{#1}\newdimen{#2}\pgfextractx{#1}{\pgfpointanchor{#3}{center}}\pgfextracty{#2}{\pgfpointanchor{#3}{center}}}

\def\Xint#1{\mathchoice
   {\XXint\displaystyle\textstyle{#1}}%
   {\XXint\textstyle\scriptstyle{#1}}%
   {\XXint\scriptstyle\scriptscriptstyle{#1}}%
   {\XXint\scriptscriptstyle\scriptscriptstyle{#1}}%
   \!\int}
\def\XXint#1#2#3{{\setbox0=\hbox{$#1{#2#3}{\int}$}
     \vcenter{\hbox{$#2#3$}}\kern-.5\wd0}}
\def\ddashint{\Xint=}
\def\dashint{\Xint-}


\begin{document}
\maketitle
{\small
  \noindent\emph{Part IB Quantum Mechanics and ``Multiparticle Systems'' from Part II Principles of Quantum Mechanics are essential}

  \vspace{10pt}
  \noindent\textbf{Fundamentals of statistical mechanics}\\
  Microcanonical ensemble. Entropy, temperature and pressure. Laws of thermodynamics. Example of paramagnetism. Boltzmann distribution and canonical ensemble. Partition function. Free energy. Specific heats. Chemical Potential. Grand Canonical Ensemble.\hspace*{\fill} [5]

  \vspace{10pt}
  \noindent\textbf{Classical gases}\\
  Density of states and the classical limit. Ideal gas. Maxwell distribution. Equipartition of energy. Diatomic gas. Interacting gases. Virial expansion. Van der Waal's equation of state. Basic kinetic theory.\hspace*{\fill} [3]

  \vspace{10pt}
  \noindent\textbf{Quantum gases}\\
  Density of states. Planck distribution and black body radiation. Debye model of phonons in solids. Bose-Einstein distribution. Ideal Bose gas and Bose-Einstein condensation. Fermi-Dirac distribution. Ideal Fermi gas. Pauli paramagnetism.\hspace*{\fill}[8]

  \vspace{10pt}
  \noindent\textbf{Thermodynamics}\\
  Thermodynamic temperature scale. Heat and work. Carnot cycle. Applications of laws of thermodynamics. Thermodynamic potentials. Maxwell relations.\hspace*{\fill} [4]

  \vspace{10pt}
  \noindent\textbf{Phase transitions}\\
  Liquid-gas transitions. Critical point and critical exponents. Ising model. Mean field theory. First and second order phase transitions. Symmetries and order parameters.\hspace*{\fill} [4]%
}

\tableofcontents
\setcounter{section}{-1}
\section{Introduction}
In previous physics courses, we mostly focused on ``microscopic'' physics. We looked at the interaction of atoms, electrons etc. We now want to look at more macroscopic things, when we have lots of atoms, and lots of molecules. The aim of the course is translate microscopic laws of physics into macroscopic laws of physics.

By ``macroscopic'', we mean a system that is made of many particles. For example, $\SI{12}{\gram}$ of Carbon-12 contains $6 \times 10^{23}$ atoms, known as a mole. This number is very big. This is the key ingredient we are going to talk about in this course. How do we go about doing that? We might want to write down the Schr\"odinger equation for all water particles in a bottle, and then see how it evolves in time, and then predict how water evolves. But this is, obviously, hopeless. Even if we have two atoms, it is very hard to solve the Schr\"odinger equation, let alone $10^{23}$. But we don't have to. We don't have to have a complete description of the molecules if we want to know how the water behaves. It doesn't matter if we swap around two water molecules. We just want to know about macroscopic properties of the substance.

For example, temperature is something we can ask about a system, and this is a property of a system. It doesn't make sense to ask the temperature of a particle.

Historically, one reason to study statistical physics is that we can use it to test our microscopic laws of matter. For example, this is how quantum mechanics was discovered. Nowadays, techniques in statistical physics have many applications. In addition to studying, say, condensed matter physics, they can be used to do many other things like studying black holes, or biology!

\section{Fundamentals of statistical mechanics}
\subsection{Microcanonical ensemble}
We are going to consider an isolated system containing $N$ particles, where $N$ is a Large Number\textsuperscript{TM}. For example, a gas in a box can be considered to be isolated. We are going to use the term \term{microstate} to describe the individual (quantum) state of the system. This gives a complete description of the system. As we would expect, this is very complicated. In statistical physics, we observe that many microstates are indistinguishable macroscopically. So we don't attempt to describe the system using the microstate. Instead, we use a probability distribution on microstates.

More precisely, we let $\{\bket{n}\}$ be a basis of normalized eigenstates, say
\[
  \hat{H}\bket{n} = E_n \bket{n}.
\]
We let $p(n)$ be the probability that the microstate is $\bket{n}$. We can then define the expectation value of an operator $\mathcal{O}$ by
\[
  \bra \mathcal{O}\ket = \sum_n p(n) \brak{n}\mathcal{O}\bket{n}.
\]
In other words, we assume the system is described by a mixed state with density operator
\[
  \rho = \sum_n p(n) \bket{n}\brak{n}.
\]
There is an equivalent way of looking at this. We can consider an \term{ensemble} consisting of $W \gg 1$ independent copies of our system such that $Wp(m)$ many copies are in the microstate $\bket{n}$. Then the expectation is just the average over the ensemble.

We will further assume our system is in \term{equilibrium}, ie. the probability distribution $p(n)$ does not change in time. So in particular $\bra O\ket$ is independent of time. Of course, this does not mean the particles stop moving. The particles are still whizzing around. It's just that the statistical distribution does not change. In this course, we will mostly be talking about equilibrium systems. When we get out of equilibrium, things become very complicated.

We may have some partial knowledge about the system. For example, we might know its total energy. The microstates that are compatible with this partial knowledge are called \emph{accessible}\index{accessible state}. We can now state our fundamental assumption of statistical mechanics:
\begin{center}
  \emph{An isolated system in equilibrium is equally likely to be in any of the accessible microstates.}
\end{center}
This is an assumption we make, and we will see that it works very well.

Thus, different probability distributions, or different ensembles, are distinguished by the partial knowledge we know.
\begin{defi}[Microcanonical ensemble]\index{microcanonical ensemble}
  In a \emph{microcanonical ensemble}, we know the energy is between $E$ and $E + \delta E$, where $\delta E$ Is the accuracy of our measuring device. The accessible microstates are those with energy $E \leq E_n \leq E + \delta E$. We let $\Omega(E)$ be the number of such states.
\end{defi}
In practice, $\delta E$ is much much larger than the spacing of energy levels, and so $\Omega(E) \gg 1$. This notation doesn't really do justice to how big $\Omega(E)$ is. We should really write something like $\Omega(E) \gg{}\gg{}\gg 1$.

Note that we are working with a quantum system here, so the possible systems is discrete, and so it makes sense to count the number of systems. We need to do more work if we want to do this classically.

\begin{eg}
  Suppose we have $N = 10^{23}$ particles, and each particle can occupy two states $\bket{\uparrow}$ and $\bket{\downarrow}$, which have the same energy $\varepsilon$. Then we always have $N\varepsilon$ total energy, and we have
  \[
    \Omega(N\varepsilon) = 2^{10^{23}}.
  \]
  This is a fantastically huge, mind-boggling number. This is the kind of number we are talking about.
\end{eg}

By the fundamental assumption, we have
\[
  p(n) =
  \begin{cases}
    \frac{1}{\Omega(E)} & \text{if } E \leq E_n \leq E + \delta E\\
    0 & \text{otherwise}
  \end{cases}
\]
This probability distribution defines what is known as as the microcanonical ensemble. % \delta E doesn't matter, because exponentials?

Since this $\Omega(E)$ is huge, we want something smaller, one that is linear in $N$. So we take the log

\begin{defi}[Boltzmann entropy]\index{Boltzmann entropy}\index{entropy}
  The \emph{(Boltzmann) entropy} is defined as
  \[
    S(E) = k \log \Omega(E),
  \]
  where $k = \SI{1.381e-23}{\joule\per\kelvin}$ is \term{Boltzmann's constant}\index{$k$}.
\end{defi}
This annoying constant $k$ is necessary because when people started doing thermodynamics, they didn't know about statistical physics, and picked weird conventions.

As we have seen previously, we expect $\Omega(E) \sim e^N$. So we would expect $S(E) \sim kN$, and this is why the entropy is more convenient.

The second nice property of the entropy is that it is additive --- if we have two \emph{non-interacting} systems with energies $E_{(1)}, E_{(2)}$. Then the total number of states of the combined system is
\[
  \Omega(E_{(1)}, E_{(2)}) = \Omega_{1}(E_{(1)})\Omega_{2} (E_{(2)}).
\]
So when we take the logarithm, we find
\[
  S(E_{(1)}, E_{(2)}) = S(E_{(1)}) + S(E_{(2)}).
\]
This is for systems that do not interact with each other.

Suppose we bring the two systems together, and let them exchange energy. Then the energy of the individual systems is no longer fixed, and only the total energy
\[
  E_{\mathrm{total}} = E_{(1)} + E_{(2)}
\]
is fixed. Then we find that
\[
  \Omega(E_{\mathrm{total}}) = \sum_{E_i} \Omega_1(E_i) \Omega_2(E_{\mathrm{total}} - E_i),
\]
where we sum over all possible energy levels of the first system. In terms of the entropy of the system, we have
\[
  \Omega(E_{\mathrm{total}}) = \sum_{E_i} \exp\left(\frac{S_1(E_i)}{k} + \frac{S_2(E_{\mathrm{total}} - E_i)}{k}\right)
\]
We can be a bit more precise with what the sum means. Here we have previously fixed an accuracy $\delta E$. So we can imagine dividing the whole energy spectrum into chunks of size $\delta E$, and here we are summing over the chunks.

We know that $S_{1, 2}/k \sim N_{1, 2} \sim 10^{23}$, which is a ridiculously large number. So the sum is overwhelmingly dominated by the term with the largest exponent. Suppose this is maximized when $E_i = E_*$.

We now want to figure out which $E_i$ maximizes this exponent. We suppose this is given by $E_i = E_*$. For the sake of this analysis, we suppose we have a continuum of energies, so that we can differentiate the expression. Then we need
\[
 \frac{\d}{\d E_i} \left(S_1(E_i) + S_2(E_{\mathrm{total}} - E_i)\right) = 0.
\]
In other words, we need
\[
 \left.\frac{\d S_1}{\d E}\right|_{E_{(1)} = E_*} - \left.\frac{\d S_i}{\d E} \right|_{E_{(2)} = E_{\mathrm{total}} - E_*} = 0
\]
Then we have
\[
  S(E_{\mathrm{total}}) = k \log \Omega(E_{\mathrm{total}}) \approx S_1(E_*) + S_2(E_{\mathrm{total}} - E_*) \geq S_1(E_{(1)}) + S_2(E_{(2)}),
\]
where the last inequality comes from the fact that we defined $E_*$ to maximize the expression.

Note that $E_{(1)}$ is not fixed, but the probability that the first system has energy $E_{(1)}$ is
\[
  \frac{\Omega_1(E_{(1)})\Omega_2(E_{\mathrm{total}} - E_{(1)})}{\Omega(E_{\mathrm{total}})} = \exp\left(\frac{1}{k}\left(S_1(E_{(1)}) + S_2(E_{(2)}) - S(E_{\mathrm{total}})\right)\right).
\]
Since the $S_i$ are incredibly large numbers, this is overwhelmingly maximized for $E_{(1)} = E_*$, so that equality almost holds, and is negligible otherwise. So for all practical purposes, the value of $E_{(1)}$ is fixed into $E_*$.

Now imagine we prepare two systems with energies $E_{(1)}$ and $E_{(2)}$ such that $E_{(1)} \not= E_*$, and then bring the system together, then we are no longer in equilibrium. $E_{(1)}$ will change until it takes value $E_*$, and then entropy of the system will increase from $S_1(E_{(1)}) + S_2(E_{(2)})$ to $S_1(E_*) + S_2(E_{\mathrm{total}} - E_*)$. This prediction is verified by virtually all observations of physics.
\begin{law}[Second law of thermodynamics]\index{second law of thermodynamics}
  The entropy of an isolated system increases (or remains the same) in any physical process. In equilibrium, the entropy attains its maximum value.
\end{law}
While our derivation did not show it is \emph{impossible} to violate the second law of thermodynamics, it is very very very very very very very very unlikely to be violated.

\subsubsection*{Temperature}
Having defined entropy, the next interesting thing we can define is the \emph{temperature}.
\begin{defi}[Temperature]\index{temperature}
  The \emph{temperature} is defined to be
  \[
    \frac{1}{T} = \frac{\d S}{\d E}.
  \]
\end{defi}
Why does this definition make sense? Recall that for two systems in equilibrium, the entropy is maximized. By definition, this means we need
\[
  \frac{1}{T_1} = \frac{1}{T_2}.
\]
In other words, we need
\[
  T_1 = T_2.
\]
So a necessary condition for two systems to be in equilibrium is that they have the same temperature. So this is at least a bit sensible.

What if $T_1 \not= T_2$? If so, then $E_{(1)} \not= E_*$. Let's assume, wlog, that $E_{(1)} + E_* + \delta E$. Then
\[
  E_{(2)} = E_{\mathrm{total}} - E_{*} - \delta E_{(1)}.
\]
So we know
\[
  \delta E_{(2)} = - \delta E_{(1)},
\]
as expected. What is the entropy of these states? We can compute
\[
  \delta S = \frac{\d S_1}{\d E} \delta E_{(1)} + \frac{\d S_2}{\d E} \delta E_{(2)} = \left(\frac{1}{T_1} - \frac{1}{T_2}\right) \delta E_{(1)}.
\]
The second law of thermodynamics says $\delta S$ will always increase. So if, for example, $T_1 > T_2$, we know $\delta E_1 < 0$. So energy flows from the system with higher temperature to the system of lower temperature.

So this notion of temperature agrees with the basic properties of temperature we expect.

Now, these properties we've derived only depends on the fact that $\frac{1}{T}$ is a monotonically decreasing function of $T$. In principle, we could have picked \emph{any} monotonically decreasing function of $T$, but we will later see that this definition will agree with the other definitions of temperature we have previously seen, eg. via the ideal gas law.

\subsubsection*{Heat capacity}
We now take another derivative, to get the heat capacity. Recall that $T$ was a function of energy, $T = T(E)$. We can try to invert this function and write $E = E(T)$.
\begin{defi}[Heat capacity]\index{heat capacity}
  The \emph{heat capacity} of a system is
  \[
    C = \frac{\d E}{\d T}.
  \]
  The \term{specific heat capacity} is
  \[
    \frac{C}{\text{mass of system}}.
  \]
\end{defi}
Unlike the entropy, this is something we can actually measure. By measuring $C$, we can indirectly measure entropy, as we have
\[
  \frac{\d S}{\d T} = \frac{\d S}{\d E} = \frac{\d E}{\d T} = \frac{C}{T}.
\]
Integrating up, if the temperature changes from $T_1$ to $T_2$, we know
\[
  \Delta S = \int_{T_1}^{T_2} \frac{C(T)}{T}\;\d T.
\]
By measuring heat capacity experimentally, we can measure the change in entropy.

The specific heat capacity is a property of the substance that makes up the system, and not how much stuff there is, as both $C$ and the mass scale linearly with the size of the system.

Now note that the previous formula
\[
  \frac{\d S_1}{\d E} - \frac{\d S_2}{\d E} = 0
\]
only gives us extrema of the entropy. We do not know if it is a maximum or not. To figure out, we take the second derivative and get
\[
  \frac{\d^2 S}{\d E^2} = \frac{\d}{\d E} \left(\frac{1}{T}\right) = -\frac{1}{T^2 C}.
\]
Applying this to two systems, entropy is maximized at $E_{(1)} = E_*$ if $C_1, C_2 > 0$.

Let's look at some actual systems and try to calculate these quantities.
\begin{eg}
  Consider a $2$-state system, where we have $N$ non-interacting particles with fixed positions. Each particle is either in $\bket{\uparrow}$ or $\bket{\downarrow}$. We can think of these as spins, for example. These two states have different energies
  \[
    E_{\uparrow} = \varepsilon,\quad E_{\downarrow} = 0.
  \]
  We let $N_{\uparrow}$ and $N_{\downarrow}$ be the number of particles in $\bket{\uparrow}$ and $\bket{\downarrow}$ respectively. Then the total energy of the system is
  \[
    E = \varepsilon N_{\uparrow}.
  \]
  We want to calculate this quantity $\Omega(E)$. Here in this very contrived example, it is convenient to pick $\delta E < \varepsilon$, so that $\Omega(E)$ is just the number of ways of choosing $N_{\uparrow}$ particles from $N$. We then have
  \[
    \Omega(E) = \frac{N!}{N_{\uparrow}! (N - N_\uparrow)!},
  \]
  and
  \[
    S(E) = k \log \left(\frac{N!}{N_{\uparrow}! (N - N_\uparrow)!}\right).
  \]
  Since we assumed that $N$ and $N_\uparrow$ are huge, we can use \term{Stirling's approximation}
  \[
    N! = \sqrt{2\pi N} N^N e^{-N} \left(1 + O\left(\frac{1}{N}\right)\right).
  \]
  Then we have
  \[
    \log N! = N \log N - N + \frac{1}{2}\log (2\pi N) + O\left(\frac{1}{N}\right).
  \]
  We just use the approximation three times to get
  \begin{align*}
    S(E) &= k\left(N \log N - N - N_\uparrow \log N_\uparrow + N_\uparrow - (N - N_\uparrow) \log(N - N_\uparrow) + N - N_\uparrow\right)\\
    &= -k \left((N - N_\uparrow) \log \left(\frac{N - N_\uparrow}{N}\right) + N_\uparrow \log\left(\frac{N_{\uparrow}}{N}\right)\right)\\
    &= -kN\left(\left(1 - \frac{E}{N\varepsilon}\right) \log\left(1 - \frac{E}{N\varepsilon}\right) + \frac{E}{N\varepsilon} \log\left(\frac{E}{N\varepsilon}\right)\right).
  \end{align*}
  This is not a particularly illuminating expressions, but if we plot it, it looks something like
  \begin{center}
    \begin{tikzpicture}
      \draw [->] (0, 0) -- (6, 0) node [right] {$E$};
      \draw [->] (0, 0) -- (0, 4) node [above] {$S(E)$};

      \draw (0, 0) parabola bend (2.5, 3) (5, 0);
      \node [left] at (0, 0) {$0$};
      \node [below] at (5, 0) {$N\varepsilon$};
      \node [below] at (2.5, 0) {$N\varepsilon/2$};
      \draw [dashed] (2.5, 0) -- (2.5, 3);

      \draw [dashed] (2.5, 3) -- (0, 3) node [left] {$Nk \log 2$};
    \end{tikzpicture}
  \end{center}% check
  The temperature is
  \[
    \frac{1}{T} = \frac{\d S}{\d T} = \frac{k}{\varepsilon} \log \left(\frac{N \varepsilon}{E} - 1\right),
  \]
  and we can invert to get
  \[
    \frac{N_{\uparrow}}{N} = \frac{E}{N\varepsilon} = \frac{1}{e^{\varepsilon /kT} + 1}.
  \]
  As $T \to 0$, we have $N_\uparrow \to 0$. So the states all try to go to the ground state. If $T \to \infty$, then we find $N_\uparrow/N \to \frac{1}{2}$, and $E \to N\varepsilon/2$.

  This is a bit weird. As $T \to \infty$, we might expect all things to go the maximum energy level, and not just half of them. What is happening?

  We can plot another graph, for $\frac{1}{T}$ vs $E$. The graph looks like
  \begin{center}
    \begin{tikzpicture}
      \draw [->] (0, -3) -- (0, 3) node [above] {$\frac{1}{T}$};
      \draw [->] (0, 0) -- (4, 0) node [right] {$E$};
      \draw [mblue, thick] (0.1, 3) .. controls (0.1, 0) and (0.4, 0) .. (1.5, 0);
      \draw [mblue, thick] (2.9, -3) .. controls (2.9, 0) and (2.6, 0) .. (1.5, 0);

      \node [circ] at (1.5, 0) {};
      \node [below] at (1.5, 0) {$N\varepsilon/2$};
    \end{tikzpicture}
  \end{center}
  We see that having energy $> N\varepsilon/2$ corresponds to negative energy, and to go from positive temperature to negative temperature, we need to pass through infinite temperature. So in some sense, negative temperature is ``hotter'' than infinite temperature.

  What is going on? For negative $T$, we know $\Omega(E)$ is a decreasing function of energy. This is a very unusual situation. What we have done in this system is that these particles are fixed, and have no kinetic energy. But if we do have kinetic energy, the kinetic energy can be arbitrarily large. So there is no upper bound on $E$, and $\Omega(E)$ is usually an increasing function of $E$.

  Negative $T$ has been observed experimentally, which requires setups where the kinetic energy is not so important in the range of energies we are talking about. For example, this happens in nuclear spins of crystals in magnetic fields. Naturally, the most of the spins will align with the field. We now suddenly flip the field, and then most of the spins are anti-aligned, and this can give us a negative temperature state.

  As we said, an experimentally interesting quantity is the heat capacity
  \[
    C = \frac{\d E}{\d T} = \frac{N\varepsilon^2}{ kT^2 } \frac{e^{\varepsilon/kT}}{(e^{\varepsilon/kT} + 1)^2}.
  \]
  Again we can sketch this, and it looks like this:
  \begin{center}
    \begin{tikzpicture}
      \draw [->] (0, 0) -- (6, 0) node [right] {$T$};
      \draw [->] (0, 0) -- (0, 3) node [above] {$C$};

      \draw (0, 0) .. controls (1, 0) and (0.5, 2) .. (1, 2) .. controls (1.5, 2) and (2, 0) .. (5, 0);
      \draw [dashed] (1, 2) -- (1, 0) node [below] {$T \sim \varepsilon/k$};
    \end{tikzpicture}
  \end{center}
  Now the maximum $T$ is related to the microscopic $\varepsilon$. So we can use the macroscopic observation of $C$ to deduce something about the microscopic $\varepsilon$.

  Note that $C$ is proportional to $N$. As $T \to 0$, we have
  \[
    C \propto T^{-2} e^{-\varepsilon kT},
  \]
  and this is a function that decreases very rapidly as $T \to 0$, and in fact this is one of the favorite examples in Analysis where all derivatives of the function at $0$ vanish. We will see that this is due to the energy gap between the ground state and the first excited state.

  The heat capacity also vanishes at high temperature, but this is due to the peculiar property of the system at high temperature.
\end{eg}

How much of this is actually physical? The answer is ``not much'', as one might expect because we didn't really do much physics. For most solids, the contribution to $C$ from spins is swamped by other effects such as contributions of phonons (quantized vibrations in the solid) or electrons. In this case, $C(T)$ is monotonic in $T$. However, there are some very peculiar materials, we obtain a small local maximum in $C(T)$ for very small $T$, which is due to the contributions of spin. % can insert picture.

\subsection{Pressure, volume and the first law of thermodynamics}
Many systems have other external parameters which can be varied. Consider a system whose total volume $V$ can vary, such as gas in a box with movable walls. As we change the volume, the allowed energies eigenstates depend on the volume, and the $\Omega$, and hence $S$, will depend on the volume:
\[
  S(E, V) = k \log \Omega(E, V).
\]
We now need to modify our definition of temperature to account for this dependence:
\begin{defi}[Temperature]\index{temperature}
  The \emph{temperature} of a system with variable volume is
  \[
    \frac{1}{T} = \left(\frac{\partial S}{\partial E}\right)_V,
  \]
  with $V$ fixed.
\end{defi}

But now we can define a different thermodynamic quantity by taking the derivative with respect to $V$.

\begin{defi}[Pressure]\index{pressure}
  We define the \emph{pressure} of a system with variable volume to be
  \[
    p = T \left(\frac{\partial S}{\partial V}\right)_E.
  \]
\end{defi}
Is this thing we call the ``pressure'' any thing like what we used to think of as pressure, namely force per unit area?

\printindex
\end{document}
