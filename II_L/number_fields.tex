\documentclass[a4paper]{article}

\def\npart {II}
\def\nterm {Lent}
\def\nyear {2016}
\def\nlecturer {I. Grojnowski}
\def\ncourse {Number fields}
\def\nlectures {TT.12}
\def\nnotready {}

% Imports
\ifx \nextra \undefined
  \usepackage[pdftex,
    hidelinks,
    pdfauthor={Dexter Chua},
    pdfsubject={Cambridge Maths Notes: Part \npart\ - \ncourse},
    pdftitle={Part \npart\ - \ncourse},
  pdfkeywords={Cambridge Mathematics Maths Math \npart\ \nterm\ \nyear\ \ncourse}]{hyperref}
  \title{Part \npart\ - \ncourse}
\else
  \usepackage[pdftex,
    hidelinks,
    pdfauthor={Dexter Chua},
    pdfsubject={Cambridge Maths Notes: Part \npart\ - \ncourse\ (\nextra)},
    pdftitle={Part \npart\ - \ncourse\ (\nextra)},
  pdfkeywords={Cambridge Mathematics Maths Math \npart\ \nterm\ \nyear\ \ncourse\ \nextra}]{hyperref}

  \title{Part \npart\ - \ncourse \\ {\Large \nextra}}
\fi

\author{Lectured by \nlecturer \\\small Notes taken by Dexter Chua}
\date{\nterm\ \nyear}

\usepackage{alltt}
\usepackage{amsfonts}
\usepackage{amsmath}
\usepackage{amssymb}
\usepackage{amsthm}
\usepackage{booktabs}
\usepackage{caption}
\usepackage{enumitem}
\usepackage{fancyhdr}
\usepackage{graphicx}
\usepackage{mathtools}
\usepackage{microtype}
\usepackage{multirow}
\usepackage{pdflscape}
\usepackage{pgfplots}
\usepackage{siunitx}
\usepackage{tabularx}
\usepackage{tikz}
\usepackage{tkz-euclide}
\usepackage[normalem]{ulem}
\usepackage[all]{xy}

\pgfplotsset{compat=1.12}

\pagestyle{fancyplain}
\lhead{\emph{\nouppercase{\leftmark}}}
\ifx \nextra \undefined
  \rhead{
    \ifnum\thepage=1
    \else
      \npart\ \ncourse
    \fi}
\else
  \rhead{
    \ifnum\thepage=1
    \else
      \npart\ \ncourse\ (\nextra)
    \fi}
\fi
\usetikzlibrary{arrows}
\usetikzlibrary{decorations.markings}
\usetikzlibrary{decorations.pathmorphing}
\usetikzlibrary{positioning}
\usetikzlibrary{fadings}
\usetikzlibrary{intersections}
\usetikzlibrary{cd}

\newcommand*{\Cdot}{\raisebox{-0.25ex}{\scalebox{1.5}{$\cdot$}}}
\newcommand {\pd}[2][ ]{
  \ifx #1 { }
    \frac{\partial}{\partial #2}
  \else
    \frac{\partial^{#1}}{\partial #2^{#1}}
  \fi
}

% Theorems
\theoremstyle{definition}
\newtheorem*{aim}{Aim}
\newtheorem*{axiom}{Axiom}
\newtheorem*{claim}{Claim}
\newtheorem*{cor}{Corollary}
\newtheorem*{defi}{Definition}
\newtheorem*{eg}{Example}
\newtheorem*{fact}{Fact}
\newtheorem*{law}{Law}
\newtheorem*{lemma}{Lemma}
\newtheorem*{notation}{Notation}
\newtheorem*{prop}{Proposition}
\newtheorem*{thm}{Theorem}

\renewcommand{\labelitemi}{--}
\renewcommand{\labelitemii}{$\circ$}
\renewcommand{\labelenumi}{(\roman{*})}

\let\stdsection\section
\renewcommand\section{\newpage\stdsection}

% Strike through
\def\st{\bgroup \ULdepth=-.55ex \ULset}

% Maths symbols
\newcommand{\bra}{\langle}
\newcommand{\ket}{\rangle}

\newcommand{\N}{\mathbb{N}}
\newcommand{\Z}{\mathbb{Z}}
\newcommand{\Q}{\mathbb{Q}}
\renewcommand{\H}{\mathbb{H}}
\newcommand{\R}{\mathbb{R}}
\newcommand{\C}{\mathbb{C}}
\newcommand{\Prob}{\mathbb{P}}
\renewcommand{\P}{\mathbb{P}}
\newcommand{\E}{\mathbb{E}}
\newcommand{\F}{\mathbb{F}}
\newcommand{\cU}{\mathcal{U}}
\newcommand{\RP}{\mathbb{RP}}
\newcommand{\CP}{\mathbb{CP}}

\newcommand{\ph}{\,\cdot\,}

\DeclareMathOperator{\sech}{sech}
\DeclareMathOperator{\cosech}{cosech}
\DeclareMathOperator{\cosec}{cosec}

\DeclareMathOperator{\covol}{covol}
\DeclareMathOperator{\vol}{vol}

\let\Im\relax
\let\Re\relax
\DeclareMathOperator{\Im}{Im}
\DeclareMathOperator{\Re}{Re}
\DeclareMathOperator{\im}{im}
\DeclareMathOperator{\image}{image}
\DeclareMathOperator{\Ann}{Ann}

\DeclareMathOperator*{\res}{res}
\DeclareMathOperator{\Res}{Res}
\DeclareMathOperator{\Ind}{Ind}

\DeclareMathOperator{\tr}{tr}
\DeclareMathOperator{\diag}{diag}
\DeclareMathOperator{\rank}{rank}
\DeclareMathOperator{\card}{card}
\DeclareMathOperator{\spn}{span}
\DeclareMathOperator{\adj}{adj}

\DeclareMathOperator{\erf}{erf}
\DeclareMathOperator{\erfc}{erfc}

\DeclareMathOperator{\ord}{ord}
\DeclareMathOperator{\Sym}{Sym}

\DeclareMathOperator{\sgn}{sgn}
\DeclareMathOperator{\orb}{orb}
\DeclareMathOperator{\stab}{stab}
\DeclareMathOperator{\ccl}{ccl}

\DeclareMathOperator{\lcm}{lcm}
\DeclareMathOperator{\hcf}{hcf}

\DeclareMathOperator{\Int}{Int}
\DeclareMathOperator{\id}{id}

\DeclareMathOperator{\betaD}{beta}
\DeclareMathOperator{\gammaD}{gamma}
\DeclareMathOperator{\Poisson}{Poisson}
\DeclareMathOperator{\binomial}{binomial}
\DeclareMathOperator{\multinomial}{multinomial}
\DeclareMathOperator{\Bernoulli}{Bernoulli}
\DeclareMathOperator{\like}{like}

\DeclareMathOperator{\var}{var}
\DeclareMathOperator{\cov}{cov}
\DeclareMathOperator{\bias}{bias}
\DeclareMathOperator{\mse}{mse}
\DeclareMathOperator{\corr}{corr}

\DeclareMathOperator{\otp}{otp}
\DeclareMathOperator{\dom}{dom}

\DeclareMathOperator{\Root}{Root}
\DeclareMathOperator{\supp}{supp}
\DeclareMathOperator{\rel}{rel}
\DeclareMathOperator{\Hom}{Hom}
\DeclareMathOperator{\Aut}{Aut}
\DeclareMathOperator{\Gal}{Gal}
\DeclareMathOperator{\Mat}{Mat}
\DeclareMathOperator{\End}{End}
\DeclareMathOperator{\Char}{char}
\DeclareMathOperator{\ev}{ev}
\DeclareMathOperator{\St}{St}
\DeclareMathOperator{\Lk}{Lk}
\DeclareMathOperator{\disc}{disc}
\DeclareMathOperator{\Isom}{Isom}
\DeclareMathOperator{\length}{length}
\DeclareMathOperator{\energy}{energy}
\DeclareMathOperator{\area}{area}
\DeclareMathOperator{\Syl}{Syl}
\DeclareMathOperator{\cl}{cl}
\DeclareMathOperator{\fix}{fix}

\newcommand{\GL}{\mathrm{GL}}
\newcommand{\SL}{\mathrm{SL}}
\newcommand{\PGL}{\mathrm{PGL}}
\newcommand{\PSL}{\mathrm{PSL}}
\newcommand{\PSU}{\mathrm{PSU}}
\newcommand{\Or}{\mathrm{O}}
\newcommand{\SO}{\mathrm{SO}}
\newcommand{\U}{\mathrm{U}}
\newcommand{\SU}{\mathrm{SU}}

\renewcommand{\d}{\mathrm{d}}
\newcommand{\D}{\mathrm{D}}

\tikzset{->/.style = {decoration={markings,
                                  mark=at position 1 with {\arrow[scale=2]{latex'}}},
                      postaction={decorate}}}
\tikzset{<-/.style = {decoration={markings,
                                  mark=at position 0 with {\arrowreversed[scale=2]{latex'}}},
                      postaction={decorate}}}
\tikzset{<->/.style = {decoration={markings,
                                   mark=at position 0 with {\arrowreversed[scale=2]{latex'}},
                                   mark=at position 1 with {\arrow[scale=2]{latex'}}},
                       postaction={decorate}}}
\tikzset{->-/.style = {decoration={markings,
                                   mark=at position #1 with {\arrow[scale=2]{latex'}}},
                       postaction={decorate}}}
\tikzset{-<-/.style = {decoration={markings,
                                   mark=at position #1 with {\arrowreversed[scale=2]{latex'}}},
                       postaction={decorate}}}

\tikzset{circ/.style = {fill, circle, inner sep = 0, minimum size = 3}}
\tikzset{mstate/.style={circle, draw, blue, text=black, minimum width=0.7cm}}

\definecolor{mblue}{rgb}{0.2, 0.3, 0.8}
\definecolor{morange}{rgb}{1, 0.5, 0}
\definecolor{mgreen}{rgb}{0.1, 0.4, 0.2}
\definecolor{mred}{rgb}{0.5, 0, 0}

\def\drawcirculararc(#1,#2)(#3,#4)(#5,#6){%
    \pgfmathsetmacro\cA{(#1*#1+#2*#2-#3*#3-#4*#4)/2}%
    \pgfmathsetmacro\cB{(#1*#1+#2*#2-#5*#5-#6*#6)/2}%
    \pgfmathsetmacro\cy{(\cB*(#1-#3)-\cA*(#1-#5))/%
                        ((#2-#6)*(#1-#3)-(#2-#4)*(#1-#5))}%
    \pgfmathsetmacro\cx{(\cA-\cy*(#2-#4))/(#1-#3)}%
    \pgfmathsetmacro\cr{sqrt((#1-\cx)*(#1-\cx)+(#2-\cy)*(#2-\cy))}%
    \pgfmathsetmacro\cA{atan2(#2-\cy,#1-\cx)}%
    \pgfmathsetmacro\cB{atan2(#6-\cy,#5-\cx)}%
    \pgfmathparse{\cB<\cA}%
    \ifnum\pgfmathresult=1
        \pgfmathsetmacro\cB{\cB+360}%
    \fi
    \draw (#1,#2) arc (\cA:\cB:\cr);%
}
\newcommand\getCoord[3]{\newdimen{#1}\newdimen{#2}\pgfextractx{#1}{\pgfpointanchor{#3}{center}}\pgfextracty{#2}{\pgfpointanchor{#3}{center}}}

\def\Xint#1{\mathchoice
   {\XXint\displaystyle\textstyle{#1}}%
   {\XXint\textstyle\scriptstyle{#1}}%
   {\XXint\scriptstyle\scriptscriptstyle{#1}}%
   {\XXint\scriptscriptstyle\scriptscriptstyle{#1}}%
   \!\int}
\def\XXint#1#2#3{{\setbox0=\hbox{$#1{#2#3}{\int}$}
     \vcenter{\hbox{$#2#3$}}\kern-.5\wd0}}
\def\ddashint{\Xint=}
\def\dashint{\Xint-}


\begin{document}
\maketitle
{\small
\noindent Definition of algebraic number fields, their integers and units. Norms, bases and discriminants.\hspace*{\fill} [3]

\vspace{5pt}
\noindent Ideals, principal and prime ideals, unique factorisation. Norms of ideals.\hspace*{\fill} [3]

\vspace{5pt}
\noindent Minkowski's theorem on convex bodies. Statement of Dirichlet's unit theorem. Determination of units in quadratic fields.\hspace*{\fill} [2]

\vspace{5pt}
\noindent Ideal classes, finiteness of the class group. Calculation of class numbers using statement of the Minkowski bound.\hspace*{\fill} [3]

\vspace{5pt}
\noindent Dedekind's theorem on the factorisation of primes. Application to quadratic fields.\hspace*{\fill} [2]

\vspace{5pt}
\noindent Discussion of the cyclotomic field and the Fermat equation or some other topic chosen by the lecturer.\hspace*{\fill} [3]}

\tableofcontents
\setcounter{section}{-1}
\section{Introduction}
Technically, IID Galois Theory is not a prerequisite of this course. However, many results we have are analogous to what we did in Galois Theory, and we will not refrain from pointing out the correspondence. If you have not learnt Galois Theory, then you can ignore them.

\section{Integrality}
We start with some definitions.

\begin{defi}[Field extension]
  A \emph{field extension} is an inclusion of fields $K \subseteq L$. We sometimes write this as $L/K$.
\end{defi}

\begin{defi}[Degree of field extension]
  Let $K \subseteq L$ be fields. Then $L$ is a field over $K$, and the \emph{degree} of the field extension is
  \[
    [L:K] = \dim_K (L).
  \]
\end{defi}

\begin{defi}[Finite extension]
  A \emph{finite field extension} is a field extension with finite degree.
\end{defi}

\begin{defi}[Number field]
  A \emph{number field} is a finite field extension over $\Q$.
\end{defi}

What is special about number fields is that there is a canonical copy of $\Z \subseteq \Q$ lying inside it. Using this $\Z \subseteq \Q$, we define
\begin{defi}[Algebraic integer]
  Let $L$ be a number field. An \emph{algebraic integer} is an $\alpha \in L$ such that there is some monic $f \in \Z[x]$ with $f(\alpha) = 0$. We write $\mathcal{O}_L$ for the set of algebraic integers in $L$. This is a generalization of $\Z \subseteq \Q$.
\end{defi}

These are what we are going to study here. Since we say this is a generalization of $\Z \subseteq \Q$, the following better be true:

\begin{lemma}
  $\mathcal{O}_\Q = \Z$, ie. $\alpha \in \Q$ is an algebraic integer if and only if $\alpha \in \Z$.
\end{lemma}

\begin{proof}
  If $\alpha \in \Z$, then $x - \alpha \in \Z[x]$ is a monic polynomial. So $\alpha \in \mathcal{O}_\Q$.

  On the other hand, let $\alpha \in \Q$. Then there is some coprime $r, s \in \Z$ such that $\alpha = \frac{r}{s}$. If it is an algebraic integer, then there is some
  \[
    f(x) = x^n + a_{n - 1} x^{n - 1} + \cdots + a_0
  \]
  with $a_i \in \Z$ such that $f(\alpha) = 0$. Substituting in and multiplying by $s^n$, we get
  \[
    r^n + \underbrace{a_{n - 1} r^{n - 1}s + \cdots + a_0 s^n}_{\text{divisible by }s} = 0,
  \]
  So $s\mid r^n$. But if $s\not= 1$, there is a prime $p$ such that $p \mid s$, and hence $p \mid r^n$. Thus $p \mid r$. So $p$ is a common factor of $s$ and $r$. This is a contradiction. So $s = 1$, and $\alpha$ is an integer.
\end{proof}

How else is this a generalization of $\Z$? We know $\Z$ is a ring. So perhaps $\mathcal{O}_L$ also is.

\begin{thm}
  $\mathcal{O}_L$ is a ring, ie. if $\alpha, \beta \in \mathcal{O}_L$, then so is $\alpha \pm \beta$ and $\alpha\beta$.
\end{thm}
Note that usually $\frac{1}{\alpha} \not \in \mathcal{O}_L$.

Before proving this, we note that this is a refinement of something we've proved in IID Galois Theory. Recall that if $L/K$ is a field extension with $\alpha, \beta \in L$ algebraic over $K$, then so is $\alpha\beta$ and $\alpha \pm \beta$, as well as $\frac{1}{\alpha}$ if $\alpha \not= 0$.

The proof is as follows: consider $K[\alpha, \beta] \subseteq L$. Then $K[\alpha, \beta]$ is a finite-dimensional vector space over $K$, spanned by $\alpha^i \beta^j$ with $1 \leq i \leq n$, $1 \leq j \leq m$, where $n$ and $m$ are the degrees of $\alpha$. So if $x \in K[\alpha, \beta]$, then $1, x, x^2, \cdots, x^N$ is linearly dependent for sufficiently large $N$. So $x$ satisfies an algebraic equation. We apply this to $\alpha \pm \beta, \alpha\beta$ etc. The crux of this proof is that finiteness implies algebraicity.

We would like to prove this theorem in an analogous way. However, in this case, we are looking at polynomials over $\Z$, which is a ring, not a field. Hence we need to extend the notion of being algebraic and finite to rings.

\begin{defi}[Integrality]
  Let $R \subseteq S$ be rings. We say $\alpha \in S$ is \emph{integral over $R$} if there is some monic polynomial $f \in R[x]$ such that $f(\alpha) = 0$.

  We say $S$ is \emph{integral over $R$} if all $\alpha \in S$ are integral over $R$.

  We say $S$ is \emph{finitely-generated} over $R$ if there exists elements $\alpha_1, \cdots, \alpha_n \in S$ such that the function $R^n \to S$ defined by $(r_1, \cdots, r_n) \mapsto \sum r_i \alpha_i$ is surjective, ie. every element of $S$ can be written as a $R$-linear combination of elements $\alpha_1, \cdots, \alpha_n$. In other words, $S$ is finitely-generated as an $R$-module.
\end{defi}
This is a refinement of the idea of being algebraic. We allow the use of rings and restrict to monic polynomials. In Galois theory, we showed that finiteness and algebraicity ``are the same thing''. We will generalize this to integrality of rings.

\begin{eg}
  In a number field $\Z \subseteq \Q \subseteq L$, $\alpha \in L$ is an algebraic integer if and only if $\alpha$ is integral over $\Z$ by definition, and $\mathcal{O}_L$ is integral over $\Z$.
\end{eg}

\begin{notation}
  If $\alpha_1, \cdots, \alpha_r \in S$, we write $R[\alpha_1, \cdots, \alpha_r]$ for the subring of $S$ generated by $R, \alpha_1, \cdots, \alpha_r$. In other words, it is the image of the homomorphism from the polynomial ring $R[x_1, \cdots, x_n] \to S$ given by $x_i \mapsto \alpha_i$.
\end{notation}

\begin{prop}\leavevmode
  \begin{enumerate}
    \item Let $R \subseteq S$ be rings. If $S = R[s]$ and $s$ is integral over $R$, then $S$ is finitely-generated over $R$.
    \item If $S = R[s_1, \cdots, s_n]$ with $s_i$ integral over $R$, then $S$ is finitely-generated over $R$.
  \end{enumerate}
\end{prop}
This is the easy direction in identifying integrality with finitely-generated.

\begin{proof}\leavevmode
  \begin{enumerate}
    \item We know $S$ is spanned by $1, s, s^2, \cdots$ over $R$. However, since $s$ is integral, there exists $a_0, \cdots, a_n \in R$ such that
      \[
        s^n = a_0 + a_1 s + \cdots + a_{n - 1}s^{n - 1}.
      \]
      So the $R$-submodule generated by $1, s, \cdots, s^{n - 1}$ is stable under multiplication by $s$. So it contains $s^n, s^{n + 1}, s^{n + 2}, \cdots$. So it is $S$.
    \item Let $S_i = R[s_1, \cdots, s_i]$. So $S_i = S_{i - 1}[s_i]$. Since $s_i$ is integral over $R$, it is integral over $s_{i - 1}$. By the previous part, $S_i$ is finitely-generated over $S_{i - 1}$. To finish, it suffices to show that being finitely-generated is transitive. More precisely, if $A \subseteq B \subseteq C$ are rings, $B$ is finitely generated over $A$ and $C$ is finitely generated over $B$, then $C$ is finitely generated over $A$. This is left as an exercise for the reader. % exercise
  \end{enumerate}
\end{proof}
The converse to this is harder.

\begin{thm}
  If $S$ is finitely-generated over $R$, then $S$ is integral over $R$.
\end{thm}
The idea of the proof is as follows: if $s \in S$, we need to find a monic polynomial which it satisfies. In Galois theory, we have fields and vector spaces, and the proof is easy. We can just consider $1, s, s^2, \cdots$, and linear dependence kicks in and gives us a relation. But even if this worked in our case, there is no way we can make this polynomial monic.

Instead, consider the multiplication by $s$ map: $m_s: S \to S$ by $\gamma \mapsto s\gamma$. If $S$ were a finite-dimensional vector space over $R$, then Cayley-Hamilton tells us $m_s$, and thus $s$, satisfies its characteristic polynomial, which is monic. Even though $S$ is not a finite-dimensional vector space, the proof of Cayley-Hamilton will work.

\begin{proof}
  Let $\alpha_1, \cdots, \alpha_n$ generate $S$ as an $R$-module. wlog take $\alpha_1 = 1 \in S$. For any $s \in S$, write
  \[
    s \alpha_i = \sum b_{ij}\alpha_j
  \]
  for some $b_{ij} \in R$. We write $B = (b_{ij})$. This is the ``matrix of multiplication by $S$''. By construction, we have
  \[
    (sI - B)
    \begin{pmatrix}
      \alpha_1\\\vdots\\a_n
    \end{pmatrix} = 0.\tag{$*$}
  \]
  Now recall for any matrix $X$, we have $\adj(X)X = (\det X) I$, where the $i, j$th entry of $\adj(X)$ is given by the determinant of the matrix obtained by removing the $i$th row and $j$th column of $X$.

  We now multiply $(*)$ by $\adj(s I - B)$. So we get
  \[
    \det(sI - B)
    \begin{pmatrix}
      \alpha_1\\\vdots\\\alpha_n
    \end{pmatrix} = 0
  \]
  In particular, $\det(sI - B) \alpha_1 = 0$. Since we picked $\alpha_1 = 1$, we get $\det(sI - B) = 0$. Hence if $f(x) = \det(xI - B)$, then $f(x) \in R[x]$, and $f(s) = 0$.
\end{proof}
\end{document}
