\documentclass[a4paper]{article}

\def\npart {II}
\def\nterm {Lent}
\def\nyear {2016}
\def\nlecturer {I. Grojnowski}
\def\ncourse {Number fields}
\def\nlectures {TT.12}
\def\nnotready {}

% Imports
\ifx \nextra \undefined
  \usepackage[pdftex,
    hidelinks,
    pdfauthor={Dexter Chua},
    pdfsubject={Cambridge Maths Notes: Part \npart\ - \ncourse},
    pdftitle={Part \npart\ - \ncourse},
  pdfkeywords={Cambridge Mathematics Maths Math \npart\ \nterm\ \nyear\ \ncourse}]{hyperref}
  \title{Part \npart\ - \ncourse}
\else
  \usepackage[pdftex,
    hidelinks,
    pdfauthor={Dexter Chua},
    pdfsubject={Cambridge Maths Notes: Part \npart\ - \ncourse\ (\nextra)},
    pdftitle={Part \npart\ - \ncourse\ (\nextra)},
  pdfkeywords={Cambridge Mathematics Maths Math \npart\ \nterm\ \nyear\ \ncourse\ \nextra}]{hyperref}

  \title{Part \npart\ - \ncourse \\ {\Large \nextra}}
\fi

\author{Lectured by \nlecturer \\\small Notes taken by Dexter Chua}
\date{\nterm\ \nyear}

\usepackage{alltt}
\usepackage{amsfonts}
\usepackage{amsmath}
\usepackage{amssymb}
\usepackage{amsthm}
\usepackage{booktabs}
\usepackage{caption}
\usepackage{enumitem}
\usepackage{fancyhdr}
\usepackage{graphicx}
\usepackage{mathtools}
\usepackage{microtype}
\usepackage{multirow}
\usepackage{pdflscape}
\usepackage{pgfplots}
\usepackage{siunitx}
\usepackage{tabularx}
\usepackage{tikz}
\usepackage{tkz-euclide}
\usepackage[normalem]{ulem}
\usepackage[all]{xy}

\pgfplotsset{compat=1.12}

\pagestyle{fancyplain}
\lhead{\emph{\nouppercase{\leftmark}}}
\ifx \nextra \undefined
  \rhead{
    \ifnum\thepage=1
    \else
      \npart\ \ncourse
    \fi}
\else
  \rhead{
    \ifnum\thepage=1
    \else
      \npart\ \ncourse\ (\nextra)
    \fi}
\fi
\usetikzlibrary{arrows}
\usetikzlibrary{decorations.markings}
\usetikzlibrary{decorations.pathmorphing}
\usetikzlibrary{positioning}
\usetikzlibrary{fadings}
\usetikzlibrary{intersections}
\usetikzlibrary{cd}

\newcommand*{\Cdot}{\raisebox{-0.25ex}{\scalebox{1.5}{$\cdot$}}}
\newcommand {\pd}[2][ ]{
  \ifx #1 { }
    \frac{\partial}{\partial #2}
  \else
    \frac{\partial^{#1}}{\partial #2^{#1}}
  \fi
}

% Theorems
\theoremstyle{definition}
\newtheorem*{aim}{Aim}
\newtheorem*{axiom}{Axiom}
\newtheorem*{claim}{Claim}
\newtheorem*{cor}{Corollary}
\newtheorem*{defi}{Definition}
\newtheorem*{eg}{Example}
\newtheorem*{fact}{Fact}
\newtheorem*{law}{Law}
\newtheorem*{lemma}{Lemma}
\newtheorem*{notation}{Notation}
\newtheorem*{prop}{Proposition}
\newtheorem*{thm}{Theorem}

\renewcommand{\labelitemi}{--}
\renewcommand{\labelitemii}{$\circ$}
\renewcommand{\labelenumi}{(\roman{*})}

\let\stdsection\section
\renewcommand\section{\newpage\stdsection}

% Strike through
\def\st{\bgroup \ULdepth=-.55ex \ULset}

% Maths symbols
\newcommand{\bra}{\langle}
\newcommand{\ket}{\rangle}

\newcommand{\N}{\mathbb{N}}
\newcommand{\Z}{\mathbb{Z}}
\newcommand{\Q}{\mathbb{Q}}
\renewcommand{\H}{\mathbb{H}}
\newcommand{\R}{\mathbb{R}}
\newcommand{\C}{\mathbb{C}}
\newcommand{\Prob}{\mathbb{P}}
\renewcommand{\P}{\mathbb{P}}
\newcommand{\E}{\mathbb{E}}
\newcommand{\F}{\mathbb{F}}
\newcommand{\cU}{\mathcal{U}}
\newcommand{\RP}{\mathbb{RP}}
\newcommand{\CP}{\mathbb{CP}}

\newcommand{\ph}{\,\cdot\,}

\DeclareMathOperator{\sech}{sech}
\DeclareMathOperator{\cosech}{cosech}
\DeclareMathOperator{\cosec}{cosec}

\DeclareMathOperator{\covol}{covol}
\DeclareMathOperator{\vol}{vol}

\let\Im\relax
\let\Re\relax
\DeclareMathOperator{\Im}{Im}
\DeclareMathOperator{\Re}{Re}
\DeclareMathOperator{\im}{im}
\DeclareMathOperator{\image}{image}
\DeclareMathOperator{\Ann}{Ann}

\DeclareMathOperator*{\res}{res}
\DeclareMathOperator{\Res}{Res}
\DeclareMathOperator{\Ind}{Ind}

\DeclareMathOperator{\tr}{tr}
\DeclareMathOperator{\diag}{diag}
\DeclareMathOperator{\rank}{rank}
\DeclareMathOperator{\card}{card}
\DeclareMathOperator{\spn}{span}
\DeclareMathOperator{\adj}{adj}

\DeclareMathOperator{\erf}{erf}
\DeclareMathOperator{\erfc}{erfc}

\DeclareMathOperator{\ord}{ord}
\DeclareMathOperator{\Sym}{Sym}

\DeclareMathOperator{\sgn}{sgn}
\DeclareMathOperator{\orb}{orb}
\DeclareMathOperator{\stab}{stab}
\DeclareMathOperator{\ccl}{ccl}

\DeclareMathOperator{\lcm}{lcm}
\DeclareMathOperator{\hcf}{hcf}

\DeclareMathOperator{\Int}{Int}
\DeclareMathOperator{\id}{id}

\DeclareMathOperator{\betaD}{beta}
\DeclareMathOperator{\gammaD}{gamma}
\DeclareMathOperator{\Poisson}{Poisson}
\DeclareMathOperator{\binomial}{binomial}
\DeclareMathOperator{\multinomial}{multinomial}
\DeclareMathOperator{\Bernoulli}{Bernoulli}
\DeclareMathOperator{\like}{like}

\DeclareMathOperator{\var}{var}
\DeclareMathOperator{\cov}{cov}
\DeclareMathOperator{\bias}{bias}
\DeclareMathOperator{\mse}{mse}
\DeclareMathOperator{\corr}{corr}

\DeclareMathOperator{\otp}{otp}
\DeclareMathOperator{\dom}{dom}

\DeclareMathOperator{\Root}{Root}
\DeclareMathOperator{\supp}{supp}
\DeclareMathOperator{\rel}{rel}
\DeclareMathOperator{\Hom}{Hom}
\DeclareMathOperator{\Aut}{Aut}
\DeclareMathOperator{\Gal}{Gal}
\DeclareMathOperator{\Mat}{Mat}
\DeclareMathOperator{\End}{End}
\DeclareMathOperator{\Char}{char}
\DeclareMathOperator{\ev}{ev}
\DeclareMathOperator{\St}{St}
\DeclareMathOperator{\Lk}{Lk}
\DeclareMathOperator{\disc}{disc}
\DeclareMathOperator{\Isom}{Isom}
\DeclareMathOperator{\length}{length}
\DeclareMathOperator{\energy}{energy}
\DeclareMathOperator{\area}{area}
\DeclareMathOperator{\Syl}{Syl}
\DeclareMathOperator{\cl}{cl}
\DeclareMathOperator{\fix}{fix}

\newcommand{\GL}{\mathrm{GL}}
\newcommand{\SL}{\mathrm{SL}}
\newcommand{\PGL}{\mathrm{PGL}}
\newcommand{\PSL}{\mathrm{PSL}}
\newcommand{\PSU}{\mathrm{PSU}}
\newcommand{\Or}{\mathrm{O}}
\newcommand{\SO}{\mathrm{SO}}
\newcommand{\U}{\mathrm{U}}
\newcommand{\SU}{\mathrm{SU}}

\renewcommand{\d}{\mathrm{d}}
\newcommand{\D}{\mathrm{D}}

\tikzset{->/.style = {decoration={markings,
                                  mark=at position 1 with {\arrow[scale=2]{latex'}}},
                      postaction={decorate}}}
\tikzset{<-/.style = {decoration={markings,
                                  mark=at position 0 with {\arrowreversed[scale=2]{latex'}}},
                      postaction={decorate}}}
\tikzset{<->/.style = {decoration={markings,
                                   mark=at position 0 with {\arrowreversed[scale=2]{latex'}},
                                   mark=at position 1 with {\arrow[scale=2]{latex'}}},
                       postaction={decorate}}}
\tikzset{->-/.style = {decoration={markings,
                                   mark=at position #1 with {\arrow[scale=2]{latex'}}},
                       postaction={decorate}}}
\tikzset{-<-/.style = {decoration={markings,
                                   mark=at position #1 with {\arrowreversed[scale=2]{latex'}}},
                       postaction={decorate}}}

\tikzset{circ/.style = {fill, circle, inner sep = 0, minimum size = 3}}
\tikzset{mstate/.style={circle, draw, blue, text=black, minimum width=0.7cm}}

\definecolor{mblue}{rgb}{0.2, 0.3, 0.8}
\definecolor{morange}{rgb}{1, 0.5, 0}
\definecolor{mgreen}{rgb}{0.1, 0.4, 0.2}
\definecolor{mred}{rgb}{0.5, 0, 0}

\def\drawcirculararc(#1,#2)(#3,#4)(#5,#6){%
    \pgfmathsetmacro\cA{(#1*#1+#2*#2-#3*#3-#4*#4)/2}%
    \pgfmathsetmacro\cB{(#1*#1+#2*#2-#5*#5-#6*#6)/2}%
    \pgfmathsetmacro\cy{(\cB*(#1-#3)-\cA*(#1-#5))/%
                        ((#2-#6)*(#1-#3)-(#2-#4)*(#1-#5))}%
    \pgfmathsetmacro\cx{(\cA-\cy*(#2-#4))/(#1-#3)}%
    \pgfmathsetmacro\cr{sqrt((#1-\cx)*(#1-\cx)+(#2-\cy)*(#2-\cy))}%
    \pgfmathsetmacro\cA{atan2(#2-\cy,#1-\cx)}%
    \pgfmathsetmacro\cB{atan2(#6-\cy,#5-\cx)}%
    \pgfmathparse{\cB<\cA}%
    \ifnum\pgfmathresult=1
        \pgfmathsetmacro\cB{\cB+360}%
    \fi
    \draw (#1,#2) arc (\cA:\cB:\cr);%
}
\newcommand\getCoord[3]{\newdimen{#1}\newdimen{#2}\pgfextractx{#1}{\pgfpointanchor{#3}{center}}\pgfextracty{#2}{\pgfpointanchor{#3}{center}}}

\def\Xint#1{\mathchoice
   {\XXint\displaystyle\textstyle{#1}}%
   {\XXint\textstyle\scriptstyle{#1}}%
   {\XXint\scriptstyle\scriptscriptstyle{#1}}%
   {\XXint\scriptscriptstyle\scriptscriptstyle{#1}}%
   \!\int}
\def\XXint#1#2#3{{\setbox0=\hbox{$#1{#2#3}{\int}$}
     \vcenter{\hbox{$#2#3$}}\kern-.5\wd0}}
\def\ddashint{\Xint=}
\def\dashint{\Xint-}


\begin{document}
\maketitle
{\small
\noindent Definition of algebraic number fields, their integers and units. Norms, bases and discriminants.\hspace*{\fill} [3]

\vspace{5pt}
\noindent Ideals, principal and prime ideals, unique factorisation. Norms of ideals.\hspace*{\fill} [3]

\vspace{5pt}
\noindent Minkowski's theorem on convex bodies. Statement of Dirichlet's unit theorem. Determination of units in quadratic fields.\hspace*{\fill} [2]

\vspace{5pt}
\noindent Ideal classes, finiteness of the class group. Calculation of class numbers using statement of the Minkowski bound.\hspace*{\fill} [3]

\vspace{5pt}
\noindent Dedekind's theorem on the factorisation of primes. Application to quadratic fields.\hspace*{\fill} [2]

\vspace{5pt}
\noindent Discussion of the cyclotomic field and the Fermat equation or some other topic chosen by the lecturer.\hspace*{\fill} [3]}

\tableofcontents
\setcounter{section}{-1}
\section{Introduction}
Technically, IID Galois Theory is not a prerequisite of this course. However, many results we have are analogous to what we did in Galois Theory, and we will not refrain from pointing out the correspondence. If you have not learnt Galois Theory, then you can ignore them.

\section{Integrality}
We start with some definitions.

\begin{defi}[Field extension]
  A \emph{field extension} is an inclusion of fields $K \subseteq L$. We sometimes write this as $L/K$.
\end{defi}

\begin{defi}[Degree of field extension]
  Let $K \subseteq L$ be fields. Then $L$ is a field over $K$, and the \emph{degree} of the field extension is
  \[
    [L:K] = \dim_K (L).
  \]
\end{defi}

\begin{defi}[Finite extension]
  A \emph{finite field extension} is a field extension with finite degree.
\end{defi}

\begin{defi}[Number field]
  A \emph{number field} is a finite field extension over $\Q$.
\end{defi}

What is special about number fields is that there is a canonical copy of $\Z \subseteq \Q$ lying inside it. Using this $\Z \subseteq \Q$, we define
\begin{defi}[Algebraic integer]
  Let $L$ be a number field. An \emph{algebraic integer} is an $\alpha \in L$ such that there is some monic $f \in \Z[x]$ with $f(\alpha) = 0$. We write $\mathcal{O}_L$ for the set of algebraic integers in $L$. This is a generalization of $\Z \subseteq \Q$.
\end{defi}

These are what we are going to study here. Since we say this is a generalization of $\Z \subseteq \Q$, the following better be true:

\begin{lemma}
  $\mathcal{O}_\Q = \Z$, ie. $\alpha \in \Q$ is an algebraic integer if and only if $\alpha \in \Z$.
\end{lemma}

\begin{proof}
  If $\alpha \in \Z$, then $x - \alpha \in \Z[x]$ is a monic polynomial. So $\alpha \in \mathcal{O}_\Q$.

  On the other hand, let $\alpha \in \Q$. Then there is some coprime $r, s \in \Z$ such that $\alpha = \frac{r}{s}$. If it is an algebraic integer, then there is some
  \[
    f(x) = x^n + a_{n - 1} x^{n - 1} + \cdots + a_0
  \]
  with $a_i \in \Z$ such that $f(\alpha) = 0$. Substituting in and multiplying by $s^n$, we get
  \[
    r^n + \underbrace{a_{n - 1} r^{n - 1}s + \cdots + a_0 s^n}_{\text{divisible by }s} = 0,
  \]
  So $s\mid r^n$. But if $s\not= 1$, there is a prime $p$ such that $p \mid s$, and hence $p \mid r^n$. Thus $p \mid r$. So $p$ is a common factor of $s$ and $r$. This is a contradiction. So $s = 1$, and $\alpha$ is an integer.
\end{proof}

How else is this a generalization of $\Z$? We know $\Z$ is a ring. So perhaps $\mathcal{O}_L$ also is.

\begin{thm}
  $\mathcal{O}_L$ is a ring, ie. if $\alpha, \beta \in \mathcal{O}_L$, then so is $\alpha \pm \beta$ and $\alpha\beta$.
\end{thm}
Note that usually $\frac{1}{\alpha} \not \in \mathcal{O}_L$.

Before proving this, we note that this is a refinement of something we've proved in IID Galois Theory. Recall that if $L/K$ is a field extension with $\alpha, \beta \in L$ algebraic over $K$, then so is $\alpha\beta$ and $\alpha \pm \beta$, as well as $\frac{1}{\alpha}$ if $\alpha \not= 0$.

The proof is as follows: consider $K[\alpha, \beta] \subseteq L$. Then $K[\alpha, \beta]$ is a finite-dimensional vector space over $K$, spanned by $\alpha^i \beta^j$ with $1 \leq i \leq n$, $1 \leq j \leq m$, where $n$ and $m$ are the degrees of $\alpha$. So if $x \in K[\alpha, \beta]$, then $1, x, x^2, \cdots, x^N$ is linearly dependent for sufficiently large $N$. So $x$ satisfies an algebraic equation. We apply this to $\alpha \pm \beta, \alpha\beta$ etc. The crux of this proof is that finiteness implies algebraicity.

We would like to prove this theorem in an analogous way. However, in this case, we are looking at polynomials over $\Z$, which is a ring, not a field. Hence we need to extend the notion of being algebraic and finite to rings.

\begin{defi}[Integrality]
  Let $R \subseteq S$ be rings. We say $\alpha \in S$ is \emph{integral over $R$} if there is some monic polynomial $f \in R[x]$ such that $f(\alpha) = 0$.

  We say $S$ is \emph{integral over $R$} if all $\alpha \in S$ are integral over $R$.

  We say $S$ is \emph{finitely-generated} over $R$ if there exists elements $\alpha_1, \cdots, \alpha_n \in S$ such that the function $R^n \to S$ defined by $(r_1, \cdots, r_n) \mapsto \sum r_i \alpha_i$ is surjective, ie. every element of $S$ can be written as a $R$-linear combination of elements $\alpha_1, \cdots, \alpha_n$. In other words, $S$ is finitely-generated as an $R$-module.
\end{defi}
This is a refinement of the idea of being algebraic. We allow the use of rings and restrict to monic polynomials. In Galois theory, we showed that finiteness and algebraicity ``are the same thing''. We will generalize this to integrality of rings.

\begin{eg}
  In a number field $\Z \subseteq \Q \subseteq L$, $\alpha \in L$ is an algebraic integer if and only if $\alpha$ is integral over $\Z$ by definition, and $\mathcal{O}_L$ is integral over $\Z$.
\end{eg}

\begin{notation}
  If $\alpha_1, \cdots, \alpha_r \in S$, we write $R[\alpha_1, \cdots, \alpha_r]$ for the subring of $S$ generated by $R, \alpha_1, \cdots, \alpha_r$. In other words, it is the image of the homomorphism from the polynomial ring $R[x_1, \cdots, x_n] \to S$ given by $x_i \mapsto \alpha_i$.
\end{notation}

\begin{prop}\leavevmode
  \begin{enumerate}
    \item Let $R \subseteq S$ be rings. If $S = R[s]$ and $s$ is integral over $R$, then $S$ is finitely-generated over $R$.
    \item If $S = R[s_1, \cdots, s_n]$ with $s_i$ integral over $R$, then $S$ is finitely-generated over $R$.
  \end{enumerate}
\end{prop}
This is the easy direction in identifying integrality with finitely-generated.

\begin{proof}\leavevmode
  \begin{enumerate}
    \item We know $S$ is spanned by $1, s, s^2, \cdots$ over $R$. However, since $s$ is integral, there exists $a_0, \cdots, a_n \in R$ such that
      \[
        s^n = a_0 + a_1 s + \cdots + a_{n - 1}s^{n - 1}.
      \]
      So the $R$-submodule generated by $1, s, \cdots, s^{n - 1}$ is stable under multiplication by $s$. So it contains $s^n, s^{n + 1}, s^{n + 2}, \cdots$. So it is $S$.
    \item Let $S_i = R[s_1, \cdots, s_i]$. So $S_i = S_{i - 1}[s_i]$. Since $s_i$ is integral over $R$, it is integral over $s_{i - 1}$. By the previous part, $S_i$ is finitely-generated over $S_{i - 1}$. To finish, it suffices to show that being finitely-generated is transitive. More precisely, if $A \subseteq B \subseteq C$ are rings, $B$ is finitely generated over $A$ and $C$ is finitely generated over $B$, then $C$ is finitely generated over $A$. This is left as an exercise for the reader. % exercise
  \end{enumerate}
\end{proof}
The converse to this is harder.

\begin{thm}
  If $S$ is finitely-generated over $R$, then $S$ is integral over $R$.
\end{thm}
The idea of the proof is as follows: if $s \in S$, we need to find a monic polynomial which it satisfies. In Galois theory, we have fields and vector spaces, and the proof is easy. We can just consider $1, s, s^2, \cdots$, and linear dependence kicks in and gives us a relation. But even if this worked in our case, there is no way we can make this polynomial monic.

Instead, consider the multiplication by $s$ map: $m_s: S \to S$ by $\gamma \mapsto s\gamma$. If $S$ were a finite-dimensional vector space over $R$, then Cayley-Hamilton tells us $m_s$, and thus $s$, satisfies its characteristic polynomial, which is monic. Even though $S$ is not a finite-dimensional vector space, the proof of Cayley-Hamilton will work.

\begin{proof}
  Let $\alpha_1, \cdots, \alpha_n$ generate $S$ as an $R$-module. wlog take $\alpha_1 = 1 \in S$. For any $s \in S$, write
  \[
    s \alpha_i = \sum b_{ij}\alpha_j
  \]
  for some $b_{ij} \in R$. We write $B = (b_{ij})$. This is the ``matrix of multiplication by $S$''. By construction, we have
  \[
    (sI - B)
    \begin{pmatrix}
      \alpha_1\\\vdots\\a_n
    \end{pmatrix} = 0.\tag{$*$}
  \]
  Now recall for any matrix $X$, we have $\adj(X)X = (\det X) I$, where the $i, j$th entry of $\adj(X)$ is given by the determinant of the matrix obtained by removing the $i$th row and $j$th column of $X$.

  We now multiply $(*)$ by $\adj(s I - B)$. So we get
  \[
    \det(sI - B)
    \begin{pmatrix}
      \alpha_1\\\vdots\\\alpha_n
    \end{pmatrix} = 0
  \]
  In particular, $\det(sI - B) \alpha_1 = 0$. Since we picked $\alpha_1 = 1$, we get $\det(sI - B) = 0$. Hence if $f(x) = \det(xI - B)$, then $f(x) \in R[x]$, and $f(s) = 0$.
\end{proof}

Hence we obtain the following:
\begin{cor}
  let $L \supseteq \Q$ be a number field. Then $\mathcal{O}_L$ is a ring.
\end{cor}

\begin{proof}
  if $\alpha, \beta \in \mathcal{O}_L$, then $\Z[\alpha, \beta]$ is finitely-generated by the proposition. But then $\Z[\alpha, \beta]$ is integral over $\Z$, by the previous theorem. So $\alpha \pm \beta, \alpha\beta \in \Z[\alpha, \beta]$.
\end{proof}

\begin{cor}
  If $A \subseteq B \subseteq C$ be ring extensions such that $B$ over $A$ and $C$ over $B$ are integral extensions. Then $C$ is integral over $A$.
\end{cor}

\begin{proof}
  If $c \in C$, let $f(x) = \sum_{0}^N b_i x^i \in B[x]$ be a monic polynomial such that $f(c) = 0$. Let $B_0 = A[b_0, \cdots, b_N]$ and let $C_0 = B_0[c]$. Then $B_0/A$ is finitely generated as $b_0, \cdots, b_N$ are integral over $A$. Also, $C_0$ is finitely-generated over $B_0$, since $c$ is integral over $B_0$. Hence $C_0$ is finitely-generated over $A$. So $c$ is integral over $A$.
\end{proof}

Note that $C$ could have been infinitely generated over $A$, such as $C = \{\alpha \in \C: \alpha\text{ is an algebraic integer}\}$, $A = \Z$ and $B = \Z[i]$. This is why we have to create our $C_0$. The point is that for any $c \in C$, only finitely many elements are involved in the $n$-degree polynomial relation on it.

Exercise: show that $\mathcal{O}_{\Q[i]} = \Z[i]$.

Recall that if $K \subseteq L$ is a field extensions with $\alpha \in L$, then the minimal polynomial is the \emph{monic polynomial} $p_\alpha(t) \in K[t]$ of minimal degree such that $p_\alpha(\alpha) = 0$.

Recall that we had the following lemma about minimal polynomials:
\begin{lemma}
  If $f \in K[x]$ with $f(\alpha) = 0$, then $p_\alpha \mid f$.
\end{lemma}

\begin{proof}
  Write $f = p_\alpha h + r$, with $r \in K[x]$ and $\deg(r) < \deg (p_\alpha)$. Then we have
  \[
    0 = f(\alpha) = p(\alpha) h(\alpha) + r(\alpha) = r(\alpha).
  \]
  So if $r \not= 0$, this contradicts the minimality of $\deg p_\alpha$.
\end{proof}

This lemma implies $p_\alpha$ is unique. For number fields, we have the following:
\begin{prop}
  Let $L$ be a number field. Then $\alpha \in \mathcal{O}_L$ if and only if the minimal polynomial $p_\alpha(x) \in \Q[z]$ for the field extension $\Q \subseteq L$ is in fact in $\Z[t]$.
\end{prop}
This is a nice proposition. This gives us an necessary and sufficient condition for whether something is algebraic.
\begin{proof}
  $(\Leftarrow)$ is trivial, since this is just the definition of an algebraic integer.

  $(\Rightarrow)$ Let $\alpha \in \mathcal{O}_L$ and $p_\alpha \in \Q[x]$ be the minimal polynomial of $\alpha$, and $h(x) \in \Z[x]$ be a monic polynomial which $\alpha$ satisfies. Now there exists a bigger field $M \supseteq L$ such that
  \[
    p_\alpha(x) = (x - \alpha_1) \cdots (x - \alpha_r)
  \]
  factors in $M[x]$. But by our lemma, $p_\alpha \mid h$. So $h(\alpha_i) = 0$ for all $\alpha_i$. So $\alpha_L \in \mathcal{O}_M$ is an algebraic integer. By $\mathcal{O}_M$ is a ring, ie. sums and products of the $\alpha_i$'s are still algebraic integers. So the coefficients of $p_\alpha$ are algebraic integers (in $\mathcal{O}_M$). But they are also in $\Q$. By the first result we proved, the $\alpha_i$ must be integers.
\end{proof}
Alternatively, we can deduce this proposition from the previous lemma plus Gauss' lemma.

Another relation between $\Z$ and $\Q$ is that $\Q$ is the fraction field of $\Z$. This is true for general number fields
\begin{lemma}
  We have
  \[
    \mathrm{Frac} \mathcal{O}_L = \left\{\frac{\alpha}{\beta}: \alpha, \beta \in \mathcal{O}_L, \beta \not= 0\right\} = L.
  \]
  In fact, for any $\alpha \in L$, there is some $n \in \Z$ such that $n\alpha \in \mathcal{O}_L$.
\end{lemma}

\begin{proof}
  if $\alpha \in L$, let $g(x) \in \Q[x]$ be its monic minimal polynomial. Then there exists $n \in \Z$ non-zero such that $ng(x) \in \Z[x]$ (pick $n$ to be the least common multiple of the denominators of the coefficients of $g(x)$). Now the magic is to put
  \[
    h(x) = n^{\deg(g)}g\left(\frac{x}{n}\right).
  \]
  Then this is a monic polynomial with integral coefficients --- in effect, we have just multiplied the coefficient of $x^i$ by $n^{\deg(g) - i}$! Then $h(n\alpha) = 0$. So $n\alpha$ is integral.
\end{proof}

\subsection{Norms and traces}
Recall from IID Galois Theory that if $K \subseteq L$ are fields, $\alpha \in L$ and $m_\alpha: L \to L$ is the function $\ell \mapsto \alpha \ell$, we define
\begin{defi}[Norm and trace]
  The \emph{norm} of $\alpha$ is
  \[
    N_{L/K}(\alpha) = \det m_\alpha.
  \]
  The \emph{trace} is
  \[
    \tr_{L/K} = \tr m_\alpha.
  \]
\end{defi}

We have also proved the proposition
\begin{prop}
  Let $p_\alpha \in K[x]$ be the minimal polynomial of $\alpha$. Then the characteristic polynomial of $m_\alpha$ is
  \[
    \det(xI - m_\alpha) = p_\alpha^{[L:K(\alpha)]}
  \]
  Hence if $p_\alpha(x)$ splits in some field $L'\supseteq K(\alpha)$, say
  \[
    p_\alpha(x) = (x - \alpha_1) \cdots (x - \alpha_r),
  \]
  then
  \[
    N_{K(\alpha)/K}(\alpha) = \prod \alpha_i,\quad \tr_{K(\alpha)/K}(\alpha) = \sum \alpha_i,
  \]
  and hence
  \[
    N_{L/K}(\alpha) = \left(\prod \alpha_i\right)^{[L:K(\alpha)]},\quad \tr_{L/K} = [L:K(\alpha)] \left(\sum \alpha_i\right).
  \]
\end{prop}
This was proved in the IID Galois Theory course, and we will just use it without proving.

\begin{prop}
  Let $L \supseteq \Q$ be a number field. Then $\alpha \in \mathcal{O}_L$ if and only if the minimal polynomial $p_\alpha \in \Z[x]$ if and only if the characteristic polynomial of $m_\alpha \in \Z[x]$. This in particular implies $N_{L/\Q}(\alpha) \in \Z$ and $\tr_{L/\Q}(\alpha) \in \Z$.
\end{prop}

\begin{proof}
  The first equivalence was proven. For the second equivalence, if $m_\alpha \in \Z[x]$, then $\alpha \in \mathcal{O}_L$ since it vanishes on a monic polynomial in $\Z[x]$. On the other hand, if $p_\alpha \in \Z[x]$, then so is the characteristic polynomial, since it is just $p_\alpha^N$.

  The final implication comes from the fact that the norm and trace are just coefficients of the characteristic polynomial.
\end{proof}

It would be nice if the last implication is an if and only if. This occurs, obviously, when the characteristic polynomial is quadratic, since the norm and trace would be the only coefficients.
\begin{eg}
  Let $L = K(\sqrt{d}) = K[z]/(z^2 - d)$, where $d$ is not a square in $K$. As a vector space over $K$, we can take $1, \sqrt{d}$ as our basis. So every $\alpha$ can be written as
  \[
    \alpha = x + y\sqrt{d}.
  \]
  Hence the matrix of multiplication by $\alpha$ is
  \[
    m_\alpha =
    \begin{pmatrix}
      x & dy\\
      y & x
    \end{pmatrix}.
  \]
  So the trace and norm are given by
  \begin{align*}
    \tr_{L/K} (x + y\sqrt{d}) &= 2x = (x + y\sqrt{d}) + (x - y\sqrt{d})\\
    N_{L/K} (x + y \sqrt{d}) &= x^2 - dy^2 = (x + y\sqrt{d})(x- y\sqrt{d})
  \end{align*}
\end{eg}

Now that we have computed the general trace and norm, we can use the proposition to find out what the algebraic integers are.
\begin{lemma}
  Let $L = \Q(\sqrt{d})$, where $d \in \Z$ is not $0, 1$ and is square-free. Then
  \[
    \mathcal{O}_L =
    \begin{cases}
      \Z[\sqrt{d}] & d \equiv 2 \text{ or }3\pmod 4\\
      \Z\left[\frac{1}{2}(1 + \sqrt{d})\right] & d \equiv 1 \pmod 4
    \end{cases}
  \]
\end{lemma}
This result is weird --- intuitively it makes sense if it is always $\Z[\sqrt{d}]$. However, it turns out sticking with our definition of algebraic integers, we get much nicer properties.

\begin{proof}
  We know $x + y \sqrt{\lambda} \in \mathcal{O}_L$ if and only if $2x, x^2 - dy^2 \in \Z$ by the previous example. These imply $4dy^2 \in \Z$. So if $y = \frac{r}{s}$ with $r, s$ coprime, $r, s \in \Z$, then we must have $s^2 \mid 4d$. But $d$ is square-free. So $s = 1$ or $2$. So
  \[
    x = \frac{u}{2},\quad y = \frac{v}{2}
  \]
  for some $u, v \in \Z$. Then we know $u^2 - dv^2 \in 4\Z$, ie. $u^2 \equiv dv^2 \mod 4$. But we know the squares mod $4$ are always $0$ and $1$. So if $d \not\equiv 1 \pmod 4$, then $u^2 \equiv dv^2\pmod 4$ imply that $u^2 = v^2 = 0\pmod 4$, and hence $u, v$ are even. So $x, y \in \Z$, giving $\mathcal{O}_L = \Z[\sqrt{d}]$.

  On the other hand, if $d \equiv 1\pmod 4$, then $u, v$ have the same parity mod 2, ie. we can write $x + y\sqrt{d}$ as a $\Z$-combination of $1$ and $\frac{1}{2}(1 + \sqrt{d})$.

  As a sanity check, we find that the minimal polynomial of $\frac{1}{2}(1 + \sqrt{d})$ is $x^2 - x + \frac{1}{4}(1 - d)$ which is in $\Z$ if and only if $d \equiv 1 \pmod 4$.
\end{proof}

We will need the following fact from Galois theory.
\begin{thm}[Primitive element theorem]
  Let $K \subseteq L$ be separable field extensions, $\beta \in L$ and $K(\beta) \subseteq L$. Then there exists an $\alpha \in L$ such that $K(\alpha) = K$.
\end{thm}

For example, $\Q(\sqrt{2}, \sqrt{3}) = \Q(\sqrt{2} + \sqrt{3})$.

Hence in particular, if $L \supseteq \Q$, a number field, there is some $\alpha$ such that $L = \Q(\alpha)$. So if $p_\alpha \in \Q[x]$ is its minimal polynomial, then
\[
  \frac{\Q[x]}{(p_\alpha)} \cong \Q(\alpha) = L.
\]
Also, we have
\[
  [L:\Q] = n = \deg p_\alpha.
\]
Then $L$ has basis $1, \alpha, \alpha^2, \cdots, \alpha^{n - 1}$. Moreover, we have
\begin{lemma}
  $n$ is the number of field embeddings $L \hookrightarrow \C$.
\end{lemma}

\begin{proof}
  Let $p_\alpha(x) \in \Q[x]$ be irreducible. So $p_\alpha$ and $p_\alpha' = \frac{\d}{\d x}p_\alpha$ have no common factors, ie.
  \[
    p_\alpha(x) = (x - \alpha_1) \cdots (x - \alpha_n)
  \]
  has $n$ distinct roots in $\C$, namely $\alpha_1, \cdots, \alpha_n \in \C$.

  A field map $\Q[x]/(p_\alpha) \to \C$ is automatically $\Q$-linear. So we must send $x$ to a root of $p_\alpha(x)$ in $\C$, ie. one of $\alpha_1, \cdots, \alpha_n$, and is determined uniquely by this. Conversely, sending $x \mapsto \alpha_i$ determines one such map.
\end{proof}

This allows us to define an invariant that refines $n = [L:\Q]$.

\begin{defi}[$r$ and $s$]
  Let $r$ be the number of real roots of $p_\alpha$ and $s$ be the number of pairs of complex conjugate roots. So
  \[
    n = r + 2s,\quad s = \frac{n - r}{2}.
  \]
  This is independent of the choice of $\alpha$, as $r$ is the number of field embeddings $L \hookrightarrow \R$.
\end{defi}
This will be really important for the second half of the course.

\begin{cor}
  Let $L \supseteq \Q$ be a number field. If $\sigma_1, \cdots, \sigma_n:L \to \C$ are the different field embeddings and $\beta \in L$, then
  \[
    \tr_{L/\Q}(\beta) = \sum \sigma_i(\beta),\quad N_{L/\Q}(\beta) = \prod_i \sigma_i(\beta).
  \]
  We call $\sigma_1(\beta), \cdots, \sigma_n (\beta)$ the \emph{conjugates} of $\beta$ in $\C$.
\end{cor}
Proof is in the Galois theory course.

\begin{prop}
  Let $L\supseteq K$ be a separable extension. Then a $K$-bilinear form $L \times L \to K$ defined by $(x, y) \mapsto \tr_{L/K}(xy)$ is non-degenerate. Equivalent, if $\alpha_1,\cdots, \alpha_n$ are a $K$-basis for $L$, the Gram matrix $(\tr(\alpha_i\alpha_j))_{i, j = 1, \cdots, n}$ has non-zero determinant.
\end{prop}
Recall from Galois theory that if $K \subseteq L$ is \emph{not} separable, then $\tr_{L/K} = 0$, and it is very \emph{very} degenerate.

We will use the following important notation:
\begin{notation}
  \[
    \Delta(\alpha_1, \cdots, \alpha_n) = \det(\tr_{L/K}(\alpha_i \alpha_j)).
  \]
\end{notation}

\begin{proof}
  Let $\sigma_1, \cdots, \sigma_n: L \to \bar{K}$ be distinct the $n$ distinct $K$-linear field embeddings $L \hookrightarrow \bar{K}$. They are distinct as $L \supseteq K$ is separable. Put
  \[
    S = (\sigma_i(\alpha_j))_{i, j = 1, \cdots, n} =
    \begin{pmatrix}
      \sigma_1(\alpha_1) & \cdots & \sigma_1(\alpha_n)\\
      \vdots & \ddots & \vdots\\
      \sigma_n(\alpha_1) & \cdots & \sigma_n(\alpha_n).
    \end{pmatrix}
  \]
  Then
  \[
    S^T S = \left(\sum_{k = 1}^n \sigma_k(\alpha_i)\sigma_k(\alpha_j)\right)_{i,j = 1, \cdots n}.
  \]
  We know $\sigma_k$ is a field homomorphism. So
  \[
    \sum_{k = 1}^n \sigma_k (\alpha_i)\sigma_k(\alpha_j) = \sum_{k = 1}^n \sigma_k(\alpha_i \alpha_j) = \tr_{L/K}(\alpha_i \alpha_j).
  \]
  So
  \[
    S^T S = (\tr(\alpha_i\alpha_j))_{i, j = 1, \cdots, n}.
  \]
  So we have
  \[
    \Delta(\alpha_1, \cdots, \alpha_n) = \det(S^T S) = \det(S)^2.
  \]
  Now we use the theorem of primitive elements to write $L = K(\theta)$ such that $1, \theta, \cdots, \theta^{n - 1}$ is a basis for $L$ over $K$, with $[L:K] = n$. Now $S$ is just
  \[
    S =
    \begin{pmatrix}
      1 & \sigma_1(\theta) & \cdots & \sigma_1(\theta)^{n - 1}\\
      \vdots & \vdots & \ddots & \vdots\\
      1 & \sigma_n(\theta) & \cdots & \sigma_n(\theta)^{n - 1}
    \end{pmatrix}.
  \]
  This is a Vandermonde matrix, and so
  \[
    \Delta(1, \theta, \cdots, \theta^{n - 1}) = (\det S)^2 = \prod_{i < j} (\sigma_i(\theta) - \sigma_j(\theta))^2.
  \]
  Since the field extension is separable, and hence $\sigma_i \not= \sigma_j$ for all $i, j$. This implies $\sigma_i (\theta) \not= \sigma_j(\theta)$, since $\theta$ generates the field. so the product above is non-zero.
\end{proof}
So we have this nice canonical bilinear map. However, this determinant is not canonical. Recall that if $\alpha_1, \cdots, \alpha_n$ is a basis for $L/K$, and $\alpha_1', \cdots, \alpha_n'$ is another basis, then
\[
  \alpha_i' = \sum a_{ij}\alpha_j
\]
for some $A = (a_{ij}) \in \GL_n(K)$. So
\[
  \Delta(\alpha_1', \cdots, \alpha_n') = (\det A)^2 \Delta(\alpha_1, \cdots, \alpha_n).
\]
However, for number fields, we will later see that we can pick a ``canonical'' basis, and get a canonical value for $\Delta$. We will call this the discriminant.

We have seen the word discriminant before, and let's make sure these concepts are more-or-less consistent.

 Recall also that the discriminant of a polynomial $f(x) = \prod (x - \alpha_i)$, then the discriminant is defined as
\[
  \disc(f) = \prod_{i \not= j}(\alpha_i - \alpha_j).
\]
If $p_\theta(x) \in K[x]$ is the minimal polynomial of $\theta$ (where $L = K[\theta]$), then
\[
  p_\theta(x) = \prod_{1}^n (x - \sigma_i(\theta))
\]
by the Galois theory result. Hence we get
\[
  \disc(p_\theta) = \prod_{i \not= j}(\sigma_i \theta - \sigma_j \theta).
\]
In other words,
\[
  \disc(p_\theta) = \Delta(1, \theta, \cdots, \theta^{n - 1}).
\]
Now back to number fields.
\begin{defi}[Integral basis]
  Let $L \supseteq \Q$ be a number field. Then a basis $\alpha_1, \cdots, \alpha_n$ of $L$ ais an \emph{integral basis} if
  \[
    \mathcal{O}_L = \left\{\sum_{i = 1}^n m_i \alpha_i: m_i \in \Z\right\} = \bigoplus_1^n \Z\alpha_i.
  \]
  In other words, it is simultaneously a basis for $L$ over $K$ and $\mathcal{O}_L$ over $\Z$.
\end{defi}
Note that integral bases are not unique, just as with usual bases. Given one basis, you can get any other by acting by $\GL_n(\Z)$.

\begin{eg}
  Consider $\Q[\sqrt{d}]$ with $d$ square-free, $d \not= 0, 1$. If $d \cong 1\pmod 4$, we've seen that $1, \frac{1}{2}(1 + \sqrt{\lambda})$ is an integral basis. Otherwise, if $d \cong 2, 3 \pmod 4$, then $1, \sqrt{d}$ is an integral basis.
\end{eg}

The important theorem is that an integral basis always exists.

\begin{thm}
  Let $\Q \subseteq L$ be a number field. Then there exists an integral basis for $\mathcal{O}_L$. In particular, $\mathcal{O}_L \cong \Z^n$ with $n = [L:\Q]$.
\end{thm}

\begin{proof}
  Let $\alpha_1, \cdots, \alpha_n$ be any basis of $L$ over $\Q$. We have proved that there is some $n_i \in \Z$ such that $n_i \alpha_i \in \mathcal{O}_L$. So wlog $\alpha_1, \cdots, \alpha_n \in \mathcal{O}_L$, and are an basis of $L$ over $\Q$. Since $\alpha_i$ are integral, so are $\alpha_i \alpha_j$, and so all these have integer trace, as we have previously shown. Hence $\Delta(\alpha_1, \cdots, \alpha_n)$, being the determinant of a matrix with integer entries, is an integer.

  Now choose a $\Q$-basis $\alpha_1, \cdots, \alpha_n \in \mathcal{O}_L$ such that $\Delta(\alpha_1, \cdots, \alpha_n) \in \Z\setminus \{0\}$ has minimal absolute value. Now let $x \in \mathcal{O}_L$. So
  \[
    x = \sum \lambda_i \alpha_i
  \]
  for some $\lambda_i \in \Q$. These $\lambda_i$ are necessarily unique since $\alpha_1, \cdots, \alpha_n$ is a basis.

  Suppose some $\lambda_i \not\in \Z$. wlog say $\lambda_1 \not \in \Z$. We write
  \[
    \lambda_1 = n_1 + \varepsilon_1,
  \]
  for $n_1 \in \Z$ and $0 < \varepsilon_1 < 1$. We put
  \[
    \alpha_1' = x - n_1 \alpha_1 = \varepsilon_1 \alpha_1 + \lambda_2 \alpha_2 + \cdots + \lambda_n \alpha_n \in \mathcal{O}_L.
  \]
  So $\alpha_1', \alpha_2, \cdots, \alpha_n$ is still a basis for $L/\Q$, and are still in $\mathcal{O}_L$. But then
  \[
    \Delta(\alpha_1', \cdots, \alpha_n) = \varepsilon_1^2 \cdot \Delta(\alpha_1, \cdots, \alpha_n) < \Delta(\alpha_1, \cdots, \alpha_n).
  \]
  This contradicts minimality. So we must have $\lambda_i \in \Z$ for all $\Z$. So this is a basis for $\mathcal{O}_L$.
\end{proof}

Now if $\alpha_1', \cdots, \alpha_n'$ is another integral basis of $L$ over $\Q$, then there is some $g \in \GL_n(\Z)$ such that $g\alpha_i = \alpha_i'$. Since $\det (g)$ is invertible in $\Z$, it must be $1$ or $-1$, and hence
\[
  \det \Delta(\alpha_1', \cdots, \alpha_n') = \det(g)^2 \Delta (\alpha_1, \cdots, \alpha_n) = \Delta(\alpha_1, \cdots, \alpha_n)
\]
and is independent of the choice of integral basis.
\begin{defi}[Discriminant]
  The \emph{discriminant} $D_L$ of a number field $L$ is defined as
  \[
    D_L = \det \Delta(\alpha_1, \cdots, \alpha_n)
  \]
  for any integral basis $\alpha_1, \cdots, \alpha_n$.
\end{defi}
\end{document}
