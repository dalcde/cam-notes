\documentclass[a4paper]{article}

\def\npart {II}
\def\nterm {Lent}
\def\nyear {2016}
\def\nlecturer {S. Martin}
\def\ncourse {Representation Theory}
\def\nlectures {TTS.9}
\def\nnotready {}

% Imports
\ifx \nextra \undefined
  \usepackage[pdftex,
    hidelinks,
    pdfauthor={Dexter Chua},
    pdfsubject={Cambridge Maths Notes: Part \npart\ - \ncourse},
    pdftitle={Part \npart\ - \ncourse},
  pdfkeywords={Cambridge Mathematics Maths Math \npart\ \nterm\ \nyear\ \ncourse}]{hyperref}
  \title{Part \npart\ - \ncourse}
\else
  \usepackage[pdftex,
    hidelinks,
    pdfauthor={Dexter Chua},
    pdfsubject={Cambridge Maths Notes: Part \npart\ - \ncourse\ (\nextra)},
    pdftitle={Part \npart\ - \ncourse\ (\nextra)},
  pdfkeywords={Cambridge Mathematics Maths Math \npart\ \nterm\ \nyear\ \ncourse\ \nextra}]{hyperref}

  \title{Part \npart\ - \ncourse \\ {\Large \nextra}}
\fi

\author{Lectured by \nlecturer \\\small Notes taken by Dexter Chua}
\date{\nterm\ \nyear}

\usepackage{alltt}
\usepackage{amsfonts}
\usepackage{amsmath}
\usepackage{amssymb}
\usepackage{amsthm}
\usepackage{booktabs}
\usepackage{caption}
\usepackage{enumitem}
\usepackage{fancyhdr}
\usepackage{graphicx}
\usepackage{mathtools}
\usepackage{microtype}
\usepackage{multirow}
\usepackage{pdflscape}
\usepackage{pgfplots}
\usepackage{siunitx}
\usepackage{tabularx}
\usepackage{tikz}
\usepackage{tkz-euclide}
\usepackage[normalem]{ulem}
\usepackage[all]{xy}

\pgfplotsset{compat=1.12}

\pagestyle{fancyplain}
\lhead{\emph{\nouppercase{\leftmark}}}
\ifx \nextra \undefined
  \rhead{
    \ifnum\thepage=1
    \else
      \npart\ \ncourse
    \fi}
\else
  \rhead{
    \ifnum\thepage=1
    \else
      \npart\ \ncourse\ (\nextra)
    \fi}
\fi
\usetikzlibrary{arrows}
\usetikzlibrary{decorations.markings}
\usetikzlibrary{decorations.pathmorphing}
\usetikzlibrary{positioning}
\usetikzlibrary{fadings}
\usetikzlibrary{intersections}
\usetikzlibrary{cd}

\newcommand*{\Cdot}{\raisebox{-0.25ex}{\scalebox{1.5}{$\cdot$}}}
\newcommand {\pd}[2][ ]{
  \ifx #1 { }
    \frac{\partial}{\partial #2}
  \else
    \frac{\partial^{#1}}{\partial #2^{#1}}
  \fi
}

% Theorems
\theoremstyle{definition}
\newtheorem*{aim}{Aim}
\newtheorem*{axiom}{Axiom}
\newtheorem*{claim}{Claim}
\newtheorem*{cor}{Corollary}
\newtheorem*{defi}{Definition}
\newtheorem*{eg}{Example}
\newtheorem*{fact}{Fact}
\newtheorem*{law}{Law}
\newtheorem*{lemma}{Lemma}
\newtheorem*{notation}{Notation}
\newtheorem*{prop}{Proposition}
\newtheorem*{thm}{Theorem}

\renewcommand{\labelitemi}{--}
\renewcommand{\labelitemii}{$\circ$}
\renewcommand{\labelenumi}{(\roman{*})}

\let\stdsection\section
\renewcommand\section{\newpage\stdsection}

% Strike through
\def\st{\bgroup \ULdepth=-.55ex \ULset}

% Maths symbols
\newcommand{\bra}{\langle}
\newcommand{\ket}{\rangle}

\newcommand{\N}{\mathbb{N}}
\newcommand{\Z}{\mathbb{Z}}
\newcommand{\Q}{\mathbb{Q}}
\renewcommand{\H}{\mathbb{H}}
\newcommand{\R}{\mathbb{R}}
\newcommand{\C}{\mathbb{C}}
\newcommand{\Prob}{\mathbb{P}}
\renewcommand{\P}{\mathbb{P}}
\newcommand{\E}{\mathbb{E}}
\newcommand{\F}{\mathbb{F}}
\newcommand{\cU}{\mathcal{U}}
\newcommand{\RP}{\mathbb{RP}}
\newcommand{\CP}{\mathbb{CP}}

\newcommand{\ph}{\,\cdot\,}

\DeclareMathOperator{\sech}{sech}
\DeclareMathOperator{\cosech}{cosech}
\DeclareMathOperator{\cosec}{cosec}

\DeclareMathOperator{\covol}{covol}
\DeclareMathOperator{\vol}{vol}

\let\Im\relax
\let\Re\relax
\DeclareMathOperator{\Im}{Im}
\DeclareMathOperator{\Re}{Re}
\DeclareMathOperator{\im}{im}
\DeclareMathOperator{\image}{image}
\DeclareMathOperator{\Ann}{Ann}

\DeclareMathOperator*{\res}{res}
\DeclareMathOperator{\Res}{Res}
\DeclareMathOperator{\Ind}{Ind}

\DeclareMathOperator{\tr}{tr}
\DeclareMathOperator{\diag}{diag}
\DeclareMathOperator{\rank}{rank}
\DeclareMathOperator{\card}{card}
\DeclareMathOperator{\spn}{span}
\DeclareMathOperator{\adj}{adj}

\DeclareMathOperator{\erf}{erf}
\DeclareMathOperator{\erfc}{erfc}

\DeclareMathOperator{\ord}{ord}
\DeclareMathOperator{\Sym}{Sym}

\DeclareMathOperator{\sgn}{sgn}
\DeclareMathOperator{\orb}{orb}
\DeclareMathOperator{\stab}{stab}
\DeclareMathOperator{\ccl}{ccl}

\DeclareMathOperator{\lcm}{lcm}
\DeclareMathOperator{\hcf}{hcf}

\DeclareMathOperator{\Int}{Int}
\DeclareMathOperator{\id}{id}

\DeclareMathOperator{\betaD}{beta}
\DeclareMathOperator{\gammaD}{gamma}
\DeclareMathOperator{\Poisson}{Poisson}
\DeclareMathOperator{\binomial}{binomial}
\DeclareMathOperator{\multinomial}{multinomial}
\DeclareMathOperator{\Bernoulli}{Bernoulli}
\DeclareMathOperator{\like}{like}

\DeclareMathOperator{\var}{var}
\DeclareMathOperator{\cov}{cov}
\DeclareMathOperator{\bias}{bias}
\DeclareMathOperator{\mse}{mse}
\DeclareMathOperator{\corr}{corr}

\DeclareMathOperator{\otp}{otp}
\DeclareMathOperator{\dom}{dom}

\DeclareMathOperator{\Root}{Root}
\DeclareMathOperator{\supp}{supp}
\DeclareMathOperator{\rel}{rel}
\DeclareMathOperator{\Hom}{Hom}
\DeclareMathOperator{\Aut}{Aut}
\DeclareMathOperator{\Gal}{Gal}
\DeclareMathOperator{\Mat}{Mat}
\DeclareMathOperator{\End}{End}
\DeclareMathOperator{\Char}{char}
\DeclareMathOperator{\ev}{ev}
\DeclareMathOperator{\St}{St}
\DeclareMathOperator{\Lk}{Lk}
\DeclareMathOperator{\disc}{disc}
\DeclareMathOperator{\Isom}{Isom}
\DeclareMathOperator{\length}{length}
\DeclareMathOperator{\energy}{energy}
\DeclareMathOperator{\area}{area}
\DeclareMathOperator{\Syl}{Syl}
\DeclareMathOperator{\cl}{cl}
\DeclareMathOperator{\fix}{fix}

\newcommand{\GL}{\mathrm{GL}}
\newcommand{\SL}{\mathrm{SL}}
\newcommand{\PGL}{\mathrm{PGL}}
\newcommand{\PSL}{\mathrm{PSL}}
\newcommand{\PSU}{\mathrm{PSU}}
\newcommand{\Or}{\mathrm{O}}
\newcommand{\SO}{\mathrm{SO}}
\newcommand{\U}{\mathrm{U}}
\newcommand{\SU}{\mathrm{SU}}

\renewcommand{\d}{\mathrm{d}}
\newcommand{\D}{\mathrm{D}}

\tikzset{->/.style = {decoration={markings,
                                  mark=at position 1 with {\arrow[scale=2]{latex'}}},
                      postaction={decorate}}}
\tikzset{<-/.style = {decoration={markings,
                                  mark=at position 0 with {\arrowreversed[scale=2]{latex'}}},
                      postaction={decorate}}}
\tikzset{<->/.style = {decoration={markings,
                                   mark=at position 0 with {\arrowreversed[scale=2]{latex'}},
                                   mark=at position 1 with {\arrow[scale=2]{latex'}}},
                       postaction={decorate}}}
\tikzset{->-/.style = {decoration={markings,
                                   mark=at position #1 with {\arrow[scale=2]{latex'}}},
                       postaction={decorate}}}
\tikzset{-<-/.style = {decoration={markings,
                                   mark=at position #1 with {\arrowreversed[scale=2]{latex'}}},
                       postaction={decorate}}}

\tikzset{circ/.style = {fill, circle, inner sep = 0, minimum size = 3}}
\tikzset{mstate/.style={circle, draw, blue, text=black, minimum width=0.7cm}}

\definecolor{mblue}{rgb}{0.2, 0.3, 0.8}
\definecolor{morange}{rgb}{1, 0.5, 0}
\definecolor{mgreen}{rgb}{0.1, 0.4, 0.2}
\definecolor{mred}{rgb}{0.5, 0, 0}

\def\drawcirculararc(#1,#2)(#3,#4)(#5,#6){%
    \pgfmathsetmacro\cA{(#1*#1+#2*#2-#3*#3-#4*#4)/2}%
    \pgfmathsetmacro\cB{(#1*#1+#2*#2-#5*#5-#6*#6)/2}%
    \pgfmathsetmacro\cy{(\cB*(#1-#3)-\cA*(#1-#5))/%
                        ((#2-#6)*(#1-#3)-(#2-#4)*(#1-#5))}%
    \pgfmathsetmacro\cx{(\cA-\cy*(#2-#4))/(#1-#3)}%
    \pgfmathsetmacro\cr{sqrt((#1-\cx)*(#1-\cx)+(#2-\cy)*(#2-\cy))}%
    \pgfmathsetmacro\cA{atan2(#2-\cy,#1-\cx)}%
    \pgfmathsetmacro\cB{atan2(#6-\cy,#5-\cx)}%
    \pgfmathparse{\cB<\cA}%
    \ifnum\pgfmathresult=1
        \pgfmathsetmacro\cB{\cB+360}%
    \fi
    \draw (#1,#2) arc (\cA:\cB:\cr);%
}
\newcommand\getCoord[3]{\newdimen{#1}\newdimen{#2}\pgfextractx{#1}{\pgfpointanchor{#3}{center}}\pgfextracty{#2}{\pgfpointanchor{#3}{center}}}

\def\Xint#1{\mathchoice
   {\XXint\displaystyle\textstyle{#1}}%
   {\XXint\textstyle\scriptstyle{#1}}%
   {\XXint\scriptstyle\scriptscriptstyle{#1}}%
   {\XXint\scriptscriptstyle\scriptscriptstyle{#1}}%
   \!\int}
\def\XXint#1#2#3{{\setbox0=\hbox{$#1{#2#3}{\int}$}
     \vcenter{\hbox{$#2#3$}}\kern-.5\wd0}}
\def\ddashint{\Xint=}
\def\dashint{\Xint-}


\begin{document}
\maketitle
{\small
\noindent\textbf{Representations of finite groups}\\
Representations of groups on vector spaces, matrix representations. Equivalence of representations. Invariant subspaces and submodules. Irreducibility and Schur's Lemma. Complete reducibility for finite groups. Irreducible representations of Abelian groups.

\vspace{10pt}
\noindent\textbf{Character theory}\\
Determination of a representation by its character. The group algebra, conjugacy classes, and orthogonality relations. Regular representation. Permutation representations and their characters. Induced representations and the Frobenius reciprocity theorem. Mackey's theorem. Frobenius's Theorem.\hspace*{\fill}[12]

\vspace{10pt}
\noindent\textbf{Arithmetic properties of characters}\\
Divisibility of the order of the group by the degrees of its irreducible characters. Burnside's $p^a q^b$ theorem.\hspace*{\fill}[2]

\vspace{10pt}
\noindent\textbf{Tensor products}\\
Tensor products of representations and products of characters. The character ring. Tensor, symmetric and exterior algebras.\hspace*{\fill}[3]

\vspace{10pt}
\noindent\textbf{Representations of $S^1$ and $SU_2$}\\
The groups $S^1$, $SU_2$ and $SO(3)$, their irreducible representations, complete reducibility. The Clebsch-Gordan formula. *Compact groups.*\hspace*{\fill}[4]

\vspace{10pt}
\noindent\textbf{Further worked examples}\\
The characters of one of $GL_2(F_q), S_n$ or the Heisenberg group.\hspace*{\fill}[3]}

\tableofcontents
\setcounter{section}{-1}
\section{Introduction}
The course studies how groups act as groups of linear transformations on vector spaces. Hopefully, you understand all the words in this sentence. If so, this is a good start.

In our case, groups are usually either finite groups or topological compact groups (to be defined later). Topological compact groups are typically subgroups of the general linear group over some infinite fields. It turns out the tools we have for finite groups often work well for these particular kinds of infinite groups. The vector spaces are always finite-dimensional, and usually over $\C$.

Prerequisites of this course include knowledge of group theory (as much as the IB Groups, Rings and Modules course), linear algebra, and, optionally, geometry, Galois theory and metric and topological spaces. There is one lemma where we must use something from Galois theory, but if you don't know about Galois theory, you can just assume the lemma to be true without bothering yourself too much about the proofs.

% C Curtis: pioneers of representation theory

\section{Group actions}

\subsection{Basic linear algebra}
We first set out our notations:

\begin{notation}
  $\F$ always represents a field.
\end{notation}

Usually, we take $\F = \C$, but sometimes it can also be $\R$ or $\Q$. These fields all have characteristic zero, and in this case, we call what we're doing \emph{ordinary} representation theory. Sometimes, we will take $F = \F_p$ or $\bar{\F}_p$, the algebraic closure of $\F_p$. This is called \emph{modular} representation theory.

\begin{notation}
  We write $V$ for a vector space over $\F$ --- this will always be finite dimensional over $\F$. We write $\GL(V)$ for the group of invertible linear maps $\theta: V \to V$. This is a group with the operation given by composition of maps, with the identity as the identity map (and inverse by inverse).
\end{notation}

\begin{notation}
  Let $V$ be a finite-dimensional vector space over $\F$. We write $\End(V)$ for the endomorphism algebra, the set of all linear maps $V \to V$.
\end{notation}

We recall a couple of facts from linear algebra:

If $\dim_\F V = n < \infty$, we can choose a basis $\mathbf{e}_1, \cdots, \mathbf{e}_n$ of $V$ over $\F$. So we can identify $V$ with $\F^n$. Then every endomorphism $\theta \in \GL(V)$ corresponds to a matrix $A_\theta = (a_{ij}) \in M_n(F)$ given by
\[
  \theta(\mathbf{e}_j) = \sum_i a_{ij} \mathbf{e}_i.
\]
In fact, we have $A_\theta \in \GL_n(\F)$, the general linear group. It is easy to see the following:
\begin{prop}
  As groups, $\GL(V) \cong \GL_n(\F)$, with the isomorphism given by $\theta \mapsto A_\theta$.
\end{prop}

Of course, picking a different basis of $V$ gives a different isomorphism to $\GL_n(\F)$, but we have the following fact:
\begin{prop}
  Matrices $A_1, A_2$ represent the same element of $\GL(V)$ with respect to different bases if and only if they are \emph{conjugate}, namely there is some $X \in \GL_n(\F)$ such that
  \[
    A_2 = XA_1 X^{-1}.
  \]
\end{prop}

Recall that $\tr(A) = \sum_i a_{ii}$, where $A = (a_{ij}) \in M_n(\F)$, is the trace of $A$. A nice property of the trace is that it doesn't notice conjugacy:
\begin{prop}
  \[
    \tr(XAX^{-1}) = \tr (A).
  \]
\end{prop}

Hence we can define the trace of an operator $\tr(\theta) = \tr(A_\theta)$, which is independent of our choice of basis. This is an important result. When we study representations, we will have matrices flying all over the place, and often we just look at the trace of these matrices. This reduces our problem of studying matrices to plain arithmetic.

When we have too many matrices, we get confused. So we want to put a matrix into a form as simple as possible. One of the simplest form a matrix can take is being diagonal. So we want to know something about diagonalizing matrices.

\begin{prop}
  Let $\alpha \in \GL(V)$, where $V$ is a finite-dimensional vector space over $\C$ and $\alpha^m = \id$ for some $m$. Then $\alpha$ is diagonalizable.
\end{prop}

\begin{prop}
  Let $V$ be a finite-dimensional vector space over $\C$, and $\alpha \in \End(V)$, not necessarily invertible. Then $\alpha$ is diagonalizable if and only if there is a polynomial $f$ with distinct linear factors with $f(\alpha) = 0$.
\end{prop}
This proposition allows us to prove the previous proposition immediately, noting that $X^m - 1 = \prod(X - \omega^j)$, where $\omega = e^{2\pi i/m}$.

Instead of just one endomorphism, we can look at many endomorphisms.

\begin{prop}
  A finite family of separately diagonalizable endomorphisms of a vector space over $\C$ can be simultaneously diagonalized if and only if they commute.
\end{prop}

\subsection{Basic group theory}
We start by listing all our favorite groups.

\begin{defi}[Symmetric group $S_n$]
  The \emph{symmetric group} $S_n$ is the set of all permutations of $X = \{1, \cdots, n\}$, ie. the set of all bijections $X \to X$. We have $|S_n| = n!$.
\end{defi}

\begin{defi}[Alternating group $A_n$]
  The \emph{alternating group} $A_n$ is the set of products of an even number of transpositions $(i\; j)$ in $S_n$. We know $|A_n| = \frac{n!}{2}$. So this is a subgroup of index 2 and hence normal.
\end{defi}

\begin{defi}[Cyclic group $C_m$]
  The \emph{cyclic group} of order $m$, written $C_m$ is
  \[
    C_m = \bra x: x^m = 1\ket.
  \]
  This also occurs naturally, as $\Z/m\Z$ over addition, and also the group of $n$th roots of unity in $\C$. We can view this as a subgroup of $\GL_1(\C) \cong \C^*$. Alternatively, this is the group of rotation symmetries of a regular $m$-gon in $\R^2$, and can be viewed as a subgroup of $\GL_2(\R)$.
\end{defi}

\begin{defi}[Dihedral group $D_{2m}$]
  The \emph{dihedral group} $D_{2m}$ of order $2m$ is
  \[
    D_{2m} = \bra x, y: x^m = y^2 = 1, yxy^{-1} = x^{-1}\ket.
  \]
  This is the symmetry group of a regular $m$-gon. The $x^i$ are the rotations and $x^i y$ are the reflections. For example, in $D_8$, $x$ is rotation by $\frac{\pi}{2}$ and $y$ is any reflection.

  This group can be viewed as a subgroup of $\GL_2(\R)$, but since it also acts on the vertices, it can be viewed as a subgroup of $S_m$.
\end{defi}

\begin{defi}[Quaternion group]
  The \emph{quaternion group} is given by
  \[
    Q_8 = \bra x, y: x^4 = 1, y^2 = x^2, yxy^{-1} = x^{-1}\ket.
  \]
  This has order $8$, and we write $i = x, j = y$, $k = ij$, $-1 = i^2$, with
  \[
    Q_8 = \{\pm 1, \pm i, \pm j, \pm k\}.
  \]
  We can view this as a subgroup of $\GL_2(\C)$ via
  \[
    1 = \begin{pmatrix}
      1&0\\0&1
    \end{pmatrix},\quad
    i = \begin{pmatrix}
      i & 0\\0&-i
    \end{pmatrix},\quad
    j = \begin{pmatrix}
      0&1\\-1&0
    \end{pmatrix},\quad
    k = \begin{pmatrix}
      0&i\\i&0
    \end{pmatrix}
  \]
\end{defi}

\begin{defi}[Conjugacy class]
  The \emph{conjugacy class} of $g \in G$ is
  \[
    \mathcal{C}_G(g)=\{xgx^{-1}: x \in G\}.
  \]
\end{defi}

\begin{defi}[Centralizer]
  The \emph{centralizer} of $g \in G$ is
  \[
    C_G(g) = \{x \in G: xg = gx\}.
  \]
\end{defi}
Then by the orbit-stabilizer theorem, we have $|\mathcal{C}_G(g)| = |G: C_G(G)|$.

\begin{defi}[Group action]
  Let $G$ be a group and $X$ a set. We say $G$ \emph{acts on} $X$ if there is a map $*: G \times X \to X$, written $(g, x) \mapsto g * x = gx$ such that
  \begin{enumerate}
    \item $1x = x$
    \item $g(hx) = (gh)x$
  \end{enumerate}
\end{defi}

The group action can also be characterised in terms of a homomorphism.
\begin{lemma}
  Given an action of $G$ on $X$, we obtain a homomorphism $\theta: G \to \Sym(X)$, where $\Sym(X)$ is the set of all permutations of $X$.
\end{lemma}

\begin{proof}
  For $g \in G$, define $\theta(g) = \theta_g \in \Sym (X)$ as the function $X \to X$ by $x \mapsto gx$. This is indeed a permutation of $X$ because $\theta_{g^{-1}}$ is an inverse.

  Moreover, for any $g_1, g_2 \in G$, we get $\theta_{g_1g_2} = \theta_{g_1} \theta_{g_2}$, since $(g_1g_2) x = g_1(g_2x)$.
\end{proof}

\begin{defi}[Permutation representation]
  The \emph{permutation representation} of a group action $G$ on $X$ is the homomorphism $\theta: G \to \Sym (X)$ obtained above.
\end{defi}

In this course, $X$ is often a finite-dimensional vector space over $F$, and we want the action to satisfy some more properties. We will require the action to be linear, ie. for all $g \in G$, $v_1, v_2 \in V$, and $\lambda \in \F$.
\[
  g(v_1 + v_2) = g v_1 + g v_2,\quad g(\lambda v_1) = \lambda (gv_1).
\]
\section{Basic definitions}
We finally get to something new. We will get some new definitions. As always, $G$ will be a finite group and $\F$ will be a field, usually $\C$.

\begin{defi}[Representation]
  Let $V$ be a finite-dimensional vector space over $\F$. A \emph{(linear) representation} of $G$ on $V$ is a group homomorphism
  \[
    \rho = \rho_V: G \to \GL(V).
  \]
  We sometimes write $\rho_g$ for $\rho_V(g)$, so for each $g \in G$, $\rho_g \in \GL(V)$, and $\rho_g \rho_h = \rho_{gh}$ and $\rho_{g^{-1}} = (\rho_g)^{-1}$ for all $g, h \in G$.
\end{defi}

\begin{defi}[Dimension or degree of representation]
  The \emph{dimension} (or \emph{degree}) of a representation $\rho: G \to \GL(V)$ is $\dim_\F(V)$.
\end{defi}

Recall that $\ker \rho \lhd G$ and $G/\ker \rho \cong \rho(G) \leq \GL(V)$. In the very special case where $\ker \rho$ is trivial, we give it a name:
\begin{defi}[Faithful representation]
  A \emph{faithful} representation is a representation $\rho$ such that $\ker \rho = 1$.
\end{defi}
These are the representations where the identity is the only element that does nothing.

An alternative (and of course equivalent) definition of a representation is to observe that a linear representation is ``the same'' as a linear action of $G$.
\begin{defi}[Linear action]
  A group $G$ \emph{acts linearly} on a vector space $V$ if it acts on $V$ such that
  \[
    g(v_1 + v_2) = g v_1 g v_2,\quad g(\lambda v_1) = \lambda (gv_1)
  \]
  for all $g \in G$, $v_1, v_2 \in V$ and $\lambda \in \F$. We call this a \emph{linear action}.
\end{defi}
Now if $g$ acts linearly on $V$, the map $G \to \GL(V)$ defined by $g \mapsto \rho_g$, with $\rho_g: v \mapsto gv$, is a representation in the previous sense. Conversely, given a representation $\rho: G \to \GL(V)$, we have a linear action of $G$ on $V$ via $gv = \rho(g) v$.

In other words, a representation is just a linear action.

\begin{defi}[$G$-space/$G$-module]
  If there is a linear action $G$ on $V$, we say $V$ is a \emph{$G$-space} or \emph{$G$-module}.
\end{defi}

Alternatively, we can define the group action using the group algebra:
\begin{defi}[Group algebra]
  The \emph{group algebra} $\F G$ is defined to be the algebra (ie. a vector space with a bilinear multiplication operation) of formal sums
  \[
    \F G = \left\{ \sum_{g \in G} \alpha_g g: \alpha_g \in F\right\}
  \]
  with the obvious addition and multiplication.
\end{defi}
Then we can regard $\F G$ as a ring, and a $G$-space is just an $\F G$-module in the sense of IB Groups, Rings and Modules.

\begin{defi}[Matrix representation]
  $R$ is a \emph{matrix representation} of $G$ of degree $n$ if $R$ Is a homomorphism $G \to \GL_n(\F)$.
\end{defi}

Given a linear representation $\rho: G \to \GL(V)$ with $\dim V = n$, we can get a matrix representation by fixing a basis $\mathcal{B}$, and then define the matrix representation $G \to \GL_n(\F)$ by $g \mapsto [\rho(g)]_{\mathcal{B}}$.

Conversely, given a matrix representation $R$, we get a linear representation $\rho$ in the obvious way --- $\rho: G \to \GL(\F^n)$ by $g \mapsto \rho_g$ via $\rho_g(v) = R_g v$.

We have defined representations in four ways --- as a homomorphism to $\GL(V)$, as linear actions, as $\F G$-modules and as matrix representations. Now let's look at some examples.

\begin{eg}[Trivial representation]
  Given any group $G$, take $V = \F$ (the one-dimensional space), and $\rho: G \to \GL(V)$ by $g \mapsto (\id: \F \to \F)$ for all $g$. This is the \emph{trivial representation} of $G$, and has degree $1$.
\end{eg}
Despite being called trivial, trivial representations are highly non-trivial in representation theory. The way they interact with other representations geometrically, topologically etc, and cannot be disregarded. This is a very important representation, despite looking silly.

\begin{eg}
  Let $G = C_4 = \bra x: x^4 = 1\ket$. Let $n = 2$, and work over $\F = \C$. Then we can define a representation by $R: x \mapsto X$, and the action of other elements follows directly by $x^j \mapsto X^j$. Of course, we need $X^4 = I_2$. So we can take
  \begin{enumerate}
    \item $X$ diagonal: any choice of diagonal entries $\{\pm 1, \pm i\}$, in which we have $16$ choices; or
    \item $X$ is not diagonal: then it will be equivalent to a diagonal matrix since $X^4 = I_2$.
  \end{enumerate}
\end{eg}
What we would like to say above that any matrix representation in which $X$ is not diagonal is ``equivalent'' to one in which $X$ is. To make this notion precise, we need to define what it means for representations to be equivalent.

\begin{defi}[$G$-homomorphism/intertwine]
  Fix a group $G$ and a field $\F$. Let $V, V'$ be finite-dimensional vector spaces over $\F$ and $\rho: G \to \GL(V)$ and $\rho': G \to \GL(V')$ be representations of $G$. The linear map $\varphi: V \to V'$ is a \emph{$G$-homomorphism} if
  \[
    \varphi \circ \rho(g) = \rho'(g) \circ \varphi.\tag{$*$}
  \]
  In other words, the following diagram commutes:
  \[
    \begin{tikzcd}
      V \ar[r, "\rho_g"] \ar[d, "\varphi"] & V \ar[d, "\varphi"]\\
      V' \ar[r, "\rho_g'"] & V'
    \end{tikzcd}
  \]
  ie. no matter which way we go form $V$ (top left) to $V'$ (bottom right), we still get the same map.

  We say $\varphi$ \emph{intertwines} $\rho$ and $\rho'$. We write $\Hom_G(V, V')$ for the $\F$-space of all these maps.
\end{defi}

\begin{defi}[$G$-isomorphism]
  A $G$-homomorphism is a \emph{$G$-isomorphism} if $\varphi$ is bijective.
\end{defi}

\begin{defi}[Equivalent/isomorphic representations]
  Two representations $\rho, \rho'$ are \emph{equivalent} or \emph{isomorphic} if there is a $G$-isomorphism between them.
\end{defi}

If $\varphi$ is a $G$-isomorphism, then we can write $(*)$ as
\[
  \rho' = \varphi \rho \varphi^{-1}.\tag{$\dagger$}
\]
\begin{lemma}
  The relation of ``being isomorphic'' is an equivalence relation on the set of all linear representations of $G$ over $\F$.
\end{lemma}
This is an easy exercise left for the reader.

\begin{lemma}
  If $\rho, \rho'$ are isomorphic representations, then they have the same dimension.
\end{lemma}

\begin{proof}
  Trivial since isomorphisms between vector spaces preserve dimension.
\end{proof}

The converse is false.
\begin{eg}
  $C_4$ has four non-isomoprhic dimensions: if $\omega = e^{2 \pi i/4}$, then we have the representations
  \[
    \rho_j (x^i) = \omega^{ij},
  \]
  for $0 \leq i \leq 3$.
\end{eg}
Note that given a group $G$, field $\F$, a vector space $V$ of dimension $n$, and a representation $\rho: G \to \GL(V)$, fix a basis $\mathcal{B}$ of $V$. Then we get a linear $G$-isomorphism $\varphi: V \to \F^n$ by $v \mapsto [v]_{\mathcal{B}}$, ie. by writing $v$ as a column vector with respect to $\mathcal{B}$. Then we get a representation $\rho': G \to \GL(\F^n)$ isomorphic to $\rho$. In other words, every representation is isomorphic to a matrix representation.
\[
  \begin{tikzcd}
    V \ar[r, "\rho"] \ar[d, "\varphi"] & V \ar[d, "\varphi"]\\
    \F^n \ar[r, "\rho'"] & \F^n
  \end{tikzcd}
\]
\end{document}
