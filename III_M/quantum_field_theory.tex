\documentclass[a4paper]{article}

\def\npart {III}
\def\nterm {Michaelmas}
\def\nyear {2016}
\def\nlecturer {B. Allanach}
\def\ncourse {Quantum Field Theory}
\def\nlectures {TTS.12}

% Imports
\ifx \nextra \undefined
  \usepackage[pdftex,
    hidelinks,
    pdfauthor={Dexter Chua},
    pdfsubject={Cambridge Maths Notes: Part \npart\ - \ncourse},
    pdftitle={Part \npart\ - \ncourse},
  pdfkeywords={Cambridge Mathematics Maths Math \npart\ \nterm\ \nyear\ \ncourse}]{hyperref}
  \title{Part \npart\ - \ncourse}
\else
  \usepackage[pdftex,
    hidelinks,
    pdfauthor={Dexter Chua},
    pdfsubject={Cambridge Maths Notes: Part \npart\ - \ncourse\ (\nextra)},
    pdftitle={Part \npart\ - \ncourse\ (\nextra)},
  pdfkeywords={Cambridge Mathematics Maths Math \npart\ \nterm\ \nyear\ \ncourse\ \nextra}]{hyperref}

  \title{Part \npart\ - \ncourse \\ {\Large \nextra}}
\fi

\author{Lectured by \nlecturer \\\small Notes taken by Dexter Chua}
\date{\nterm\ \nyear}

\usepackage{alltt}
\usepackage{amsfonts}
\usepackage{amsmath}
\usepackage{amssymb}
\usepackage{amsthm}
\usepackage{booktabs}
\usepackage{caption}
\usepackage{enumitem}
\usepackage{fancyhdr}
\usepackage{graphicx}
\usepackage{mathtools}
\usepackage{microtype}
\usepackage{multirow}
\usepackage{pdflscape}
\usepackage{pgfplots}
\usepackage{siunitx}
\usepackage{tabularx}
\usepackage{tikz}
\usepackage{tkz-euclide}
\usepackage[normalem]{ulem}
\usepackage[all]{xy}

\pgfplotsset{compat=1.12}

\pagestyle{fancyplain}
\lhead{\emph{\nouppercase{\leftmark}}}
\ifx \nextra \undefined
  \rhead{
    \ifnum\thepage=1
    \else
      \npart\ \ncourse
    \fi}
\else
  \rhead{
    \ifnum\thepage=1
    \else
      \npart\ \ncourse\ (\nextra)
    \fi}
\fi
\usetikzlibrary{arrows}
\usetikzlibrary{decorations.markings}
\usetikzlibrary{decorations.pathmorphing}
\usetikzlibrary{positioning}
\usetikzlibrary{fadings}
\usetikzlibrary{intersections}
\usetikzlibrary{cd}

\newcommand*{\Cdot}{\raisebox{-0.25ex}{\scalebox{1.5}{$\cdot$}}}
\newcommand {\pd}[2][ ]{
  \ifx #1 { }
    \frac{\partial}{\partial #2}
  \else
    \frac{\partial^{#1}}{\partial #2^{#1}}
  \fi
}

% Theorems
\theoremstyle{definition}
\newtheorem*{aim}{Aim}
\newtheorem*{axiom}{Axiom}
\newtheorem*{claim}{Claim}
\newtheorem*{cor}{Corollary}
\newtheorem*{defi}{Definition}
\newtheorem*{eg}{Example}
\newtheorem*{fact}{Fact}
\newtheorem*{law}{Law}
\newtheorem*{lemma}{Lemma}
\newtheorem*{notation}{Notation}
\newtheorem*{prop}{Proposition}
\newtheorem*{thm}{Theorem}

\renewcommand{\labelitemi}{--}
\renewcommand{\labelitemii}{$\circ$}
\renewcommand{\labelenumi}{(\roman{*})}

\let\stdsection\section
\renewcommand\section{\newpage\stdsection}

% Strike through
\def\st{\bgroup \ULdepth=-.55ex \ULset}

% Maths symbols
\newcommand{\bra}{\langle}
\newcommand{\ket}{\rangle}

\newcommand{\N}{\mathbb{N}}
\newcommand{\Z}{\mathbb{Z}}
\newcommand{\Q}{\mathbb{Q}}
\renewcommand{\H}{\mathbb{H}}
\newcommand{\R}{\mathbb{R}}
\newcommand{\C}{\mathbb{C}}
\newcommand{\Prob}{\mathbb{P}}
\renewcommand{\P}{\mathbb{P}}
\newcommand{\E}{\mathbb{E}}
\newcommand{\F}{\mathbb{F}}
\newcommand{\cU}{\mathcal{U}}
\newcommand{\RP}{\mathbb{RP}}
\newcommand{\CP}{\mathbb{CP}}

\newcommand{\ph}{\,\cdot\,}

\DeclareMathOperator{\sech}{sech}
\DeclareMathOperator{\cosech}{cosech}
\DeclareMathOperator{\cosec}{cosec}

\DeclareMathOperator{\covol}{covol}
\DeclareMathOperator{\vol}{vol}

\let\Im\relax
\let\Re\relax
\DeclareMathOperator{\Im}{Im}
\DeclareMathOperator{\Re}{Re}
\DeclareMathOperator{\im}{im}
\DeclareMathOperator{\image}{image}
\DeclareMathOperator{\Ann}{Ann}

\DeclareMathOperator*{\res}{res}
\DeclareMathOperator{\Res}{Res}
\DeclareMathOperator{\Ind}{Ind}

\DeclareMathOperator{\tr}{tr}
\DeclareMathOperator{\diag}{diag}
\DeclareMathOperator{\rank}{rank}
\DeclareMathOperator{\card}{card}
\DeclareMathOperator{\spn}{span}
\DeclareMathOperator{\adj}{adj}

\DeclareMathOperator{\erf}{erf}
\DeclareMathOperator{\erfc}{erfc}

\DeclareMathOperator{\ord}{ord}
\DeclareMathOperator{\Sym}{Sym}

\DeclareMathOperator{\sgn}{sgn}
\DeclareMathOperator{\orb}{orb}
\DeclareMathOperator{\stab}{stab}
\DeclareMathOperator{\ccl}{ccl}

\DeclareMathOperator{\lcm}{lcm}
\DeclareMathOperator{\hcf}{hcf}

\DeclareMathOperator{\Int}{Int}
\DeclareMathOperator{\id}{id}

\DeclareMathOperator{\betaD}{beta}
\DeclareMathOperator{\gammaD}{gamma}
\DeclareMathOperator{\Poisson}{Poisson}
\DeclareMathOperator{\binomial}{binomial}
\DeclareMathOperator{\multinomial}{multinomial}
\DeclareMathOperator{\Bernoulli}{Bernoulli}
\DeclareMathOperator{\like}{like}

\DeclareMathOperator{\var}{var}
\DeclareMathOperator{\cov}{cov}
\DeclareMathOperator{\bias}{bias}
\DeclareMathOperator{\mse}{mse}
\DeclareMathOperator{\corr}{corr}

\DeclareMathOperator{\otp}{otp}
\DeclareMathOperator{\dom}{dom}

\DeclareMathOperator{\Root}{Root}
\DeclareMathOperator{\supp}{supp}
\DeclareMathOperator{\rel}{rel}
\DeclareMathOperator{\Hom}{Hom}
\DeclareMathOperator{\Aut}{Aut}
\DeclareMathOperator{\Gal}{Gal}
\DeclareMathOperator{\Mat}{Mat}
\DeclareMathOperator{\End}{End}
\DeclareMathOperator{\Char}{char}
\DeclareMathOperator{\ev}{ev}
\DeclareMathOperator{\St}{St}
\DeclareMathOperator{\Lk}{Lk}
\DeclareMathOperator{\disc}{disc}
\DeclareMathOperator{\Isom}{Isom}
\DeclareMathOperator{\length}{length}
\DeclareMathOperator{\energy}{energy}
\DeclareMathOperator{\area}{area}
\DeclareMathOperator{\Syl}{Syl}
\DeclareMathOperator{\cl}{cl}
\DeclareMathOperator{\fix}{fix}

\newcommand{\GL}{\mathrm{GL}}
\newcommand{\SL}{\mathrm{SL}}
\newcommand{\PGL}{\mathrm{PGL}}
\newcommand{\PSL}{\mathrm{PSL}}
\newcommand{\PSU}{\mathrm{PSU}}
\newcommand{\Or}{\mathrm{O}}
\newcommand{\SO}{\mathrm{SO}}
\newcommand{\U}{\mathrm{U}}
\newcommand{\SU}{\mathrm{SU}}

\renewcommand{\d}{\mathrm{d}}
\newcommand{\D}{\mathrm{D}}

\tikzset{->/.style = {decoration={markings,
                                  mark=at position 1 with {\arrow[scale=2]{latex'}}},
                      postaction={decorate}}}
\tikzset{<-/.style = {decoration={markings,
                                  mark=at position 0 with {\arrowreversed[scale=2]{latex'}}},
                      postaction={decorate}}}
\tikzset{<->/.style = {decoration={markings,
                                   mark=at position 0 with {\arrowreversed[scale=2]{latex'}},
                                   mark=at position 1 with {\arrow[scale=2]{latex'}}},
                       postaction={decorate}}}
\tikzset{->-/.style = {decoration={markings,
                                   mark=at position #1 with {\arrow[scale=2]{latex'}}},
                       postaction={decorate}}}
\tikzset{-<-/.style = {decoration={markings,
                                   mark=at position #1 with {\arrowreversed[scale=2]{latex'}}},
                       postaction={decorate}}}

\tikzset{circ/.style = {fill, circle, inner sep = 0, minimum size = 3}}
\tikzset{mstate/.style={circle, draw, blue, text=black, minimum width=0.7cm}}

\definecolor{mblue}{rgb}{0.2, 0.3, 0.8}
\definecolor{morange}{rgb}{1, 0.5, 0}
\definecolor{mgreen}{rgb}{0.1, 0.4, 0.2}
\definecolor{mred}{rgb}{0.5, 0, 0}

\def\drawcirculararc(#1,#2)(#3,#4)(#5,#6){%
    \pgfmathsetmacro\cA{(#1*#1+#2*#2-#3*#3-#4*#4)/2}%
    \pgfmathsetmacro\cB{(#1*#1+#2*#2-#5*#5-#6*#6)/2}%
    \pgfmathsetmacro\cy{(\cB*(#1-#3)-\cA*(#1-#5))/%
                        ((#2-#6)*(#1-#3)-(#2-#4)*(#1-#5))}%
    \pgfmathsetmacro\cx{(\cA-\cy*(#2-#4))/(#1-#3)}%
    \pgfmathsetmacro\cr{sqrt((#1-\cx)*(#1-\cx)+(#2-\cy)*(#2-\cy))}%
    \pgfmathsetmacro\cA{atan2(#2-\cy,#1-\cx)}%
    \pgfmathsetmacro\cB{atan2(#6-\cy,#5-\cx)}%
    \pgfmathparse{\cB<\cA}%
    \ifnum\pgfmathresult=1
        \pgfmathsetmacro\cB{\cB+360}%
    \fi
    \draw (#1,#2) arc (\cA:\cB:\cr);%
}
\newcommand\getCoord[3]{\newdimen{#1}\newdimen{#2}\pgfextractx{#1}{\pgfpointanchor{#3}{center}}\pgfextracty{#2}{\pgfpointanchor{#3}{center}}}

\def\Xint#1{\mathchoice
   {\XXint\displaystyle\textstyle{#1}}%
   {\XXint\textstyle\scriptstyle{#1}}%
   {\XXint\scriptstyle\scriptscriptstyle{#1}}%
   {\XXint\scriptscriptstyle\scriptscriptstyle{#1}}%
   \!\int}
\def\XXint#1#2#3{{\setbox0=\hbox{$#1{#2#3}{\int}$}
     \vcenter{\hbox{$#2#3$}}\kern-.5\wd0}}
\def\ddashint{\Xint=}
\def\dashint{\Xint-}


\begin{document}
\maketitle
{\small
\setlength{\parindent}{0em}
\setlength{\parskip}{1em}
\emph{The following description is the official description, but is totally wrong.}

Quantum Field Theory is the language in which modern particle physics is formulated. It represents the marriage of quantum mechanics with special relativity and provides the mathematical framework in which to describe the interactions of elementary particles.

This first Quantum Field Theory course introduces the basic types of fields which play an important role in high energy physics: scalar, spinor (Dirac), and vector (gauge) fields. The relativistic invariance and symmetry properties of these fields are discussed using Lagrangian language and Noether's theorem.

The quantisation of the basic non-interacting free fields is firstly developed using the Hamiltonian and canonical methods in terms of operators which create and annihilate particles and anti-particles. The associated Fock space of quantum physical states is explained together with ideas about how particles propagate in spacetime and their statistics. How these fields interact with a classical electromagnetic field is described.

Next, we introduce the path integral which is an alternative way of describing quantum fields. The path integral is fundamental in introducing interaction into quantum field theory. Interactions are described using perturbative theory and Feynman diagrams. This is first illustrated for theories with a purely scalar field interaction, and then for a couplings between scalar fields and fermions. Finally Quantum Electrodynamics, the theory of interacting photons, electrons and positrons, is introduced and elementary scattering processes are computed.

Finally, the idea of loops in Feynman diagrams are explored and the question of the consequent infinities looked at. Ways of dealing with the infinities will be explored in the Advanced Quantum Field Theory course which follows on directly from this one.

\subsubsection*{Pre-requisites}
You will need to be comfortable with the Lagrangian and Hamiltonian formulations of classical mechanics and with special relativity. You will also need to have taken an advanced course on quantum mechanics.
}
\tableofcontents

\setcounter{section}{-1}
\section{Introduction}
\emph{The greyed out parts are my own attempts to make more sense of the theory from a more mathematical/differential-geometric point of view}

The idea of quantum mechanics is that photons and electrons behave similarly. We can make a photon interfere with itself in double-slit experiments, and similarly an electron can interfere with itself. However, as we know, lights are ripples in an electromagnetic field. So photons should arise from the quantization of the electromagnetic field. If electrons are like photons, should we then have an electron field? The answer is yes!

Quantum field theory is a quantization of a classical field. Recall that in quantum mechanics, we promote degrees of freedom to operators. Basic degrees of freedom of a quantum field theory are operator-valued functions of spacetime. Since there are infinitely many points in spacetime, there is an infinite number of degrees of freedom. This infinity will come back and bite us as we try to develop quantum field theory.

Quantum field theory describes creation and annihilation of particles. The interactions are governed by several basic principles --- locality, symmetry and \emph{renormalization group flow}. What the renormalization group flow describes is the decoupling of low and high energy processes.

\subsubsection*{Why quantum field theory?}
It appears that all particles of the same type are indistinguishable, eg. all electrons are the same. It is difficult to justify why this is the case if each particle is considered individually, but if we view all electrons as excitations of the same field, this is (almost) automatic.

Secondly, if we want to combine special relativity and quantum mechanics, then the number of particles is not conserved. Indeed, consider a particle trapped in a box of size $L$. By the Heisenberg uncertainty principle, we have $\Delta p \gtrsim \bar{h}/L$. We choose a particle with small rest mass so that $m \ll E$. Then we have
\[
  \Delta E = \Delta p \cdot c \gtrsim \frac{\hbar c}{L}.
\]
When $\Delta E \gtrsim 2 mc^2$, then we can pop a particle-antiparticle pair out of the vacuum. So when $L \lesssim \frac{\hbar}{2mc}$, we can't say for sure that there is only one particle.

We say $\lambda = \hbar/(mc)$ is the ``compton wavelength'' --- the minimum distance at which it makes sense to localize a particle. This is also the scale at which quantum kick in.

This is argument is somewhat circular, since we just assumed that if we have enough energy, then particle-antiparticle pairs would just pop out of existence. This is in fact something we can prove in quantum field theory.

To reconcile quantum mechanics and special relativity, we can try to write a relativistic version of Schr\"odinger's equation for a single particle, but something goes wrong. Either the energy is unbounded from below, or we end up with some causality violation. This is bad. These are all fixed by quantum field theory by the introduction of particle creation and annihilation.

\subsubsection*{What is quantum field theory good for?}
Quantum field theory is used in (non-relativistic) condensed matter systems. It describes simple phenomena such as phonons, superconductivity, and the fractional quantum hall effect.

Quantum field theory is also used in high energy physics. The standard model of particle physics consists of electromagnetism (quantum electrodynamics), quantum chromodynamics and the weak forces. The standard model is tested to very high precision by experiments, sometimes up to $1$ part in $10^{10}$. So it is good. While there are many attempts to go beyond the standard model, eg. Grand Unified Theories, they are mostly also quantum field theories.

In cosmology, quantum field theory is used in cosmology to explain the density perturbations. In quantum gravity, string theory is also primarily a quantum field theory in some aspects. It is even used in pure mathematics, with applications in topology and geometry.

\subsubsection*{History of quantum field theory}
In the 1930's, the basics of quantum field theory were laid down by Jordan, Pauli, Heisenberg, Dirac, Weisskopf etc. They encountered all sorts of infinities, which scared them. Back then, these sorts of infinities seemed impossible to work with.

Fortunately, in the 1940's, renormalization and quantum electrodynamics were invented by Tomonaga, Schwinger, Feynman, Dyson, which managed to deal with the infinities. It was a sloppy process, and there was no understanding of why we can subtract infinities and get a sensible finite result. Yet, they managed to make experimental predictions, which were subsequently verified by actual experiments.

In the 1960's, quantum field theory fell out of favour as new particles such as mesons and baryons were discovered. But in the 1970's, it had a golden age when the renormalization group was developed by Kadanoff and Wilson, which was really when the infinities became understood. At the same time, the standard model was invented, and a connection between quantum field theory and geometry was developed.

\subsubsection*{Units and scales}
We are going to do a lot of computations in the course, which are reasonably long. We do not want to have loads of $\hbar$ and $c$'s all over the place when we do the calculations. So we pick convenient units so that they all vanish.

The nature presents us with three fundamental dimensionful constants that are relevant to us:
\begin{enumerate}
  \item The speed of light $c$ with dimensions $LT^{-1}$;
  \item Planck's constant $\hbar$ with dimensions $L^2 MT^{-1}$;
  \item The gravitational constant $G$ with dimensions $L^3 M^{-1} T^{-2}$.
\end{enumerate}
We see that these dimensions are independent. So we define units such that $c = \hbar = 1$. So we can express everything in terms of a mass, or an energy, as we now have $E = m$. For example, instead of $\lambda = \hbar/(mc)$, we just write $\lambda = m^{-1}$. We will work with electron volts $eV$. To convert back to the conventional SI units, we must insert the relevant powers of $c$ and $\hbar$. For example, for a mass of $m_e = \SI{e6}\electronvolt$, we have $\lambda_e = \SI{2e-12}{\meter}$.

After getting rid of all factors of $\hbar$ and $c$, if a quantity $X$ has a mass dimension $d$, we write $[x] = d$. For example, we have $[G] = 2$, since we have
\[
  G = \frac{\hbar c}{M_p^2} = \frac{1}{M_p^2},
\]
where $M_p \sim \SI{e19}{\giga\electronvolt}$ is the \emph{Planck scale}.

\section{Classical field theory}
\begin{own}
  We suppose the universe is given by a spacetime manifold $\mathcal{M}$ with a pseudo-Riemannian metric of signature $(+1, -1, -1 ,-1)$. All maps are assumed to be smooth.
\end{own}

\begin{defi}[Field]\index{field}
  A \emph{field} is a physical quantity defined at every point of spacetime $(\mathbf{x}, t)$.
\end{defi}

\begin{own}
  Mathematically, this a field is a function $\mathcal{M} \to V$, where $V$ is some vector space. This is, of course, equivalent to a section of the trivial bundle $V \times \mathcal{M} \to \mathcal{M}$. We will consider more general bundles later when we do gauge theory.
\end{own}

In classical point mechanics, we have a finite number of generalized coordinates $q_a(t)$. In field theory, we are interested in the dynamics of $\phi_a(\mathbf{x}, t)$, where $a$ and $\mathbf{x}$ are \emph{both} labels. Since $\mathbf{x}$ is a continuous variable, we now have an infinite number of degrees of freedom! Note that here position has been relegated from a dynamical variable (ie. one of the $q_a$) to a mere label.

\begin{eg}
  The \term{electric field} $E_i(\mathbf{x}, t)$ and \term{magnetic field} $B_i(\mathbf{x}, t)$, for $i = 1, 2, 3$, are examples of fields. These six fields can in fact be derived from 4 fields $A_\mu(\mathbf{x}, t)$, for $\mu = 0, 1, 2, 3$, where
  \[
    E_i = \frac{\partial A_i}{\partial t} - \frac{\partial A_0}{\partial x_i},\quad B_i = \frac{1}{2} \varepsilon_{ijk}\frac{\partial A_k}{\partial x^j}.
  \]
  Often, we write
  \[
    A^\mu = (\phi, \mathbf{A}).
  \]
\end{eg}

\begin{defi}[Lagrangian density]\index{Lagrangian density}
  Given a field $\phi (\mathbf{x}, t)$, a \emph{Lagrangian density} is a function $\mathcal{L}(\phi, \partial_\mu \phi)$ of $\phi$ and its derivative.
\end{defi}

\begin{own}
  Mathematically, a (local) Lagrangian density is given by a function from the $n$th jet bundle $\mathcal{L}: j^n(\mathcal{M}, V) \to \R$ ($n$ can be taken to be infinite if one is not taunted by the horrors of infinite-dimensional spaces). Given a field $\phi: \mathcal{M} \to V$, we write $\mathcal{L}(\phi)$ for the composition $\mathcal{L} \circ j^n \phi$, where $j^n$ is the $n$th jet prolongation of $\phi$.

  It happens that nature seems to prefer Lagrangians with $n = 1$, and this is the only case we will consider.
\end{own}

\begin{defi}[Lagrangian]\index{Lagrangian}
  Given a Lagrangian density, the \emph{Lagrangian} is defined by
  \[
    L = \int \d^3 x\; \mathcal{L}(\phi, \partial_\mu \phi).
  \]
  \begin{own}
    Mathematically, given a preferred time axis $T$, we obtain a projection $\mathcal{M} \to T$. Given a field $\phi: \mathcal{M} \to V$, integrating $\mathcal{L}(\phi)$ along fibers gives a function $L: T \to V$.
  \end{own}
\end{defi}
For most of the course, we will only care about the Lagrangian density, and call it the Lagrangian.

\begin{defi}[Action]\index{action}
  Given a Lagrangian and a time interval $[t_1, t_2]$, the \emph{action} is defined by
  \[
    S = \int_{t_1}^{t_2} \;\d t\; L(t) = \int \d^4 x\; \mathcal{L}.
  \]
\end{defi}
In particle mechanics, $\mathcal{L}$ depends on $\dot{q}$ and not $\ddot{q}$. In field theory, $\mathcal{L}$ should depend on $\dot{\phi}$ but could depend on $\nabla \ph, \nabla^2\phi$ etc. However, with an eye to Lorentz invariance, we should not treat space and time differently. So we'll only consider $\mathcal{L}$ depending on $\nabla \phi$.

In general, we would want $[S] = 0$. Since we have $[\d^4 x] = -4$, we must have $[\mathcal{L}] = 4$.

The equations of motion is given by the principle of least action.
\begin{defi}[Principle of least action]\index{principle of least action}
  The equation of motion of a Lagrangian system is given by the \emph{principle of least action} --- we vary the path of integration, keeping the end points fixed, and require the first-order change $\delta S = 0$.

  \begin{own}
    More formally, We say $\phi:\mathcal{M} \to V$ is a solution if for any compact region $C \subseteq \mathcal{M}$ and any $F: C \times (-\varepsilon, \varepsilon) \to V$ such that $F(\mathbf{x}, 0) = \phi(\mathbf{x})$ for all $\mathbf{x} \in C$, and $F(\mathbf{x}, s) = \phi(\mathbf{x})$ for all $\mathbf{x} \in \partial C$, we have
    \[
      0 = \left.\frac{\d}{\d s}\right|_{s = 0} \int_C \d^4 x\; \mathcal{L}(F(\mathbf{x}, s)).
    \]
  \end{own}
\end{defi}

We can compute
\begin{align*}
  \delta S &= \int d^4 x\left(\frac{\partial \mathcal{L}}{\partial \phi_a} \delta \phi_a + \frac{\partial \mathcal{L}}{\partial (\partial_\mu \phi_a)} \delta(\partial_\mu\phi_a)\right)\\
  &= \int \d^4 x \left\{\left(\frac{\partial \mathcal{L}}{\partial \phi_a} - \partial_\mu\left(\frac{\partial \mathcal{L}}{\partial (\partial_\mu\phi_a)}\right)\right) \delta \phi_a + \partial_\mu \left(\frac{\partial \mathcal{L}}{\partial (\partial_\mu\phi_a)} \delta\phi_a\right)\right\}.
\end{align*}
We see that the last term vanishes for any term that decays at spatial infinity, and obeys $\delta \phi_a(\mathbf{x}, t_1) = \delta\phi_a(\mathbf{x}, t_2) = 0$.

Requiring $\delta S = 0$ means that we need
\begin{prop}[Euler-Lagrange equation]
  The equations of motion for a field are given by the \term{Euler-Lagrange equations}:
  \[
    \partial_\mu\left(\frac{\partial \mathcal{L}}{\partial (\partial_\mu \phi_a)}\right) - \frac{\partial \mathcal{L}}{\partial \phi_a} = 0.
  \]
\end{prop}

\begin{eg}
  Consider the \term{Klein-Gordon equation} for a real scalar field $\phi(\mathbf{x}, t)$ is given by the Lagrangian
  \[
    \mathcal{L} = \frac{1}{2} \partial_\mu \phi \partial^\mu \phi - \frac{1}{2}m^2 \phi^2 = \frac{1}{2} \dot{\phi}^2 - \frac{1}{2} (\nabla \phi)^2 - \frac{1}{2} m^2 \phi^2,
  \]
  where $\eta^{\mu\nu}$ is the \term{Minkowski metric} given by
  \[
    \eta^{\mu\nu} =
    \begin{pmatrix}
      1 & 0 & 0 & 0\\
      0 & -1 & 0 & 0\\
      0 & 0 & -1 & 0\\
      0 & 0 & 0 & -1
    \end{pmatrix}.
  \]
  We can view this as $\mathcal{L} = T - V$, where
  \[
    T = \frac{1}{2} \dot{\phi}^2
  \]
  is the kinetic energy, and
  \[
    V = \frac{1}{2} (\nabla \phi)^2 + \frac{1}{2}m^2 \phi^2
  \]
  is the potential energy.

  To find the Euler-Lagrange equations, we can compute
  \[
    \frac{\partial \mathcal{L}}{\partial(\partial_\mu \phi)} = \partial^\mu \phi = (\dot{\phi}, -\nabla \phi)
  \]
  and
  \[
    \frac{\partial \mathcal{L}}{\partial \phi} = -m^2 \phi.
  \]
  So the Euler-Lagrange equation says
  \[
    \partial_\mu \partial^\mu \phi + m^2 \phi = 0.
  \]
  More explicitly, this says
  \[
    \ddot{\phi} - \nabla^2 \phi + m^2 \phi = 0.
  \]
\end{eg}
We could generalize this and add more terms to the Lagrangian. An obvious generalization would be
\[
  \mathcal{L} = \frac{1}{2} \partial_\mu \phi \partial^\mu \phi - V(\phi),
\]
where $V(\phi)$ is an arbitrary potential function. Then we similarly obtain the equation
\[
  \partial_\mu \partial^\mu \phi + \frac{V}{\partial \phi} = 0.
\]

\begin{eg}
  \term{Maxwell's equations} in vacuum is given by the Lagrangian
  \[
    \mathcal{L} = -\frac{1}{2} (\partial_\mu A_\nu)(\partial^\mu A^\nu) + \frac{1}{2}(\partial_\mu A^\mu)^2.
  \]
  To find out the Euler-Lagrange equation, we need to figure out the derivatives with respect to each component of the $A$ field. We obtain
  \[
    \frac{\partial \mathcal{L}}{\partial(\partial_\mu A_\nu)} = -\partial^\mu A^\nu + (\partial_\rho A^\rho) \eta^{\mu\nu}.
  \]
  So we obtain
  \[
    \partial_\mu \left(\frac{\partial \mathcal{L}}{\partial(\partial_\mu A_\nu)}\right) = - \partial_\mu\partial^\mu + \partial^\nu (\partial_\rho A^\rho) = - \partial_\mu (\partial^\mu A^\nu - \partial^\nu A^\mu).
  \]
  We write
  \[
    F^{\mu\nu} = \partial^\mu A^\nu - \partial^\nu A^\mu.
  \]
  So we are left with
  \[
    \partial_\mu \left(\frac{\partial \mathcal{L}}{\partial(\partial_\mu A_\nu)}\right) -\partial_\mu F^{\mu\nu}.
  \]
  It is an exercise to check that these Euler-Lagrange equations reproduce
  \[
    \partial_i E_i = 0,\quad \dot{E}_i = \varepsilon_{ijk} \partial_j B_k.
  \]
  Using our $F^{\mu\nu}$, we can rewrite the Lagrangian as
  \[
    \mathcal{L} = -\frac{1}{4} F_{\mu\nu}F^{\mu\nu}.
  \]
\end{eg}
How did we come up with these Lagrangians? In general, it is guided by two principles --- one is symmetry, and the other is renormalizability. We will discuss symmetries shortly, and renormalizaility would be done in the III Advanced Quantum Field Theory course.

In all of these examples, the Lagrangian is local. In other words, the terms don't couple $\phi(\mathbf{x}, t)$ to $\phi(\mathbf{y}, t)$ if $\mathbf{x} \not= \mathbf{y}$.
\begin{eg}
  An example of a non-local Lagrangian would be
  \[
    \int \d^3 x \int \d^3 y \phi(\mathbf{x}) \phi(\mathbf{x} - \mathbf{y})
  \]
\end{eg}
A priori, there is no reason for this --- $x$ is merely a label, and we do couple other indices together, eg. in $\partial_3 A_0$, we mix the labels $3$ and $0$. However, it happens that nature seems to be local. So we shall only consider local Lagrangian.

Note that locality does \emph{not} mean that the Lagrangian at a point only depends on the value at the point. Indeed, it also depends on the derivatives at $\mathbf{x}$. So we can view this as saying the value of $\mathcal{L}$ at $\mathbf{x}$ only depends on the value of $\phi$ at an infinitesimal neighbourhood of $\mathbf{x}$ (formally, the jet at $\mathbf{x}$).

\subsection{Lorentz invariance}
if we wish to construct relativistic field theories such that $\mathbf{x}$ and $t$ are on an equal footing, the Lagrangian should be invariant under \term{Lorentz transformations} $x^\mu \mapsto x'^\mu = \Lambda^\mu\!_\nu x^\nu$, where
\[
  \Lambda^\mu\!_\sigma \eta^{\sigma\tau} \Lambda^\nu\!_\tau = \eta^{\mu\nu}.
\]
\begin{eg}
  The transformation
  \[
    \Lambda^\mu\!_\sigma =
    \begin{pmatrix}
      1 & 0 & 0 & 0\\
      0 & 1 & 0 & 0\\
      0 & 0 & \cos \theta & -\sin \theta\\
      0 & 0 & \sin \theta & \cos \theta
    \end{pmatrix}
  \]
  describes a rotation by an angle around the $x$-axis.
\end{eg}

\begin{eg}
  The transformation
  \[
    \Lambda^\mu\!_\sigma =
    \begin{pmatrix}
      \gamma & -\gamma v & 0 & 0\\
      -\gamma v & \gamma & 0 & 0\\
      0 & 0 & 1 & 0\\
      0 & 0 & 0 & 1
    \end{pmatrix}
  \]
  describes a \term{boost} by $v$ along the $x$-axis.
\end{eg}
The Lorentz transformations form a Lie group under matrix multiplication --- see III Symmetries Field and Particles.

The Lorentz transformations have a representation on the fields. For a scalar field, this is given by
\[
  \phi(x) \mapsto \phi'(x) = \phi(\Lambda^{-1}x),
\]
where the indices are suppressed. This is an active transformation --- say $\mathbf{x}_0$ is the point at which, say, the field is a maximum. Then after applying the Lorentz transformation, the position of the new maximum is $\Lambda x_0$. We could alternatively have used passive transformations, where we just relabel the points. In this case, we have
\[
  \phi(x) \mapsto \phi(\Lambda (x)).
\]
However, this doesn't really matter, since if $\Lambda$ is a Lorentz transformation, then so is $\Lambda^{-1}$.

A Lorentz invariant theory should have equations of motions such that if $\phi(x)$ is a solution, then so is $\phi(\Lambda^{-1} x)$. This can be achieved by requiring that the action $S$ is invariant under Lorentz transformations.

\begin{eg}
  In the Klein-Gordon field, we have
  \[
    \mathcal{L} = \frac{1}{2} \partial_\mu \phi \partial^\mu \phi - \frac{1}{2}m^2 \phi^2.
  \]
  The Lorentz transformation is given by
  \[
    \phi(x) \mapsto \phi'(x) = \phi(\Lambda^{-1}x) = \phi(y),
  \]
  where
  \[
    y^\mu = (\Lambda^{-1})^\mu\!_\nu x^\nu.
  \]
  We then check that
  \begin{align*}
    \partial_\mu \phi(x) &\mapsto \frac{\partial}{\partial x^\mu}(\phi(\Lambda^{-1}x)) \\
    &= \frac{\partial}{\partial x^\mu} (\phi(y))\\
    &= \frac{\partial y^\nu}{\partial x^\mu} \frac{\partial}{\partial y^\nu} (\phi(y))\\
    &= (\Lambda^{-1})^\nu\!_\mu(\partial_\nu \phi)(y).
  \end{align*}
  Since $\Lambda^{-1}$ is a Lorentz transformation, we have
  \[
    \partial_\mu \phi \partial^\mu \phi = \partial_\mu \phi' \partial^\mu \phi'.
  \]
\end{eg}
In general, as long as we write everything in terms of tensors, we get a Lorentz invariant theories.

Symmetries play an important role in QFT. Different kinds of symmetries include Lorentz symmetries, gauge symmetries, global symmetries and supersymmetries (SUSY).

\begin{eg}
  Protons and neutrons are made up of quarks. Each type of quark comes in three flavors, which are called red, blue and green (these are arbitrary names). If we swap around red and blue everywhere in the universe, then the laws don't change. This is known as a global symmetry.

  But if we make this change differently at different points, the equations don't a priori remain invariant, unless we introduce a \emph{gauge boson}. More of this will be explored in the AQFT course.
\end{eg}

\subsection{Noether's theorem for field theories}
\begin{own}
  \begin{defi}[Continuous symmetry]\index{continuous symmetry}
    A \emph{continuous symmetry} of the Lagrangian is an action of a Lie group $G$ on $j^n(\mathcal{M}, V)$ such that for every $g \in G$ and every compact set $C \subseteq \mathcal{M}$, we have
    \[
      \int_C \mathcal{L}(j^1\phi(\mathbf{x})) \;\d g = \int_C \mathcal{L}(g \cdot (j^1\phi(\mathbf{x})))\;\d g.
    \]
    In particular, if $L(g \cdot a) = L(a)$ for all $a \in j^n(\mathcal{M}, V)$, then this is a symmetry.

    A \emph{one-parameter symmetry} is one where the Lie algebra of $G$ is $\R$.
  \end{defi}
\end{own}
\begin{thm}[Noether's theorem]\index{Noether's theorem}
  Every continuous symmetry of $\mathcal{L}$ gives rise to a \term{conserved current} $j^\mu(x)$ such that the equation of motion implies that
  \[
    \partial_\mu j^\mu = 0.
  \]
  More explicitly, this gives
  \[
    \partial_0 j^0 + \nabla \cdot \mathbf{j} = 0.
  \]
  A conserved current gives rise to a \emph{conserved charge}
  \[
    Q = \int_{\R^3} j^0 \d^3 \mathbf{x},
  \]
  since
  \begin{align*}
    \frac{\d Q}{\d t} &= \int_{\R^3} \frac{\d j^0}{\d t} \;\d^3 \mathbf{x}\\
    &= -\int_{\R^3} \nabla \cdot \mathbf{j} \;\d ^3 \mathbf{x}\\
    &= 0,
  \end{align*}
  assuming that $j^i \to 0$ as $\mathbf{x} \to \infty$.
\end{thm}

\begin{proof}
  Consider making an arbitrary transformation of the field $\phi_a \mapsto \phi_a + \delta \phi_a$. We then have
  \begin{align*}
    \delta \mathcal{L} &= \frac{\partial \mathcal{L}}{\partial \phi_a} \delta \phi_a + \frac{\partial \mathcal{L}}{\partial(\partial_\mu \phi_a)} \delta(\partial_\mu \phi_a)\\
    &= \left[\frac{\partial \mathcal{L}}{\partial \phi_a} - \partial_\mu \frac{\partial \mathcal{L}}{\partial(\partial_\mu \phi_a)}\right] \delta \phi_a + \partial_\mu \left(\frac{\partial \mathcal{L}}{\partial(\partial_\mu \phi_a)} \delta \phi_a\right).
  \end{align*}
  When the equations of motion are satisfied, we know the first term always vanishes. So we are left with
  \[
    \delta \mathcal{L} = \partial_\mu \left(\frac{\partial \mathcal{L}}{\partial(\partial_\mu \phi_a)} \delta \phi_a\right).
  \]
  If the specific transformation $\delta \phi_a = X_a$ we are considering is a \term{symmetry}, then $\delta\mathcal{L} = 0$ (this is the definition of a symmetry). In this case, we can define a conserved current by
  \[
    j^\mu = \frac{\partial \mathcal{L}}{\partial (\partial_\mu \phi_a)}X_a,
  \]
  and by the equations above, this is actually conserved.
\end{proof}
We can have a slight generalization where we relax the condition for a symmetry and still get a conserved current. We say that $X_a$ is a symmetry if $\delta \mathcal{L} = \partial_\mu F^\mu(\phi)$ for some $F^\mu(\phi)$, ie. a total derivative. Replaying the calculations, we get
\[
  j^\mu = \frac{\partial \mathcal{L}}{\partial (\partial_\mu \phi_a)} X_a - F^\mu.
\]
\begin{eg}[Space-time]
  Recall that in classical dynamics, spatial invariance implies the conservation of momentum, and invariance wrt to time translation implies the conservation of energy. We'll see something similar in field theory. Consider $x^\mu \mapsto x^\mu + \varepsilon^\mu$. Then we obtain
  \[
    \phi_a(x) \mapsto \phi_a(x) + \varepsilon^\nu \partial_\nu \phi_a(x).
  \]
  A Lagrangian that has no explicit $x^\mu$ dependence transforms as
  \[
    \mathcal{L}(x) \mapsto \mathcal{L}(x) + \varepsilon^\nu \partial_\nu \mathcal{L}(x),
  \]
  giving rise to 4 currents --- one for each $\nu = 0, 1, 2, 3$. We have
  \[
    (j^\mu)_\nu = \frac{\partial \mathcal{L}}{\partial (\partial_\mu \phi_a)} \partial_\nu \phi_a - \delta^\mu\!_\nu \mathcal{L},
  \]
  This particular current is called $T^\mu\!_\nu$, the \term{energy-momentum tensor}\index{$T^\mu_\nu$}. This satisfies
  \[
    \partial_\mu T^\mu\!_\nu = 0,
  \]
  We obtain conserved quantities, namely the \term{energy}
  \[
    E = \int \d^3 x\;T^{00},
  \]
  and the total \term{momentum}
  \[
    \mathbf{P}^i = \int \d^3 x\; T^{0i}.
  \]
\end{eg}

\begin{eg}
  Consider the Klein-Gordon field, with
  \[
    \mathcal{L} = \frac{1}{2} \partial_\mu \phi \partial^\mu \phi - \frac{1}{2}m^2 \phi^2.
  \]
  We then obtain
  \[
    T^{\mu\nu} = \partial^\mu \partial^\nu \phi - \eta^{\mu\nu} \mathcal{L}.
  \]
  So we have
  \[
    E = \int \d^3 \mathbf{x} \left(\frac{1}{2}\dot{\phi}^2 + \frac{1}{2}(\nabla \phi)^2 + \frac{1}{2}m^2 \phi^2\right).
  \]
  The momentum is given by
  \[
    \mathbf{P}^i = \int\d^3 x\; \dot{\phi} \partial^i \phi.
  \]
\end{eg}
In this example, $T^{\mu\nu}$ comes out symmetric in $\mu$ and $\nu$. In general, it would not be, but we can always massage it into a symmetric form by adding
\[
  \sigma^{\mu\nu} = T^{\mu\nu} + \partial_\rho \Gamma^{\rho\mu\nu}
\]
with $\Gamma^{\rho\mu\nu}$ a tensor antisymmetric in $\rho$ and $\mu$. Then we have
\[
  \partial_\mu \partial_\rho T^{\rho\mu\nu} = 0.
\]
So this $\sigma^{\mu\nu}$ is also invariant.

A symmetric energy-momentum tensor of this form on the RHS of Einstein's field equation.

\begin{eg}[Internal symmetries]
  Consider a complex scalar field
  \[
    \psi(x) = \frac{1}{\sqrt{2}} (\phi_1 + i \phi_2).
  \]
  We put
  \[
    \mathcal{L} = \partial_\mu \psi^* \partial^\mu \psi - V(\psi^*\psi),
  \]
  where
  \[
    V (\psi^*\psi) + m^2 \psi^* \psi + \frac{\lambda}{2}(\psi^*\psi)^2 + \cdots
  \]
  is some potential term.

  To find the equations of motions, if we do all the complex analysis required, we will figure that we will obtain the same equations as the real case if we treat $\psi$ and $\psi^*$ as independent variables. In this case, we obtain
  \[
    \partial_\mu \partial^\mu \psi + m^2 \psi + \lambda (\psi^*\psi) \psi + \cdots = 0.
  \]
  The $\mathcal{L}$ has a symmetry given by $\psi \mapsto e^{i\alpha} \psi$. Infinitesimally, we have $\delta \psi = i\alpha \psi$, and $\delta \psi^* = -i\alpha \psi^*$.

  This gives a current
  \[
    j^\mu = i(\partial^\mu \psi^*) \psi - i (\partial^\mu \psi)\psi^*.
  \]
  We will later see that associated charges of this type has an interpretation of electric charge (or particle number, eg. baryon number or lepton number).
\end{eg}

Note that this symmetry is an abelian symmetry, since it is a symmetry under the action of the abelian group $\U(1)$. There is a generalization to a non-abelian case.

\begin{own}
  This example doesn't make much sense to me. Need to figure out some time later.
\end{own}
\begin{eg}[Non-abelian internal symmetries]
  Suppose we have a theory with many fields, with the Lagrangian given by
  \[
    \mathcal{L} = \sum_{a = 1}^N \partial_\mu \phi_a \partial^\mu \phi_a - \frac{1}{2} \sum_{a = 1}^N \phi_a^2 - g \left(\sum_{a = 1}^N \phi_a^2\right)^2.
  \]
  This theory is invariannt under the bigger symmetry group $G = \SO(N)$, or $\U(N/2)$ or even $\SU(N/2)$, if the field is suitably complexified. For example, the symmetry group $\SU(3)$ gives the \term{8-fold way}.

  There exists a trick to determine the current in this case. We know that $\delta \mathcal{L} = 0$. We re-do the transformation with a perturbation $\alpha = \alpha(x)$. We'll find that the change in Lagrangian is of the form
  \[
    \delta \mathcal{L} = (\partial_\mu \alpha(x)) h^\mu(\phi).
  \]
  Since we know that when $\alpha$ is a constant, we ahve $\delta \mathcal{L} = 0$, we must have
  \[
    \delta S = \int \delta \mathcal{L} = - \int \alpha(x) \partial_\mu h^\mu,
  \]
  which means that the equations of motions are satisfied. So $\delta S = 0$ for all variations of the field, including when we choose $\delta \alpha = \alpha(x) \phi)$. So we have $\partial_\mu h^\mu = 0$. So we can identify $h^\mu$ with the conserved current.
\end{eg}

We can now talk about the Hamiltonian formulation. This can be done for field theories as well. We define
\begin{defi}[Conjugate momentum]\index{conjugate momentum}
  Given a Lagrangian system for a field $\phi$, we define the \emph{conjugate momentum} by
  \[
    \pi(x) = \frac{\partial \mathcal{L}}{\partial \dot{\phi}}.
  \]
\end{defi}
This is not to be confused with the total momentum $\mathbf{P}_i$.

\begin{defi}[Hamiltonian density]\index{Hamiltonian density}
  The \emph{Hamiltonian density} is given by
  \[
    \mathcal{H} = \pi(x) \dot{\phi}(x) - \mathcal{L}(x),
  \]
  where we replace all occurences of $\dot{\phi}$ with $\pi(x)$.
\end{defi}

\begin{eg}
  Suppose we have a field Lagrangian of the form
  \[
    \mathcal{L} = \frac{1}{2} \dot{\phi}^2 - \frac{1}{2} (\nabla \phi)^2 - V(\phi)..
  \]
  Then we can compute that
  \[
    \pi = \dot\phi.
  \]
  So we can easily find
  \[
    \mathcal{H} = \frac{1}{2}\pi^2 + \frac{1}{2}(\nabla \phi)^2 + V(\phi).
  \]
\end{eg}

\begin{defi}[Hamiltonian]\index{Hamiltonian}
  The \emph{Hamiltonian} of a Hamiltonian system is
  \[
    H = \int\;\d^3 x\; \mathcal{H}.
  \]
\end{defi}
This agrees with the field energy we computed using Noether's theorem.

\begin{defi}[Hamilton's equation]\index{Hamilton's equation}
  Hamilton's equations are
  \[
    \dot{\phi} = \frac{\partial \mathcal{H}}{\partial \pi},\quad \dot{\pi} = -\frac{\partial \mathcal{H}}{\partial \phi}.
  \]
  These give us the equations of motions of $\phi$.
\end{defi}
There is an obvious problem with this, that the Hamiltonian formulation is not manifestly Lorentz invariant. However, we know it is because we derived it as an equivalent formulation of a Lorentz invariant theory.

So far, we've just been talking about classical field theory. We now want to quantize this, and actually do \emph{quantum} field theory.

\section{Canonical quantization}
Recall that in quantum mechanics, canonical quantization tells us to take generalized coordinates $q_a$ and momenta $p_a$, and then promote them to operators. We then replace the Poisson brackets with commutators
\[
  [q_a, p^b] = i \delta_a^b.
\]
We are just going to do the same for $\phi_a(\mathbf{x})$ and $\pi_b(\mathbf{x})$.

\begin{defi}[Quantum field]\index{quantum field}
  A \emph{quantum field} is an operator-valued functions of space $\phi_a, \pi_b$ with $a, b \in I$ such that
  \[
    [\phi_a(\mathbf{x}), \phi_b(\mathbf{y})] = 0 = [\pi^a(\mathbf{x}), \pi^b(\mathbf{y})]
  \]
  and
  \[
    [\phi_a(\mathbf{x}), \pi^b(\mathbf{y})] = i \delta^3(\mathbf{x} - \mathbf{y}) \delta_a^b.
  \]
\end{defi}
Again, this is not manifestly Lorentz invariant. Here we are in the Schrodinger picture, where states evolve with time, and the operators do not depend on time.

The evolution of states is again given by Schr\"odinger equations.
\begin{defi}[Schr\"odinger equations]\index{Schr\"odinger equations}
  The \emph{Schr\"odinger equation} says
  \[
    i \frac{\d}{\d t}\bket{\psi} = H \bket{\psi}.
  \]
\end{defi}
We want to know the spectrum of $H$. Typically, this is very hard --- we have an infinite number of degrees of freedom, one for each point $\mathbf{x}$ in space.

For certain theories (known as \term{free theories}), this is easier. We can find new degrees of freedom in which all states evolve independently.

Free field theories have Lagrangians which are quadratic in the fields. So the equations of motions are linear.

\begin{eg}
  The simplest free theory is the classic Klein-Gordon theory for a real scalar field $\phi(x\mathbf{x}, t)$. The equations of motion are
  \[
    \partial_\mu \partial^\mu \phi + m^2 \phi = 0.
  \]
  To see why this is free, we take the Fourier transform so that
  \[
    \phi(\mathbf{x}, t) = \int \frac{\d^3 p}{(2\pi)^3} e^{i \mathbf{p}\cdot \mathbf{x}} \tilde{\phi}(\mathbf{p}, t).
  \]
  We substitute this into the Klein-Gordon equation to obtain
  \[
    \left(\frac{\partial^2}{\partial t^2} + (|\mathbf{p}|^2 + m^2)\right)\tilde{\phi}(\mathbf{p}, t) = 0.
  \]
  This is just the usual equation for a simple harmonic oscillator for each $\mathbf{p}$, independently, with frequency $\omega_p = \sqrt{|\mathbf{p}|^2 + m^2}$. So the solutions to the classical Klein-Gordon equation is a superposition of simple harmonic oscillators, each vibrating at a different frequency (and a different amplitude).

  So to quantize the Klein-Gordon field, we just have to quantize this infinite number of harmonic oscillators.
\end{eg}

\subsection{Review of simple harmonic oscillator}
We work in the Schr\"odinger picture with
\[
  H = \frac{1}{2}p^2 + \frac{1}{2} \omega^2 q^2,
\]
where we have our commutation relation
\[
  [q, p] = i.
\]
To find the spectrum, we define the \term{creation} and \term{annihilation} operators (``raising'' and ``lowering'') given by
\[
  a = \frac{i}{\sqrt{2 \omega}}p + \sqrt{\frac{\omega}{2}}q,\quad a^\dagger = \frac{-i}{\sqrt{2\omega}}p + \sqrt{\frac{\omega}{2}}q.
\]
We can invert these to find
\[
  p = \frac{1}{\sqrt{2\omega}}(a + a^\dagger),\quad p = -i\sqrt{\frac{\omega}{2}} (a - a^\dagger).
\]
We can substitute these equations into the commutator relations to obtain
\[
  [a, a^\dagger] = 1.
\]
Putting them into the Hamiltonian, we obtain
\[
  H = \frac{1}{2}\omega(a a^\dagger + a^\dagger a) = \omega \left(a^\dagger a + \frac{1}{2}\right).
\]
We can now compute
\[
  [H, a^\dagger] = \omega a^\dagger,\quad [H, a] = -\omega a.
\]
These ensure that $a, a^\dagger$ take us between energy eigenstates --- if
\[
  H\bket{E} = E \bket{E},
\]
then
\[
  Ha^\dagger\bket{E} = (a^\dagger H + [H, a]) \bket{E} = (E + \omega)a^\dagger \bket{E}.
\]
Similarly, we have
\[
  Ha\bket{E} = (E - \omega)a\bket{E}.
\]
So we get a ladder of energy eigenstates with eigenvalues
\[
  \cdots, E - 2\omega, E - \omega, E, E + \omega, E + 2\omega, \cdots.
\]
If the energy is bounded below, then there must be a ground state $\bket{0}$ satisfying $a \bket{0} = 0$. Excited states arise from repeated applications of $a^\dagger$, labelled by
\[
  \bket{n} = (a^\dagger)^n \bket{0},
\]
with
\[
  H\bket{n} = \left(n + \frac{1}{2}\right)\omega \bket{n}.
\]
Note that we were lazy and ignored normalization, so we have $\braket{n}{n} \not= 1$.

This algebraic approach tells us the spectrum, but not the explicit form of the wavefunction. In the Schr\"odinger representation, we have
\[
  \hat{p} = -\frac{\partial}{\partial q}.
\]
Since $a \bket{0} = 0$, we have
\[
  \left(\frac{1}{\sqrt{2\omega}}\frac{\partial}{\partial q} + \sqrt{\frac{\omega}{2}} q\right)\psi_0(q) = 0.
\]
This simplifies to
\[
  \left(\frac{\partial}{\partial q} + \omega q\right) \psi_0(q) = 0.
\]
So we have
\[
  \psi_0 \propto e^{-\omega q^2/2}.
\]
\subsection{Free field theory}
We are going to apply this to free fields. By some Fourier transform or other magic trickery, we suppose we can find operators $a_\mathbf{p}$ such that we have
\begin{align*}
  \phi(\mathbf{x}) &= \int \frac{\d^3 \mathbf{p}}{(2\pi)^3} \frac{1}{\sqrt{2 \omega_p}} \left(a_\mathbf{p} e^{i \mathbf{p} \cdot \mathbf{x}} + a_\mathbf{p}^\dagger e^{-i \mathbf{p}\cdot \mathbf{x}}\right)\\
  \pi(\mathbf{x}) &= \int \frac{\d^3 \mathbf{p}}{(2\pi)^3} (-i) \sqrt{\frac{\omega_\mathbf{p}}{2}}\left(a_\mathbf{p} e^{i\mathbf{p}\cdot \mathbf{x}} - a_\mathbf{p}^\dagger e^{-i \mathbf{p}\cdot \mathbf{x}}\right).
\end{align*}
Note that the way we have written this shows that $\phi$ is real, since we are adding $a_\mathbf{p}e^{i\mathbf{p}\cdot \mathbf{x}}$ to its conjugate (and similarly for $\pi$).

On the other hand, since we are integrating over all $\mathbf{p}$, we can relabel $\mathbf{p} \mapsto -\mathbf{p}$ for the second term in the expression for $\phi$, we can write
\[
  \phi(x) = \int \frac{\d^3 \mathbf{p}}{(2\pi)^3} \frac{e^{i \mathbf{p}\cdot \mathbf{x}}}{\sqrt{2 \omega_\mathbf{p}}} (a_\mathbf{p} + a_\mathbf{p}^\dagger).
\]
This then looks more like our good old
\[
  q = \frac{1}{\sqrt{2\omega}}(a + a^\dagger)
\]
for a normal harmonic oscillator.
\begin{prop}
  The canonical commutation relations of $\phi, \pi$, namely
  \begin{align*}
    [\phi(\mathbf{x}), \phi(\mathbf{y})] &= 0\\
    [\pi(\mathbf{x}), \pi(\mathbf{y})] &= 0\\
    [\phi(\mathbf{x}), \pi(\mathbf{y})] &= i \delta^3(\mathbf{x} - \mathbf{y})
    \intertext{are equivalent to}
    [a_\mathbf{p}, a_\mathbf{q}] &= 0\\
    [a_\mathbf{p}^\dagger, a_\mathbf{q}^\dagger] &= 0\\
    [a_\mathbf{p}, a_\mathbf{q}^\dagger] &= (2\pi)^3 \delta^3(\mathbf{p} - \mathbf{q}).
  \end{align*}
\end{prop}

\begin{proof}
  We will only prove one direction of the equivalence. We will use the commutation relations for the $a_\mathbf{p}$ to obtain the commutation relations for $\phi$ and $\pi$. We compute
  \begin{align*}
    &[\phi(\mathbf{x}), \pi(\mathbf{y})]\\
    ={}& \int \frac{\d^3 \mathbf{p}\; \d^3 \mathbf{q}}{(2\pi)^6} \frac{(-i)}{2}\sqrt{\frac{\omega_\mathbf{q}}{\omega_\mathbf{p}}}\left(-[a_\mathbf{p}, a_\mathbf{q}^\dagger] e^{i\mathbf{p} \cdot \mathbf{x} - i \mathbf{q}\cdot \mathbf{y}} + [a_\mathbf{p}^\dagger, a_\mathbf{q}] e^{i\mathbf{p}\cdot \mathbf{x} + i \mathbf{q}\cdot \mathbf{y}}\right)\\
    ={}& \int \frac{\d^3 \mathbf{p}\; \d^3 \mathbf{q}}{(2\pi)^6} \frac{(-i)}{2}\sqrt{\frac{\omega_\mathbf{q}}{\omega_\mathbf{p}}}(2\pi)^3\left(-\delta^3(\mathbf{p} - \mathbf{q}) e^{i\mathbf{p} \cdot \mathbf{x} - i \mathbf{q}\cdot \mathbf{y}} - \delta^3(\mathbf{q} - \mathbf{p}) e^{i\mathbf{p}\cdot \mathbf{x} + i \mathbf{q}\cdot \mathbf{y}}\right)\\
    ={}& \int \frac{\d^3 \mathbf{p}}{(2\pi)^3} \left(-e^{-i\mathbf{p}\cdot (\mathbf{x} - \mathbf{y})} - e^{i\mathbf{p} \cdot(\mathbf{y} - \mathbf{x})}\right)\\
    ={}& i \delta^3(\mathbf{x} - \mathbf{y}).
  \end{align*}
  The other ones are trivial.
\end{proof}

Let's now compute the Hamiltonian in terms of $a_\mathbf{p}$ and $a_\mathbf{p}^\dagger$. We have
\begin{align*}
  H ={}& \frac{1}{2}\int \d^3 \mathbf{x}\;(\pi^2 + (\nabla \phi)^2 + m^2 \phi^2)\\
  ={}&\int \frac{\d^3 \mathbf{x}\; \d^3 \mathbf{p}\; \d^3 \mathbf{q}}{(2\pi)^6} \left(-\frac{\sqrt{\omega_p \omega_q}}{2} \left(a_\mathbf{p} e^{i\mathbf{p}\cdot \mathbf{x}} - a_\mathbf{p}^\dagger e^{-i\mathbf{p}\cdot \mathbf{x}}\right)\left(a_\mathbf{q} e^{i\mathbf{q}\cdot \mathbf{x}} - a_\mathbf{q}^\dagger e^{-i\mathbf{q}\cdot \mathbf{x}}\right)\right.\\
  &\quad+ \frac{1}{2\sqrt{\omega_\mathbf{p} \omega_\mathbf{q}}}\left(i\mathbf{p} a_\mathbf{p} e^{i\mathbf{p}\cdot \mathbf{x}} - i\mathbf{p} a_\mathbf{p}^\dagger e^{-i\mathbf{p}\cdot \mathbf{x}}\right)\left(i\mathbf{q} a_\mathbf{q} e^{i\mathbf{q}\cdot \mathbf{x}} - i\mathbf{q} a_\mathbf{q}^\dagger e^{-i\mathbf{q}\cdot \mathbf{x}}\right)\\
  &\quad+\left. \frac{m^2}{2\sqrt{\omega_\mathbf{p}\omega_\mathbf{q}}} \left(a_\mathbf{p} e^{i\mathbf{p}\cdot \mathbf{x}} + a_\mathbf{p}^\dagger e^{-i\mathbf{p}\cdot \mathbf{x}}\right)\left(a_\mathbf{q} e^{i\mathbf{q}\cdot \mathbf{x}} + a_\mathbf{q}^\dagger e^{-i\mathbf{q}\cdot \mathbf{x}}\right)\right)\\
  ={}& \frac{1}{4}\int \frac{\d^3 \mathbf{p}}{(2\pi)^3}\left(-\omega_\mathbf{p} + \frac{\mathbf{p}^2}{\omega_\mathbf{p}} + \frac{m^2}{\omega_\mathbf{p}}\right)(a_\mathbf{p} a_{-\mathbf{p}} + a_\mathbf{p}^\dagger a_{-\mathbf{p}}^\dagger)\\
  &\quad+ \left(\omega_\mathbf{p} + \frac{\mathbf{p}^2}{\omega_\mathbf{p}} + \frac{m^2}{\omega_\mathbf{p}}\right) (a_\mathbf{p} a_\mathbf{p}^\dagger + a_\mathbf{p}^\dagger a_\mathbf{p})\\
  \intertext{Now the first term vanishes, since we have $\omega_\mathbf{p}^2 = p^2 + m^2$. So we are left with}
  ={}& \frac{1}{4}\int \frac{\d^3 \mathbf{p}}{(2\pi)^3}\frac{1}{\omega_\mathbf{p}} (\omega_\mathbf{p}^2 + p^2 + m^2)(a_\mathbf{p} a_\mathbf{p}^\dagger + a_\mathbf{p}^\dagger a_\mathbf{p})\\
  ={}& \int \frac{\d^3 p}{(2\pi)^3} \omega_\mathbf{p}\left(a_\mathbf{p}^\dagger a_\mathbf{p} + \frac{1}{2}[a_\mathbf{p}, a_\mathbf{p}^\dagger]\right)\\
  ={}& \int \frac{\d^3 p}{(2\pi)^3} \omega_\mathbf{p}\left(a_\mathbf{p}^\dagger a_\mathbf{p} + \frac{1}{2}(2\pi)^3 \delta^3(\mathbf{0})\right),
\end{align*}
which is simply the Hamiltonian for an infinite number of uncoupled simple harmonic oscillators, each with frequency $\omega_\mathbf{p} = \sqrt{\mathbf{p}^2 + m^2}$.

Following the simple harmonic oscillator, we define the vacuum $\bket{0}$ such that
\[
  a_\mathbf{p} \bket{0} = 0
\]
for all $\mathbf{p}$.

When $H$ acts on this, the $a_\mathbf{p}^\dagger a_\mathbf{p}$ terms all vanish. So the energy of this ground state comes from the second term only, and we have
\[
  H\bket{0} = \frac{1}{2}\int \frac{\d^3 \mathbf{p}}{(2\pi)^3} \omega_\mathbf{p} (2\pi)^3 \delta^3(\mathbf{0}) \bket{0} = \infty\bket{0}.
\]
since all $\omega_\mathbf{p}$ are non-negative.

Quantum field theory is always full of these infinities! But they tell us something important. Often, they tell us that we are asking a stupid question.

Let's take a minute to explore this infinity and see what it means. This is bad, since what we want to do at the end is to compute some actual probabilities in real life, and infinities aren't exactly easy to calculate with.

In fact, there are two infinities in this. The first is due to space being big. These are known as ``\term{infrared divergences}''. To solve this, we simply put the universe in a box of side length $L$, and impose periodic boundary conditions on the field. If we do that, then we can write
\[
  (2\pi)^3 \delta^3(\mathbf{0}) = \lim_{L \to \infty} \left.\int_{-L/2}^{L/2} e^{i\mathbf{p}\cdot \mathbf{x}}\right|_{p = 0} = V,
\]
the volume of the universe. So the $\delta$-function arises just because we are computing the total energy of the whole universe, which is not surprisingly infinite. However, we are actually fine as long as the energy density $\mathcal{E}_0$ is finite. Here we have
\[
  \mathcal{E}_0 = \frac{E}{V} = \int \frac{\d^3 \mathbf{p}}{(2\pi)^3} \frac{1}{2} \omega_\mathbf{p},
\]
which are \emph{still} infinite, since $\omega_\mathbf{p}$ gets unbounded as $\mathbf{p} \to \infty$. In other words, the ground state energies for each simple harmonic oscillator add up to infinity.

These are high frequency divergences at short distances. These are called \term{ultraviolet divergences}. This arises because we assumed that our theory is valid to arbitrarily short distances, or equivalently, high momentum. Instead, if we just view our theory as a low-energy approximation of the real world, we really should cut off the integral at high momentum in some way. While the cut-off point is arbitrary, it doesn't really matter. In (non-gravitational) physics, we only care about energy differences, and picking a different cut-off point would just add a constant energy to everything.

Even more straightforwardly, if we just care about energy differences, we can just forget about the infinite term, and write
\[
  H = \int \frac{\d^3 p}{(2\pi)^3} \omega_\mathbf{p} a_\mathbf{p}^\dagger a_\mathbf{p}.
\]
Then we have
\[
  H \bket{0} = 0.
\]
While all these infinity cancelling sound like bonkers, the theory actually fits experimental data very well. So we have to live with it.

The difference between this $H$ and the previous one with infinities is an ordering ambiguity in going from the classical to the quantum theory. Recall that we did the quantization by replacing the terms in the classical Hamiltonian with operators. However, terms in the classical Hamiltonian are commutative, but not in the quantum theory. So if we write the classical Hamiltonian in a different way, we get a different quantized theory. Indeed, if we initially wrote
\[
  H = \frac{1}{2} (\omega q - ip)(\omega q + ip),
\]
for the classical Hamiltonian for a single harmonic operator, and then do the quantization, we obtain
\[
  H = \omega a^\dagger a.
\]
\begin{defi}[Normal order]\index{normal order}
  Given a string of operators
  \[
    \phi_1(\mathbf{x}_1) \cdots \phi_n(\mathbf{x}_n),
  \]
  the \emph{normal order} is what you obtain when you put all the annihilation operators in the RHS. This is written as
  \[
    :\phi_1(\mathbf{x}_1) \cdots \phi_n(\mathbf{x}_n):.
  \]
\end{defi}
So we could write
\[
  :H: = \int \frac{\d^3 \mathbf{p}}{(2\pi)^3} a_\mathbf{p}^\dagger a_\mathbf{p}.
\]
\printindex
\end{document}
