\documentclass[a4paper]{article}

\def\npart {III}
\def\nterm {Michaelmas}
\def\nyear {2016}
\def\nlecturer {R. Jozsa}
\def\ncourse {Quantum Computation}
\def\nofficial {https://www.qi.damtp.cam.ac.uk/node/261}
\def\nlectures {TT.9}

% Imports
\ifx \nextra \undefined
  \usepackage[pdftex,
    hidelinks,
    pdfauthor={Dexter Chua},
    pdfsubject={Cambridge Maths Notes: Part \npart\ - \ncourse},
    pdftitle={Part \npart\ - \ncourse},
  pdfkeywords={Cambridge Mathematics Maths Math \npart\ \nterm\ \nyear\ \ncourse}]{hyperref}
  \title{Part \npart\ - \ncourse}
\else
  \usepackage[pdftex,
    hidelinks,
    pdfauthor={Dexter Chua},
    pdfsubject={Cambridge Maths Notes: Part \npart\ - \ncourse\ (\nextra)},
    pdftitle={Part \npart\ - \ncourse\ (\nextra)},
  pdfkeywords={Cambridge Mathematics Maths Math \npart\ \nterm\ \nyear\ \ncourse\ \nextra}]{hyperref}

  \title{Part \npart\ - \ncourse \\ {\Large \nextra}}
\fi

\author{Lectured by \nlecturer \\\small Notes taken by Dexter Chua}
\date{\nterm\ \nyear}

\usepackage{alltt}
\usepackage{amsfonts}
\usepackage{amsmath}
\usepackage{amssymb}
\usepackage{amsthm}
\usepackage{booktabs}
\usepackage{caption}
\usepackage{enumitem}
\usepackage{fancyhdr}
\usepackage{graphicx}
\usepackage{mathtools}
\usepackage{microtype}
\usepackage{multirow}
\usepackage{pdflscape}
\usepackage{pgfplots}
\usepackage{siunitx}
\usepackage{tabularx}
\usepackage{tikz}
\usepackage{tkz-euclide}
\usepackage[normalem]{ulem}
\usepackage[all]{xy}

\pgfplotsset{compat=1.12}

\pagestyle{fancyplain}
\lhead{\emph{\nouppercase{\leftmark}}}
\ifx \nextra \undefined
  \rhead{
    \ifnum\thepage=1
    \else
      \npart\ \ncourse
    \fi}
\else
  \rhead{
    \ifnum\thepage=1
    \else
      \npart\ \ncourse\ (\nextra)
    \fi}
\fi
\usetikzlibrary{arrows}
\usetikzlibrary{decorations.markings}
\usetikzlibrary{decorations.pathmorphing}
\usetikzlibrary{positioning}
\usetikzlibrary{fadings}
\usetikzlibrary{intersections}
\usetikzlibrary{cd}

\newcommand*{\Cdot}{\raisebox{-0.25ex}{\scalebox{1.5}{$\cdot$}}}
\newcommand {\pd}[2][ ]{
  \ifx #1 { }
    \frac{\partial}{\partial #2}
  \else
    \frac{\partial^{#1}}{\partial #2^{#1}}
  \fi
}

% Theorems
\theoremstyle{definition}
\newtheorem*{aim}{Aim}
\newtheorem*{axiom}{Axiom}
\newtheorem*{claim}{Claim}
\newtheorem*{cor}{Corollary}
\newtheorem*{defi}{Definition}
\newtheorem*{eg}{Example}
\newtheorem*{fact}{Fact}
\newtheorem*{law}{Law}
\newtheorem*{lemma}{Lemma}
\newtheorem*{notation}{Notation}
\newtheorem*{prop}{Proposition}
\newtheorem*{thm}{Theorem}

\renewcommand{\labelitemi}{--}
\renewcommand{\labelitemii}{$\circ$}
\renewcommand{\labelenumi}{(\roman{*})}

\let\stdsection\section
\renewcommand\section{\newpage\stdsection}

% Strike through
\def\st{\bgroup \ULdepth=-.55ex \ULset}

% Maths symbols
\newcommand{\bra}{\langle}
\newcommand{\ket}{\rangle}

\newcommand{\N}{\mathbb{N}}
\newcommand{\Z}{\mathbb{Z}}
\newcommand{\Q}{\mathbb{Q}}
\renewcommand{\H}{\mathbb{H}}
\newcommand{\R}{\mathbb{R}}
\newcommand{\C}{\mathbb{C}}
\newcommand{\Prob}{\mathbb{P}}
\renewcommand{\P}{\mathbb{P}}
\newcommand{\E}{\mathbb{E}}
\newcommand{\F}{\mathbb{F}}
\newcommand{\cU}{\mathcal{U}}
\newcommand{\RP}{\mathbb{RP}}
\newcommand{\CP}{\mathbb{CP}}

\newcommand{\ph}{\,\cdot\,}

\DeclareMathOperator{\sech}{sech}
\DeclareMathOperator{\cosech}{cosech}
\DeclareMathOperator{\cosec}{cosec}

\DeclareMathOperator{\covol}{covol}
\DeclareMathOperator{\vol}{vol}

\let\Im\relax
\let\Re\relax
\DeclareMathOperator{\Im}{Im}
\DeclareMathOperator{\Re}{Re}
\DeclareMathOperator{\im}{im}
\DeclareMathOperator{\image}{image}
\DeclareMathOperator{\Ann}{Ann}

\DeclareMathOperator*{\res}{res}
\DeclareMathOperator{\Res}{Res}
\DeclareMathOperator{\Ind}{Ind}

\DeclareMathOperator{\tr}{tr}
\DeclareMathOperator{\diag}{diag}
\DeclareMathOperator{\rank}{rank}
\DeclareMathOperator{\card}{card}
\DeclareMathOperator{\spn}{span}
\DeclareMathOperator{\adj}{adj}

\DeclareMathOperator{\erf}{erf}
\DeclareMathOperator{\erfc}{erfc}

\DeclareMathOperator{\ord}{ord}
\DeclareMathOperator{\Sym}{Sym}

\DeclareMathOperator{\sgn}{sgn}
\DeclareMathOperator{\orb}{orb}
\DeclareMathOperator{\stab}{stab}
\DeclareMathOperator{\ccl}{ccl}

\DeclareMathOperator{\lcm}{lcm}
\DeclareMathOperator{\hcf}{hcf}

\DeclareMathOperator{\Int}{Int}
\DeclareMathOperator{\id}{id}

\DeclareMathOperator{\betaD}{beta}
\DeclareMathOperator{\gammaD}{gamma}
\DeclareMathOperator{\Poisson}{Poisson}
\DeclareMathOperator{\binomial}{binomial}
\DeclareMathOperator{\multinomial}{multinomial}
\DeclareMathOperator{\Bernoulli}{Bernoulli}
\DeclareMathOperator{\like}{like}

\DeclareMathOperator{\var}{var}
\DeclareMathOperator{\cov}{cov}
\DeclareMathOperator{\bias}{bias}
\DeclareMathOperator{\mse}{mse}
\DeclareMathOperator{\corr}{corr}

\DeclareMathOperator{\otp}{otp}
\DeclareMathOperator{\dom}{dom}

\DeclareMathOperator{\Root}{Root}
\DeclareMathOperator{\supp}{supp}
\DeclareMathOperator{\rel}{rel}
\DeclareMathOperator{\Hom}{Hom}
\DeclareMathOperator{\Aut}{Aut}
\DeclareMathOperator{\Gal}{Gal}
\DeclareMathOperator{\Mat}{Mat}
\DeclareMathOperator{\End}{End}
\DeclareMathOperator{\Char}{char}
\DeclareMathOperator{\ev}{ev}
\DeclareMathOperator{\St}{St}
\DeclareMathOperator{\Lk}{Lk}
\DeclareMathOperator{\disc}{disc}
\DeclareMathOperator{\Isom}{Isom}
\DeclareMathOperator{\length}{length}
\DeclareMathOperator{\energy}{energy}
\DeclareMathOperator{\area}{area}
\DeclareMathOperator{\Syl}{Syl}
\DeclareMathOperator{\cl}{cl}
\DeclareMathOperator{\fix}{fix}

\newcommand{\GL}{\mathrm{GL}}
\newcommand{\SL}{\mathrm{SL}}
\newcommand{\PGL}{\mathrm{PGL}}
\newcommand{\PSL}{\mathrm{PSL}}
\newcommand{\PSU}{\mathrm{PSU}}
\newcommand{\Or}{\mathrm{O}}
\newcommand{\SO}{\mathrm{SO}}
\newcommand{\U}{\mathrm{U}}
\newcommand{\SU}{\mathrm{SU}}

\renewcommand{\d}{\mathrm{d}}
\newcommand{\D}{\mathrm{D}}

\tikzset{->/.style = {decoration={markings,
                                  mark=at position 1 with {\arrow[scale=2]{latex'}}},
                      postaction={decorate}}}
\tikzset{<-/.style = {decoration={markings,
                                  mark=at position 0 with {\arrowreversed[scale=2]{latex'}}},
                      postaction={decorate}}}
\tikzset{<->/.style = {decoration={markings,
                                   mark=at position 0 with {\arrowreversed[scale=2]{latex'}},
                                   mark=at position 1 with {\arrow[scale=2]{latex'}}},
                       postaction={decorate}}}
\tikzset{->-/.style = {decoration={markings,
                                   mark=at position #1 with {\arrow[scale=2]{latex'}}},
                       postaction={decorate}}}
\tikzset{-<-/.style = {decoration={markings,
                                   mark=at position #1 with {\arrowreversed[scale=2]{latex'}}},
                       postaction={decorate}}}

\tikzset{circ/.style = {fill, circle, inner sep = 0, minimum size = 3}}
\tikzset{mstate/.style={circle, draw, blue, text=black, minimum width=0.7cm}}

\definecolor{mblue}{rgb}{0.2, 0.3, 0.8}
\definecolor{morange}{rgb}{1, 0.5, 0}
\definecolor{mgreen}{rgb}{0.1, 0.4, 0.2}
\definecolor{mred}{rgb}{0.5, 0, 0}

\def\drawcirculararc(#1,#2)(#3,#4)(#5,#6){%
    \pgfmathsetmacro\cA{(#1*#1+#2*#2-#3*#3-#4*#4)/2}%
    \pgfmathsetmacro\cB{(#1*#1+#2*#2-#5*#5-#6*#6)/2}%
    \pgfmathsetmacro\cy{(\cB*(#1-#3)-\cA*(#1-#5))/%
                        ((#2-#6)*(#1-#3)-(#2-#4)*(#1-#5))}%
    \pgfmathsetmacro\cx{(\cA-\cy*(#2-#4))/(#1-#3)}%
    \pgfmathsetmacro\cr{sqrt((#1-\cx)*(#1-\cx)+(#2-\cy)*(#2-\cy))}%
    \pgfmathsetmacro\cA{atan2(#2-\cy,#1-\cx)}%
    \pgfmathsetmacro\cB{atan2(#6-\cy,#5-\cx)}%
    \pgfmathparse{\cB<\cA}%
    \ifnum\pgfmathresult=1
        \pgfmathsetmacro\cB{\cB+360}%
    \fi
    \draw (#1,#2) arc (\cA:\cB:\cr);%
}
\newcommand\getCoord[3]{\newdimen{#1}\newdimen{#2}\pgfextractx{#1}{\pgfpointanchor{#3}{center}}\pgfextracty{#2}{\pgfpointanchor{#3}{center}}}

\def\Xint#1{\mathchoice
   {\XXint\displaystyle\textstyle{#1}}%
   {\XXint\textstyle\scriptstyle{#1}}%
   {\XXint\scriptstyle\scriptscriptstyle{#1}}%
   {\XXint\scriptscriptstyle\scriptscriptstyle{#1}}%
   \!\int}
\def\XXint#1#2#3{{\setbox0=\hbox{$#1{#2#3}{\int}$}
     \vcenter{\hbox{$#2#3$}}\kern-.5\wd0}}
\def\ddashint{\Xint=}
\def\dashint{\Xint-}


\begin{document}
\maketitle
{\small
\setlength{\parindent}{0em}
\setlength{\parskip}{1em}

Quantum mechanical processes can be exploited to provide new modes of information processing that are beyond the capabilities of any classical computer. This leads to remarkable new kinds of algorithms (so-called quantum algorithms) that can offer a dramatically increased efficiency for the execution of some computational tasks. Notable examples include integer factorisation (and consequent efficient breaking of commonly used public key crypto systems) and database searching. In addition to such potential practical benefits, the study of quantum computation has great theoretical interest, combining concepts from computational complexity theory and quantum physics to provide striking fundamental insights into the nature of both disciplines.

The course will cover the following topics:

Notion of qubits, quantum logic gates, circuit model of quantum computation. Basic notions of quantum computational complexity, oracles, query complexity.

The quantum Fourier transform. Exposition of fundamental quantum algorithms including the Deutsch-Jozsa algorithm, Shors factoring algorithm, Grovers searching algorithm.

A selection from the following further topics (and possibly others):
\begin{enumerate}
  \item Quantum teleportation and the measurement-based model of quantum computation;
  \item Lower bounds on quantum query complexity;
  \item Phase estimation and applications in quantum algorithms;
  \item Quantum simulation for local hamiltonians.
\end{enumerate}

\subsubsection*{Pre-requisites}
It is desirable to have familiarity with the basic formalism of quantum mechanics especially in the simple context of finite dimensional state spaces (state vectors, Dirac notation, composite systems, unitary matrices, Born rule for quantum measurements). Prerequisite notes will be provided on the course webpage giving an account of the necessary material including exercises on the use of notations and relevant calculational techniques of linear algebra. It would be desirable for you to look through this material at (or slightly before) the start of the course. Any encounter with basic ideas of classical theoretical computer science (complexity theory) would be helpful but is not essential.
}
\tableofcontents

\setcounter{section}{-1}
\section{Introduction}
% Significance and Context
Quantum computation is currently a highly significant and important subject, and is very active in international research.

First of all, it is a fundamental connection between physics and computing. We can think of physics as computing, where in physics, we label states with parameters (ie. numbers), and physical evolution changes these parameters. So we can think of these parameters as encoding information, and physical evolution changes the information. Thus, this evolution can be thought of as a computational process.

More strikingly, we can also view computing as physics! We all have computers, and usually represent information as bits, 0 or 1. We often think of computation as manipulation of these bits, ie. as discrete maths. However, there is no actual discrete bits --- when we build a computer, we need physical devices to represent these bits. When we run a computation on a computer, it has to obey the laws of physics. So we arrive at the idea that the limits of computation are not a part of mathematics, but depend on the laws of physics. Thus, we can associate a ``computing power'' with any theory of physics!

On the other hand, there is also a technology/engineering aspect of quantum computation. Historically, we have been trying to reduce the size of computers. Eventually, we will want to try to achieve miniaturization of computer components to essentially the subatomic scale. The usual boolean operations we base our computations on do not work so well on this small scale, since quantum effects start to kick in. We could try to mitigate these quantum issues and somehow force the bits to act classically, but we can also embrace the quantum effects, and build a quantum computer! There is a lot of recent progress in quantum technology. We are now expecting a 50-qubit quantum computer in full coherent control soon. However, we are not going to talk about implementation in this course.

Finally, apart from the practical problem of building quantum computers, we also have theoretical quantum computer science, where we try to understand how quantum algorithms behave. This is about how we can actually exploit quantum physical facts for computational possibilities beyond classical computers. This will be the focus of the course.

\section{Computational complexity theory}
To appreciate the difference between quantum and classical computing, we need to know how to evaluate an algorithm. This involves understanding some basic complexity theory. But before we could do that, we need to know about computability versus uncomputability.

We will not provide a proper definition of ``computable''. To do so, one comes up with a sensible mathematical model of a computer, and ``computable'' means that theoretical computer can compute it. By the Church-Turing thesis, any two sensible definitions of computability should be equivalent, so we will not spend too much time working out the details.

\begin{eg}
  Let $N$ be an integer. We want to figure out if $N$ a prime. This is clearly computable, since we can try all numbers less than $N$ and see if it divides $N$.
\end{eg}

This is not too surprising, but it turns out there are some problems that are not computable! Most famously, we have the Halting problem.
\begin{eg}[Halting problem]\index{Halting problem}
  Given the code of a computer program, we want to figure out if the computer will eventually halt. In 1936, Turing proved that this problem is uncomputable! So we cannot have a program that determines if an arbitrary program halts!
\end{eg}

For a less arbitrary problem, we have
\begin{eg}
  Given a polynomial with integer coefficients with many variables, eg. $2x^2 y - 17 zw^{19} + x^5 w^3 + 1$, does this have a root in the integers? It was shown in 1976 that this problem is uncomputable as well!
\end{eg}

These results are all for classical computing. If quantum computing is somehow different, can we get around this problems? This turns out not to be the case, for the very reason that all the laws of quantum physics (eg. state descriptions, evolution equations) are all computable on a classical computer (in principle). So it follows that quantum computing, being a quantum process, cannot compute any classical uncomputable problem.

Despite this limitation, quantum computation is still interesting! In practice, there is also the problem of complexity --- the complexity (ie. how hard, how much effort is needed to do the computation) of a quantum computation might be much simpler than the classical counterpart.

To formalize this, we make precise the notion of a computational task.

\begin{defi}[Input string]\index{input string}
  An \emph{input bit string} is a sequence of bits $x = i_1i_2 \cdots i_n$, where each $i_k$ is either $0$ or $1$. We write $B_n$ for the set of all $n$-bit string, and $B = \bigcup_{n \in \N} B_n$. The \emph{input size} is the length $n$. So in particular, if the input is regarded as an actual number, the size is not the number itself, but its logarithm.
\end{defi}

\begin{defi}[Language]
  A \term{language} is a subset $L\subseteq B$.
\end{defi}

\begin{defi}[Decision problem]
  Given a language $L$, the \term{decision problem} is to determine whether an arbitrary $x \in B$ is a member of $L$. The output is thus 1 bit of information, namely yes or no.
\end{defi}
Of course, we can have a more general task with multiple outputs, but for simplicity, we will not consider that case here.

\begin{eg}
  If $L$ is the set of all prime numbers, then the corresponding decision problem is determining whether a number is prime.
\end{eg}

We also have to talk about models of computations. We will only give an intuitive and classical description of it.
\begin{defi}[Computational model]
  A \term{computational model} is a process with discrete steps (elementary computational steps), where each step requires a constant amount of effort/resources to implement.
\end{defi}
If we think about actual computers that works with bits, we can imagine a step as an operation such as ``and'' or ``or''. Note that addition and multiplication are \emph{not} considered a single step --- as the number gets larger, it takes more effort to add or multiply them.

\begin{defi}[Randomized/probabilistic computation]
  This is the same as a usual computational model, but the process also has access to a string $r_1, r_2, r_3, \cdots$ of independent, uniform random bits. In this case, we will often require the answer/output to be correct with ``suitably good'' probability.
\end{defi}

In computer science, there is a separate notion of ``non-deterministic'' computation, which is \emph{different} from probabilistic computation. In probabilistic computation, every time we ask for a random number, we just pick one of the possible output and follows that. With a non-deterministic computer, we simultaneously consider \emph{all} possible paths you can take.

\begin{defi}[Complexity of a computational task (or an algorithm)]
  The \term{complexity} of a computational task or algorithm is the ``consumption of resources as a function of input size $n$''. The resources are usually the time
  \[
    T(n) = \text{number of computational steps needed},
  \]
  and space
  \[
    Sp(n) = \text{number of memory/work space needed}.
  \]
  In each case, we take the worse case input of a given size $n$.
\end{defi}
We usually consider the worst-case scenario, since, eg. for primality testing, there are always some numbers which we can easily rule out as being not prime (eg. even numbers). Sometimes, we will also want to study the average complexity.

In the course, we will mostly focus on the time complexity, and not work with the space complexity itself.

As one would imagine, the actual time or space taken would vary a lot on the actual computational model. Thus, the main question we ask will be whether $T(n)$ grows polynomially or super-polynomially (``exponentially'') with $n$.
\begin{defi}[Polynomial growth]\index{polynomial growth}
  We say $T(n)$ \emph{grows polynomially}, and write
  \[
    T(n) = O(\poly(n)) = O(n^k)
  \]
  for some $k$, if there is some constant $c$, and some integer $k$ and some integer $n_0$ such that $T(n) < c n^k$ for all $n > n_0$.
\end{defi}

The other possible cases are exponential growth, eg. $T(n) = c_1 2^{c_2 n}$, or super-polynomial and sub-exponential growth such as $T(n) = 2^{\sqrt{n}}$ or $n^{\log n}$.

We will usually regard polynomial time processes as ``feasible in practice'', while super-polynomial ones are considered ``infeasible''. Of course, this is not always actually true. For example, we might have a polynomial time of $n^{10^{10^10}}$, or an exponential time of $2^{0.0000\ldots0001 n}$. However, this distinction of polynomial vs non-polynomial is robust, since any computational model can ``simulate'' other computational models in polynomial time. So if something is polynomial in one computational model, it is polynomial in all models.

In general, we can have a more refined complexity classes of decision problems:
\begin{enumerate}
  \item \term{\textbf{P}} (\term{polynomial time}): The class of decision problems having \emph{deterministic} polynomial-time algorithm.
  \item \term{\textbf{BPP}} (\term{bounded error, probabilistic polynomial time}): The class of decision problems having \emph{probabilistic} polynomial time algorithms such that for every input,
    \[
      \mathrm{Prob}(\text{answer is correct}) \geq \frac{2}{3}.
    \]
    The number $\frac{2}{3}$ is sort of arbitrary --- we see that we cannot put $\frac{1}{2}$, or else we can just randomly guess a number. So we need something greater than $\frac{1}{}$, and ``bounded'' refers to it being bounded away from $\frac{1}{2}$. We could replace $\frac{2}{3}$ with any other constant $\frac{1}{2} + \delta$ with $0 < \delta < \frac{1}{2}$, and \textbf{BPP} is the same. This is because if we have a $\frac{1}{2} + \delta$ algorithm, we simply repeat the algorithm $K$ times, and take the majority vote. By the Chernoff bound (a result in probability), the probability that the majority vote is correct is $> 1 - e^{-2 \delta^2 K}$. So as we do more and more runs, the probability of getting a right answer grows exponentially. This can be bigger than an $1 - \varepsilon$ by a suitably large $K$. Since $K$ times a polynomial time is still polynomial time, we still have a polynomial time algorithm.

    These are often considered as ``classically feasible computations'', or ``computable in practice''. In this case, we tolerate small errors, but that is fine in practice, since in genuine computers, random cosmic rays and memory failures can also cause small errors in the result, even for a deterministic algorithm.
\end{enumerate}
It is not known whether \textbf{P} and \textbf{BPP} are the same --- in general, not much is known about whether two complexity classes are the same.

\begin{eg}[Primality testing]
  Let $N$ be an integer. We want to determine if it is prime. The input size is $\log_2 N$. The naive method of primality testing is to test all numbers and see if it divides $N$. We only need to test up to $\sqrt{N}$, since if $N$ has a factor, there must be one below $\sqrt{N}$. The is \emph{not} polynomial time, since we need $\sqrt{N} = 2^{\frac{1}{2} \log N}$ operations, we see that this is exponential time.

  How about a probabilistic algorithm? We can choose a random $k < N$, and see if $k$ divides $N$. This is a probabilistic, polynomial time algorithm, but it is not bounded, since the probability of getting a correct answer is not $> \frac{1}{2}$.

  In reality, primality testing is known to be in \textbf{BPP} (1976), and it is also known to be in \textbf{P} (2004).
\end{eg}

The most famous model for computation is probably a \emph{Turing machine}. However, for our purposes, it is much simpler to work with the \emph{circuit model}.

Classically, for each size $n$, we have a prescribed circuit of Boolean $\AND$, $\OR$ and $\NOT$ gates. We have a \term{circuit family} $\mathcal{C}_1, \mathcal{C}_2, \cdots$, with one for each input size.
% insert diagram

We can think of this as a ``program in machine language''. The computational steps are the gates. The time $T(n)$ is the size of the circuit $\mathcal{C}_n$, ie. the total number of gates in the circuit. A polynomial time computation is one where the size of $\mathcal{C}_n$ grows polynomially in $n$.

We have chosen or primitive gates to be $\AND$, $\OR$ and $\NOT$. It is a fact that this is a \emph{universal set} of gates --- we can make any boolean function $f: B_m \to B_n$ as a circuit of gates from $G$.

Now we go on to quantum stuff!
\section{Quantum computation}
We are again going to introduce our quantum computational model with the circuit model. A single qubit is an element of $\C^2$, with basis vectors
\[
  \bket{0} =
  \begin{pmatrix}
    1 \\ 0
  \end{pmatrix},\quad
  \bket{1} =
  \begin{pmatrix}
    0 \\ 1
  \end{pmatrix}.
\]
For an input $x = i_1i_2\cdots i_n$, we encode this as a qubit $\bket{i_1} \bket{i_2}\cdots\bket{i_n}\bket{0}\cdots\bket{0}$. Now the computational steps are quantum gates, namely unitary operations on designated (few) qubits. For a single qubit, namely an element of $\C^2$, a unitary operation is a $2 \times 2$ matrix. Basic unitary gates commonly used include
\[
  \qX =
  \begin{pmatrix}
    0 & 1\\
    1 & 0
  \end{pmatrix},\quad
  \qZ =
  \begin{pmatrix}
    1 & 0\\
    0 & -1
  \end{pmatrix},\quad
  \qH = \frac{1}{\sqrt{2}}
  \begin{pmatrix}
    1 & 1\\
    1 & -1
  \end{pmatrix},\quad
  \qP_\varphi =
  \begin{pmatrix}1 & 0\\0 & e^{\varphi}\end{pmatrix}
\]
We also have two-qubit gates
\[
  \qCX\bket{i}\bket{j} = \bket{i} X^i\bket{j},\quad CZ
\]
More explicitly, we have
\[
  \qCX =
  \begin{pmatrix}
    1 & 0 & 0 & 0\\
    0 & 1 & 0 & 0\\
    0 & 0 & 0 & 1\\
    0 & 0 & 1 & 0
  \end{pmatrix},\quad
  \qCZ =
  \begin{pmatrix}
    1 & 0 & 0 & 0\\
    0 & 1 & 0 & 0\\
    0 & 0 & 1 & 0\\
    0 & 0 & 0 & -1
  \end{pmatrix},
\]
These will be taken as the basic unitary gates.

After applying a number of these gates, we measure one or more qubit at the end. We also allow intermediate measurements as steps too. These are \emph{not} given by unitary matrices, but in fact these add nothing new (see example sheet 1). This is known as the \term{principle of deferred measurements}. Moreover, we can allow making classical decisions based on the result of the measurement, ie. take different routes depending on what we measure, but again this adds nothing new.

Now what is the analogous notion of a universal set of gates? In the classical case, the set of boolean functions is discrete, and therefore a finite set of gates can be universal. However, in the quantum case, the possible unitary matrices are continuous, so no finite set can be universal.

The idea is to consider approximate universality. We can take the example of the rotations --- there is no single rotation that generates all possible rotations in $\R^2$. However, we can pick a rotation by an irrational angle, and then the set of rotations generated by this rotation is dense in the set of all rotations, and this is good enough.

\begin{defi}[Approximate universality]\index{approximate universality}
  A collection of gates is \emph{approximately universal} if for any unitary matrix $U$, there is some circuit $\tilde{U}$ such that
  \[
    \norm{U - \tilde{U}} < \varepsilon.
  \]
  In other words, we have
  \[
    \sup_{\norm{\psi} = 1} \norm{U\bket{\psi} - \tilde{U} \bket{\psi}} < \varepsilon,
  \]
  where we take the usual norm on the vectors (any two norms are equivalence if the state space is finite dimensional, so it doesn't really matter).
\end{defi}

We will provide some examples without proof.
\begin{eg}
  The infinite set $\{\qCX\} \cup \{\text{all $1$-qubit states}\}$ is exactly universal.
\end{eg}

\begin{eg}
  The collection
  \[
    \{\qH, \qT = \qP_{\pi/4}, \qCX\}
  \]
  is approximately universal.
\end{eg}

\begin{defi}[\textbf{BQP}]\index{\textbf{BQP}}
  The complexity class \textbf{BQP} (bounded error, quantum polynomial time) is the class of all decision problems computable with polynomial quantum circuits with at least $2/3$ probability of being correct.
\end{defi}
We can show that \textbf{BQP} is independent of choice of approximately universal gate set. This is not as trivial as the classical case, since when we switch to a different set, we cannot just replace a gate with an equivalent circuit --- we can only do so approximately, and we have to make sure we control the error appropriately to maintain the bound of $2/3$.

We consider \textbf{BQP} to be the feasible computations with quantum computations.

It is also a fact that \textbf{BPP} is a subset of \textbf{BQP}. This, again, is not a trivial result. Note that in a quantum computation, we act by unitary matrices, which are invertible. However, boolean functions in classical computing are not invertible in general.

However, it turns out that for any classical computation, there is an equivalent computation that uses reversible/invertible boolean gates, with a modest (ie. polynomial) overhead of both space and time resources. Indeed, let $f: B_m \to B_n$ be a boolean function. We consider the function
\begin{align*}
  \tilde{f}: B_{m + n} &\to B_{m + n}\\
  (x, y) &\mapsto (x, y \oplus f(x)),
\end{align*}
where $\oplus$ is the bitwise addition (ie. addition in $(\Z/2\Z)^n$, eg. $011 \oplus 110 = 101$). We call $x$ and $y$ the \term{input register} and \term{output register} respectively.

Now if we set $y = 0$, then we get $f(x)$ in the second component of $\tilde{f}$. So we can easily obtain $f$ from $\tilde{f}$, and vice versa.

\begin{lemma}
  For any boolean function $f: B_m \to B_n$, the function
  \begin{align*}
    \tilde{f}: B_{m + n} &\to B_{m + n}\\
    (x, y) &\mapsto (x, y \oplus f(x)),
  \end{align*}
  is invertible, and in fact an involution, ie. is its own inverse.
\end{lemma}

\begin{proof}
  Simply note that $x \oplus x = 0$ for any $x$, and bitwise addition is associative.
\end{proof}

So we can just consider boolean functions that are invertible. There is an easy way of making this a unitary matrix.
\begin{lemma}
  Let $g: B_k \to B_k$ be a reversible permutation of $k$-bit strings. Then the linear map on $\C^k$ defined by
  \[
    A:\bket{x} \mapsto \bket{g(x)}
  \]
  on $k$ qubits is unitary.
\end{lemma}

\begin{proof}
  This is because the $x$th column of the matrix of $A$ is in fact $A\bket{x} = \bket{g(x)}$, and since $g$ is bijective, the collection of all $\bket{g(x)}$ are all orthonormal.
\end{proof}

Thus, given any $f: B_m \to B_n$, we get the $n + m$-qubit unitary matrix denoted by $U_f$, given by
\[
  U_f\bket{x}\bket{y} = \bket{x}\bket{y \oplus f(x)}.
\]
\subsection{Computation by quantum parallelism}
Suppose we have a function $f: B_m \to B_n$. If we start with $\bket{x}\bket{0\cdots}$, we map
\[
  \begin{tikzcd}
    \bket{x} \bket{0\cdots0} \ar[r, "U_f"] & \bket{x}\bket{f(x)}
  \end{tikzcd}.
\]
So by linearity, we have
\[
  \begin{tikzcd}
    \frac{1}{\sqrt{2^n}}\sum_x \bket{x} \bket{0\cdots0} \ar[r, "U_f"] & \frac{1}{\sqrt{2^n}}\sum_x \bket{x}\bket{f(x)}
  \end{tikzcd}.
\]
We call the final state $\bket{f}$. Now \emph{one} run of $U_f$ gives us the $\bket{f}$ that embodies all exponentially many values of $f(x)$'s. While the state
\[
  \bket{\psi}\frac{1}{\sqrt{2^n}}\sum_x \bket{x}
\]
has exponentially many terms, it can be made in polynomial (and in fact linear) time by $n$ applications of $\qH$, where
\[
  \begin{tikzcd}
    \bket{0} \ar[r, "\qH"] & \frac{1}{\sqrt{2}}(\bket{0} + \bket{1})
  \end{tikzcd}
\]
So for an $n$-qubit state, we have
\[
  \begin{tikzcd}
    \bket{0}\cdots\bket{0} \ar[r, "\qH\otimes \cdots\otimes \qH"] &\frac{1}{\sqrt{2^n}}(\bket{0} + \bket{1})\cdots(\bket{0} + \bket{1})
  \end{tikzcd},
\]
and the result is exactly the starting state we were looking for.

\section{Query complexity}
We are going to come to our first quantum algorithm. Here the input to the computational task is slightly different from what we've got before. Instead of an input $i_1, \cdots, i_n \in B_n$, we have a \emph{black box}/\term{oracle} that computes some $f: B_m \to B_n$. We may have some \emph{a priori} promise on $f$, and we want to determine some property of the function $f$. The only access to $f$ is querying the oracle with its inputs.

The use of $f$ (classical) or $U_f$ (quantum) counts as one step of computation. The \term{query complexity} of this task is the least number of times the oracle needs to be queried. Usually, we do not care much about how many times the other gates are used.

Obviously, if we just query all the values of the function, then we can determine anything about the function, since we have complete information. But we want to see if we can use fewer queries.

\begin{eg}[Balanced vs constant problem]\index{balanced vs constant problem}
  The input black box is a function $f: B_n \to B_1$. The promise is that $f$ is either
  \begin{enumerate}
    \item a \emph{constant} function, ie. $f(x) = 0$ for all $x$, or $f(0) = 1$ for all $x$; or
    \item a \emph{balanced} function, ie. exactly half of its values in $B_n$ are sent to $1$.
  \end{enumerate}
  We want to determine if $f$ is (i) or (ii) with certainty.

  Classically, if we want to find the answer, in the worst case scenario, we will have to perform $2^n/2 + 1$ queries --- if you are really unlucky, you might query a balanced function $2^n/2$ times and get $0$ results.

  Quantumly, we have the \term{Deutsch-Jozsa algorithm}, that answers the question in 1 query!

  A trick we are going to use is something known as ``\term{phase kickback}''. Instead of encoding the result as a single bit, we encode them as $\pm$ signs, ie. as phases of the quantum bits. The ``kickback'' part is about using the fact that we have
  \[
    \bket{a} (e^{i\theta}\bket{b}) = (e^{i\theta}\bket{a})\bket{b},
  \]
  So we might do something to $\bket{b}$ to give it a phase, and then we ``kick back'' the phase on $\bket{a}$, and hopefully obtain something when we measure $\bket{a}$.

  Recall that we have
  \[
    U_f \bket{x}\bket{y} = \bket{x} \bket{y \oplus f(x)}.
  \]
  Here $\bket{x}$ has $n$ qubits, and $\bket{y}$ has 1 qubit.

  The non-obvious thing to do is to set the output register to
  \[
    \bket{\alpha} = \frac{\bket{0} - \bket{1}}{2} = \qH\bket{1} = \qH \qX \bket{0}.
  \]
  We then note that $U_f$ acts by
  \begin{align*}
    \bket{x}\left(\frac{\bket{0} - \bket{1}}{2}\right) &\mapsto \bket{x} \frac{\bket{f(x)} - \bket{1 \oplus f(x)}}{\sqrt{2}}\\
    &=
    \begin{cases}
      \bket{x}\frac{\bket{0} - \bket{1}}{\sqrt{2}}& f(x) = 0\\
      \bket{x}\frac{\bket{1} - \bket{0}}{\sqrt{2}}& f(x) = 1\\
    \end{cases}\\
    &= (-1)^{f(x)}\bket{x}\bket{\alpha}.
  \end{align*}
  Now we do this to the superposition over all possible $x$:
  \[
    \frac{1}{\sqrt{2^n}} \sum \bket{x}\bket{\alpha} \mapsto \left(\frac{1}{\sqrt{2^n}} \sum (-1)^{f(x)}\bket{x}\right) \bket{\alpha}.
  \]
  So one query gives us
  \[
    \bket{\xi_f} = \frac{1}{\sqrt{2^n}} \sum (-1)^{f(x)}\bket{x}.
  \]
  The key observation now is simply that if $f$ is constant, then all the signs are the same. If $x$ is balanced, then exactly half of the signs are $+$ and $-$. The crucial thing is that $\bket{\xi_{f_\text{const}}}$ is orthogonal to $\bket{\xi_{f_\text{balanced}}}$. This is a good thing, since orthogonality is something we can perfectly distinguish with a quantum measurement.

  There is a slight technical problem here. We allow only measurements in the standard $\bket{0}$, $\bket{1}$ basis. So we need want to ``rotate'' our states to the standard basis. Fortunately, recall that
  \[
    \begin{tikzcd}
      \bket{0}\cdots\bket{0} \ar[r, "\qH\otimes \cdots\otimes \qH"] &\frac{1}{\sqrt{2^n}}\sum \bket{x}
    \end{tikzcd},
  \]
  Now recall that $\qH$ is self-inverse, so $H^2 = I$. Thus, if we apply $\qH \otimes\cdots \otimes \qH$ to $\frac{1}{\sqrt{2^n}}\sum \bket{x}$, then we obtain $\bket{0}\cdots\bket{0}$.

  We write
  \[
    \bket{\eta_f} = \qH \otimes \cdots \otimes \qH \bket{\xi_f}.
  \]
  Since $\qH$ is unitary, we still have
  \[
    \bket{\eta_{f_\mathrm{const}}} \perp \bket{\eta_{f_\mathrm{balanced}}}.
  \]
  Now we note that if $f$ is constant, then
  \[
    \eta_{f_\mathrm{const}} = \pm \bket{0}\cdots\bket{0}.
  \]
  If we look at what $\bket{\eta_{f_\mathrm{balanced}}}$ is, it will be a huge mess, but it doesn't really matter --- all that matters is that it is perpendicular to $\bket{0}\cdots\bket{0}$.

  Now when we measure $\eta_f$, if $f$ is a constant function, then we obtain $0\cdots0$ with probability $1$. If it is balanced, then we obtain something that is not $0$ with probability $1$. So we can determine the result with probability $1$.

  \makeatletter
  \DeclareRobustCommand{\rvdots}{%
    \vbox{
      \baselineskip4\p@\lineskiplimit\z@
      \kern-\p@
      \hbox{.}\hbox{.}\hbox{.}
    }}
  \newcommand\addstate[3]{
    \pgfmathsetmacro{\y@y}{-#2 * 0.7};
    \pgfmathsetmacro{\x@x}{#3};
    \node [left] at (0, \y@y) {#1};
    \draw (0, \y@y) -- (\x@x, \y@y);
  }
  \newcommand\addoperator[3]{
    \node [draw, rectangle, fill=white] at (#3, -#2 * 0.7) {#1};
  }
  \newcommand\addbigoperator[5]{
    \pgfmathsetmacro{\y@s}{-#2 * 0.7 + 0.2};
    \pgfmathsetmacro{\y@t}{-#3 * 0.7 - 0.2};
    \pgfmathsetmacro{\x@s}{#4};
    \pgfmathsetmacro{\x@t}{#5};
    \pgfmathsetmacro{\x@c}{(\x@s + \x@t)/2};
    \pgfmathsetmacro{\y@c}{(\y@s + \y@t)/2};

    \draw [fill=white] (\x@s, \y@s) rectangle (\x@t, \y@t);
    \node at (\x@c, \y@c) {#1};
  }
  \makeatother
  \begin{center}
    \begin{tikzpicture}
      \addstate{$\bket{0}$}{0}{5};
      \addstate{$\rvdots\;$}{1}{5};
      \addstate{$\bket{0}$}{2}{5};
      \addstate{$\bket{0}$}{3}{5};
      \addoperator{$\qH$}{0}{1};
      \addoperator{$\qH$}{1}{1};
      \addoperator{$\qH$}{2}{1};
      \addoperator{$\qX$}{3}{0.6};
      \addoperator{$\qH$}{3}{1.4};
      \addbigoperator{$U_f$}{0}{3}{2}{3};
      \addoperator{$\qH$}{0}{4};
      \addoperator{$\qH$}{1}{4};
      \addoperator{$\qH$}{2}{4};

      \draw [decorate, decoration={brace, amplitude=6pt}] (-0.8, -1.7) -- (-0.8, 0.2) node [pos=0.5, left] {input\;\;};

      \node [left] at (-0.8, -2.1) {output};

      \draw [decorate, decoration={brace, amplitude=6pt}] (5.2, 0.2) -- (5.2, -1.7) node [pos=0.5, right] {\;\;measure};

      \node [right] at (5.2, -2.1) {discard};
    \end{tikzpicture}
  \end{center}
  This uses exactly one query, with $1 + (n + 1) + n + n = O(n)$ elementary gates.

  Note that for each $a \in B_n$, we can find a special balanced function $f_a$ such that
  \[
    \bket{\eta_{f_a}} = \bket{a}.
  \]
  This is the so-called BV-problem, on the first example sheet. In fact, these are given by
  \[
    f_a(x) = \sum a_i x_i.
  \]
  What if we tolerate error in the balanced vs constant problem? In other words, we only require that the answer is correct with probability $1 - \varepsilon$ with $0 < \varepsilon < \frac{1}{2}$.

  In the quantum case, nothing much changes, since we are probably not going to do better than $1$ query. However, we no longer have a huge benefit over classical algorithms. There is a classical randomized algorithm with $O(\log(1/\varepsilon))$ queries, and in particular does not depend on $n$.

  Indeed, we do it the obvious way --- we choose some $K$ $x$-values uniformly at random from $B_n$, say $x_1, \cdots, x_n$ (where $K$ is fixed and determined later). We then evaluate $f(x_1), \cdots, f(x_n)$.

  If all the outputs are the same, then we say $f$ is constant. If they are not the same, then we say $f$ is balanced.

  If $f$ actually is constant, then the answer is correct with probability $1$. If $f$ is balanced, then each $f(x_i)$ is $0$ or $1$ with equal probability. So the probability of getting the same values for all $x_i$ is
  \[
    \frac{2}{2^K} = 2^{1 - K}.
  \]
  This is our failure probability. So if we pick
  \[
    K > \log_2 (\varepsilon^{-1}) + 1,
  \]
  then we have a failure probability of less than $\varepsilon$.
\end{eg}

Can we decide \emph{every} yes/no question about $f: B_n \to B_1$'s by quantum algorithms with ``a few'' queries? The answer is no. One prominent example is the \term{SAT problem} (\term{satisfiability problem}) --- given $f$, is there an $x$ such that $f(x) = 1$? This is an NP-complete problem. It can be shown that any quantum algorithm (even with a probability $1 - \varepsilon$) needs at least $O(\sqrt{2^n})$, which is achieved by Grover's algorithm. Classically, we need $O(2^n)$ queries, so we have achieved a square root speedup, but not as good as the Deutsch-Jozsa algorithm.

In any case, the Deutsch-Jozsa algorithm demonstrates how we can achieve an exponential benefit with quantum algorithms, but it happens only when we have no error tolerance. In real life scenario, external factors will lead to potential errors anyway, and requiring that we are always correct is not a sensible requirement.

\begin{eg}[Simon's algorithm]\index{Simon's algorithm}
  The \term{Simon's problem} is a promise problem about $f: B_n \to B_n$ with provably exponential separation between claissical ($O(2^{n/4})$) and quantum ($O(n)$) query complexity. The details are on the first example sheet.
\end{eg}

So these are nice examples of benefits of quantum algorithms. However, oracles are rather unnatural problems --- it is rare to just have a black-box access to a function without knowing anything else about the function.

How about more ``normal'' problems? The issue with trying to compare quantum and classical algorithms for ``normal'' problems is that we don't actually have any method to find the lower bound for the computation complexity. For example, while we have not managed to find polynomial prime factorization algorithms, we cannot prove for sure that there isn't any classical algorithm that is polynomial time. However, for the prime factorization problem, we \emph{do} have a quantum algorithm that does much better than all known classical algorithms. This is \emph{Shor's algorithm}, which relies on the toolkit of the quantum Fourier transform.

\section{Quantum Fourier transform and periodicities}

\begin{defi}[Quantum Fourier transform mod $N$]\index{quantum Fourier transform}
  Suppose we have an $N$-dimensional state space with basis $\bket{0}, \bket{1}, \cdots, \bket{N - 1}$ labelled by $\Z/N\Z$. The \emph{quantum Fourier transform mod $N$} is defined by
  \[
    \qQFT: \bket{a} \mapsto \frac{1}{\sqrt{N}} \sum_{b = 0}^{N - 1} e^{2\pi i ab/N} \bket{b}.
  \]
  The matrix entries are
  \[
    [\qQFT]_{ab} = \frac{1}{\sqrt{N}} \omega^{ab},\quad \omega = e^{2\pi i/N},
  \]
  where $a, b = 0, 1, \cdots, N - 1$. We write \term{$\qQFT_n$} for the quantum Fourier transform mod $n$.
\end{defi}
Note that we start counting at $0$, not $1$.

We observe that the matrix $\sqrt{N}\qQFT$ is
\begin{enumerate}
  \item Symmetric
  \item The first (ie. $0$th) row and column are all $1$'s.
  \item Each row and column is a geometric progression $1, r, r^2, \cdots, r^{n - 1}$, where $r = \omega^k$ for the $k$th row or column.
\end{enumerate}

\begin{eg}
  If we look at $\qQFT_2$, then we get our good old $\qH$. However, $\qQFT_4$ is not $H \otimes H$.
\end{eg}

\begin{prop}
  $\qQFT$ is unitary.
\end{prop}

\begin{proof}
  We use the fact that
  \[
    1 + r + \cdots + r^{N - 1} =
    \begin{cases}
      \frac{1 - r^N}{1 - r} & r \not= 1\\
      N & r = 1
    \end{cases}.
  \]
  So if $r = \omega^k$, then we get
  \[
    1 + r + \cdots + r^{N - 1} =
    \begin{cases}
      0 & k \not\equiv 0 \mod N\\
      N & k \equiv 0 \mod N
    \end{cases}.
  \]
  Then we have
  \[
    (\qQFT^\dagger \qQFT)_{ij} = \frac{1}{\sqrt{N}^2} \sum_k \omega^{-ik} \omega^{jk} =\frac{1}{N} \sum_k \omega^{(j-i)k} =
    \begin{cases}
      1 & i = j\\
      0 & i \not= j
    \end{cases}.
  \]
\end{proof}

Now let's consider the periodicity problem.
\begin{eg}\index{Periodicity problem}
  Suppose we are given $f: \Z/N\Z \to Y$. We are promised that $f$ is periodic with some period $r \mid N$, so that
  \[
    f(x + r) = f(x)
  \]
  for all $x$. We also assume that $f$ is injective in each period, so that
  \[
    0 \leq x_1 \not= x_2 \leq r - 1\quad\text{ implies }\quad f(x_1) \not= f(x_2).
  \]
  The problem is to find $r$, with any constant level of error $1 - \varepsilon$ independent of $N$. Since this is not a decision problem, we can allow $\varepsilon > \frac{1}{2}$.

  In the classical setting, if $f$ is viewed as an oracle, then $O(\sqrt{N})$ queries are necessary and sufficient. We are going to show that quantumly, $O(\log \log N)$ queries with $O(\mathrm{poly}(\log N))$ processing steps suffice. In later applications, we will see that the relevant input size is $\log N$, not $N$. So the classical algorithm is exponential time, while the quantum algorithm is polynomial time.

  Note that even if $f$ is given by a simple number-theoretic formula, we still often do not have good classical algorithms to find the period.

  The quantum algorithm is given as follows:
  \begin{enumerate}
    \item Make $\frac{1}{\sqrt{N}} \sum_{x = 0}^{N - 1} \bket{x}$. For example, if $N = 2^n$, then we can make this using $\qH \otimes \cdots \otimes \qH$. If $N$ is not a power of $2$, it is not immediately obvious how we can make this state, but we will discuss this problem later.

    \item We make one query to get
      \[
        \bket{f} = \frac{1}{\sqrt{N}} \sum \bket{x} \bket{f(x)}.
      \]

    \item We now recall that $r \mid N$, Write $N = Ar$, so that $A$ is the number of periods. We measure the second register, and we will see some $y = f(x_0)$ with $x_0$ being the \emph{least} $x$ with $f(x) = y$, ie. it is in the first period. Note that we don't know what $x_0$ is. We just know what $y$ is.

      By periodicity, we know there are exactly $A$ values of $x$ such that $f(x) = y$, namely
      \[
        x_0, x_0 + r, x_0 + 2r, \cdots, x_0 + (A - 1)r.
      \]
      By the Born rule, the first register is collapsed to
      \[
        \bket{\mathrm{per}} = \left(\frac{1}{\sqrt{A}} \sum_{j = 0}^{A - 1} \bket{x_0 + jr}\right) \bket{f(x_0)}.
      \]
      We throw the second register away. Note that $x_0$ is chosen randomly from the first period $0, 1, \cdots, r - 1$.

      What do we do next? If we measure $\bket{\mathrm{per}}$, we obtain a random $j$-value, so what we actually get is a random element ($x_0$th) of a random period ($j$th), namely a uniformly chosen random number in $0, 1, \cdots, N$. This is not too useful.

      The solution is the use the quantum Fourier transform, which is not surprising, since Fourier transforms are classically used to extract periodicity information.
  \end{enumerate}
\end{eg}


\end{document}
