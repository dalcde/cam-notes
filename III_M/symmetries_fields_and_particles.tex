\documentclass[a4paper]{article}

\def\npart {III}
\def\nterm {Michaelmas}
\def\nyear {2016}
\def\nlecturer {N. Dorey}
\def\ncourse {Symmetries, Fields and Particles}
\def\nlectures {TTS.10}

% Imports
\ifx \nextra \undefined
  \usepackage[pdftex,
    hidelinks,
    pdfauthor={Dexter Chua},
    pdfsubject={Cambridge Maths Notes: Part \npart\ - \ncourse},
    pdftitle={Part \npart\ - \ncourse},
  pdfkeywords={Cambridge Mathematics Maths Math \npart\ \nterm\ \nyear\ \ncourse}]{hyperref}
  \title{Part \npart\ - \ncourse}
\else
  \usepackage[pdftex,
    hidelinks,
    pdfauthor={Dexter Chua},
    pdfsubject={Cambridge Maths Notes: Part \npart\ - \ncourse\ (\nextra)},
    pdftitle={Part \npart\ - \ncourse\ (\nextra)},
  pdfkeywords={Cambridge Mathematics Maths Math \npart\ \nterm\ \nyear\ \ncourse\ \nextra}]{hyperref}

  \title{Part \npart\ - \ncourse \\ {\Large \nextra}}
\fi

\author{Lectured by \nlecturer \\\small Notes taken by Dexter Chua}
\date{\nterm\ \nyear}

\usepackage{alltt}
\usepackage{amsfonts}
\usepackage{amsmath}
\usepackage{amssymb}
\usepackage{amsthm}
\usepackage{booktabs}
\usepackage{caption}
\usepackage{enumitem}
\usepackage{fancyhdr}
\usepackage{graphicx}
\usepackage{mathtools}
\usepackage{microtype}
\usepackage{multirow}
\usepackage{pdflscape}
\usepackage{pgfplots}
\usepackage{siunitx}
\usepackage{tabularx}
\usepackage{tikz}
\usepackage{tkz-euclide}
\usepackage[normalem]{ulem}
\usepackage[all]{xy}

\pgfplotsset{compat=1.12}

\pagestyle{fancyplain}
\lhead{\emph{\nouppercase{\leftmark}}}
\ifx \nextra \undefined
  \rhead{
    \ifnum\thepage=1
    \else
      \npart\ \ncourse
    \fi}
\else
  \rhead{
    \ifnum\thepage=1
    \else
      \npart\ \ncourse\ (\nextra)
    \fi}
\fi
\usetikzlibrary{arrows}
\usetikzlibrary{decorations.markings}
\usetikzlibrary{decorations.pathmorphing}
\usetikzlibrary{positioning}
\usetikzlibrary{fadings}
\usetikzlibrary{intersections}
\usetikzlibrary{cd}

\newcommand*{\Cdot}{\raisebox{-0.25ex}{\scalebox{1.5}{$\cdot$}}}
\newcommand {\pd}[2][ ]{
  \ifx #1 { }
    \frac{\partial}{\partial #2}
  \else
    \frac{\partial^{#1}}{\partial #2^{#1}}
  \fi
}

% Theorems
\theoremstyle{definition}
\newtheorem*{aim}{Aim}
\newtheorem*{axiom}{Axiom}
\newtheorem*{claim}{Claim}
\newtheorem*{cor}{Corollary}
\newtheorem*{defi}{Definition}
\newtheorem*{eg}{Example}
\newtheorem*{fact}{Fact}
\newtheorem*{law}{Law}
\newtheorem*{lemma}{Lemma}
\newtheorem*{notation}{Notation}
\newtheorem*{prop}{Proposition}
\newtheorem*{thm}{Theorem}

\renewcommand{\labelitemi}{--}
\renewcommand{\labelitemii}{$\circ$}
\renewcommand{\labelenumi}{(\roman{*})}

\let\stdsection\section
\renewcommand\section{\newpage\stdsection}

% Strike through
\def\st{\bgroup \ULdepth=-.55ex \ULset}

% Maths symbols
\newcommand{\bra}{\langle}
\newcommand{\ket}{\rangle}

\newcommand{\N}{\mathbb{N}}
\newcommand{\Z}{\mathbb{Z}}
\newcommand{\Q}{\mathbb{Q}}
\renewcommand{\H}{\mathbb{H}}
\newcommand{\R}{\mathbb{R}}
\newcommand{\C}{\mathbb{C}}
\newcommand{\Prob}{\mathbb{P}}
\renewcommand{\P}{\mathbb{P}}
\newcommand{\E}{\mathbb{E}}
\newcommand{\F}{\mathbb{F}}
\newcommand{\cU}{\mathcal{U}}
\newcommand{\RP}{\mathbb{RP}}
\newcommand{\CP}{\mathbb{CP}}

\newcommand{\ph}{\,\cdot\,}

\DeclareMathOperator{\sech}{sech}
\DeclareMathOperator{\cosech}{cosech}
\DeclareMathOperator{\cosec}{cosec}

\DeclareMathOperator{\covol}{covol}
\DeclareMathOperator{\vol}{vol}

\let\Im\relax
\let\Re\relax
\DeclareMathOperator{\Im}{Im}
\DeclareMathOperator{\Re}{Re}
\DeclareMathOperator{\im}{im}
\DeclareMathOperator{\image}{image}
\DeclareMathOperator{\Ann}{Ann}

\DeclareMathOperator*{\res}{res}
\DeclareMathOperator{\Res}{Res}
\DeclareMathOperator{\Ind}{Ind}

\DeclareMathOperator{\tr}{tr}
\DeclareMathOperator{\diag}{diag}
\DeclareMathOperator{\rank}{rank}
\DeclareMathOperator{\card}{card}
\DeclareMathOperator{\spn}{span}
\DeclareMathOperator{\adj}{adj}

\DeclareMathOperator{\erf}{erf}
\DeclareMathOperator{\erfc}{erfc}

\DeclareMathOperator{\ord}{ord}
\DeclareMathOperator{\Sym}{Sym}

\DeclareMathOperator{\sgn}{sgn}
\DeclareMathOperator{\orb}{orb}
\DeclareMathOperator{\stab}{stab}
\DeclareMathOperator{\ccl}{ccl}

\DeclareMathOperator{\lcm}{lcm}
\DeclareMathOperator{\hcf}{hcf}

\DeclareMathOperator{\Int}{Int}
\DeclareMathOperator{\id}{id}

\DeclareMathOperator{\betaD}{beta}
\DeclareMathOperator{\gammaD}{gamma}
\DeclareMathOperator{\Poisson}{Poisson}
\DeclareMathOperator{\binomial}{binomial}
\DeclareMathOperator{\multinomial}{multinomial}
\DeclareMathOperator{\Bernoulli}{Bernoulli}
\DeclareMathOperator{\like}{like}

\DeclareMathOperator{\var}{var}
\DeclareMathOperator{\cov}{cov}
\DeclareMathOperator{\bias}{bias}
\DeclareMathOperator{\mse}{mse}
\DeclareMathOperator{\corr}{corr}

\DeclareMathOperator{\otp}{otp}
\DeclareMathOperator{\dom}{dom}

\DeclareMathOperator{\Root}{Root}
\DeclareMathOperator{\supp}{supp}
\DeclareMathOperator{\rel}{rel}
\DeclareMathOperator{\Hom}{Hom}
\DeclareMathOperator{\Aut}{Aut}
\DeclareMathOperator{\Gal}{Gal}
\DeclareMathOperator{\Mat}{Mat}
\DeclareMathOperator{\End}{End}
\DeclareMathOperator{\Char}{char}
\DeclareMathOperator{\ev}{ev}
\DeclareMathOperator{\St}{St}
\DeclareMathOperator{\Lk}{Lk}
\DeclareMathOperator{\disc}{disc}
\DeclareMathOperator{\Isom}{Isom}
\DeclareMathOperator{\length}{length}
\DeclareMathOperator{\energy}{energy}
\DeclareMathOperator{\area}{area}
\DeclareMathOperator{\Syl}{Syl}
\DeclareMathOperator{\cl}{cl}
\DeclareMathOperator{\fix}{fix}

\newcommand{\GL}{\mathrm{GL}}
\newcommand{\SL}{\mathrm{SL}}
\newcommand{\PGL}{\mathrm{PGL}}
\newcommand{\PSL}{\mathrm{PSL}}
\newcommand{\PSU}{\mathrm{PSU}}
\newcommand{\Or}{\mathrm{O}}
\newcommand{\SO}{\mathrm{SO}}
\newcommand{\U}{\mathrm{U}}
\newcommand{\SU}{\mathrm{SU}}

\renewcommand{\d}{\mathrm{d}}
\newcommand{\D}{\mathrm{D}}

\tikzset{->/.style = {decoration={markings,
                                  mark=at position 1 with {\arrow[scale=2]{latex'}}},
                      postaction={decorate}}}
\tikzset{<-/.style = {decoration={markings,
                                  mark=at position 0 with {\arrowreversed[scale=2]{latex'}}},
                      postaction={decorate}}}
\tikzset{<->/.style = {decoration={markings,
                                   mark=at position 0 with {\arrowreversed[scale=2]{latex'}},
                                   mark=at position 1 with {\arrow[scale=2]{latex'}}},
                       postaction={decorate}}}
\tikzset{->-/.style = {decoration={markings,
                                   mark=at position #1 with {\arrow[scale=2]{latex'}}},
                       postaction={decorate}}}
\tikzset{-<-/.style = {decoration={markings,
                                   mark=at position #1 with {\arrowreversed[scale=2]{latex'}}},
                       postaction={decorate}}}

\tikzset{circ/.style = {fill, circle, inner sep = 0, minimum size = 3}}
\tikzset{mstate/.style={circle, draw, blue, text=black, minimum width=0.7cm}}

\definecolor{mblue}{rgb}{0.2, 0.3, 0.8}
\definecolor{morange}{rgb}{1, 0.5, 0}
\definecolor{mgreen}{rgb}{0.1, 0.4, 0.2}
\definecolor{mred}{rgb}{0.5, 0, 0}

\def\drawcirculararc(#1,#2)(#3,#4)(#5,#6){%
    \pgfmathsetmacro\cA{(#1*#1+#2*#2-#3*#3-#4*#4)/2}%
    \pgfmathsetmacro\cB{(#1*#1+#2*#2-#5*#5-#6*#6)/2}%
    \pgfmathsetmacro\cy{(\cB*(#1-#3)-\cA*(#1-#5))/%
                        ((#2-#6)*(#1-#3)-(#2-#4)*(#1-#5))}%
    \pgfmathsetmacro\cx{(\cA-\cy*(#2-#4))/(#1-#3)}%
    \pgfmathsetmacro\cr{sqrt((#1-\cx)*(#1-\cx)+(#2-\cy)*(#2-\cy))}%
    \pgfmathsetmacro\cA{atan2(#2-\cy,#1-\cx)}%
    \pgfmathsetmacro\cB{atan2(#6-\cy,#5-\cx)}%
    \pgfmathparse{\cB<\cA}%
    \ifnum\pgfmathresult=1
        \pgfmathsetmacro\cB{\cB+360}%
    \fi
    \draw (#1,#2) arc (\cA:\cB:\cr);%
}
\newcommand\getCoord[3]{\newdimen{#1}\newdimen{#2}\pgfextractx{#1}{\pgfpointanchor{#3}{center}}\pgfextracty{#2}{\pgfpointanchor{#3}{center}}}

\def\Xint#1{\mathchoice
   {\XXint\displaystyle\textstyle{#1}}%
   {\XXint\textstyle\scriptstyle{#1}}%
   {\XXint\scriptstyle\scriptscriptstyle{#1}}%
   {\XXint\scriptscriptstyle\scriptscriptstyle{#1}}%
   \!\int}
\def\XXint#1#2#3{{\setbox0=\hbox{$#1{#2#3}{\int}$}
     \vcenter{\hbox{$#2#3$}}\kern-.5\wd0}}
\def\ddashint{\Xint=}
\def\dashint{\Xint-}


\begin{document}
\maketitle
{\small
\setlength{\parindent}{0em}
\setlength{\parskip}{1em}

This course introduces the theory of Lie groups and Lie algebras and their applications to high energy physics. The course begins with a brief overview of the role of symmetry in physics. After reviewing basic notions of group theory we define a Lie group as a manifold with a compatible group structure. We give the abstract definition of a Lie algebra and show that every Lie group has an associated Lie algebra corresponding to the tangent space at the identity element. Examples arising from groups of orthogonal and unitary matrices are discussed. The case of $\SU(2)$, the group of rotations in three dimensions is studied in detail. We then study the representations of Lie groups and Lie algebras. We discuss reducibility and classify the finite dimensional, irreducible representations of $\SU(2)$ and introduce the tensor product of representations. The next part of the course develops the theory of complex simple Lie algebras. We define the Killing form on a Lie algebra. We introduce the Cartan-Weyl basis and discuss the properties of roots and weights of a Lie algebra. We cover the Cartan classification of simple Lie algebras in detail. We describe the finite dimensional, irreducible representations of simple Lie algebras, illustrating the general theory for the Lie algebra of $\SU(3)$. The last part of the course discusses some physical applications. After a general discussion of symmetry in quantum mechanical systems, we review the approximate $\SU(3)$ global symmetry of the strong interactions and its consequences for the observed spectrum of hadrons. We introduce gauge symmetry and construct a gauge-invariant Lagrangian for Yang-Mills theory coupled to matter. The course ends with a brief introduction to the Standard Model of particle physics.

\subsubsection*{Pre-requisites}
Basic finite group theory, including subgroups and orbits. Special relativity and quantum theory, including orbital angular momentum theory and Pauli spin matrices. Basic ideas about manifolds, including coordinates, dimension, tangent spaces.
}
\tableofcontents

\section{Introduction}
\emph{This chapter is currently a mess.}

What is a symmetry? There are many definitions, but we are going to pick one that is relevant to physics.
\begin{defi}[Symmetry]\index{symmetry}
  A \emph{symmetry} is a transformation of the dynamical variables which leaves the physical laws invariant.
\end{defi}

When we first view symmetry, it is easy to overlook its importance.
\begin{eg}
  In Newtonian physics, we have symmetries under rotation and translation. In special relativity, we have invariance under Lorentz transformations.

  To be precise, we view rotation as an action on the coordinates of a particle:
  \[
    \mathbf{x} \in \R^3 \mapsto \mathbf{x}' = M \mathbf{x} \in \R^3.
  \]
  For a matrix $M$ to represent a rotation, it has to be a real $3 \times 3$ matrix that satisfies:
  \begin{enumerate}
    \item $MM^T = 1$, ie. it is orthogonal;
    \item $\det M = 1$, ie. it is ``special''.
  \end{enumerate}
  The fact that the equations are invariant under rotation is immediate as soon as we write down the equations of motion in vector form. For example, Newton's second law says
  \[
    \mathbf{F} = m \mathbf{a}.
  \]
  After rotation, we have
  \[
    \mathbf{F}' = m \mathbf{a}',
  \]
  where
  \[
    \mathbf{F}' = M \mathbf{F},\quad \mathbf{a} = M\mathbf{a}.
  \]
  Similarly, in special relativity, symmetry manifests itself as long as we write things in terms of 4-vectors.
\end{eg}

Mathematically, the symmetries form a \emph{group}.
\begin{defi}[Group]\index{group}
  A \emph{group} is a set $G$ of elements with a multiplication rule, obeying the axioms
  \begin{enumerate}
    \item For all $g_1, g_2 \in G$, we have $g_1 g_2 \in G$.\hfill (closure)
    \item There is a (necessarily unique) element $e \in G$ such that for all $g \in G$, we have $eg = ge = g$.\hfill (identity)
    \item For every $g \in G$, there exists some (necessarily unique) $g^{-1} \in G$ such that $gg^{-1} = g^{-1}g = e$.\hfill (inverse)
    \item For every $g_1, g_2, g_3 \in G$, we have $g_1 (g_2 g_3) = (g_1 g_2) g_3$.\hfill (associativity)
  \end{enumerate}
\end{defi}
Physically, these mean
\begin{enumerate}
  \item The composition of two symmetries is also a symmetry.
  \item ``Doing nothing'' is a symmetry.
  \item A symmetry can be ``undone''.
  \item Composing functions is always associative.
\end{enumerate}

Note that the set of elements $G$ may be finite or infinite.
\begin{defi}[Commutative/abelian group]\index{commutative group}\index{abelian group}
  A group is \emph{abelian} or \emph{commutative} if $g_1 g_2 = g_2 g_1$ for all $g_1, g_2 \in G$. A group is \emph{non-abelian} if it is not abelian.
\end{defi}
Abelian groups are typically relatively dull.

One interesting thing about the rotation group in 3 dimensions, namely, $\SO(3)$ is that a rotation depends \emph{continuous} on 3 real parameters --- we pick a unit vector $\mathbf{n} \in S^2$, and an angle $\theta = [0, \pi]$. Moreover, such a dependence is smooth. We say they form \emph{manifolds}, and we call such groups \emph{Lie groups}.

\begin{defi}[Manifold]\index{manifold}
  A \emph{manifold} is a (topological) space that looks locally like $\R^n$, by may be globally non-trivial.
\end{defi}

Since this is not a course on differential topology, we will talk about manifolds rather informally.

\begin{defi}[Lie group]\index{Lie group}
  A \emph{Lie group} is a group that is also a manifold, where the group structure is compatible with the smooth structure. More precisely, multiplication and inverse has to be a smooth map. In \emph{really} fancy language, we say a Lie group is a group object in the category of smooth manifolds.
\end{defi}

It turns out this requirement is hugely restrictive. Since multiplying by the inverse sends any element to the identity, the Lie group is (almost!) completely determined by the behaviour of group multiplication and inversion near the identity element, ie. by the ``infinitesimal transformations''. Mathematically, these correspond to vectors in the tangent space to $G$ at $e$, written $T_e(G)$. The tangent space $T_e(G)$ is equipped with a \emph{Lie bracket}\index{Lie bracket}
\begin{align*}
  T_e(G) \times T_e(G) &\to T_e(G)\\
  \mathbf{v}_1, \mathbf{v}_2 \mapsto [\mathbf{v}_1, \mathbf{v}_2]
\end{align*}
This gives $T_e(G)$ the structure of a \emph{Lie algebra}\index{Lie algebra}, written $\mathcal{L}(G)$ or $\mathfrak{g}$.

Classifying Lie groups reduce (almost) to classifying lie algebras. This leads to the Cartan classification of simple Lie algebras --- all simple finite-dimensional complex Lie algebras belong to the four infinite families $A_n$, $B_n$, $C_n$, $D_n$, where $n \in \N$, and five exceptional cases $E_6, E_7, E_8, G_2, F_4$. This Cartan classification gives us the basic building blocks for gauge theories.

That's the mathematical part of the theory. We now move on to see how we can apply these to physics.

In classic physics, the main way we meet symmetries is that by Noether's theorem, symmetries of a classical system give rise to conserved charges. For example, invariance under rotations imply the conservation of angular momentum $\mathbf{L} = (L_1, L_2, L_3)$. However, despite this fact, there is not much motivation for thinking of the group structure of the symmetries.

In quantum mechanics, instead of describing states by dynamical variables, we describe states by vectors in Hilbert spaces $|\psi\ket \in \mathcal{H}$. The observables in the system correspond to Hermitian operators on $\mathcal{H}$. Famously, these operators have non-commutative multiplication. This allows a much more direct connection with the symmetry structure. For example, the angular momentum operators $\hat{L}_1, \hat{L}_2, \hat{L}_3$ satisfy
\[
  [\hat{L}_i, \hat{L}_j] = i \varepsilon_{ijk} \hat{L}_k.
\]
This commutation relation exactly describes the Lie algebra of $\SO(3)$, written $\mathcal{L}(\SO(3))$, where the commutator takes the role of the Lie bracket.

Angular momentum operators often act on finite-dimensional spaces, of which the canonical example is the spin of the electron, in which case $\mathcal{H} = \C^2$, corresponding to
\[
  |\uparrow\ket =
  \begin{pmatrix}
    1\\0
  \end{pmatrix}
  ,\quad |\downarrow\ket =
  \begin{pmatrix}
    0\\1
  \end{pmatrix}.
\]
When the Lie algebra acts on $\mathcal{H}$, we can represent the operators $\hat{L}_i$ as matrices $\Sigma_i$, where
\[
  [\Sigma_i, \Sigma_j] = i \varepsilon_{ijk} \Sigma_k.
\]
This gives a \emph{representation} of $\mathcal{L}(\SO(3))$ (or $\so(3)$).

In quantum mechanics, the statement that we have a rotational symmetry is the statement that
\[
  [\hat{H}, \hat{L}_i] = 0,
\]
ie. that $\hat{H}$ and $\hat{L}_i$ commute for all $i$. This in particular means that acting with $\hat{L}_i$ preserves the energy of the state. In particular, this means that $|\uparrow\ket$ and $|\downarrow\ket$ have the same energy.

% something something eightfold way, Gell-Mann, approximate symmetries based on the Lie group SU(3). This is an eight-dimensional representation of su(3).

Previously, we have been talking about global symmetries. These are symmetry actions that act directly on the spacetime coordinates. We have the familiar symmetries such as the rotation group, Lorentz group, Poincare group, and in more modern developments, we try to find supersymmetry. Apart from these, we also have some internal symmetries, in the sense that they don't act on the spacetime coordinates, but act on the field describing the particles. For example, these give rise to electric charges and flavor.

On the other hand, we also have gauge symmetry\index{gauge symmetry}. This is more subtle, since it does not act on the physical dynamical variables. They act on things we cannot physically measure. These are ``redundancies'' in our mathematical description of physics.

\begin{eg}
  The phase of a wavefunction in quantum mechanics is a gauge symmetry. If we act by $\psi \mapsto e^{i\delta}\psi$, then we get a different wavefunction, but it describes the same physics, and we cannot detect it.
\end{eg}

\begin{eg}
  In electromagnetism, we have a magnetic potential $\mathbf{A}$. We can change it by any transformation of the form $\mathbf{A} \mapsto \mathbf{A} + \nabla \chi$, for $\chi$ an arbitrary real-valued function, and the magnetic field $\mathbf{B} = \nabla \times \mathbf{A}$ is not changed.
\end{eg}

Although the gauge symmetries seem to be ``redundancies'', they are indeed very important. In fact, the standard model of particle physics is a gauge theory. It is given by specifying a Lie group as the group of gauge symmetries ($\SU(3) \times \SU(2) \times U(1)$), and perform the same constructions as we do to construct electromagnetism.

\section{Lie groups}
Recall that a \term{Lie group} is a group and a manifold, where the group operations define smooth maps. We will start by making these concepts a bit more precise, but we will not go into full formality.

\begin{defi}[Manifold]\index{manifold} % Improve
  A \emph{manifold} is a topological space that is locally diffeomorphic to $\R^n$ for some fixed $n$. We call $n$ the \term{dimension}.
\end{defi}

\begin{defi}[Lie group]\index{Lie group}
  A \emph{Lie group} is a group $G$ whose underlying set is given a manifold structure, and such that the multiplication map $m: G \times G \to G$ is a smooth map. We sometimes write $\mathcal{M}(G)$ for the underlying manifold of $G$.
\end{defi}

\begin{eg}
  The unit $2$-sphere
  \[
    S^2 = \{(x, y, z) \in \R^3: x^2 + y^2 + z^2 = 1\}
  \]
  is a manifold. Indeed, we can construct a coordinate patch near $N = (0, 0, 1)$. Near this point, we have $z = \sqrt{1 - x^2 - y^2}$. This works, since near the north pole, the $z$-coordinate is always positive. In this case, $x$ and $y$ are good coordinates near the north pole.
\end{eg}

In general, most of our manifolds will be given by subsets of Euclidean space specified by certain equations.

\begin{defi}[Dimension of Lie group]\index{Dimension of Lie group}\index{Lie group, dimension}
  The \emph{dimension} of a Lie group $G$ is the dimension of the underlying manifold.
\end{defi}

Given a Lie group $G$, we can introduce coordinates $\boldsymbol\theta = \{\theta^i\}_{i = 1, \ldots, D}$, where $D = \dim(G)$, in a patch $P$ containing $e$. We write $g(\boldsymbol\theta) \in G$ for the element of $G$ specified by the coordinates $\boldsymbol\theta$. We by convention set $g(0) = e$.

Suppose we have two elements $g(\boldsymbol\theta), g(\boldsymbol\theta') \in G$. Suppose they are small enough so that their product is also in the patch $P$. So we an write
\[
  g(\boldsymbol\theta) g(\boldsymbol\theta') = g(\boldsymbol\varphi)
\]
This corresponds to a smooth map $G \times G \to G$. In coordinates, we can write this as
\[
  \varphi^i = \varphi^i(\boldsymbol\theta, \boldsymbol\theta').
\]
This map is smooth.

Similarly, group inversion also defines a smooth map.

\begin{eg}
  Let $G = (\R^D, +)$ be the $D$-dimensional Euclidean space with addition as the group operation. The inverse of a vector $\mathbf{x}$ is $-\mathbf{x}$, and the identity is $\mathbf{0}$. This is obviously locally homeomorphic to $\R^D$, since it \emph{is} $\R^D$, and addition and negation are obviously smooth.
\end{eg}

This is a rather boring example, since $\R^D$ is a rather trivial manifold, and the operation is commutative.

\begin{eg}[Matrix groups]\index{matrix group}
  Let $M_n(\F)$ denote the set of $n\times n$ matrices with entries in a field $\F$ (usually $\R$ or $\C$). Matrix multiplication is certainly associative, and has an identity, namely the identity. However, it doesn't always have inverses --- not all matrices are invertible! So this is not a group. (Instead, we call it a monoid)
\end{eg}
Thus, we are lead to consider the \emph{general linear group}:

\begin{defi}[General linear group]\index{general linear group}
  The \emph{general linear group} is
  \[
    \GL(n, \F) = \{M \in \Mat_n(\F): \det M \not= 0\}.
  \]
\end{defi}
This is closed under multiplication since the determinant is multiplicative, and matrices with non-zero determinant are invertible.

\begin{defi}[Special linear group]\index{special linear group}
  The \emph{special linear group} is
  \[
    \SL(n, \F) = \{M \in \Mat_n(\F): \det M =1\} \leq \GL(n, \F).
  \]
\end{defi}
While these are obviously groups, less obviously, these are in fact Lie groups!

\begin{eg}
  Explicitly, we have
  \[
    \SL(2, \R) = \left\{
      \begin{pmatrix}
        a & b\\
        c & d
      \end{pmatrix}:
    a, b, c, d \in \R, ad - bc = 1\right\}>
  \]
  The identity is the matrix with $a = d = 1$ and $b = c = 0$. For $a \not= 0$, we have
  \[
    d = \frac{1 + bc}{a}.
  \]
  This gives us a coordinate patch for all points where $a \not= 0$ in terms of $b, c, a$, which, in particular, contains the origin $e$. By considering the case where $b \not= 0$, we obtain a separate coordinate chart, and these together cover all of $\SL(2, \R)$, as a matrix in $\SL(2, \R)$ cannot have $a = b = 0$.

  Thus, we see that $\SL(2, \R)$ has dimension $3$.
\end{eg}

In general, by a similar counting argument, we have
\begin{align*}
  \dim (\SL(n, \R)) &= n^2 - 1& \dim (\SL(n, \C)) &= 2n^2 - 2\\
  \dim (\GL(n, \R)) &= n^2,& \dim(\GL(n, \C)) &= 2n^2.
\end{align*}

\begin{defi}[Subgroup]\index{subgroup}
  A \emph{subgroup} $H$ of $G$ is a subset of $G$ that is also a group under the same operations.
\end{defi}

The really interesting is when the subgroup is also a manifold!
\begin{defi}[Lie subgroup]\index{Lie subgroup}
  A subgroup is a \emph{Lie subgroup} if it is also a manifold (under the induced smooth structure).
\end{defi}

\subsection{Subgroups of \texorpdfstring{$\GL(n, R)$}{GL(n, R)}}
\begin{lemma}
  The \term{general linear group}\index{$\GL(n, \R)$}:
  \[
    \GL(n, \R) = \{M \in M_n(\R): \det M \not= 0\}
  \]
  and \term{orthogonal group}\index{$\Or(n)$}:
  \[
    \Or(n) = \{M \in \GL(n, \R): M^TM = 1\}
  \]
  are Lie groups.
\end{lemma}
Note that we write $\Or(n)$ instead of $\Or(n, \R)$ since orthogonal matrices make sense only when talking about real matrices.

The orthogonal matrices are those that preserve the lengths of vectors. Indeed, for $\mathbf{v}\in \R^n$, we have
\[
  |M\mathbf{v}|^2 = \mathbf{v}^T M^T M \mathbf{v} = \mathbf{v}^T \mathbf{v} = |\mathbf{v}|^2.
\]
We notice something interesting. If $M \in \Or(n)$, we have
\[
  1 = \det(I) = \det(M^TM) = \det(M)^2.
\]
So $\det(M) = \pm 1$. Now $\det$ is a continuous function, and it is easy to see that $\det$ takes both $\pm 1$. So $\Or(n)$ has (at least) two connected components. Only one of these pieces contain the identity, namely the piece $\det M = 1$. We might expect this to be a group on its own right, and indeed it is, because $\det$ is multiplicative.
\begin{lemma}
  The \term{special orthogonal group}\term{$\SO(n)$}:
  \[
    \SO(n) = \{M \in \Or(n): \det M = 1\}
  \]
  is a Lie group.
\end{lemma}

Given a frame $\{\mathbf{v}_1, \cdots, \mathbf{v}_n\}$ in $\R^n$ (ie. an ordered basis), any orthogonal matrix $M \in \Or(n)$ acts on it to give another frame $\mathbf{v}_a \in \R^n \mapsto \mathbf{v}_a' \mapsto M\mathbf{v}_a \in \R^n$.
\begin{defi}[Volume element]\index{volume element}
  Given a frame $\{\mathbf{v}_1, \cdots, \mathbf{v}_n\}$ in $\R^n$, the \emph{volume element} is
  \[
    \Omega = \varepsilon_{i_1 \ldots i_n} v_1^{i_1} v_2^{i_2} \cdots v_n^{i_n}.
  \]
\end{defi}

By direct computation, we see that an orthogonal matrix preserves the sign of the volume element iff its determinant is $+1$, ie. $M \in \SO(n)$.

\begin{defi}[Eigenvalue]\index{eigenvalue}
  A complex number $\lambda$ is an eigenvalue of $M \in M(n)$ if there is some (possibly complex) vector $\mathbf{v}_\lambda \not= 0$ such that
  \[
    M \mathbf{v}_\lambda = \lambda \mathbf{v}_\lambda.
  \]
\end{defi}

\begin{thm}
  Let $M$ be an orthogonal matrix. Then $\lambda$ is an eigenvalue iff $\lambda^*$ is an eigenvalue. Moreover, if $\lambda$ is an eigenvalue, then $|\lambda|^2 = 1$.
\end{thm}

\begin{proof}
  Suppose $M \mathbf{v}_\lambda = \lambda \mathbf{v}_\lambda$. Then applying the complex conjugate gives $M \mathbf{v}_\lambda^* = \lambda^* \mathbf{v}_\lambda^*$.

  Now suppose $\lambda$ is an eigenvalue. Then $M\mathbf{v}_\lambda = \lambda \mathbf{v}_\lambda$ for some non-zero $\mathbf{v}_\lambda$. We take the norm to obtain $|M\mathbf{v}_\lambda| = |\lambda| |\mathbf{v}_\lambda|$. Using the fact that $|\mathbf{v}_\lambda| = |M\mathbf{v}_\lambda|$, we have $|\lambda| = 1$. So done.
\end{proof}

\begin{eg}
  Let $M \in \SO(2)$. Since $\det M = 1$, the eigenvalues must be of the form $\lambda = e^{i\theta}, e^{-i\theta}$. In this case, we have
  \[
    M = M(\theta) =
    \begin{pmatrix}
      \cos \theta & -\sin \theta\\
      \sin \theta & \cos \theta
    \end{pmatrix},
  \]
  where $\theta$ is the rotation angle in $S^1$. Here we have
  \[
    M(\theta_1)M(\theta_2) = M(\theta_2) M(\theta_1) = M(\theta_1 + \theta_2).
  \]
  So we have $\mathcal{M}(\SO(2)) = S^1$.
\end{eg}

\begin{eg}
  Consider $G = \SO(3)$. Suppose $M \in \SO(3)$. Since $\det M = +1$, and the eigenvalues have to come in complex conjugate pairs, we know one of them must be $1$. Then the other two must be of the form $e^{i\theta}, e^{-i\theta}$, where $\theta \in S^1$.

  We pick a normalized eigenvector $\mathbf{n}$ for $\lambda = 1$. Then $M\mathbf{n} = \mathbf{n}$, and $\mathbf{n} \cdot \mathbf{n} = 1$. This is known as the \emph{axis of rotation}. Similarly, $\theta$ is the angle of rotation. We write $M(\mathbf{n}, \theta)$ for this matrix, and it turns out this is
  \[
    M(\mathbf{n}, \theta)_{ij} = \cos \theta \delta_{ij} + (1 - \cos \theta)n_i n_j - \sin \theta \varepsilon_{ijk} n_k.
  \]
  Note that this does not uniquely specify a matrix. We have
  \[
    M(\mathbf{n}, 2\pi - \theta) = M(-\mathbf{n}, \theta).
  \]
  Thus, to uniquely specify a matrix, we need to restrict the range of $\theta$ to $0 \leq \theta \leq \pi$, with the further identification that
  \[
    (\mathbf{n}, \pi) \sim (-\mathbf{n}, -\pi).
  \]
  Also note that $M(\mathbf{n}, 0) = I$ for any $\mathbf{n}$.

  Consider a vector $\mathbf{w} = \theta \mathbf{n}$. Then there is a single value of $\mathbf{w}$ that corresponds to the identity matrix. If we require $\mathbf{w}$ to lie in the region
  \[
    B_3 = \{\mathbf{w} \in \R^3: \norm{\mathbf{w}} \leq \pi\} \subseteq \R^3.
  \]
  This has a boundary
  \[
    \partial B_3 = \{\mathbf{w} \in \R^3: \norm{\mathbf{w}} = \pi\} \cong S^2.
  \]
  Now we identify antipodal points on $\partial B_3$.

  The resulting manifold is \emph{compact} (ie. ``closed and bounded subset of $\R^n$''), and connected.
\end{eg}
For completeness, we provide a formal, but not too intuitive definition of compactness.
\begin{defi}[Compact]\index{compact}
  A manifold $X$ (or topological space) is \emph{compact} if every open cover of $X$ has a finite subcover.
\end{defi}

\begin{eg}
  The sphere $S^2$ is compact, but the hyperboloid given by $x^2 - y^2 - z^2 = 1$ (as a subset of $\R^3$) is not.
\end{eg}

\begin{defi}[Simply connected]\index{simply connected}
  A manifold $M$ if every loop $l: S^1 \to M$ can be contracted to a point.
\end{defi}

\begin{eg}
  The $2$-sphere $S^2$ is simply connected, but the torus is not. $\SO(3)$ is also not simply connected. We can define the map by
  \[
    l(\theta) =
    \begin{cases}
      \theta \mathbf{n} & \theta \in [0, \pi)\\
      -(2\pi - \theta) \mathbf{n} & \theta \in [\pi, 2\pi)
    \end{cases}
  \]
  This works precisely because we identify antipodal points.
\end{eg}

The failure of simply-connectedness is measured by the \emph{fundamental group}.
\begin{defi}[Fundamental group/First homotopy group]\index{fundamental group}\index{homotopy group}\index{first homotopy group}
  Let $M$ be a manifold, and $x_0 \in M$ be a preferred point. We define $\pi_1(M)$ to be the equivalence classes of loops starting and ending at $x_0$, where two loops are considered equivalent if they can be continuously deformed to the other.

  This has a group structure, with the identity given by the ``loop'' that stays at $x_0$ all the time, and composition given by doing one after the other.
\end{defi}

\begin{eg}
  $\pi_1(S^2) = \emptyset$ and $\pi_1(T^2) = \Z \times \Z$. We also have $\pi_1(\SO(2)) = \Z/2\Z$.
\end{eg}

\end{document}
