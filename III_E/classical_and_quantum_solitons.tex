\documentclass[a4paper]{article}

\def\npart {III}
\def\nterm {Easter}
\def\nyear {2017}
\def\nlecturer {N.\ S.\ Manton}
\def\ncourse {Classical and Quantum Solitons}

% Imports
\ifx \nextra \undefined
  \usepackage[pdftex,
    hidelinks,
    pdfauthor={Dexter Chua},
    pdfsubject={Cambridge Maths Notes: Part \npart\ - \ncourse},
    pdftitle={Part \npart\ - \ncourse},
  pdfkeywords={Cambridge Mathematics Maths Math \npart\ \nterm\ \nyear\ \ncourse}]{hyperref}
  \title{Part \npart\ - \ncourse}
\else
  \usepackage[pdftex,
    hidelinks,
    pdfauthor={Dexter Chua},
    pdfsubject={Cambridge Maths Notes: Part \npart\ - \ncourse\ (\nextra)},
    pdftitle={Part \npart\ - \ncourse\ (\nextra)},
  pdfkeywords={Cambridge Mathematics Maths Math \npart\ \nterm\ \nyear\ \ncourse\ \nextra}]{hyperref}

  \title{Part \npart\ - \ncourse \\ {\Large \nextra}}
\fi

\author{Lectured by \nlecturer \\\small Notes taken by Dexter Chua}
\date{\nterm\ \nyear}

\usepackage{alltt}
\usepackage{amsfonts}
\usepackage{amsmath}
\usepackage{amssymb}
\usepackage{amsthm}
\usepackage{booktabs}
\usepackage{caption}
\usepackage{enumitem}
\usepackage{fancyhdr}
\usepackage{graphicx}
\usepackage{mathtools}
\usepackage{microtype}
\usepackage{multirow}
\usepackage{pdflscape}
\usepackage{pgfplots}
\usepackage{siunitx}
\usepackage{tabularx}
\usepackage{tikz}
\usepackage{tkz-euclide}
\usepackage[normalem]{ulem}
\usepackage[all]{xy}

\pgfplotsset{compat=1.12}

\pagestyle{fancyplain}
\lhead{\emph{\nouppercase{\leftmark}}}
\ifx \nextra \undefined
  \rhead{
    \ifnum\thepage=1
    \else
      \npart\ \ncourse
    \fi}
\else
  \rhead{
    \ifnum\thepage=1
    \else
      \npart\ \ncourse\ (\nextra)
    \fi}
\fi
\usetikzlibrary{arrows}
\usetikzlibrary{decorations.markings}
\usetikzlibrary{decorations.pathmorphing}
\usetikzlibrary{positioning}
\usetikzlibrary{fadings}
\usetikzlibrary{intersections}
\usetikzlibrary{cd}

\newcommand*{\Cdot}{\raisebox{-0.25ex}{\scalebox{1.5}{$\cdot$}}}
\newcommand {\pd}[2][ ]{
  \ifx #1 { }
    \frac{\partial}{\partial #2}
  \else
    \frac{\partial^{#1}}{\partial #2^{#1}}
  \fi
}

% Theorems
\theoremstyle{definition}
\newtheorem*{aim}{Aim}
\newtheorem*{axiom}{Axiom}
\newtheorem*{claim}{Claim}
\newtheorem*{cor}{Corollary}
\newtheorem*{defi}{Definition}
\newtheorem*{eg}{Example}
\newtheorem*{fact}{Fact}
\newtheorem*{law}{Law}
\newtheorem*{lemma}{Lemma}
\newtheorem*{notation}{Notation}
\newtheorem*{prop}{Proposition}
\newtheorem*{thm}{Theorem}

\renewcommand{\labelitemi}{--}
\renewcommand{\labelitemii}{$\circ$}
\renewcommand{\labelenumi}{(\roman{*})}

\let\stdsection\section
\renewcommand\section{\newpage\stdsection}

% Strike through
\def\st{\bgroup \ULdepth=-.55ex \ULset}

% Maths symbols
\newcommand{\bra}{\langle}
\newcommand{\ket}{\rangle}

\newcommand{\N}{\mathbb{N}}
\newcommand{\Z}{\mathbb{Z}}
\newcommand{\Q}{\mathbb{Q}}
\renewcommand{\H}{\mathbb{H}}
\newcommand{\R}{\mathbb{R}}
\newcommand{\C}{\mathbb{C}}
\newcommand{\Prob}{\mathbb{P}}
\renewcommand{\P}{\mathbb{P}}
\newcommand{\E}{\mathbb{E}}
\newcommand{\F}{\mathbb{F}}
\newcommand{\cU}{\mathcal{U}}
\newcommand{\RP}{\mathbb{RP}}
\newcommand{\CP}{\mathbb{CP}}

\newcommand{\ph}{\,\cdot\,}

\DeclareMathOperator{\sech}{sech}
\DeclareMathOperator{\cosech}{cosech}
\DeclareMathOperator{\cosec}{cosec}

\DeclareMathOperator{\covol}{covol}
\DeclareMathOperator{\vol}{vol}

\let\Im\relax
\let\Re\relax
\DeclareMathOperator{\Im}{Im}
\DeclareMathOperator{\Re}{Re}
\DeclareMathOperator{\im}{im}
\DeclareMathOperator{\image}{image}
\DeclareMathOperator{\Ann}{Ann}

\DeclareMathOperator*{\res}{res}
\DeclareMathOperator{\Res}{Res}
\DeclareMathOperator{\Ind}{Ind}

\DeclareMathOperator{\tr}{tr}
\DeclareMathOperator{\diag}{diag}
\DeclareMathOperator{\rank}{rank}
\DeclareMathOperator{\card}{card}
\DeclareMathOperator{\spn}{span}
\DeclareMathOperator{\adj}{adj}

\DeclareMathOperator{\erf}{erf}
\DeclareMathOperator{\erfc}{erfc}

\DeclareMathOperator{\ord}{ord}
\DeclareMathOperator{\Sym}{Sym}

\DeclareMathOperator{\sgn}{sgn}
\DeclareMathOperator{\orb}{orb}
\DeclareMathOperator{\stab}{stab}
\DeclareMathOperator{\ccl}{ccl}

\DeclareMathOperator{\lcm}{lcm}
\DeclareMathOperator{\hcf}{hcf}

\DeclareMathOperator{\Int}{Int}
\DeclareMathOperator{\id}{id}

\DeclareMathOperator{\betaD}{beta}
\DeclareMathOperator{\gammaD}{gamma}
\DeclareMathOperator{\Poisson}{Poisson}
\DeclareMathOperator{\binomial}{binomial}
\DeclareMathOperator{\multinomial}{multinomial}
\DeclareMathOperator{\Bernoulli}{Bernoulli}
\DeclareMathOperator{\like}{like}

\DeclareMathOperator{\var}{var}
\DeclareMathOperator{\cov}{cov}
\DeclareMathOperator{\bias}{bias}
\DeclareMathOperator{\mse}{mse}
\DeclareMathOperator{\corr}{corr}

\DeclareMathOperator{\otp}{otp}
\DeclareMathOperator{\dom}{dom}

\DeclareMathOperator{\Root}{Root}
\DeclareMathOperator{\supp}{supp}
\DeclareMathOperator{\rel}{rel}
\DeclareMathOperator{\Hom}{Hom}
\DeclareMathOperator{\Aut}{Aut}
\DeclareMathOperator{\Gal}{Gal}
\DeclareMathOperator{\Mat}{Mat}
\DeclareMathOperator{\End}{End}
\DeclareMathOperator{\Char}{char}
\DeclareMathOperator{\ev}{ev}
\DeclareMathOperator{\St}{St}
\DeclareMathOperator{\Lk}{Lk}
\DeclareMathOperator{\disc}{disc}
\DeclareMathOperator{\Isom}{Isom}
\DeclareMathOperator{\length}{length}
\DeclareMathOperator{\energy}{energy}
\DeclareMathOperator{\area}{area}
\DeclareMathOperator{\Syl}{Syl}
\DeclareMathOperator{\cl}{cl}
\DeclareMathOperator{\fix}{fix}

\newcommand{\GL}{\mathrm{GL}}
\newcommand{\SL}{\mathrm{SL}}
\newcommand{\PGL}{\mathrm{PGL}}
\newcommand{\PSL}{\mathrm{PSL}}
\newcommand{\PSU}{\mathrm{PSU}}
\newcommand{\Or}{\mathrm{O}}
\newcommand{\SO}{\mathrm{SO}}
\newcommand{\U}{\mathrm{U}}
\newcommand{\SU}{\mathrm{SU}}

\renewcommand{\d}{\mathrm{d}}
\newcommand{\D}{\mathrm{D}}

\tikzset{->/.style = {decoration={markings,
                                  mark=at position 1 with {\arrow[scale=2]{latex'}}},
                      postaction={decorate}}}
\tikzset{<-/.style = {decoration={markings,
                                  mark=at position 0 with {\arrowreversed[scale=2]{latex'}}},
                      postaction={decorate}}}
\tikzset{<->/.style = {decoration={markings,
                                   mark=at position 0 with {\arrowreversed[scale=2]{latex'}},
                                   mark=at position 1 with {\arrow[scale=2]{latex'}}},
                       postaction={decorate}}}
\tikzset{->-/.style = {decoration={markings,
                                   mark=at position #1 with {\arrow[scale=2]{latex'}}},
                       postaction={decorate}}}
\tikzset{-<-/.style = {decoration={markings,
                                   mark=at position #1 with {\arrowreversed[scale=2]{latex'}}},
                       postaction={decorate}}}

\tikzset{circ/.style = {fill, circle, inner sep = 0, minimum size = 3}}
\tikzset{mstate/.style={circle, draw, blue, text=black, minimum width=0.7cm}}

\definecolor{mblue}{rgb}{0.2, 0.3, 0.8}
\definecolor{morange}{rgb}{1, 0.5, 0}
\definecolor{mgreen}{rgb}{0.1, 0.4, 0.2}
\definecolor{mred}{rgb}{0.5, 0, 0}

\def\drawcirculararc(#1,#2)(#3,#4)(#5,#6){%
    \pgfmathsetmacro\cA{(#1*#1+#2*#2-#3*#3-#4*#4)/2}%
    \pgfmathsetmacro\cB{(#1*#1+#2*#2-#5*#5-#6*#6)/2}%
    \pgfmathsetmacro\cy{(\cB*(#1-#3)-\cA*(#1-#5))/%
                        ((#2-#6)*(#1-#3)-(#2-#4)*(#1-#5))}%
    \pgfmathsetmacro\cx{(\cA-\cy*(#2-#4))/(#1-#3)}%
    \pgfmathsetmacro\cr{sqrt((#1-\cx)*(#1-\cx)+(#2-\cy)*(#2-\cy))}%
    \pgfmathsetmacro\cA{atan2(#2-\cy,#1-\cx)}%
    \pgfmathsetmacro\cB{atan2(#6-\cy,#5-\cx)}%
    \pgfmathparse{\cB<\cA}%
    \ifnum\pgfmathresult=1
        \pgfmathsetmacro\cB{\cB+360}%
    \fi
    \draw (#1,#2) arc (\cA:\cB:\cr);%
}
\newcommand\getCoord[3]{\newdimen{#1}\newdimen{#2}\pgfextractx{#1}{\pgfpointanchor{#3}{center}}\pgfextracty{#2}{\pgfpointanchor{#3}{center}}}

\def\Xint#1{\mathchoice
   {\XXint\displaystyle\textstyle{#1}}%
   {\XXint\textstyle\scriptstyle{#1}}%
   {\XXint\scriptstyle\scriptscriptstyle{#1}}%
   {\XXint\scriptscriptstyle\scriptscriptstyle{#1}}%
   \!\int}
\def\XXint#1#2#3{{\setbox0=\hbox{$#1{#2#3}{\int}$}
     \vcenter{\hbox{$#2#3$}}\kern-.5\wd0}}
\def\ddashint{\Xint=}
\def\dashint{\Xint-}


\begin{document}
\maketitle
{\small
\setlength{\parindent}{0em}
\setlength{\parskip}{1em}
Solitons are solutions of classical field equations with particle-like properties. They are localised in space, have finite energy and are stable against decay into radiation. The stability usually has a topological explanation. After quantisation, they give rise to new particle states in the underlying quantum field theory that are not seen in perturbation theory. We will focus mainly on kink solitons in one space dimension, and on Skyrmions in three dimensions. Solitons in gauge theories will also be mentioned.

\subsubsection*{Pre-requisites}
This course assumes you have taken Quantum Field Theory and Symmetries, Fields and Particles. The small amount of topology that is needed will be developed during the course.
}
\tableofcontents
\section{Introduction}
Given a classical field theory, if we want to ``quantize'' it, then we find the vacuum of the theory, and then do perturbation theory around this vacuum. If there are multiple vacua, then what we did was that we arbitrarily picked a vacuum, and then expanded around that vacuum.

However, these field theories with multiple vacua often contain \emph{soliton} solutions. These are localized, smooth solutions of the classical field equations, and they ``connect multiple vacuums''. To quantize these solitons solutions, we fix such a soliton, and use it as the ``background''. We then do perturbation theory around these solutions, but this is rather tricky to do. Thus, in a lot of the course, we will just look at the classical part of the theory.

Recall that when quantizing our field theories in perturbation theory, we obtain particles in the quantum theory, despite the classical theory being completely about fields. It turns our solitons also behave like particles, and they are a \emph{new} type of particles. These are non-perturbative phenomenon. If we want to do the quantum field theory properly, we have to include these solitons in the quantum field theory. In general this is hard, and so we are not going to develop this a lot.

What does it mean to say that solitons are like particles? In relativistic field theories, we find these solitons have a classical energy. We define the ``mass'' $M$ of the soliton to be the energy in the ``rest frame''. Since this is relativistic, we can do a Lorentz boost, and we obtain a moving soliton. Then we obtain a relation of the form
\[
  E^2 - \mathbf{P} \cdot \mathbf{P} = M^2.
\]
This is a Lorentz-invariant property of the soliton. Together with the fact that the soliton is localized, this is a justification for thinking of them as particles.

These particles differ from the particles perturbative quantum fields, as they have rather different properties. Interesting solitons have a \emph{topological} character different from the classical vacuum. Thus, at least naively, they cannot be thought of perturbatively.

There are also non-relativistic solitons, and they usually don't have interpretations of particles. These appear, for example, as defects in solids. We will not be interested in these much.

What kinds of theories have solitons? To obtain solitons, we definitely need a non-linear field structure and/or non-linear equations. Thus, free field theories with quadratic Lagrangians such as Maxwell theory do not have solitons. We need interaction terms.

Note that in QFT, we did interactions using the interaction picture. We split the Hamiltonian into a ``free field'' part, which we solve exactly, and the ``interaction'' part. However, to quantize solitons, we need to solve the full interacting Lagrangian \emph{exactly}.

Having interactions is not enough for solitons to appear. To obtain solitons, we also need some non-trivial vacuum topology. In other words, we need more than one vacuum. This usually comes from symmetry breaking, and often gauge symmetries are involved.

In this course, we will focus on three types of solitons.
\begin{itemize}
  \item In one (space) dimension, we have kinks. We will spend $4$ lectures on this.

  \item In two dimensions, we have vortices. We will spend $6$ lectures on this.

  \item In three dimensions, there are monopoles and Skyrmions. We will only study Skyrmions, and will spend $6$ lectures on it.
\end{itemize}
These examples are all relativistic. Non-relativistic solitons include \emph{domain walls}, which occur in ferromagnets, but we will not study these.

In general, solitons appear in all sorts of different actual, physical scenarios such as in condensed matter physics, optical fibers, superconductors, ``cosmic strings'' etc. Since we are mathematicians, we probably will not put much focus on these actual applications. However, we can talk a bit more about Skyrmions.

Skyrmions are solitons in an \emph{effective field theory} of interacting pions, which are thought to be the most important baryons because they are the lightest. This happens in spite of the lack of a gauge symmetry. While pions have no baryon number, the associated solitons have a topological charge identified with baryon number. This baryon number is conserved for topological reasons.

Note that in QCD, baryon number is conserved because the quark number is conserved. We tried extremely hard to find proton decay, which would be a process that involves baryon numbers, but we cannot find such examples. We have very high experimental certainty that baryon number is conserved. And if baryon number is topological, then this is a very good reason for the conservation of baryon numbers.

Skyrmions give a model of low-energy interactions of baryons. This leads to an (approximate) theory of nucleons (proton and neutron) and larger nuclei, which are bound states of any number of protons and neutrons. % This is really up-to-date stuff.

There is a whole other set of Skyrmions studied, which are two-dimensional. These are structure in exotic magnets, and they have actually been seen.

For these ideas to work out well, we need to eventually do quantization. For example, Skyrmions by themselves do not have spin. We need to quantize the theory before these come out. Also, Skyrmions cannot distinguish between protons and neutrons. These differences only come up after we quantize.
\section{\tph{$\phi^4$}{phi4}{&phi;<sup>4</sup>} kinks}
In this chapter, we are going to study \term{$\phi^4$ kinks}\index{kink}\index{kink!$\phi^4$}. This happens in $1 + 1$ dimensions, and involves a single scalar field $\phi(x, t)$. In higher dimensions, we often need many fields to obtain solitons, but in the case of 1 dimension, we can get away with a single field.

In general, the \term{Lagrangian density} of such a scalar field theory is of the form
\[
  \mathcal{L} = \frac{1}{2} \partial_\mu \phi \partial^\mu \phi - U(\phi)
\]
for some potential $U(\phi)$ polynomial in $\phi$. Note that in $1 + 1$ dimensions, any such theory is renormalizable. Here we will choose the Minkowski metric to be
\[
  \eta^{\mu\nu} = 
  \begin{pmatrix}
    1 & 0\\
    0 & -1
  \end{pmatrix}.
\]
Then the \term{Lagrangian} is given by
\[
  L = \int_{-\infty}^\infty \mathcal{L}\;\d x = \int_{-\infty}^\infty \left(\frac{1}{2} \partial_\mu \phi \partial^\mu \phi - U(\phi)\right)\;\d x,
\]
and then the \term{action} is
\[
  S[\phi] = \int L\;\d t = \int \mathcal{L} \;\d x\;\d t.
\]

There is a non-linearity in the field equations due to a potential $U(\phi)$ with \emph{multiple vacua}. We need this multiple vacua to obtain a soliton. The kink stability comes from the \emph{topology}. It is very simple here, and just comes from counting the discrete, distinct vacua.

As usual, we will write
\[
  \dot{\phi} = \frac{\partial \phi}{\partial t},\quad \phi' = \frac{\partial \phi}{\partial x}.
\]
Often it is convenient to (non-relativistically) split the Lagrangian as
\[
  L = T - V,
\]
where
\[
  T = \int \frac{1}{2} \dot{\phi}^2\;\d x,\quad V = \int \left(\frac{1}{2} \phi'^2 + U(\phi)\right)\;\d x.
\]
In higher dimensions, we separate out $\partial_\mu \phi$ into $\dot{\phi}$ and $\nabla \phi$.

The classical field equation comes from the condition that $S[\phi]$ is stationary under variations of $\phi$. By a standard manipulation, the field equation turns out to be
\[
  \partial_\mu \partial^\mu \phi + \frac{\d U}{\d \phi} = 0.
\]
This is an example of a Klein--Gordon type of field equation, but is non-linear if $U$ is not quadratic. This is known as the \term{non-linear Klein--Gordon equation}\index{Klein--Gordon equation!non-linear}.

We are interested in a soliton that is a static solution. For a static field, the time derivatives can be dropped, and this equation becomes
\[
  \frac{\d^2 \phi}{\partial x^2} = \frac{\d U}{\d \phi}.
\]
Of course, the important part is the choice of $U$! In $\phi^4$ theory, we choose
\[
  U(\phi) = \frac{1}{2} (1 - \phi^2)^2.
\]
This is mathematically the simplest version, because we set all coupling constants to $1$. If we want to quantize this (or just want to be more sophisticated), then we should include some parameters. However, this tends to become a complete mess.

The importance of this $U$ is that it has two minima:
\begin{center}
  \begin{tikzpicture}
    \draw [->](-3, -1) -- (3, -1) node [right] {$\phi$};
    \draw [->] (0, -1.5) -- (0, 3) node [above] {$U(\phi)$};

    \draw [mblue, thick, domain=-2.4:2.4, samples=50] plot [smooth] (\x, {4 * ((\x/2)^4 - (\x/2)^2)});

    \node at (-1.414, -1) [below] {$-1$};
    \node at (1.414, -1) [below] {$1$};
  \end{tikzpicture}
\end{center}
The two classical vacuum are
\[
  \phi(x) \equiv 1,\quad \phi(x) \equiv -1.
\]
This is, of course, not the only possible choice. We can, for example, include some parameters and set
\[
  U(\phi) = \lambda(m^2 - \phi^2)^2.
\]
If we are more adventurous, we can talk about a $\phi^6$ theory with
\[
  U(\phi) = \lambda \phi^2 (m^2 - \phi^2)^2.
\]
In this case, we have $3$ minima, instead of $2$. Even braver people can choose
\[
  U(\phi) =  1 - \cos \phi.
\]
This has \emph{infinitely} many minima. The field equation involves a $\sin \phi$ term, and is hence this theory is called the \term{sine-Gordon theory} (a pun on the name Klein--Gordon, of course).

The sine-Gordon theory is a special case. While it seems like the most complicated potential so far, it is actually \emph{integrable}. This implies we can find explicit exact solutions involving multiple, interacting solitons in a rather easy way. However, integrable systems is a topic for another course, namely the IID Integrable Systems course.

For now, we will focus on our simplistic $\phi^4$ theory. As mentioned, there are two vacuum field configurations, both of zero energy. We will in general use the term ``\term{field configuration}'' to refer to fields that are not necessarily solutions to the classical field equation, but in this case, the vacua are indeed solutions.

If we wanted to quantize this $\phi^4$ theory, then we have to pick one of the vacua and do perturbation theory around it. This is known as \term{spontaneous symmetry breaking}. Of course, by symmetry, we obtain the same quantum theory regardless of which vacuum we expand around.

However, as we mentioned, when we want to study solitons, we have to involve \emph{both} vacua. We want to consider solutions that ``connect'' these two vacua. In other words, we are looking for solutions that look like
\begin{center}
  \begin{tikzpicture}
    \draw [->] (-3, 0) -- (3, 0) node [right] {$x$};
    \draw [->] (0, -2) -- (0, 2) node [above] {$m$};

    \draw [mblue, thick, domain=-3:3] plot [smooth] (\x, {tanh(1.5*(\x + 1))});

    \draw [dashed] (-3, 1) -- (3, 1);
    \draw [dashed] (-3, -1) -- (3, -1);

    \node [circ] at (-1, 0) {};
    \node [anchor = north west] at (-1, 0) {$a$};
  \end{tikzpicture}
\end{center}
This is known as a \term{kink solution}. Note that this kink solution is obviously not unique. We can translate this and obtain a new kink. Thus, we have a \term{moduli}, i.e.\ a parameter of the solution. In this case, it is the intersection with the $x$-axis, which we call $a$.

To actually find such a solution, we need the full field equation, given by
\[
  \frac{\d^2 \phi}{ \d x^2} = -2 (1 - \phi^2) \phi.
\]
Instead of solving this directly, we will find the kink solutions by considering the energy, since this generalizes better.
\printindex
\end{document}
