\documentclass[a4paper]{article}

\def\npart {III}
\def\nterm {Easter}
\def\nyear {2017}
\def\nlecturer {N.\ S.\ Manton}
\def\ncourse {Classical and Quantum Solitons}

% Imports
\ifx \nextra \undefined
  \usepackage[pdftex,
    hidelinks,
    pdfauthor={Dexter Chua},
    pdfsubject={Cambridge Maths Notes: Part \npart\ - \ncourse},
    pdftitle={Part \npart\ - \ncourse},
  pdfkeywords={Cambridge Mathematics Maths Math \npart\ \nterm\ \nyear\ \ncourse}]{hyperref}
  \title{Part \npart\ - \ncourse}
\else
  \usepackage[pdftex,
    hidelinks,
    pdfauthor={Dexter Chua},
    pdfsubject={Cambridge Maths Notes: Part \npart\ - \ncourse\ (\nextra)},
    pdftitle={Part \npart\ - \ncourse\ (\nextra)},
  pdfkeywords={Cambridge Mathematics Maths Math \npart\ \nterm\ \nyear\ \ncourse\ \nextra}]{hyperref}

  \title{Part \npart\ - \ncourse \\ {\Large \nextra}}
\fi

\author{Lectured by \nlecturer \\\small Notes taken by Dexter Chua}
\date{\nterm\ \nyear}

\usepackage{alltt}
\usepackage{amsfonts}
\usepackage{amsmath}
\usepackage{amssymb}
\usepackage{amsthm}
\usepackage{booktabs}
\usepackage{caption}
\usepackage{enumitem}
\usepackage{fancyhdr}
\usepackage{graphicx}
\usepackage{mathtools}
\usepackage{microtype}
\usepackage{multirow}
\usepackage{pdflscape}
\usepackage{pgfplots}
\usepackage{siunitx}
\usepackage{tabularx}
\usepackage{tikz}
\usepackage{tkz-euclide}
\usepackage[normalem]{ulem}
\usepackage[all]{xy}

\pgfplotsset{compat=1.12}

\pagestyle{fancyplain}
\lhead{\emph{\nouppercase{\leftmark}}}
\ifx \nextra \undefined
  \rhead{
    \ifnum\thepage=1
    \else
      \npart\ \ncourse
    \fi}
\else
  \rhead{
    \ifnum\thepage=1
    \else
      \npart\ \ncourse\ (\nextra)
    \fi}
\fi
\usetikzlibrary{arrows}
\usetikzlibrary{decorations.markings}
\usetikzlibrary{decorations.pathmorphing}
\usetikzlibrary{positioning}
\usetikzlibrary{fadings}
\usetikzlibrary{intersections}
\usetikzlibrary{cd}

\newcommand*{\Cdot}{\raisebox{-0.25ex}{\scalebox{1.5}{$\cdot$}}}
\newcommand {\pd}[2][ ]{
  \ifx #1 { }
    \frac{\partial}{\partial #2}
  \else
    \frac{\partial^{#1}}{\partial #2^{#1}}
  \fi
}

% Theorems
\theoremstyle{definition}
\newtheorem*{aim}{Aim}
\newtheorem*{axiom}{Axiom}
\newtheorem*{claim}{Claim}
\newtheorem*{cor}{Corollary}
\newtheorem*{defi}{Definition}
\newtheorem*{eg}{Example}
\newtheorem*{fact}{Fact}
\newtheorem*{law}{Law}
\newtheorem*{lemma}{Lemma}
\newtheorem*{notation}{Notation}
\newtheorem*{prop}{Proposition}
\newtheorem*{thm}{Theorem}

\renewcommand{\labelitemi}{--}
\renewcommand{\labelitemii}{$\circ$}
\renewcommand{\labelenumi}{(\roman{*})}

\let\stdsection\section
\renewcommand\section{\newpage\stdsection}

% Strike through
\def\st{\bgroup \ULdepth=-.55ex \ULset}

% Maths symbols
\newcommand{\bra}{\langle}
\newcommand{\ket}{\rangle}

\newcommand{\N}{\mathbb{N}}
\newcommand{\Z}{\mathbb{Z}}
\newcommand{\Q}{\mathbb{Q}}
\renewcommand{\H}{\mathbb{H}}
\newcommand{\R}{\mathbb{R}}
\newcommand{\C}{\mathbb{C}}
\newcommand{\Prob}{\mathbb{P}}
\renewcommand{\P}{\mathbb{P}}
\newcommand{\E}{\mathbb{E}}
\newcommand{\F}{\mathbb{F}}
\newcommand{\cU}{\mathcal{U}}
\newcommand{\RP}{\mathbb{RP}}
\newcommand{\CP}{\mathbb{CP}}

\newcommand{\ph}{\,\cdot\,}

\DeclareMathOperator{\sech}{sech}
\DeclareMathOperator{\cosech}{cosech}
\DeclareMathOperator{\cosec}{cosec}

\DeclareMathOperator{\covol}{covol}
\DeclareMathOperator{\vol}{vol}

\let\Im\relax
\let\Re\relax
\DeclareMathOperator{\Im}{Im}
\DeclareMathOperator{\Re}{Re}
\DeclareMathOperator{\im}{im}
\DeclareMathOperator{\image}{image}
\DeclareMathOperator{\Ann}{Ann}

\DeclareMathOperator*{\res}{res}
\DeclareMathOperator{\Res}{Res}
\DeclareMathOperator{\Ind}{Ind}

\DeclareMathOperator{\tr}{tr}
\DeclareMathOperator{\diag}{diag}
\DeclareMathOperator{\rank}{rank}
\DeclareMathOperator{\card}{card}
\DeclareMathOperator{\spn}{span}
\DeclareMathOperator{\adj}{adj}

\DeclareMathOperator{\erf}{erf}
\DeclareMathOperator{\erfc}{erfc}

\DeclareMathOperator{\ord}{ord}
\DeclareMathOperator{\Sym}{Sym}

\DeclareMathOperator{\sgn}{sgn}
\DeclareMathOperator{\orb}{orb}
\DeclareMathOperator{\stab}{stab}
\DeclareMathOperator{\ccl}{ccl}

\DeclareMathOperator{\lcm}{lcm}
\DeclareMathOperator{\hcf}{hcf}

\DeclareMathOperator{\Int}{Int}
\DeclareMathOperator{\id}{id}

\DeclareMathOperator{\betaD}{beta}
\DeclareMathOperator{\gammaD}{gamma}
\DeclareMathOperator{\Poisson}{Poisson}
\DeclareMathOperator{\binomial}{binomial}
\DeclareMathOperator{\multinomial}{multinomial}
\DeclareMathOperator{\Bernoulli}{Bernoulli}
\DeclareMathOperator{\like}{like}

\DeclareMathOperator{\var}{var}
\DeclareMathOperator{\cov}{cov}
\DeclareMathOperator{\bias}{bias}
\DeclareMathOperator{\mse}{mse}
\DeclareMathOperator{\corr}{corr}

\DeclareMathOperator{\otp}{otp}
\DeclareMathOperator{\dom}{dom}

\DeclareMathOperator{\Root}{Root}
\DeclareMathOperator{\supp}{supp}
\DeclareMathOperator{\rel}{rel}
\DeclareMathOperator{\Hom}{Hom}
\DeclareMathOperator{\Aut}{Aut}
\DeclareMathOperator{\Gal}{Gal}
\DeclareMathOperator{\Mat}{Mat}
\DeclareMathOperator{\End}{End}
\DeclareMathOperator{\Char}{char}
\DeclareMathOperator{\ev}{ev}
\DeclareMathOperator{\St}{St}
\DeclareMathOperator{\Lk}{Lk}
\DeclareMathOperator{\disc}{disc}
\DeclareMathOperator{\Isom}{Isom}
\DeclareMathOperator{\length}{length}
\DeclareMathOperator{\energy}{energy}
\DeclareMathOperator{\area}{area}
\DeclareMathOperator{\Syl}{Syl}
\DeclareMathOperator{\cl}{cl}
\DeclareMathOperator{\fix}{fix}

\newcommand{\GL}{\mathrm{GL}}
\newcommand{\SL}{\mathrm{SL}}
\newcommand{\PGL}{\mathrm{PGL}}
\newcommand{\PSL}{\mathrm{PSL}}
\newcommand{\PSU}{\mathrm{PSU}}
\newcommand{\Or}{\mathrm{O}}
\newcommand{\SO}{\mathrm{SO}}
\newcommand{\U}{\mathrm{U}}
\newcommand{\SU}{\mathrm{SU}}

\renewcommand{\d}{\mathrm{d}}
\newcommand{\D}{\mathrm{D}}

\tikzset{->/.style = {decoration={markings,
                                  mark=at position 1 with {\arrow[scale=2]{latex'}}},
                      postaction={decorate}}}
\tikzset{<-/.style = {decoration={markings,
                                  mark=at position 0 with {\arrowreversed[scale=2]{latex'}}},
                      postaction={decorate}}}
\tikzset{<->/.style = {decoration={markings,
                                   mark=at position 0 with {\arrowreversed[scale=2]{latex'}},
                                   mark=at position 1 with {\arrow[scale=2]{latex'}}},
                       postaction={decorate}}}
\tikzset{->-/.style = {decoration={markings,
                                   mark=at position #1 with {\arrow[scale=2]{latex'}}},
                       postaction={decorate}}}
\tikzset{-<-/.style = {decoration={markings,
                                   mark=at position #1 with {\arrowreversed[scale=2]{latex'}}},
                       postaction={decorate}}}

\tikzset{circ/.style = {fill, circle, inner sep = 0, minimum size = 3}}
\tikzset{mstate/.style={circle, draw, blue, text=black, minimum width=0.7cm}}

\definecolor{mblue}{rgb}{0.2, 0.3, 0.8}
\definecolor{morange}{rgb}{1, 0.5, 0}
\definecolor{mgreen}{rgb}{0.1, 0.4, 0.2}
\definecolor{mred}{rgb}{0.5, 0, 0}

\def\drawcirculararc(#1,#2)(#3,#4)(#5,#6){%
    \pgfmathsetmacro\cA{(#1*#1+#2*#2-#3*#3-#4*#4)/2}%
    \pgfmathsetmacro\cB{(#1*#1+#2*#2-#5*#5-#6*#6)/2}%
    \pgfmathsetmacro\cy{(\cB*(#1-#3)-\cA*(#1-#5))/%
                        ((#2-#6)*(#1-#3)-(#2-#4)*(#1-#5))}%
    \pgfmathsetmacro\cx{(\cA-\cy*(#2-#4))/(#1-#3)}%
    \pgfmathsetmacro\cr{sqrt((#1-\cx)*(#1-\cx)+(#2-\cy)*(#2-\cy))}%
    \pgfmathsetmacro\cA{atan2(#2-\cy,#1-\cx)}%
    \pgfmathsetmacro\cB{atan2(#6-\cy,#5-\cx)}%
    \pgfmathparse{\cB<\cA}%
    \ifnum\pgfmathresult=1
        \pgfmathsetmacro\cB{\cB+360}%
    \fi
    \draw (#1,#2) arc (\cA:\cB:\cr);%
}
\newcommand\getCoord[3]{\newdimen{#1}\newdimen{#2}\pgfextractx{#1}{\pgfpointanchor{#3}{center}}\pgfextracty{#2}{\pgfpointanchor{#3}{center}}}

\def\Xint#1{\mathchoice
   {\XXint\displaystyle\textstyle{#1}}%
   {\XXint\textstyle\scriptstyle{#1}}%
   {\XXint\scriptstyle\scriptscriptstyle{#1}}%
   {\XXint\scriptscriptstyle\scriptscriptstyle{#1}}%
   \!\int}
\def\XXint#1#2#3{{\setbox0=\hbox{$#1{#2#3}{\int}$}
     \vcenter{\hbox{$#2#3$}}\kern-.5\wd0}}
\def\ddashint{\Xint=}
\def\dashint{\Xint-}


\begin{document}
\maketitle
{\small
\setlength{\parindent}{0em}
\setlength{\parskip}{1em}
Solitons are solutions of classical field equations with particle-like properties. They are localised in space, have finite energy and are stable against decay into radiation. The stability usually has a topological explanation. After quantisation, they give rise to new particle states in the underlying quantum field theory that are not seen in perturbation theory. We will focus mainly on kink solitons in one space dimension, and on Skyrmions in three dimensions. Solitons in gauge theories will also be mentioned.

\subsubsection*{Pre-requisites}
This course assumes you have taken Quantum Field Theory and Symmetries, Fields and Particles. The small amount of topology that is needed will be developed during the course.
}
\tableofcontents
\section{Introduction}
Given a classical field theory, if we want to ``quantize'' it, then we find the vacuum of the theory, and then do perturbation theory around this vacuum. If there are multiple vacua, then what we did was that we arbitrarily picked a vacuum, and then expanded around that vacuum.

However, these field theories with multiple vacua often contain \emph{soliton} solutions. These are localized, smooth solutions of the classical field equations, and they ``connect multiple vacuums''. To quantize these solitons solutions, we fix such a soliton, and use it as the ``background''. We then do perturbation theory around these solutions, but this is rather tricky to do. Thus, in a lot of the course, we will just look at the classical part of the theory.

Recall that when quantizing our field theories in perturbation theory, we obtain particles in the quantum theory, despite the classical theory being completely about fields. It turns our solitons also behave like particles, and they are a \emph{new} type of particles. These are non-perturbative phenomenon. If we want to do the quantum field theory properly, we have to include these solitons in the quantum field theory. In general this is hard, and so we are not going to develop this a lot.

What does it mean to say that solitons are like particles? In relativistic field theories, we find these solitons have a classical energy. We define the ``mass'' $M$ of the soliton to be the energy in the ``rest frame''. Since this is relativistic, we can do a Lorentz boost, and we obtain a moving soliton. Then we obtain a relation of the form
\[
  E^2 - \mathbf{P} \cdot \mathbf{P} = M^2.
\]
This is a Lorentz-invariant property of the soliton. Together with the fact that the soliton is localized, this is a justification for thinking of them as particles.

These particles differ from the particles perturbative quantum fields, as they have rather different properties. Interesting solitons have a \emph{topological} character different from the classical vacuum. Thus, at least naively, they cannot be thought of perturbatively.

There are also non-relativistic solitons, and they usually don't have interpretations of particles. These appear, for example, as defects in solids. We will not be interested in these much.

What kinds of theories have solitons? To obtain solitons, we definitely need a non-linear field structure and/or non-linear equations. Thus, free field theories with quadratic Lagrangians such as Maxwell theory do not have solitons. We need interaction terms.

Note that in QFT, we did interactions using the interaction picture. We split the Hamiltonian into a ``free field'' part, which we solve exactly, and the ``interaction'' part. However, to quantize solitons, we need to solve the full interacting Lagrangian \emph{exactly}.

Having interactions is not enough for solitons to appear. To obtain solitons, we also need some non-trivial vacuum topology. In other words, we need more than one vacuum. This usually comes from symmetry breaking, and often gauge symmetries are involved.

In this course, we will focus on three types of solitons.
\begin{itemize}
  \item In one (space) dimension, we have kinks. We will spend $4$ lectures on this.

  \item In two dimensions, we have vortices. We will spend $6$ lectures on this.

  \item In three dimensions, there are monopoles and Skyrmions. We will only study Skyrmions, and will spend $6$ lectures on it.
\end{itemize}
These examples are all relativistic. Non-relativistic solitons include \emph{domain walls}, which occur in ferromagnets, but we will not study these.

In general, solitons appear in all sorts of different actual, physical scenarios such as in condensed matter physics, optical fibers, superconductors, ``cosmic strings'' etc. Since we are mathematicians, we probably will not put much focus on these actual applications. However, we can talk a bit more about Skyrmions.

Skyrmions are solitons in an \emph{effective field theory} of interacting pions, which are thought to be the most important baryons because they are the lightest. This happens in spite of the lack of a gauge symmetry. While pions have no baryon number, the associated solitons have a topological charge identified with baryon number. This baryon number is conserved for topological reasons.

Note that in QCD, baryon number is conserved because the quark number is conserved. We tried extremely hard to find proton decay, which would be a process that involves baryon numbers, but we cannot find such examples. We have very high experimental certainty that baryon number is conserved. And if baryon number is topological, then this is a very good reason for the conservation of baryon numbers.

Skyrmions give a model of low-energy interactions of baryons. This leads to an (approximate) theory of nucleons (proton and neutron) and larger nuclei, which are bound states of any number of protons and neutrons. % This is really up-to-date stuff.

There is a whole other set of Skyrmions studied, which are two-dimensional. These are structure in exotic magnets, and they have actually been seen.

For these ideas to work out well, we need to eventually do quantization. For example, Skyrmions by themselves do not have spin. We need to quantize the theory before these come out. Also, Skyrmions cannot distinguish between protons and neutrons. These differences only come up after we quantize.
\section{\tph{$\phi^4$}{phi4}{&straightphi;<sup>4</sup>} kinks}
\subsection{Kink solutions}
In this chapter, we are going to study \term{$\phi^4$ kinks}\index{kink}\index{kink!$\phi^4$}. This happens in $1 + 1$ dimensions, and involves a single scalar field $\phi(x, t)$. In higher dimensions, we often need many fields to obtain solitons, but in the case of 1 dimension, we can get away with a single field.

In general, the \term{Lagrangian density} of such a scalar field theory is of the form
\[
  \mathcal{L} = \frac{1}{2} \partial_\mu \phi \partial^\mu \phi - U(\phi)
\]
for some potential $U(\phi)$ polynomial in $\phi$. Note that in $1 + 1$ dimensions, any such theory is renormalizable. Here we will choose the Minkowski metric to be
\[
  \eta^{\mu\nu} = 
  \begin{pmatrix}
    1 & 0\\
    0 & -1
  \end{pmatrix}.
\]
Then the \term{Lagrangian} is given by
\[
  L = \int_{-\infty}^\infty \mathcal{L}\;\d x = \int_{-\infty}^\infty \left(\frac{1}{2} \partial_\mu \phi \partial^\mu \phi - U(\phi)\right)\;\d x,
\]
and then the \term{action} is
\[
  S[\phi] = \int L\;\d t = \int \mathcal{L} \;\d x\;\d t.
\]

There is a non-linearity in the field equations due to a potential $U(\phi)$ with \emph{multiple vacua}. We need this multiple vacua to obtain a soliton. The kink stability comes from the \emph{topology}. It is very simple here, and just comes from counting the discrete, distinct vacua.

As usual, we will write
\[
  \dot{\phi} = \frac{\partial \phi}{\partial t},\quad \phi' = \frac{\partial \phi}{\partial x}.
\]
Often it is convenient to (non-relativistically) split the Lagrangian as
\[
  L = T - V,
\]
where
\[
  T = \int \frac{1}{2} \dot{\phi}^2\;\d x,\quad V = \int \left(\frac{1}{2} \phi'^2 + U(\phi)\right)\;\d x.
\]
In higher dimensions, we separate out $\partial_\mu \phi$ into $\dot{\phi}$ and $\nabla \phi$.

The classical field equation comes from the condition that $S[\phi]$ is stationary under variations of $\phi$. By a standard manipulation, the field equation turns out to be
\[
  \partial_\mu \partial^\mu \phi + \frac{\d U}{\d \phi} = 0.
\]
This is an example of a Klein--Gordon type of field equation, but is non-linear if $U$ is not quadratic. This is known as the \term{non-linear Klein--Gordon equation}\index{Klein--Gordon equation!non-linear}.

We are interested in a soliton that is a static solution. For a static field, the time derivatives can be dropped, and this equation becomes
\[
  \frac{\d^2 \phi}{\partial x^2} = \frac{\d U}{\d \phi}.
\]
Of course, the important part is the choice of $U$! In $\phi^4$ theory, we choose
\[
  U(\phi) = \frac{1}{2} (1 - \phi^2)^2.
\]
This is mathematically the simplest version, because we set all coupling constants to $1$. If we want to quantize this (or just want to be more sophisticated), then we should include some parameters. However, this tends to become a complete mess.

The importance of this $U$ is that it has two minima:
\begin{center}
  \begin{tikzpicture}
    \draw [->](-3, -1) -- (3, -1) node [right] {$\phi$};
    \draw [->] (0, -1.5) -- (0, 3) node [above] {$U(\phi)$};

    \draw [mblue, thick, domain=-2.4:2.4, samples=50] plot [smooth] (\x, {4 * ((\x/2)^4 - (\x/2)^2)});

    \node at (-1.414, -1) [below] {$-1$};
    \node at (1.414, -1) [below] {$1$};
  \end{tikzpicture}
\end{center}
The two classical vacuum are
\[
  \phi(x) \equiv 1,\quad \phi(x) \equiv -1.
\]
This is, of course, not the only possible choice. We can, for example, include some parameters and set
\[
  U(\phi) = \lambda(m^2 - \phi^2)^2.
\]
If we are more adventurous, we can talk about a $\phi^6$ theory with
\[
  U(\phi) = \lambda \phi^2 (m^2 - \phi^2)^2.
\]
In this case, we have $3$ minima, instead of $2$. Even braver people can choose
\[
  U(\phi) =  1 - \cos \phi.
\]
This has \emph{infinitely} many minima. The field equation involves a $\sin \phi$ term, and is hence this theory is called the \term{sine-Gordon theory} (a pun on the name Klein--Gordon, of course).

The sine-Gordon theory is a special case. While it seems like the most complicated potential so far, it is actually \emph{integrable}. This implies we can find explicit exact solutions involving multiple, interacting solitons in a rather easy way. However, integrable systems is a topic for another course, namely the IID Integrable Systems course.

For now, we will focus on our simplistic $\phi^4$ theory. As mentioned, there are two vacuum field configurations, both of zero energy. We will in general use the term ``\term{field configuration}'' to refer to fields that are not necessarily solutions to the classical field equation, but in this case, the vacua are indeed solutions.

If we wanted to quantize this $\phi^4$ theory, then we have to pick one of the vacua and do perturbation theory around it. This is known as \term{spontaneous symmetry breaking}. Of course, by symmetry, we obtain the same quantum theory regardless of which vacuum we expand around.

However, as we mentioned, when we want to study solitons, we have to involve \emph{both} vacua. We want to consider solutions that ``connect'' these two vacua. In other words, we are looking for solutions that look like
\begin{center}
  \begin{tikzpicture}
    \draw [->] (-3, 0) -- (3, 0) node [right] {$x$};
    \draw [->] (0, -2) -- (0, 2) node [above] {$m$};

    \draw [mblue, thick, domain=-3:3] plot [smooth] (\x, {tanh(1.5*(\x + 1))});

    \draw [dashed] (-3, 1) -- (3, 1);
    \draw [dashed] (-3, -1) -- (3, -1);

    \node [circ] at (-1, 0) {};
    \node [anchor = north west] at (-1, 0) {$a$};
  \end{tikzpicture}
\end{center}
This is known as a \term{kink solution}.

To actually find such a solution, we need the full field equation, given by
\[
  \frac{\d^2 \phi}{\d x^2} = -2 (1 - \phi^2) \phi.
\]
Instead of solving this directly, we will find the kink solutions by considering the energy, since this method generalizes better.

We will work with a general potential $U$ with minimum $0$. From Noether's theorem, we obtain a conserved energy
\[
  E = \int \left(\frac{1}{2} \dot{\phi}^2 + \frac{1}{2} \phi'^2 + U(\phi)\right)\;\d x.
\]
For a static field, we drop the $\dot{\phi}^2$ term. Then this is just the $V$ appearing in the Lagrangian. By definition, the field equation tells us the field is a stationary point of this energy. To find the kink solution, we will in fact find a \emph{minimum} of the energy.

Of course, the global minimum is attained when we have a vacuum field, in which case $E = 0$. However, this is just the global minimum only if we don't impose any boundary conditions. In our case, the kinks satisfy the boundary conditions ``$\phi(\infty) = 1$'', ``$\phi(-\infty) = -1$'' (interpreted in terms of limits, of course). The kinks will minimize energy subject to these boundary conditions.

These boundary conditions are important, because they are ``topological''. Eventually, we will want to understand the dynamics of solitons, so we will want to consider fields that evolve with time. From physical considerations, for any fixed $t$, the field $\phi(x, t)$ must satisfy $\phi(x, t) \to \text{vacuum}$ as $x \to \pm \infty$, or else the field will have infinite energy. However, the vacuum of our potential $U$ is discrete. Thus, if $\phi$ were to evolve continuously with time, the boundary conditions must not evolve with time! At least, this is what we expect classically. Who knows what weird tunnelling can happen in quantum field theory.

So from now on, we fix a some boundary conditions $\phi(\infty)$ and $\phi(-\infty)$, and focus on fields that satisfy these boundary conditions. The trick is to write the potential in the form
\[
  U (\phi) = \frac{1}{2} \left(\frac{\d W (\phi)}{\d \phi}\right)^2.
\]
If $U$ is always non-negative, then we can always find $W$ in principle --- we take the square root and then integrate it. However, in practice, this is useful only if we can find a simple form for $W$. Let's assume we've done that. Then we can write 
\begin{align*}
  E &= \frac{1}{2} \int \left(\phi'^2 + \left(\frac{\d W}{\d \phi}\right)^2 \right)\;\d x\\
  &= \frac{1}{2} \int \left(\phi' \mp \frac{\d W}{\d \phi}\right)^2\;\d x \pm \int \frac{\d W}{\d \phi} \frac{\d \phi}{\d x} \;\d x\\
  &= \frac{1}{2} \int \left(\phi' \mp \frac{\d W}{\d \phi}\right)^2\;\d x \pm \int \d W\\
  &= \frac{1}{2} \int \left(\phi' \mp \frac{\d W}{\d \phi}\right)^2\;\d x \pm (W(\phi(\infty)) - W(\phi(-\infty))).
\end{align*}
The second term depends purely on the boundary conditions, which we have fixed. Thus, we can minimize energy if we can make the first term vanish! Note that when completing the square, the choice of the signs is arbitrary. However, if we want to set the first term to be $0$, the second term had better be non-positive, since the energy itself is non-negative! Hence, we will pick the sign such that the second term is $\geq 0$, and then the energy is minimized when
\[
  \phi' = \pm \frac{\d W}{\d \phi}.
\]
In this case, we have
\[
  E = \pm (W(\infty) - W(-\infty)).
\]
These are known as the \term{Bogomolny equation} and the \term{Bogomolny energy bound}. Note that if we picked the other sign, then we cannot solve the differential equation $\phi' = \pm \frac{\d W}{\d \phi}$, because we know the energy must be non-negative.

For the $\phi^4$ kink, we have
\[
  \frac{\d W}{\d \phi} = 1 - \phi^2.
\]
So we pick
\[
  W = \phi - \frac{1}{3} \phi^3.
\]
So when $W = \pm 1$, we have $W = \pm \frac{2}{3}$. We need to choose the $+$ sign, and then we know the energy (and hence mass) of the kink is
\[
  E \equiv M = \frac{4}{3}.
\]
We now solve for $\phi$. The equation we have is
\[
  \phi' = 1 - \phi^2.
\]
Rearranging gives
\[
  \frac{1}{1 - \phi^2} \d \phi = \d x,
\]
which we can integrate to give
\[
  \phi (x) = \tanh (x - a).
\]
This $a$ is an arbitrary constant of integration, labelling the intersection with the $x$-axis. We think of this as the ``\term{location}'' of the kink.

Note that there is not a unique solution, which is not unexpected by translation invariance. Instead, the solutions are labeled by a parameter $a$. This is known as a \term{modulus} of the solution. In general, there can be multiple moduli, and the space of all possible values of the moduli is known as the \term{moduli space}. In this case, the moduli space is just $\R$.

Is this solution stable? We obtained this kink solution by minimizing the energy within this topological class of solutions (i.e.\ among all solutions with the prescribed boundary conditions). Since a field cannot change the boundary conditions during evolution, it follows that the kink must be stable.

Are there other soliton solutions to the field equations? The solutions are determined by the boundary conditions. Thus, we can classify all soliton solutions by counting all possible combinations of the boundary conditions. We have, of course, two vacuum solutions $\phi \equiv 1$ and $\phi \equiv -1$. There is also an \term{anti-kink}\index{kink!anti-} solution obtained by inverting the kink:
\[
  \phi(x) = - \tanh (x - b).
\]
This also has energy $\frac{4}{3}$.

\subsection{Dynamic kink}
We now want to look at kinks that move. Given what we have done so far, this is trivial. Our theory is Lorentz invariant, so we simply apply a Lorentz boost. Then we obtain a field
\[
  \phi(x, t) = \tanh \gamma (x - vt),
\]
where, as usual
\[
  \gamma = (1 - v^2)^{-1/2}.
\]
But this isn't it. Notice that for small $v$, we can approximate the solution simply by
\[
  \phi(x, t) = \tanh (x - vt).
\]
This looks like a kink solution with a moduli that varies with time slowly. This is known as the \term{adiabatic} point of view.

More generally, let's consider a ``moving kink'' field
\[
  \phi(x, t) = \tanh (x - a(t))
\]
for some function $a(t)$. In general, this is not a solution to the field equation, but if $\dot{a}$ is small, then it is ``approximately a solution''.

We can now explicitly compute that
\[
  \dot{\phi} = - \frac{\d a}{\d t} \phi'.
\]
Let's consider fields of this type, and look at the Lagrangian of the field theory. The kinetic term is given by
\[
  T = \int \frac{1}{2} \dot{\phi}^2\;\d x = \frac{1}{2} \left(\frac{\d a}{\d t}\right)^2  \int \phi'^2 \;\d x = \frac{1}{2} M \left(\frac{\d a}{\d t}\right)^2.
\]
To derive this result, we had to perform the integral $\int \phi'^2 \;\d x$, and if we do that horrible integral, we will find a value that happens to be equal to $M = \frac{4}{3}$. Of course, this is not a coincidence. We can derive this result from more general principles to see that the result of integration is manifestly $M$.

The remaining part of the Lagrangian is less interesting. Since it does not involve taking time derivatives, the time variation of $a$ is not seen by it, and we simply have a constant
\[
  V = \frac{4}{3}.
\]
Then the original field Lagrangian becomes a particle Lagrangian 
\[
  L = \frac{1}{2}M \dot{a}^2 - \frac{4}{3}.
\]

Note that when we first formulated the field theory, the action principle required us to find a field that extremizes the action \emph{among all fields}. However, what we are doing now is to restrict to the set of kink solutions only, and then when we solve the variational problem arising from this Lagrangian, we are extremizing the action among fields of the form $\tanh (x - a(t))$. We can think of this as motion in a ``valley'' in the field configuration space. In general, these solutions will not also extremize the action among all fields. However, as we said, it will do so ``approximately'' if $\dot{a}$ is small.

We can obtain an effective equation of motion
\[
  M \ddot{a} = 0,
\]
which is an equation of motion for the variable $a(t)$ \emph{in the moduli space}.

Of course, the solution is just given by
\[
  a(t) = vt + \mathrm{const},
\]
where $v$ is an arbitrary constant, which we interpret as the velocity. In this formulation, we do not have any restrictions on $v$, because we took the ``non-relativistic approximation''. This approximation breaks down when $v$ is large.

There is a geometric interpretation to this. We can view the equation of motion $M\ddot{a} = 0$ as the \emph{geodesic equation} in the moduli space $\R$, and we can think of the coefficient $M$ as specifying a Riemannian metric on the moduli space. In this case, the metric is (a scalar multiple of) the usual Euclidean metric.

This seems like a complicated way of describing such a simple system, but this picture generalizes to higher-dimensional systems and allows us to analyze multi-soliton dynamics. 

%We can find the dynamics of $a(t)$ from the Lagrangian of the field theory. Thus, we are reducing the ``field dynamics'' to the ``particle dynamics''. We have
%\[
%  T = \int \frac{1}{2} \dot{\phi}^2\;\d x = \frac{1}{2} \left(\frac{\d a}{\d t}\right)^2  \int \phi'^2 \;\d x = \frac{1}{2} M \left(\frac{\d a}{\d t}\right)^2.
%\]
%The factor of $M$ just comes from doing the (tricky) integral explicitly, but we can also work it out from more general principles to make it manifestly $M$, and this is known as Derrick's theorem.
%
%Thus, the kink behaves like a particle with mass $M$! How about the potential energy? The potential energy is \emph{not} time-dependent. We simply integrate some polynomial of $\phi$ over all $x$, and the shift by $a$ does not make a difference. In this case, we have
%\[
%  V = \frac{4}{3}.
%\]
%So we've reduced the field Lagrangian to a particle Lagrangian
%\[
%  L = \frac{1}{2}M \dot{a}^2 - \frac{4}{3}.
%\]
%We can think of this as motion in a \term{valley} in the field configuration space. We are drifting in the energy minima in the field configuration space.
%
%This method is powerful, and applies to multi-soliton dynamics in high dimensions. From this, we obtain an effective equation of motion in moduli space
%\[
%  M \ddot{a} = 0.
%\]
%This has \emph{geometric} interpretation as geometric motion in the moduli space. The moduli space is just the real line $\R$ with its standard metric. We can think of the coefficient $M$ as a Riemannian metric, which happens to be constant (as a function of $a$) in this case.
%
%Of course, the solution is
%\[
%  a(t) = vt + \mathrm{const},
%\]
%where $v$ is an arbitrary constant, namely velocity.
%
%In this approximation, $v$ can be anything, and the approximation does not see the speed of light. However, as we plug this back into the actual field equation, we see that the approximation breaks down when $v$ is large.
%
%This motion in moduli space is not exact, but is accurate in the non-relativistic approximation.
%
%This was all rather trivial in our case of kinks. However, it is also important and allows us to analyze the motion of several solitons in higher dimension.
%
%Note that here we started with a Lagrangian that is quadratic in time derivatives of the field. When we pass on to solitons, we have a term that is quadratic in the time derivative in the moduli, and the coefficients provide a Riemannian geometry on the moduli space.

We might ask ourselves if there are multi-kinks in our theory. There aren't in the $\phi^4$ theory, because we saw that the solutions are classified by the boundary conditions, and there are only that many boundary conditions we've completely enumerated.  In more complicated theories like sine-Gordon theory, multiple kinks are possible.

However, while we cannot have two kinks, we can have a kink followed by an anti-kink. This actually lies in the ``vacuum sector'' of the theory, but it still looks like it's made up of kinks and anti-kinks, and it is interesting to study these.
\subsection{Momentum and forces}

In $1 + 1$ dimensions, we have a single scalar field, with Lagrangian density
\[
  \mathcal{L} = \frac{1}{2} \partial_\mu \phi \partial^\mu \phi - U(\phi).
\]
From Noether's theorem, we obtain the energy-momentum tensor
\[
  T^\mu_\nu = \frac{\partial \mathcal{L}}{\partial (\partial_\mu \phi)} \partial_\nu \phi - \delta^\mu_\nu \mathcal{L}.
\]
The conserved energy is
\[
  E = \int_{-\infty}^\infty T^0\!_0 \;\d x = \int_{-\infty}^\infty \left(\frac{1}{2}\dot{\phi}^2 + \frac{1}{2} \phi'^2 + U(\phi)\right)\;\d x.
\]
The conserved momentum is
\[
  P = - \int_{-\infty}^\infty T^0_1 \;\d x = - \int_{-\infty}^\infty \dot{\phi} \phi' \;\d x.
\]
Suppose we have a moving kink in the adiabatic approximation. Then
\[
  \phi = \tanh (x - a(t)).
\]
Doing another horrible integral, we find that
\[
  P = M \dot{a}.
\]
Note that $P$ is positive if $\dot{a}$ is positive.
\printindex
\end{document}
