\documentclass[a4paper]{article}

\def\npart {III}
\def\nterm {Lent}
\def\nyear {2017}
\def\nlecturer {A. J. Scholl}
\def\ncourse {Modular Forms and L-functions}
\def\nlectures {TTW.11}

% Imports
\ifx \nextra \undefined
  \usepackage[pdftex,
    hidelinks,
    pdfauthor={Dexter Chua},
    pdfsubject={Cambridge Maths Notes: Part \npart\ - \ncourse},
    pdftitle={Part \npart\ - \ncourse},
  pdfkeywords={Cambridge Mathematics Maths Math \npart\ \nterm\ \nyear\ \ncourse}]{hyperref}
  \title{Part \npart\ - \ncourse}
\else
  \usepackage[pdftex,
    hidelinks,
    pdfauthor={Dexter Chua},
    pdfsubject={Cambridge Maths Notes: Part \npart\ - \ncourse\ (\nextra)},
    pdftitle={Part \npart\ - \ncourse\ (\nextra)},
  pdfkeywords={Cambridge Mathematics Maths Math \npart\ \nterm\ \nyear\ \ncourse\ \nextra}]{hyperref}

  \title{Part \npart\ - \ncourse \\ {\Large \nextra}}
\fi

\author{Lectured by \nlecturer \\\small Notes taken by Dexter Chua}
\date{\nterm\ \nyear}

\usepackage{alltt}
\usepackage{amsfonts}
\usepackage{amsmath}
\usepackage{amssymb}
\usepackage{amsthm}
\usepackage{booktabs}
\usepackage{caption}
\usepackage{enumitem}
\usepackage{fancyhdr}
\usepackage{graphicx}
\usepackage{mathtools}
\usepackage{microtype}
\usepackage{multirow}
\usepackage{pdflscape}
\usepackage{pgfplots}
\usepackage{siunitx}
\usepackage{tabularx}
\usepackage{tikz}
\usepackage{tkz-euclide}
\usepackage[normalem]{ulem}
\usepackage[all]{xy}

\pgfplotsset{compat=1.12}

\pagestyle{fancyplain}
\lhead{\emph{\nouppercase{\leftmark}}}
\ifx \nextra \undefined
  \rhead{
    \ifnum\thepage=1
    \else
      \npart\ \ncourse
    \fi}
\else
  \rhead{
    \ifnum\thepage=1
    \else
      \npart\ \ncourse\ (\nextra)
    \fi}
\fi
\usetikzlibrary{arrows}
\usetikzlibrary{decorations.markings}
\usetikzlibrary{decorations.pathmorphing}
\usetikzlibrary{positioning}
\usetikzlibrary{fadings}
\usetikzlibrary{intersections}
\usetikzlibrary{cd}

\newcommand*{\Cdot}{\raisebox{-0.25ex}{\scalebox{1.5}{$\cdot$}}}
\newcommand {\pd}[2][ ]{
  \ifx #1 { }
    \frac{\partial}{\partial #2}
  \else
    \frac{\partial^{#1}}{\partial #2^{#1}}
  \fi
}

% Theorems
\theoremstyle{definition}
\newtheorem*{aim}{Aim}
\newtheorem*{axiom}{Axiom}
\newtheorem*{claim}{Claim}
\newtheorem*{cor}{Corollary}
\newtheorem*{defi}{Definition}
\newtheorem*{eg}{Example}
\newtheorem*{fact}{Fact}
\newtheorem*{law}{Law}
\newtheorem*{lemma}{Lemma}
\newtheorem*{notation}{Notation}
\newtheorem*{prop}{Proposition}
\newtheorem*{thm}{Theorem}

\renewcommand{\labelitemi}{--}
\renewcommand{\labelitemii}{$\circ$}
\renewcommand{\labelenumi}{(\roman{*})}

\let\stdsection\section
\renewcommand\section{\newpage\stdsection}

% Strike through
\def\st{\bgroup \ULdepth=-.55ex \ULset}

% Maths symbols
\newcommand{\bra}{\langle}
\newcommand{\ket}{\rangle}

\newcommand{\N}{\mathbb{N}}
\newcommand{\Z}{\mathbb{Z}}
\newcommand{\Q}{\mathbb{Q}}
\renewcommand{\H}{\mathbb{H}}
\newcommand{\R}{\mathbb{R}}
\newcommand{\C}{\mathbb{C}}
\newcommand{\Prob}{\mathbb{P}}
\renewcommand{\P}{\mathbb{P}}
\newcommand{\E}{\mathbb{E}}
\newcommand{\F}{\mathbb{F}}
\newcommand{\cU}{\mathcal{U}}
\newcommand{\RP}{\mathbb{RP}}
\newcommand{\CP}{\mathbb{CP}}

\newcommand{\ph}{\,\cdot\,}

\DeclareMathOperator{\sech}{sech}
\DeclareMathOperator{\cosech}{cosech}
\DeclareMathOperator{\cosec}{cosec}

\DeclareMathOperator{\covol}{covol}
\DeclareMathOperator{\vol}{vol}

\let\Im\relax
\let\Re\relax
\DeclareMathOperator{\Im}{Im}
\DeclareMathOperator{\Re}{Re}
\DeclareMathOperator{\im}{im}
\DeclareMathOperator{\image}{image}
\DeclareMathOperator{\Ann}{Ann}

\DeclareMathOperator*{\res}{res}
\DeclareMathOperator{\Res}{Res}
\DeclareMathOperator{\Ind}{Ind}

\DeclareMathOperator{\tr}{tr}
\DeclareMathOperator{\diag}{diag}
\DeclareMathOperator{\rank}{rank}
\DeclareMathOperator{\card}{card}
\DeclareMathOperator{\spn}{span}
\DeclareMathOperator{\adj}{adj}

\DeclareMathOperator{\erf}{erf}
\DeclareMathOperator{\erfc}{erfc}

\DeclareMathOperator{\ord}{ord}
\DeclareMathOperator{\Sym}{Sym}

\DeclareMathOperator{\sgn}{sgn}
\DeclareMathOperator{\orb}{orb}
\DeclareMathOperator{\stab}{stab}
\DeclareMathOperator{\ccl}{ccl}

\DeclareMathOperator{\lcm}{lcm}
\DeclareMathOperator{\hcf}{hcf}

\DeclareMathOperator{\Int}{Int}
\DeclareMathOperator{\id}{id}

\DeclareMathOperator{\betaD}{beta}
\DeclareMathOperator{\gammaD}{gamma}
\DeclareMathOperator{\Poisson}{Poisson}
\DeclareMathOperator{\binomial}{binomial}
\DeclareMathOperator{\multinomial}{multinomial}
\DeclareMathOperator{\Bernoulli}{Bernoulli}
\DeclareMathOperator{\like}{like}

\DeclareMathOperator{\var}{var}
\DeclareMathOperator{\cov}{cov}
\DeclareMathOperator{\bias}{bias}
\DeclareMathOperator{\mse}{mse}
\DeclareMathOperator{\corr}{corr}

\DeclareMathOperator{\otp}{otp}
\DeclareMathOperator{\dom}{dom}

\DeclareMathOperator{\Root}{Root}
\DeclareMathOperator{\supp}{supp}
\DeclareMathOperator{\rel}{rel}
\DeclareMathOperator{\Hom}{Hom}
\DeclareMathOperator{\Aut}{Aut}
\DeclareMathOperator{\Gal}{Gal}
\DeclareMathOperator{\Mat}{Mat}
\DeclareMathOperator{\End}{End}
\DeclareMathOperator{\Char}{char}
\DeclareMathOperator{\ev}{ev}
\DeclareMathOperator{\St}{St}
\DeclareMathOperator{\Lk}{Lk}
\DeclareMathOperator{\disc}{disc}
\DeclareMathOperator{\Isom}{Isom}
\DeclareMathOperator{\length}{length}
\DeclareMathOperator{\energy}{energy}
\DeclareMathOperator{\area}{area}
\DeclareMathOperator{\Syl}{Syl}
\DeclareMathOperator{\cl}{cl}
\DeclareMathOperator{\fix}{fix}

\newcommand{\GL}{\mathrm{GL}}
\newcommand{\SL}{\mathrm{SL}}
\newcommand{\PGL}{\mathrm{PGL}}
\newcommand{\PSL}{\mathrm{PSL}}
\newcommand{\PSU}{\mathrm{PSU}}
\newcommand{\Or}{\mathrm{O}}
\newcommand{\SO}{\mathrm{SO}}
\newcommand{\U}{\mathrm{U}}
\newcommand{\SU}{\mathrm{SU}}

\renewcommand{\d}{\mathrm{d}}
\newcommand{\D}{\mathrm{D}}

\tikzset{->/.style = {decoration={markings,
                                  mark=at position 1 with {\arrow[scale=2]{latex'}}},
                      postaction={decorate}}}
\tikzset{<-/.style = {decoration={markings,
                                  mark=at position 0 with {\arrowreversed[scale=2]{latex'}}},
                      postaction={decorate}}}
\tikzset{<->/.style = {decoration={markings,
                                   mark=at position 0 with {\arrowreversed[scale=2]{latex'}},
                                   mark=at position 1 with {\arrow[scale=2]{latex'}}},
                       postaction={decorate}}}
\tikzset{->-/.style = {decoration={markings,
                                   mark=at position #1 with {\arrow[scale=2]{latex'}}},
                       postaction={decorate}}}
\tikzset{-<-/.style = {decoration={markings,
                                   mark=at position #1 with {\arrowreversed[scale=2]{latex'}}},
                       postaction={decorate}}}

\tikzset{circ/.style = {fill, circle, inner sep = 0, minimum size = 3}}
\tikzset{mstate/.style={circle, draw, blue, text=black, minimum width=0.7cm}}

\definecolor{mblue}{rgb}{0.2, 0.3, 0.8}
\definecolor{morange}{rgb}{1, 0.5, 0}
\definecolor{mgreen}{rgb}{0.1, 0.4, 0.2}
\definecolor{mred}{rgb}{0.5, 0, 0}

\def\drawcirculararc(#1,#2)(#3,#4)(#5,#6){%
    \pgfmathsetmacro\cA{(#1*#1+#2*#2-#3*#3-#4*#4)/2}%
    \pgfmathsetmacro\cB{(#1*#1+#2*#2-#5*#5-#6*#6)/2}%
    \pgfmathsetmacro\cy{(\cB*(#1-#3)-\cA*(#1-#5))/%
                        ((#2-#6)*(#1-#3)-(#2-#4)*(#1-#5))}%
    \pgfmathsetmacro\cx{(\cA-\cy*(#2-#4))/(#1-#3)}%
    \pgfmathsetmacro\cr{sqrt((#1-\cx)*(#1-\cx)+(#2-\cy)*(#2-\cy))}%
    \pgfmathsetmacro\cA{atan2(#2-\cy,#1-\cx)}%
    \pgfmathsetmacro\cB{atan2(#6-\cy,#5-\cx)}%
    \pgfmathparse{\cB<\cA}%
    \ifnum\pgfmathresult=1
        \pgfmathsetmacro\cB{\cB+360}%
    \fi
    \draw (#1,#2) arc (\cA:\cB:\cr);%
}
\newcommand\getCoord[3]{\newdimen{#1}\newdimen{#2}\pgfextractx{#1}{\pgfpointanchor{#3}{center}}\pgfextracty{#2}{\pgfpointanchor{#3}{center}}}

\def\Xint#1{\mathchoice
   {\XXint\displaystyle\textstyle{#1}}%
   {\XXint\textstyle\scriptstyle{#1}}%
   {\XXint\scriptstyle\scriptscriptstyle{#1}}%
   {\XXint\scriptscriptstyle\scriptscriptstyle{#1}}%
   \!\int}
\def\XXint#1#2#3{{\setbox0=\hbox{$#1{#2#3}{\int}$}
     \vcenter{\hbox{$#2#3$}}\kern-.5\wd0}}
\def\ddashint{\Xint=}
\def\dashint{\Xint-}


\begin{document}
\maketitle
{\small
\setlength{\parindent}{0em}
\setlength{\parskip}{1em}
Modular Forms are classical objects that appear in many areas of mathematics, from number theory to representation theory and mathematical physics. Most famous is, of course, the role they played in the proof of Fermat's Last Theorem, through the conjecture of Shimura-Taniyama-Weil that elliptic curves are modular. One connection between modular forms and arithmetic is through the medium of $L$-functions, the basic example of which is the Riemann Riemann $\zeta$-function. We will discuss various types of $L$-function in this course and give arithmetic applications.

\subsubsection*{Pre-requisites}
Prerequisites for the course are fairly modest; from number theory, apart from basic elementary notions, some knowledge of quadratic fields is desirable. A fair chunk of the course will involve (fairly 19th-century) analysis, so we will assume the basic theory of holomorphic functions in one complex variable, such as are found in a first course on complex analysis (eg. the 2nd year Complex Analysis course of the Tripos).
}
\tableofcontents

\setcounter{section}{-1}
\section{Introduction}
One of the big problems in number theory is the so-called Langland's programme, which is relates ``arithmetic objects'' such as representations of the Galois group and elliptic curves over $\Q$, with ``analytic objects'' such as modular forms and more generally automorphic forms and representations.

\begin{eg}
  $y^2 + y = x^3 - x$ is an elliptic curve, and we can associate to it the function
  \[
    f(z) = q\prod_{n \geq 1} (1 - q^n)^2 (1 - q^{11n})^2 = \sum_{n = 1}^\infty a_n q^n,\quad q= e^{2\pi i z},
  \]
  where we assume $\Im z > 0$, so that $|q| < 1$. The relation between these two objects is that the number of points of $E$ over $\F_p$ is equal to $1 + p - a_p$, for $p \not= 11$. This strange function $f$ is a modular form, and is actually cooked up from the slightly easier function
  \[
    \eta(z) = q^{1/24} \prod_{n = 1}^\infty (1 - q^n)
  \]
  by
  \[
    f(z) = (\eta(z)\eta(11z))^2.
  \]
  This function $\eta$ is called the \emph{Dedekind eta function}, and is one of the simplest example modular forms, in the sense that we can write it down easily. This satisfies the following two identities:
  \[
    \eta(z + 1) = e^{i \pi/12}\eta(z),\quad \eta\left(\frac{-1}{z}\right) = \sqrt{\frac{z}{i}} \eta(z).
  \]
  The first is clear, and the second takes some work to show. These transformation laws are exactly what makes this thing a modular form.

  Another way to link $E$ and $f$ is via the \emph{$L$-series}
  \[
    L(E, s) = \sum_{n = 1}^\infty \frac{a_n}{n^s},
  \]
  which is a generalization of the Riemann $\zeta$-function
  \[
    \zeta(s) = \sum_{n = 1}^\infty \frac{1}{n^s}.
  \]
\end{eg}
We are in fact not going to study elliptic curves, as there is another course on that, but we are going to study the modular forms and these $L$-series. We are going to do this in a fairly classical way, without using algebraic number theory.

\section{Some useful tools}
\subsection{Characters of abelian groups}
We start by going through the theory of characters of abelian groups, which is basically what Fourier analysis is about. These will come up useful later on.

\begin{defi}[Character]\index{character}
  Let $G$ be an abelian topological group. A (unitary) \emph{character} of $G$ is a continuous homomorphism $\chi: G \to \U(1)$, where $\U(1) = \{z \in \C \mid |z| = 1\}$.

  The set of such characters of $G$ forms a group under multiplication, written $\hat{G}$\index{$\hat{G}$} is called the \term{character group} (or \term{Pontryagin dual}) of $G$.
\end{defi}

\begin{eg}
  Let $G = \R$. For $y \in \R$, we let $\chi_y; \R \to \U(1)$ be
  \[
    \chi_y(x) = e^{2\pi i xy}.
  \]
  For each $y \in \R$, this is a character, and all characters are of this form, and $\hat{\R} \cong \R$ under this correspondence.
\end{eg}

\begin{eg}
  Take $G = \Z$ with the discrete topology. A character is uniquely determined by the image of $1$, and any element of $\U(1)$ can be the image of $1$. So we have $\hat{G} \cong \U(1)$.
\end{eg}

\begin{eg}
  Take $G = \Z/N\Z$. Then the character is again determined by the image of $1$, and the allowed values are exactly the $N$th roots of unity. So
  \[
    \hat{G} \cong \mu_N = \{\zeta \in \C^\times: \zeta^N = 1\}.
  \]
\end{eg}

\begin{eg}
  Let $G = G_1 \times G_2$. Then $\hat{G} \cong \hat{G}_1 \times \hat{G}_2$. So, for example, $\hat{\R^n} = \R^n$. Under this correspondence, a $y \in \R^n$ corresponds to
  \[
    \chi_y(x) = e^{2\pi x\cdot y}.
  \]
\end{eg}

\begin{eg}
  Take $G = \R^\times$. We have
  \[
    G \cong \{\pm 1\} \times \R^\times_{>0} \cong \{\pm 1\} \times \R,
  \]
  where we have an isomorphism between $\R^{\times}_{> 0} \cong \R$ by the exponential map. So we have
  \[
    \hat{G} \cong \Z/2\Z \times \R.
  \]
  Explicitly, given $(\varepsilon, \sigma) \in \Z/2\Z \times \R$, then character is given by
  \[
    x \mapsto \sgn(x)^\varepsilon |x|^{i\sigma}.
  \]
\end{eg}

Note that $\hat{G}$ has a natural topology for which the evaluation maps $(\chi \in \hat{G}) \mapsto \chi(g) \in \U(1)$ are all continuous for all $g$. Evaluation gives us a map $G \to \hat{\hat{G}}$. This is an isomorphism if $G$ is locally compact. This is called \term{Pontryagin duality}. Since this is a course on number theory, and not topological groups, we will not prove this.

\begin{prop}
  Let $G$ be a finite abelian group. Then $|\hat{G}| = |G|$, and $G$ and $\hat{G}$ are in fact isomorphic, but not canonically.
\end{prop}

\begin{proof}
  By the classification of finite-abelian groups, we know $G$ is a product of cyclic groups. So it suffices to prove the result for cyclic groups $\Z/N\Z$, and the result is clear since
  \[
    \widehat{\Z/N\Z} = \mu_N \cong \Z/N\Z.
  \]
\end{proof}

\subsection{Fourier transforms}
In the early years of the undergraduate curriculum, we learnt about the Fourier transformed, and then planned to completely forget it, and hoped that we will never meet it again. However, we shall remind ourselves of what the Fourier transform is.

\begin{defi}[Fourier transform]\index{Fourier transform}
  Let $f: \R \to \C$ be an $L^1$ function, ie. $\int |f| \;\d x < \infty$. The \emph{Fourier transform} is
  \[
    \hat{f}(y) = \int_{-\infty}^\infty e^{-2\pi i xy} f(x) \;\d x = \int_{-\infty}^\infty \chi_y(x)^{-1} f(x) \;\d x.
  \]
  This is a bounded and continuous function on $\R$.
\end{defi}
Really, the Fourier transform is not a function on $\R$, but a function on the character group of $\R$, but they are the same thing.

We will only be interested in a very special class of functions $f$, for which the analytic properties are all completely trivial.
\begin{defi}[Schwarz space]\index{Schwarz space}\index{$\mathcal{S}(\R)$}
  The \term{Schwarz space} is defined by
  \[
    \mathcal{S}(\R) = \{f \in C^\infty(\R) : x^n f^{(k)}(x) \to 0\text{ as }x \to \infty\text{ for all }k, n\geq 0\}.
  \]
\end{defi}

\begin{eg}
  The function
  \[
    f(x) = e^{-\pi x^2}.
  \]
  is in the Schwarz space.
\end{eg}

\begin{prop}
  If $f \in \mathcal{S}(\R)$, then $\hat{f} \in \mathcal{S}(\R)$, and the \term{Fourier inversion formula}
  \[
    \hat{\hat{f}} = f(-x)
  \]
  holds.
\end{prop}

The same is true for functions on $\R^n$, where we take
\[
  \hat{f}(y) = \int_{\R^n} e^{-2\pi i x\cdot y} f(x) \;\d x = \int_{\R^n} \chi_y(x)^{-1} f(x)\;\d x.
\]
We've also seen things called Fourier series. They are another version of this, where do the Fourier transform on $G = \R/\Z$ instead of $G$. For $n \in \Z$, we let $\chi_n \in \hat{G}$ by
\[
  \chi_n(x) = e^{2\pi i nx}.
\]
These are exactly all the elements of $\hat{G}$, and $\hat{G} \cong \Z$. We then define the Fourier series of a periodic function $f: \R/\Z \to \C$ by
\[
  c_n(f) = \int_0^1 e^{-2\pi i n x} f(x) \;\d x = \int_{\R/\Z} \chi_n(x)^{-1} f(x) \;\d x.
\]
Again, under suitable regularity conditions on $f$, eg. if $f \in C^\infty(\R/\Z)$, we have
\[
  f(x) = \sum_{n \in \Z} c_n(f) e^{2\pi i nx} = \sum_{n \in \Z \cong \hat{G}} c_n(f) \chi_n(x).
\]
This is the Fourier inversion formula for $G = \R/\Z$.

Finally, in the case when $G = \Z/N\Z$, given a function $f: \Z/N\Z \to \C$, we define $\hat{f}: \mu_N \to \C$ by
\[
  \hat{f}(\zeta) = \sum_{a \in \Z/N\Z} \zeta^{-a} f(a).
\]
\begin{prop}
  For a function $f: \Z/N\Z \to \C$, we have
  \[
    f(x) = \frac{1}{N} \sum_{\zeta \in \mu_N} \zeta^x \hat{f}(\zeta).
  \]
\end{prop}

\begin{proof}
  We see that both sides are linear in $f$, and we can write each function $f$ as
  \[
    f = \sum_{a \in \Z/N\Z} f(a) \delta_a,
  \]
  where
  \[
    \delta_a(x) =
    \begin{cases}
      1 & x = a\\
      0 & x \not= a
    \end{cases}.
  \]
  So we wlog $f = \delta_a$. Thus we have
  \[
    \hat{f}(\zeta) = \zeta^{-a},
  \]
  and the RHS gives
  \[
    \frac{1}{N} \sum_{\zeta \in \mu_N} \zeta^{x - a}.
  \]
  We have to see that this is equal to $\delta_a$. This follows from the fact that
  \[
    \sum_{\zeta \in \mu_N} \zeta^k =
    \begin{cases}
      N & k \equiv 0 \pmod N\\
      0 & \text{otherwise}
    \end{cases}.
  \]
\end{proof}
The general picture is for any locally compact abelian group $G$, there is a canonical way to integrate functions defined on $G$ given by the \term{Haar measure}, which is a translation-invariant measure. For example, on $G = \R$, this is the usual Lebesgue measure, and if $G$ is discrete, this is just the counting measure, and
\[
  \int f = \sum_{g \in G} f(g).
\]
If $G = \R_{> 0}^\times$, then the integral is given by
\[
  \int f(x) \frac{\d x}{x},
\]
since $\frac{\d x}{x}$ is invariant under multiplication of $x$ by a constant.

Then given any $f \in L^1(G)$, we have
\[
  \hat{f}(X) = \int_G \chi(g)^{-1}f(g) \;\d g.
\]
Then one shows that for suitable $f$, we have $\hat{\hat{f}}(g) = C f(-g)$, using the canonical isomorphism $G \to \hat{\hat{G}}$, where $C$ is some constant independent of $f$. This constant is necessary, because the measure is only defined up to a multiplicative constant.

\begin{thm}[Poisson summation formula]\index{Poisson summation formula}
  Let $f \in \mathcal{S}(\R^n)$. Then
  \[
    \sum_{a \in \Z^n} f(a) = \sum_{b \in \Z^n} \hat{f}(b).
  \]
\end{thm}
This is in fact true for more general functions $f$, at the expense of doing more analysis. The proof involves connecting to the Fourier series.

\begin{proof}
  Let
  \[
    g(x) = \sum_{a \in \Z^n} f(x + a).
  \]
  This is now a function that is invariant under translation of $\Z^n$. It is easy to check this is a well-defined $C^\infty$ function on $\R^n/\Z^n$, and so has a Fourier series. We write
  \[
    g(x) = \sum_{b \in \Z^n} c_b(g) e^{2pi i b \cdot x},
  \]
  with
  \[
    c_b(g) = \int_{\R^n/\Z^n} e^{-2\pi i b\cdot x} g(x) \;\d x = \sum_{a \in \Z^n} \int_{[0, 1]^n} e^{2\pi i b\cdot x} f(x + a) \;\d x.
  \]
  We can then do a change of variables $x \mapsto x - a$, which does not change the exponential term, and get that
  \[
    c_b(g) = \int_{\R^n} e^{-2\pi i b \cdot x} f(x) \;\d x = \hat{f}(b).
  \]
  Finally, we have
  \[
    \sum_{a \in \Z^n} f(a) = g(0) = \sum_{b \in \Z^n} c_b(x) = \sum_{b \in \Z^n} \hat{f}(b).
  \]
\end{proof}

\section{Mellin transform and \texorpdfstring{$\Gamma$}{Gamma}-function}
There is one more analytic thing we have to do, which involves the Mellin transform. The general setting is that we have a function $f: \R_{>0} \to \C$.

\begin{defi}[Mellin transform]\index{Mellin transform}\index{$M(f, s)$}
  Let $f: \R_{>0} \to\C$ be a function. We define
  \[
    M(f, s) = \int_0^\infty y^s f(y) \frac{\d y}{y}.
  \]
  with the domain specified by the following lemma:
\end{defi}

\begin{lemma}
  Suppose $f$ is rapidly decreasing at $\infty$, so that $y^N f(y) \to 0$ as $y \to \infty$ for all $N$, and has ``moderate growth'' at $0$, so that there exists $m$ such that $|y^m f(y)|$ is bounded as $y \to 0$. Then $M(f, s)$ converges and is an analytic function of $s$ for $\Re(s) > m$.
\end{lemma}

\begin{proof}
  We know that for any $0 < r < R < \infty$, the integral
  \[
    \int_r^R y^s f(y) \frac{\d y}{y}
  \]
  is analytic for all $s$ since $f$ is continuous. Also, we know $\int_R^\infty \to 0$ as $R \to \infty$ uniformly on compact subsets of $\C$, and on the other the integral $\int_0^r$ as $r \to 0$ converges uniformly on compact subsets of $\{s \in \C: \Re(s) > m\}$.
\end{proof}

This transform might seem a bit strange, but we can think of this as an analytic continuation of the Fourier transform. Suppose we are in the rather good situation that
\[
  \int_0^\infty |f| \frac{\d y}{y} < \infty.
\]
In practice, this will hardly ever be the case, but this is a good place to start exploring. In this case, the integral actually converges on $i\R$, and equals the Fourier transform of $f \in L^1(G) = L^1(\R^\times_{>0})$. Indeed, we find
\[
  \hat{G} = \{y \mapsto y^{i\sigma} : \sigma \in \R\}s,.
\]
and $\frac{\d y}{y}$ is just the invariant measure on $G$. So the formula for the Mellin transform is exactly the formula for the Fourier transform, and we can view the Mellin transform as an analytic continuation of the Fourier transform.

We now move on to explore properties of the Mellin transform. When we make a change of variables $y \leftrightarrow \alpha y$, by inspection of the formula, we find
\[
  M(f(\alpha y), s) = \alpha^{-s} M(f, s)
\]
for $\alpha > 0$.

\begin{eg}
  Consider
  \[
    f(y) = e^{-y}.
  \]
  Then
  \[
    M(f, s) = \int_0^\infty e^{-y} y^s \frac{\d y}{y} = \Gamma(s).
  \]
  This is the \term{Gamma function}\index{$\Gamma$ function}. This is analytic for $\Re(s) > 0$.
\end{eg}

If we just integrate by parts, we find
\[
  \int_0^\infty e^{-y} y^{s - 1}\;\d y = \left[e^{-y} \frac{y^s}{s} \right]_0^\infty + \frac{1}{s}\int_0^\infty e^{-y} y^s \;\d y.
\]
So we find that
\begin{prop}
  \[
    s \Gamma(s) = \Gamma(s + 1).
  \]
\end{prop}
Moreover, we can compute
\[
  \Gamma(1) = \int_0^\infty e^{-y}\;\d y = 1.
\]
So we get
\begin{prop}
  For an integer $n \geq 1$, we have
  \[
    \Gamma(n) = (n - 1)!.
  \]
\end{prop}
In general, we have
\[
  \Gamma(s) = \frac{1}{s (s + 1) \cdots (s + N - 1)} \Gamma(s + N).
\]
We can thus use this to define an analytic continuation of $\Gamma(s)$ to $\{\Re(s) > -N\}$, with simple poles at $0, 1, \cdots, 1 - N$, with
\[
  \res_{s = 1 - N} \Gamma(s) = \frac{(-1)^{N - 1}}{(N - 1)!}.
\]
Here are two facts about the $\Gamma$ function that we are not going to prove, because, even if the current experience might suggest otherwise, this is not an analysis course.
\begin{prop}\leavevmode
  \begin{enumerate}
    \item The \term{Weierstrass product}: We have
      \[
        \Gamma(s)^{-1} = e^{\gamma s} s \prod_{n \geq 1} \left(1 + \frac{s}{n}\right) e^{-s/n}
      \]
      for all $s \in \C$. In particular, $\Gamma(s)$ is never zero. Here $\gamma$ is the \term{Euler-Mascheroni constant}, given by
      \[
        \gamma = \lim_{n \to \infty}\left(1 + \frac{1}{2} + \cdots + \frac{1}{n} - \log n\right).
      \]
    \item \emph{Duplication and reflection formulae}\index{duplication formula}\index{reflection formula}:
      \[
        \pi^{\frac{1}{2}} \Gamma(2s) = 2^{s - 1} \Gamma(s) \Gamma\left(s + \frac{1}{2}\right)
      \]
      and
      \[
        \Gamma(s) \Gamma(1 - s) = \frac{\pi}{\sin \pi z}.
      \]
  \end{enumerate}
\end{prop}

One useful property of the Mellin transform is that it converts power series into \emph{Dirichlet series}. Let
\[
  f(z) = \sum_{n \geq 0} a_n e^{2\pi i n z}.
\]
where $|a_n| < c n^K$ for some $c, K$, say. Then
\[
  f(iy) = \sum a_n e^{-2\pi n y}.
\]
This will converge for all $y > 0$ since the $a_n$ only have polynomial growth. Then if $\Re(s) > \max(0, K + 1)$ (so that the series we are going to write down is absolutely convergent), we find
\begin{align*}
  (2\pi)^{-s} \Gamma(s)\sum_{n = 1}^\infty \frac{a_n}{n^s} &= \sum_{n = 1}^\infty a_n (2\pi n)^{-s} M(e^{-y}, s)\\
  &= \sum_{n = 1}^\infty M(e^{-2 \pi n y}, s) \\
  &= M(f(iy), s).
\end{align*}
The series $\sum a_n/n^s$ is known as a \term{Dirichlet series}. The first example of such is the \emph{Riemann zeta function}
\[
  \zeta(s) = \sum_{n \geq 1} \frac{1}{n^s}.
\]
\section{Riemann \texorpdfstring{$\zeta$}{zeta}-function}
\begin{defi}[Riemann $\zeta$-function]\index{Riemann $\zeta$-function}
  The \emph{Riemann $\zeta$-function} is defined by
  \[
    \zeta(s) = \sum_{n \geq 1} \frac{1}{n^s}
  \]
  for $\Re(s) > 1$.
\end{defi}

\begin{prop}[Euler product formula]\index{Euler index formula}
  We have
  \[
    \zeta(s) = \prod_{p \text{ prime}} \frac{1}{1 - p^{-s}}.
  \]
\end{prop}

\begin{proof}
  Euler's proof was purely formal, without worrying about convergence. We have
  \[
    \prod_{p\text{ prime}} \frac{1}{1 - p^{-s}} = \prod_p (1 + p^{-s} + (p^2)^{-s} + \cdots) = \sum_{n \geq 1} n^{-s},
  \]
  where the last equality follows by unique factorization in $\Z$. However, to prove this properly, we need to be a bit more careful and make sure things converge.

  Saying the infinite product $\prod_p$ convergence is the same as saying $\sum p^{-s}$ converges, by basic analysis, which is okay since we know $\zeta(s)$ converges absolutely when $\Re(s) > 1$. Then we can look at the difference
  \begin{align*}
    \zeta(s) - \prod_{p \leq X} \frac{1}{1 - p^{-s}} &= \zeta(s) - \prod_{p \leq X} (1 + p^{-s} + p^{-2s} + \cdots)\\
    &= \prod_{n \in \mathcal{N}_X} n^{-s},
  \end{align*}
  where $\mathcal{N}_X$ is the set of all $n \geq 1$ such that at least one prime factor is $\geq X$. In particular, we know
  \[
    \left|\zeta(s) - \prod_{p \leq X} \frac{1}{1 - p^{-s}}\right| \leq \sum_{n \geq X} |n^{-s}| \to 0
  \]
  as $X \to \infty$.
\end{proof}
The Euler product formula is the beginning of the connection between the $\zeta$-function and the distribution of primes.

As the product converges for $\Re(s) > 1$, this shows that $\zeta(s) \not= 0$ for all $s$ when $\Re(s) > 1$. It turns out the non-vanishing of $\Re(s)$ at other places has got to do with number theory.

Next, we will give an analytic continuation for $\zeta(s)$ by writing it as a Mellin transform.
\begin{thm}
  If $\Re(s) > 1$, then
  \[
    (2\pi)^{-s} \Gamma(s)  \zeta(s) = \int_0^\infty \frac{y^s}{e^{2\pi y} - 1} \frac{\d y}{y} = M(f, s),
  \]
  where
  \[
    f(y) = \frac{1}{e^{2 \pi y} - 1}.
  \]
\end{thm}
This is just a simple computation

\begin{proof}
  We can write
  \[
    f(y) = \frac{e^{-2\pi y}}{1 - e^{-2 \pi y}} = \sum_{n \geq 1} e^{-2\pi n y}
  \]
  for $y > 0$.

  As $y \to 0$, we find
  \[
    f(y) \sim \frac{1}{2\pi y}.
  \]
  So when $\Re(s) > 1$, the Mellin transform converges, and equals
  \[
    \sum_{n \geq 1} M(e^{-2\pi n y}, s) = \sum_{n \geq 1} (2\pi n)^{-2} M(e^{-y}, s) = (2\pi)^{-s} \Gamma(s) \zeta(s).
  \]
\end{proof}

\begin{cor}
  $\zeta(s)$ has a meromorphic continuation to $\C$ with a simple pole at $s = 1$ as its only singularity, and
  \[
    \res_{s = 1}\zeta(s) = 1.
  \]
\end{cor}

\begin{proof}
  We can write
  \[
    M(f, s) = M_0 + M_\infty = \left(\int_0^1 + \int_1^\infty\right) \frac{y^s}{e^{2\pi y} - 1} \frac{\d y}{y}.
  \]
  The second integral $M_\infty$ is convergent for all $s \in \C$, hence defines a holomorphic function.

  We can expand
  \[
    f(y) = \sum_{n = -1}^{N - 1} c_n y^n + y^N g_N(u)
  \]
  for some $g \in C^\infty(\R)$ as $f$ has a simple pole at $y = 0$, and
  \[
    c_{-1} = \frac{1}{2\pi}.
  \]
  So we have
  \begin{align*}
    M_0 &= \sum_{n = -1}^{N - 1} c_n \int_0^1 y^{n + s - 1} \;\d y + \int_0^N y^{N + s - 1} g_N(y) \;\d y\\
    &= \sum_{n = -1}^{N - 1} \frac{c_n}{s + n} y^{s + n} + \int_0^1 g_N(y) y^{s + N - 1} \;\d y.
  \end{align*}
  When deriving this, we assumed $\Re(s) > 1$. But now this formula makes sense for $\Re(s) > -N$. Thus we have found a meromorphic continuation of
  \[
    (2\pi)^{-s} \Gamma(s) \zeta(s)
  \]
  to $\{\Re(s) > N\}$, with at worst simple poles at $s = 1 - N, 2 - N,  \cdots, 0, 1$, and the residue at $s = n$ is $c_n$. Also, we know $\Gamma(s)$ has a simple pole at $s = 0, -1, -2, \cdots$. Also, we know
  \[
    \res_{s = 1 - n} \Gamma(s) = \frac{(-)^{n - 1}}{(n - 1)!}.
  \]
  So $\zeta(s)$ is analytic at $s = 0, -1, -2, \cdots$. Since $c_{-1} = \frac{1}{2\pi}$ and $\Gamma(1) = 1$, we get
  \[
    \res_{s = 1} \zeta(s) = 1.
  \]
\end{proof}

\begin{defi}[Bernoulli numbers]\index{Bernoulli numbers}\index{$B_n$}
  The \emph{Bernoulli numbers} are defined by a generating function
  \[
    \sum_{n = 0}^\infty B_n \frac{t^n}{n!} = \frac{t}{e^t - 1} = \left(1 + \frac{t}{2!} + \frac{t^2}{3!} + \cdots\right)^{-1}.
  \]
\end{defi}

Clearly, all Bernoulli numbers are rational. We have
\[
  B_0 = 1, \quad B_1 = -\frac{1}{2}, \cdots.
\]
The first thing to note about this is the following:
\begin{prop}
  We have $B_n = 0$ if $n$ is odd and $n \geq 3$.
\end{prop}

\begin{proof}
  Consider
  \[
    f(t) = \frac{t}{e^t - 1} + \frac{t}{2} = \sum_{n \geq 0, n \not= 1} B_n \frac{t^n}{n!}.
  \]
  We find that
  \[
    f(t) = \frac{t}{2} \frac{e^t + 1}{e^t - 1} = f(-t).
  \]
  So all the odd coefficients must vanish.
\end{proof}

\begin{cor}
  We have
  \[
    \zeta(0) = B_1 = -\frac{1}{2},\quad \zeta(1 - n)= - \frac{B_n}{n}
  \]
  for $n > 1$. In particular, for all $n \geq 1$ integer, we know $\zeta(1 - n) \in \Q$ and vanishes if $n > 1$ is odd.
\end{cor}

\begin{proof}
  We know
  \[
    (2\pi)^{-s} \Gamma(s) \zeta(s)
  \]
  has a simple pole at $s = 1 - n$, and the residue is $c_{n - 1}$, where
  \[
    \frac{1}{e^{2\pi y} - 1} = \sum_{n \geq -1} c_n y^n.
  \]
  So we know
  \[
    c_{n - 1} = (2\pi)^{n - 1} \frac{B_n}{n!}.
  \]
  We also know that
  \[
    \res_{s = 1 - n} \Gamma(s) = \frac{(-1)^{n - 1}}{(n - 1)!},
  \]
  we get that
  \[
    \zeta(1 - n) = (-1)^{n - 1} \frac{B_n}{n}.
  \]
  If $n = 1$, then this gives $-\frac{1}{2}$. If $n$ is odd but $> 1$, then this vanishes. If $n$ is even, then this is $-\frac{B_n}{n}$, as desired.
\end{proof}

Note that there are many theorems and conjectures concerning the values at integers of $L$-functions, which are Dirichlet series like the $\zeta$-function, which relates to subtle number-theoretic quantities. For example, the numbers $B_n$ are closely related to the class numbers of the cyclotomic fields $\Q(\zeta_p)$. This is also related to early (partial) proofs of Fermat's last theorem. Also, things like the Birch--Swinnerton-Dyer conjecture on elliptic curves are also related to these things.

We are going to look at another way of representing the $\zeta$-function, and this will give us more interesting properties of the $\zeta$-function. Let
\[
  \Theta(y) = \sum_{n \in \Z} e^{-\pi n^2 y} = 1 + 2 \sum_{n \geq 1} e^{-\pi n^2 y}.
\]
This is convergent for for $y > 0$. So we can write
\[
  \Theta(y) = \vartheta(iy),
\]
where
\[
  \vartheta(z) = \sum_{n \in \Z} e^{\pi i n^2 z},
\]
which is analytic for $|e^{\pi i z}| < 1$, ie. $\Im(z) > 0$. This is \term{Jacobi's $\mathcal{\Theta}$-function}. This function is also important in algebraic geometry, representation theory, and even applied mathematics. But we will just use it for number theory. We note that
\[
  \Theta(y) \to 1
\]
as $y \to \infty$, so we can't take its Mellin transform. What we \emph{can} do is
\begin{prop}
  \[
    M\left(\frac{\Theta(y) - 1}{2}, s\right) = \pi^{-s/2} \Gamma\left(\frac{s}{2}\right) \zeta(s).
  \]
\end{prop}

The proof is again just do it.
\begin{proof}
  The left hand side is
  \[
    \sum_{n \geq 1} M\left(e^{-\pi n^2 y}, \frac{s}{2}\right) = \sum_{n \geq 1} (\pi n^2) ^{-s/2} M\left(e^{-y}, \frac{s}{2}\right) = \pi^{-s/2} \Gamma\left(\frac{s}{2}\right) \zeta(s).
  \]
\end{proof}
We will use this to prove a \emph{functional equation} relating $\zeta(s)$ and $\zeta(1 - s)$.

\printindex
\end{document}
