\documentclass[a4paper]{article}

\def\npart {III}
\def\nterm {Lent}
\def\nyear {2017}
\def\nlecturer {A. J. Scholl}
\def\ncourse {Modular Forms and L-functions}

% Imports
\ifx \nextra \undefined
  \usepackage[pdftex,
    hidelinks,
    pdfauthor={Dexter Chua},
    pdfsubject={Cambridge Maths Notes: Part \npart\ - \ncourse},
    pdftitle={Part \npart\ - \ncourse},
  pdfkeywords={Cambridge Mathematics Maths Math \npart\ \nterm\ \nyear\ \ncourse}]{hyperref}
  \title{Part \npart\ - \ncourse}
\else
  \usepackage[pdftex,
    hidelinks,
    pdfauthor={Dexter Chua},
    pdfsubject={Cambridge Maths Notes: Part \npart\ - \ncourse\ (\nextra)},
    pdftitle={Part \npart\ - \ncourse\ (\nextra)},
  pdfkeywords={Cambridge Mathematics Maths Math \npart\ \nterm\ \nyear\ \ncourse\ \nextra}]{hyperref}

  \title{Part \npart\ - \ncourse \\ {\Large \nextra}}
\fi

\author{Lectured by \nlecturer \\\small Notes taken by Dexter Chua}
\date{\nterm\ \nyear}

\usepackage{alltt}
\usepackage{amsfonts}
\usepackage{amsmath}
\usepackage{amssymb}
\usepackage{amsthm}
\usepackage{booktabs}
\usepackage{caption}
\usepackage{enumitem}
\usepackage{fancyhdr}
\usepackage{graphicx}
\usepackage{mathtools}
\usepackage{microtype}
\usepackage{multirow}
\usepackage{pdflscape}
\usepackage{pgfplots}
\usepackage{siunitx}
\usepackage{tabularx}
\usepackage{tikz}
\usepackage{tkz-euclide}
\usepackage[normalem]{ulem}
\usepackage[all]{xy}

\pgfplotsset{compat=1.12}

\pagestyle{fancyplain}
\lhead{\emph{\nouppercase{\leftmark}}}
\ifx \nextra \undefined
  \rhead{
    \ifnum\thepage=1
    \else
      \npart\ \ncourse
    \fi}
\else
  \rhead{
    \ifnum\thepage=1
    \else
      \npart\ \ncourse\ (\nextra)
    \fi}
\fi
\usetikzlibrary{arrows}
\usetikzlibrary{decorations.markings}
\usetikzlibrary{decorations.pathmorphing}
\usetikzlibrary{positioning}
\usetikzlibrary{fadings}
\usetikzlibrary{intersections}
\usetikzlibrary{cd}

\newcommand*{\Cdot}{\raisebox{-0.25ex}{\scalebox{1.5}{$\cdot$}}}
\newcommand {\pd}[2][ ]{
  \ifx #1 { }
    \frac{\partial}{\partial #2}
  \else
    \frac{\partial^{#1}}{\partial #2^{#1}}
  \fi
}

% Theorems
\theoremstyle{definition}
\newtheorem*{aim}{Aim}
\newtheorem*{axiom}{Axiom}
\newtheorem*{claim}{Claim}
\newtheorem*{cor}{Corollary}
\newtheorem*{defi}{Definition}
\newtheorem*{eg}{Example}
\newtheorem*{fact}{Fact}
\newtheorem*{law}{Law}
\newtheorem*{lemma}{Lemma}
\newtheorem*{notation}{Notation}
\newtheorem*{prop}{Proposition}
\newtheorem*{thm}{Theorem}

\renewcommand{\labelitemi}{--}
\renewcommand{\labelitemii}{$\circ$}
\renewcommand{\labelenumi}{(\roman{*})}

\let\stdsection\section
\renewcommand\section{\newpage\stdsection}

% Strike through
\def\st{\bgroup \ULdepth=-.55ex \ULset}

% Maths symbols
\newcommand{\bra}{\langle}
\newcommand{\ket}{\rangle}

\newcommand{\N}{\mathbb{N}}
\newcommand{\Z}{\mathbb{Z}}
\newcommand{\Q}{\mathbb{Q}}
\renewcommand{\H}{\mathbb{H}}
\newcommand{\R}{\mathbb{R}}
\newcommand{\C}{\mathbb{C}}
\newcommand{\Prob}{\mathbb{P}}
\renewcommand{\P}{\mathbb{P}}
\newcommand{\E}{\mathbb{E}}
\newcommand{\F}{\mathbb{F}}
\newcommand{\cU}{\mathcal{U}}
\newcommand{\RP}{\mathbb{RP}}
\newcommand{\CP}{\mathbb{CP}}

\newcommand{\ph}{\,\cdot\,}

\DeclareMathOperator{\sech}{sech}
\DeclareMathOperator{\cosech}{cosech}
\DeclareMathOperator{\cosec}{cosec}

\DeclareMathOperator{\covol}{covol}
\DeclareMathOperator{\vol}{vol}

\let\Im\relax
\let\Re\relax
\DeclareMathOperator{\Im}{Im}
\DeclareMathOperator{\Re}{Re}
\DeclareMathOperator{\im}{im}
\DeclareMathOperator{\image}{image}
\DeclareMathOperator{\Ann}{Ann}

\DeclareMathOperator*{\res}{res}
\DeclareMathOperator{\Res}{Res}
\DeclareMathOperator{\Ind}{Ind}

\DeclareMathOperator{\tr}{tr}
\DeclareMathOperator{\diag}{diag}
\DeclareMathOperator{\rank}{rank}
\DeclareMathOperator{\card}{card}
\DeclareMathOperator{\spn}{span}
\DeclareMathOperator{\adj}{adj}

\DeclareMathOperator{\erf}{erf}
\DeclareMathOperator{\erfc}{erfc}

\DeclareMathOperator{\ord}{ord}
\DeclareMathOperator{\Sym}{Sym}

\DeclareMathOperator{\sgn}{sgn}
\DeclareMathOperator{\orb}{orb}
\DeclareMathOperator{\stab}{stab}
\DeclareMathOperator{\ccl}{ccl}

\DeclareMathOperator{\lcm}{lcm}
\DeclareMathOperator{\hcf}{hcf}

\DeclareMathOperator{\Int}{Int}
\DeclareMathOperator{\id}{id}

\DeclareMathOperator{\betaD}{beta}
\DeclareMathOperator{\gammaD}{gamma}
\DeclareMathOperator{\Poisson}{Poisson}
\DeclareMathOperator{\binomial}{binomial}
\DeclareMathOperator{\multinomial}{multinomial}
\DeclareMathOperator{\Bernoulli}{Bernoulli}
\DeclareMathOperator{\like}{like}

\DeclareMathOperator{\var}{var}
\DeclareMathOperator{\cov}{cov}
\DeclareMathOperator{\bias}{bias}
\DeclareMathOperator{\mse}{mse}
\DeclareMathOperator{\corr}{corr}

\DeclareMathOperator{\otp}{otp}
\DeclareMathOperator{\dom}{dom}

\DeclareMathOperator{\Root}{Root}
\DeclareMathOperator{\supp}{supp}
\DeclareMathOperator{\rel}{rel}
\DeclareMathOperator{\Hom}{Hom}
\DeclareMathOperator{\Aut}{Aut}
\DeclareMathOperator{\Gal}{Gal}
\DeclareMathOperator{\Mat}{Mat}
\DeclareMathOperator{\End}{End}
\DeclareMathOperator{\Char}{char}
\DeclareMathOperator{\ev}{ev}
\DeclareMathOperator{\St}{St}
\DeclareMathOperator{\Lk}{Lk}
\DeclareMathOperator{\disc}{disc}
\DeclareMathOperator{\Isom}{Isom}
\DeclareMathOperator{\length}{length}
\DeclareMathOperator{\energy}{energy}
\DeclareMathOperator{\area}{area}
\DeclareMathOperator{\Syl}{Syl}
\DeclareMathOperator{\cl}{cl}
\DeclareMathOperator{\fix}{fix}

\newcommand{\GL}{\mathrm{GL}}
\newcommand{\SL}{\mathrm{SL}}
\newcommand{\PGL}{\mathrm{PGL}}
\newcommand{\PSL}{\mathrm{PSL}}
\newcommand{\PSU}{\mathrm{PSU}}
\newcommand{\Or}{\mathrm{O}}
\newcommand{\SO}{\mathrm{SO}}
\newcommand{\U}{\mathrm{U}}
\newcommand{\SU}{\mathrm{SU}}

\renewcommand{\d}{\mathrm{d}}
\newcommand{\D}{\mathrm{D}}

\tikzset{->/.style = {decoration={markings,
                                  mark=at position 1 with {\arrow[scale=2]{latex'}}},
                      postaction={decorate}}}
\tikzset{<-/.style = {decoration={markings,
                                  mark=at position 0 with {\arrowreversed[scale=2]{latex'}}},
                      postaction={decorate}}}
\tikzset{<->/.style = {decoration={markings,
                                   mark=at position 0 with {\arrowreversed[scale=2]{latex'}},
                                   mark=at position 1 with {\arrow[scale=2]{latex'}}},
                       postaction={decorate}}}
\tikzset{->-/.style = {decoration={markings,
                                   mark=at position #1 with {\arrow[scale=2]{latex'}}},
                       postaction={decorate}}}
\tikzset{-<-/.style = {decoration={markings,
                                   mark=at position #1 with {\arrowreversed[scale=2]{latex'}}},
                       postaction={decorate}}}

\tikzset{circ/.style = {fill, circle, inner sep = 0, minimum size = 3}}
\tikzset{mstate/.style={circle, draw, blue, text=black, minimum width=0.7cm}}

\definecolor{mblue}{rgb}{0.2, 0.3, 0.8}
\definecolor{morange}{rgb}{1, 0.5, 0}
\definecolor{mgreen}{rgb}{0.1, 0.4, 0.2}
\definecolor{mred}{rgb}{0.5, 0, 0}

\def\drawcirculararc(#1,#2)(#3,#4)(#5,#6){%
    \pgfmathsetmacro\cA{(#1*#1+#2*#2-#3*#3-#4*#4)/2}%
    \pgfmathsetmacro\cB{(#1*#1+#2*#2-#5*#5-#6*#6)/2}%
    \pgfmathsetmacro\cy{(\cB*(#1-#3)-\cA*(#1-#5))/%
                        ((#2-#6)*(#1-#3)-(#2-#4)*(#1-#5))}%
    \pgfmathsetmacro\cx{(\cA-\cy*(#2-#4))/(#1-#3)}%
    \pgfmathsetmacro\cr{sqrt((#1-\cx)*(#1-\cx)+(#2-\cy)*(#2-\cy))}%
    \pgfmathsetmacro\cA{atan2(#2-\cy,#1-\cx)}%
    \pgfmathsetmacro\cB{atan2(#6-\cy,#5-\cx)}%
    \pgfmathparse{\cB<\cA}%
    \ifnum\pgfmathresult=1
        \pgfmathsetmacro\cB{\cB+360}%
    \fi
    \draw (#1,#2) arc (\cA:\cB:\cr);%
}
\newcommand\getCoord[3]{\newdimen{#1}\newdimen{#2}\pgfextractx{#1}{\pgfpointanchor{#3}{center}}\pgfextracty{#2}{\pgfpointanchor{#3}{center}}}

\def\Xint#1{\mathchoice
   {\XXint\displaystyle\textstyle{#1}}%
   {\XXint\textstyle\scriptstyle{#1}}%
   {\XXint\scriptstyle\scriptscriptstyle{#1}}%
   {\XXint\scriptscriptstyle\scriptscriptstyle{#1}}%
   \!\int}
\def\XXint#1#2#3{{\setbox0=\hbox{$#1{#2#3}{\int}$}
     \vcenter{\hbox{$#2#3$}}\kern-.5\wd0}}
\def\ddashint{\Xint=}
\def\dashint{\Xint-}


\begin{document}
\maketitle
{\small
\setlength{\parindent}{0em}
\setlength{\parskip}{1em}
Modular Forms are classical objects that appear in many areas of mathematics, from number theory to representation theory and mathematical physics. Most famous is, of course, the role they played in the proof of Fermat's Last Theorem, through the conjecture of Shimura-Taniyama-Weil that elliptic curves are modular. One connection between modular forms and arithmetic is through the medium of $L$-functions, the basic example of which is the Riemann Riemann $\zeta$-function. We will discuss various types of $L$-function in this course and give arithmetic applications.

\subsubsection*{Pre-requisites}
Prerequisites for the course are fairly modest; from number theory, apart from basic elementary notions, some knowledge of quadratic fields is desirable. A fair chunk of the course will involve (fairly 19th-century) analysis, so we will assume the basic theory of holomorphic functions in one complex variable, such as are found in a first course on complex analysis (e.g. the 2nd year Complex Analysis course of the Tripos).
}
\tableofcontents

\setcounter{section}{-1}
\section{Introduction}
One of the big problems in number theory is the so-called Langland's programme, which is relates ``arithmetic objects'' such as representations of the Galois group and elliptic curves over $\Q$, with ``analytic objects'' such as modular forms and more generally automorphic forms and representations.

\begin{eg}
  $y^2 + y = x^3 - x$ is an elliptic curve, and we can associate to it the function
  \[
    f(z) = q\prod_{n \geq 1} (1 - q^n)^2 (1 - q^{11n})^2 = \sum_{n = 1}^\infty a_n q^n,\quad q= e^{2\pi i z},
  \]
  where we assume $\Im z > 0$, so that $|q| < 1$. The relation between these two objects is that the number of points of $E$ over $\F_p$ is equal to $1 + p - a_p$, for $p \not= 11$. This strange function $f$ is a modular form, and is actually cooked up from the slightly easier function
  \[
    \eta(z) = q^{1/24} \prod_{n = 1}^\infty (1 - q^n)
  \]
  by
  \[
    f(z) = (\eta(z)\eta(11z))^2.
  \]
  This function $\eta$ is called the \emph{Dedekind eta function}, and is one of the simplest example modular forms, in the sense that we can write it down easily. This satisfies the following two identities:
  \[
    \eta(z + 1) = e^{i \pi/12}\eta(z),\quad \eta\left(\frac{-1}{z}\right) = \sqrt{\frac{z}{i}} \eta(z).
  \]
  The first is clear, and the second takes some work to show. These transformation laws are exactly what makes this thing a modular form.

  Another way to link $E$ and $f$ is via the \emph{$L$-series}
  \[
    L(E, s) = \sum_{n = 1}^\infty \frac{a_n}{n^s},
  \]
  which is a generalization of the Riemann $\zeta$-function
  \[
    \zeta(s) = \sum_{n = 1}^\infty \frac{1}{n^s}.
  \]
\end{eg}
We are in fact not going to study elliptic curves, as there is another course on that, but we are going to study the modular forms and these $L$-series. We are going to do this in a fairly classical way, without using algebraic number theory.

\section{Some useful tools}
\subsection{Characters of abelian groups}
We start by going through the theory of characters of abelian groups, which is basically what Fourier analysis is about. These will come up useful later on.

\begin{defi}[Character]\index{character}
  Let $G$ be an abelian topological group. A (unitary) \emph{character} of $G$ is a continuous homomorphism $\chi: G \to \U(1)$, where $\U(1) = \{z \in \C \mid |z| = 1\}$.

  The set of such characters of $G$ forms a group under multiplication, written $\hat{G}$\index{$\hat{G}$} is called the \term{character group} (or \term{Pontryagin dual}) of $G$.
\end{defi}

\begin{eg}
  Let $G = \R$. For $y \in \R$, we let $\chi_y; \R \to \U(1)$ be
  \[
    \chi_y(x) = e^{2\pi i xy}.
  \]
  For each $y \in \R$, this is a character, and all characters are of this form, and $\hat{\R} \cong \R$ under this correspondence.
\end{eg}

\begin{eg}
  Take $G = \Z$ with the discrete topology. A character is uniquely determined by the image of $1$, and any element of $\U(1)$ can be the image of $1$. So we have $\hat{G} \cong \U(1)$.
\end{eg}

\begin{eg}
  Take $G = \Z/N\Z$. Then the character is again determined by the image of $1$, and the allowed values are exactly the $N$th roots of unity. So
  \[
    \hat{G} \cong \mu_N = \{\zeta \in \C^\times: \zeta^N = 1\}.
  \]
\end{eg}

\begin{eg}
  Let $G = G_1 \times G_2$. Then $\hat{G} \cong \hat{G}_1 \times \hat{G}_2$. So, for example, $\hat{\R^n} = \R^n$. Under this correspondence, a $y \in \R^n$ corresponds to
  \[
    \chi_y(x) = e^{2\pi x\cdot y}.
  \]
\end{eg}

\begin{eg}
  Take $G = \R^\times$. We have
  \[
    G \cong \{\pm 1\} \times \R^\times_{>0} \cong \{\pm 1\} \times \R,
  \]
  where we have an isomorphism between $\R^{\times}_{> 0} \cong \R$ by the exponential map. So we have
  \[
    \hat{G} \cong \Z/2\Z \times \R.
  \]
  Explicitly, given $(\varepsilon, \sigma) \in \Z/2\Z \times \R$, then character is given by
  \[
    x \mapsto \sgn(x)^\varepsilon |x|^{i\sigma}.
  \]
\end{eg}

Note that $\hat{G}$ has a natural topology for which the evaluation maps $(\chi \in \hat{G}) \mapsto \chi(g) \in \U(1)$ are all continuous for all $g$. Evaluation gives us a map $G \to \hat{\hat{G}}$. This is an isomorphism if $G$ is locally compact. This is called \term{Pontryagin duality}. Since this is a course on number theory, and not topological groups, we will not prove this.

\begin{prop}
  Let $G$ be a finite abelian group. Then $|\hat{G}| = |G|$, and $G$ and $\hat{G}$ are in fact isomorphic, but not canonically.
\end{prop}

\begin{proof}
  By the classification of finite-abelian groups, we know $G$ is a product of cyclic groups. So it suffices to prove the result for cyclic groups $\Z/N\Z$, and the result is clear since
  \[
    \widehat{\Z/N\Z} = \mu_N \cong \Z/N\Z.
  \]
\end{proof}

\subsection{Fourier transforms}
In the early years of the undergraduate curriculum, we learnt about the Fourier transformed, and then planned to completely forget it, and hoped that we will never meet it again. However, we shall remind ourselves of what the Fourier transform is.

\begin{defi}[Fourier transform]\index{Fourier transform}
  Let $f: \R \to \C$ be an $L^1$ function, ie. $\int |f| \;\d x < \infty$. The \emph{Fourier transform} is
  \[
    \hat{f}(y) = \int_{-\infty}^\infty e^{-2\pi i xy} f(x) \;\d x = \int_{-\infty}^\infty \chi_y(x)^{-1} f(x) \;\d x.
  \]
  This is a bounded and continuous function on $\R$.
\end{defi}
Really, the Fourier transform is not a function on $\R$, but a function on the character group of $\R$, but they are the same thing.

We will only be interested in a very special class of functions $f$, for which the analytic properties are all completely trivial.
\begin{defi}[Schwarz space]\index{Schwarz space}\index{$\mathcal{S}(\R)$}
  The \term{Schwarz space} is defined by
  \[
    \mathcal{S}(\R) = \{f \in C^\infty(\R) : x^n f^{(k)}(x) \to 0\text{ as }x \to \infty\text{ for all }k, n\geq 0\}.
  \]
\end{defi}

\begin{eg}
  The function
  \[
    f(x) = e^{-\pi x^2}.
  \]
  is in the Schwarz space.
\end{eg}

\begin{prop}
  If $f \in \mathcal{S}(\R)$, then $\hat{f} \in \mathcal{S}(\R)$, and the \term{Fourier inversion formula}
  \[
    \hat{\hat{f}} = f(-x)
  \]
  holds.
\end{prop}

The same is true for functions on $\R^n$, where we take
\[
  \hat{f}(y) = \int_{\R^n} e^{-2\pi i x\cdot y} f(x) \;\d x = \int_{\R^n} \chi_y(x)^{-1} f(x)\;\d x.
\]

We've also seen things called Fourier series. They are another version of this, where do the Fourier transform on $G = \R/\Z$ instead of $G$. For $n \in \Z$, we let $\chi_n \in \hat{G}$ by
\[
  \chi_n(x) = e^{2\pi i nx}.
\]
These are exactly all the elements of $\hat{G}$, and $\hat{G} \cong \Z$. We then define the Fourier series of a periodic function $f: \R/\Z \to \C$ by
\[
  c_n(f) = \int_0^1 e^{-2\pi i n x} f(x) \;\d x = \int_{\R/\Z} \chi_n(x)^{-1} f(x) \;\d x.
\]
Again, under suitable regularity conditions on $f$, eg. if $f \in C^\infty(\R/\Z)$, we have
\[
  f(x) = \sum_{n \in \Z} c_n(f) e^{2\pi i nx} = \sum_{n \in \Z \cong \hat{G}} c_n(f) \chi_n(x).
\]
This is the Fourier inversion formula for $G = \R/\Z$.

Finally, in the case when $G = \Z/N\Z$, given a function $f: \Z/N\Z \to \C$, we define $\hat{f}: \mu_N \to \C$ by
\[
  \hat{f}(\zeta) = \sum_{a \in \Z/N\Z} \zeta^{-a} f(a).
\]
\begin{prop}
  For a function $f: \Z/N\Z \to \C$, we have
  \[
    f(x) = \frac{1}{N} \sum_{\zeta \in \mu_N} \zeta^x \hat{f}(\zeta).
  \]
\end{prop}

The general picture is for any locally compact abelian group $G$, there is a canonical way to integrate functions defined on $G$ given by the \term{Haar measure}, which is a translation-invariant measure. For example, on $G = \R$, this is the usual Lebesgue measure, and if $G$ is discrete, this is just the counting measure, and
\[
  \int f = \sum_{g \in G} f(g).
\]
If $G = \R_{> 0}^\times$, then the integral is given by
\[
  \int f(x) \frac{\d x}{x},
\]
since $\frac{\d x}{x}$ is invariant under multiplication of $x$ by a constant.

Then given any $f \in L^1(G)$, we have
\[
  \hat{f}(X) = \int_G \chi(g)^{-1}f(g) \;\d g.
\]
Then one shows that for suitable $f$, we have $\hat{\hat{f}}(g) = C f(-g)$, using the canonical isomorphism $G \to \hat{\hat{G}}$, where $C$ is some constant independent of $f$. This constant is necessary, because the measure is only defined up to a multiplicative constant.
\printindex
\end{document}
