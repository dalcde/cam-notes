\documentclass[a4paper]{article}

\def\npart {III}
\def\nterm {Lent}
\def\nyear {2018}
\def\nlecturer {S.\ Svaldi}
\def\ncourse {Positivity in Algebraic Geometry}

% Imports
\ifx \nextra \undefined
  \usepackage[pdftex,
    hidelinks,
    pdfauthor={Dexter Chua},
    pdfsubject={Cambridge Maths Notes: Part \npart\ - \ncourse},
    pdftitle={Part \npart\ - \ncourse},
  pdfkeywords={Cambridge Mathematics Maths Math \npart\ \nterm\ \nyear\ \ncourse}]{hyperref}
  \title{Part \npart\ - \ncourse}
\else
  \usepackage[pdftex,
    hidelinks,
    pdfauthor={Dexter Chua},
    pdfsubject={Cambridge Maths Notes: Part \npart\ - \ncourse\ (\nextra)},
    pdftitle={Part \npart\ - \ncourse\ (\nextra)},
  pdfkeywords={Cambridge Mathematics Maths Math \npart\ \nterm\ \nyear\ \ncourse\ \nextra}]{hyperref}

  \title{Part \npart\ - \ncourse \\ {\Large \nextra}}
\fi

\author{Lectured by \nlecturer \\\small Notes taken by Dexter Chua}
\date{\nterm\ \nyear}

\usepackage{alltt}
\usepackage{amsfonts}
\usepackage{amsmath}
\usepackage{amssymb}
\usepackage{amsthm}
\usepackage{booktabs}
\usepackage{caption}
\usepackage{enumitem}
\usepackage{fancyhdr}
\usepackage{graphicx}
\usepackage{mathtools}
\usepackage{microtype}
\usepackage{multirow}
\usepackage{pdflscape}
\usepackage{pgfplots}
\usepackage{siunitx}
\usepackage{tabularx}
\usepackage{tikz}
\usepackage{tkz-euclide}
\usepackage[normalem]{ulem}
\usepackage[all]{xy}

\pgfplotsset{compat=1.12}

\pagestyle{fancyplain}
\lhead{\emph{\nouppercase{\leftmark}}}
\ifx \nextra \undefined
  \rhead{
    \ifnum\thepage=1
    \else
      \npart\ \ncourse
    \fi}
\else
  \rhead{
    \ifnum\thepage=1
    \else
      \npart\ \ncourse\ (\nextra)
    \fi}
\fi
\usetikzlibrary{arrows}
\usetikzlibrary{decorations.markings}
\usetikzlibrary{decorations.pathmorphing}
\usetikzlibrary{positioning}
\usetikzlibrary{fadings}
\usetikzlibrary{intersections}
\usetikzlibrary{cd}

\newcommand*{\Cdot}{\raisebox{-0.25ex}{\scalebox{1.5}{$\cdot$}}}
\newcommand {\pd}[2][ ]{
  \ifx #1 { }
    \frac{\partial}{\partial #2}
  \else
    \frac{\partial^{#1}}{\partial #2^{#1}}
  \fi
}

% Theorems
\theoremstyle{definition}
\newtheorem*{aim}{Aim}
\newtheorem*{axiom}{Axiom}
\newtheorem*{claim}{Claim}
\newtheorem*{cor}{Corollary}
\newtheorem*{defi}{Definition}
\newtheorem*{eg}{Example}
\newtheorem*{fact}{Fact}
\newtheorem*{law}{Law}
\newtheorem*{lemma}{Lemma}
\newtheorem*{notation}{Notation}
\newtheorem*{prop}{Proposition}
\newtheorem*{thm}{Theorem}

\renewcommand{\labelitemi}{--}
\renewcommand{\labelitemii}{$\circ$}
\renewcommand{\labelenumi}{(\roman{*})}

\let\stdsection\section
\renewcommand\section{\newpage\stdsection}

% Strike through
\def\st{\bgroup \ULdepth=-.55ex \ULset}

% Maths symbols
\newcommand{\bra}{\langle}
\newcommand{\ket}{\rangle}

\newcommand{\N}{\mathbb{N}}
\newcommand{\Z}{\mathbb{Z}}
\newcommand{\Q}{\mathbb{Q}}
\renewcommand{\H}{\mathbb{H}}
\newcommand{\R}{\mathbb{R}}
\newcommand{\C}{\mathbb{C}}
\newcommand{\Prob}{\mathbb{P}}
\renewcommand{\P}{\mathbb{P}}
\newcommand{\E}{\mathbb{E}}
\newcommand{\F}{\mathbb{F}}
\newcommand{\cU}{\mathcal{U}}
\newcommand{\RP}{\mathbb{RP}}
\newcommand{\CP}{\mathbb{CP}}

\newcommand{\ph}{\,\cdot\,}

\DeclareMathOperator{\sech}{sech}
\DeclareMathOperator{\cosech}{cosech}
\DeclareMathOperator{\cosec}{cosec}

\DeclareMathOperator{\covol}{covol}
\DeclareMathOperator{\vol}{vol}

\let\Im\relax
\let\Re\relax
\DeclareMathOperator{\Im}{Im}
\DeclareMathOperator{\Re}{Re}
\DeclareMathOperator{\im}{im}
\DeclareMathOperator{\image}{image}
\DeclareMathOperator{\Ann}{Ann}

\DeclareMathOperator*{\res}{res}
\DeclareMathOperator{\Res}{Res}
\DeclareMathOperator{\Ind}{Ind}

\DeclareMathOperator{\tr}{tr}
\DeclareMathOperator{\diag}{diag}
\DeclareMathOperator{\rank}{rank}
\DeclareMathOperator{\card}{card}
\DeclareMathOperator{\spn}{span}
\DeclareMathOperator{\adj}{adj}

\DeclareMathOperator{\erf}{erf}
\DeclareMathOperator{\erfc}{erfc}

\DeclareMathOperator{\ord}{ord}
\DeclareMathOperator{\Sym}{Sym}

\DeclareMathOperator{\sgn}{sgn}
\DeclareMathOperator{\orb}{orb}
\DeclareMathOperator{\stab}{stab}
\DeclareMathOperator{\ccl}{ccl}

\DeclareMathOperator{\lcm}{lcm}
\DeclareMathOperator{\hcf}{hcf}

\DeclareMathOperator{\Int}{Int}
\DeclareMathOperator{\id}{id}

\DeclareMathOperator{\betaD}{beta}
\DeclareMathOperator{\gammaD}{gamma}
\DeclareMathOperator{\Poisson}{Poisson}
\DeclareMathOperator{\binomial}{binomial}
\DeclareMathOperator{\multinomial}{multinomial}
\DeclareMathOperator{\Bernoulli}{Bernoulli}
\DeclareMathOperator{\like}{like}

\DeclareMathOperator{\var}{var}
\DeclareMathOperator{\cov}{cov}
\DeclareMathOperator{\bias}{bias}
\DeclareMathOperator{\mse}{mse}
\DeclareMathOperator{\corr}{corr}

\DeclareMathOperator{\otp}{otp}
\DeclareMathOperator{\dom}{dom}

\DeclareMathOperator{\Root}{Root}
\DeclareMathOperator{\supp}{supp}
\DeclareMathOperator{\rel}{rel}
\DeclareMathOperator{\Hom}{Hom}
\DeclareMathOperator{\Aut}{Aut}
\DeclareMathOperator{\Gal}{Gal}
\DeclareMathOperator{\Mat}{Mat}
\DeclareMathOperator{\End}{End}
\DeclareMathOperator{\Char}{char}
\DeclareMathOperator{\ev}{ev}
\DeclareMathOperator{\St}{St}
\DeclareMathOperator{\Lk}{Lk}
\DeclareMathOperator{\disc}{disc}
\DeclareMathOperator{\Isom}{Isom}
\DeclareMathOperator{\length}{length}
\DeclareMathOperator{\energy}{energy}
\DeclareMathOperator{\area}{area}
\DeclareMathOperator{\Syl}{Syl}
\DeclareMathOperator{\cl}{cl}
\DeclareMathOperator{\fix}{fix}

\newcommand{\GL}{\mathrm{GL}}
\newcommand{\SL}{\mathrm{SL}}
\newcommand{\PGL}{\mathrm{PGL}}
\newcommand{\PSL}{\mathrm{PSL}}
\newcommand{\PSU}{\mathrm{PSU}}
\newcommand{\Or}{\mathrm{O}}
\newcommand{\SO}{\mathrm{SO}}
\newcommand{\U}{\mathrm{U}}
\newcommand{\SU}{\mathrm{SU}}

\renewcommand{\d}{\mathrm{d}}
\newcommand{\D}{\mathrm{D}}

\tikzset{->/.style = {decoration={markings,
                                  mark=at position 1 with {\arrow[scale=2]{latex'}}},
                      postaction={decorate}}}
\tikzset{<-/.style = {decoration={markings,
                                  mark=at position 0 with {\arrowreversed[scale=2]{latex'}}},
                      postaction={decorate}}}
\tikzset{<->/.style = {decoration={markings,
                                   mark=at position 0 with {\arrowreversed[scale=2]{latex'}},
                                   mark=at position 1 with {\arrow[scale=2]{latex'}}},
                       postaction={decorate}}}
\tikzset{->-/.style = {decoration={markings,
                                   mark=at position #1 with {\arrow[scale=2]{latex'}}},
                       postaction={decorate}}}
\tikzset{-<-/.style = {decoration={markings,
                                   mark=at position #1 with {\arrowreversed[scale=2]{latex'}}},
                       postaction={decorate}}}

\tikzset{circ/.style = {fill, circle, inner sep = 0, minimum size = 3}}
\tikzset{mstate/.style={circle, draw, blue, text=black, minimum width=0.7cm}}

\definecolor{mblue}{rgb}{0.2, 0.3, 0.8}
\definecolor{morange}{rgb}{1, 0.5, 0}
\definecolor{mgreen}{rgb}{0.1, 0.4, 0.2}
\definecolor{mred}{rgb}{0.5, 0, 0}

\def\drawcirculararc(#1,#2)(#3,#4)(#5,#6){%
    \pgfmathsetmacro\cA{(#1*#1+#2*#2-#3*#3-#4*#4)/2}%
    \pgfmathsetmacro\cB{(#1*#1+#2*#2-#5*#5-#6*#6)/2}%
    \pgfmathsetmacro\cy{(\cB*(#1-#3)-\cA*(#1-#5))/%
                        ((#2-#6)*(#1-#3)-(#2-#4)*(#1-#5))}%
    \pgfmathsetmacro\cx{(\cA-\cy*(#2-#4))/(#1-#3)}%
    \pgfmathsetmacro\cr{sqrt((#1-\cx)*(#1-\cx)+(#2-\cy)*(#2-\cy))}%
    \pgfmathsetmacro\cA{atan2(#2-\cy,#1-\cx)}%
    \pgfmathsetmacro\cB{atan2(#6-\cy,#5-\cx)}%
    \pgfmathparse{\cB<\cA}%
    \ifnum\pgfmathresult=1
        \pgfmathsetmacro\cB{\cB+360}%
    \fi
    \draw (#1,#2) arc (\cA:\cB:\cr);%
}
\newcommand\getCoord[3]{\newdimen{#1}\newdimen{#2}\pgfextractx{#1}{\pgfpointanchor{#3}{center}}\pgfextracty{#2}{\pgfpointanchor{#3}{center}}}

\def\Xint#1{\mathchoice
   {\XXint\displaystyle\textstyle{#1}}%
   {\XXint\textstyle\scriptstyle{#1}}%
   {\XXint\scriptstyle\scriptscriptstyle{#1}}%
   {\XXint\scriptscriptstyle\scriptscriptstyle{#1}}%
   \!\int}
\def\XXint#1#2#3{{\setbox0=\hbox{$#1{#2#3}{\int}$}
     \vcenter{\hbox{$#2#3$}}\kern-.5\wd0}}
\def\ddashint{\Xint=}
\def\dashint{\Xint-}


\newcommand\A{\mathbb{A}}
\newcommand\val{\mathrm{val}}
\renewcommand\div{\mathrm{div}}
\newcommand\Cl{\mathrm{Cl}}
\newcommand\WDiv{\mathrm{WDiv}}

\begin{document}
\maketitle
{\small
\setlength{\parindent}{0em}
\setlength{\parskip}{1em}
This class aims at giving an introduction to the theory of divisors, linear systems and their positivity properties on projective algebraic varieties.

The first part of the class will be dedicated to introducing the basic notions and results regarding these objects and special attention will be devoted to discussing examples in the case of curves and surfaces.

In the second part, the course will cover classical results from the theory of divisors and linear systems and their applications to the study of the geometry of algebraic varieties.

If time allows and based on the interests of the participants, there are a number of more advanced topics that could possibly be covered: Reider's Theorem for surfaces, geometry of linear systems on higher dimensional varieties, multiplier ideal sheaves and invariance of plurigenera, higher dimensional birational geometry.

\subsubsection*{Pre-requisites}
The minimum requirement for those students wishing to enroll in this class is their knowledge of basic concepts from the Algebraic Geometry Part 3 course, i.e.\ roughly Chapters 2 and 3 of Hartshorne's Algebraic Geometry.

Familiarity with the basic concepts of the geometry of algebraic varieties of dimension 1 and 2 --- e.g.\ as covered in the preliminary sections of Chapters 4 and 5 of Hartshorne's Algebraic Geometry --- would be useful but will not be assumed --- besides what was already covered in the Michaelmas lectures.

Students should have also some familiarity with concepts covered in the Algebraic Topology Part 3 course such as cohomology, duality and characteristic classes.
}
\tableofcontents

\section{Introduction}
Fix a field $K$. The first problem we want to look at is the following:
\begin{problem}
  Classify all finitely-generated field extension $L/K$.
\end{problem}
A closely related problem is
\begin{problem}
  Classify all towers of field extensions $L/M/K$.
\end{problem}
In general, we don't want to consider finite extensions --- this falls within the scope of Galois theory instead.

In general, given a finite extension $L/K$, we can factorize it into a tower $L/M/K$, where $M/K$ is purely transcendental, i.e.\ $M \cong K(x_1, \ldots, x_n)$ is a field of rational functions over $K$.

A large class of examples comes from \term{function fields} of algebraic varieties, $K(X)$.
\begin{eg}
  Recall that if $X$ is affine, then $K(X)$ is the field of fractions of the ring of functions of $X$, i.e.\ $Q(K[X])$.
\end{eg}

\begin{question}
  Are these all the examples of finitely-generated field extensions $L/K$?
\end{question}


\begin{lemma}
  Let $\Char K = 0$. If $L/K$ is finitely-generated and not finite, then there exists an algebraic variety $X/K$ such that $K(X) \cong L$.
\end{lemma}
Note that the condition is satisfied in particular if $K$ is algebraically closed.

\begin{proof}
  We factor $L/M/K$, where $M/K$ is purely transcendental, say $M = K(x_1, \ldots, x_n) = K(\A^n)$. Since $\Char(K) = 0$, we can use the primitive element theorem to get that $L = M(\alpha)$ for some $\alpha \in L$. Then if $f$ is the minimal polynomial for $\alpha$, then
  \[
    L \cong Q(K[x_1, \ldots, x_n, x]/(f(x))).\qedhere
  \]
\end{proof}


Now suppose $\varphi: X \to Y$ is a dominant birational map, % make birational
i.e.\ $\overline{\varphi(X)} = Y$. Then this induces a map $\varphi^* : K(Y) \to K(X)$.

Indeed, an element of $K(Y)$ is just a rational function $Y \to \A^1_K$, and pre-composing with $\varphi$ gives the desired function. The dominance ensures the composition is well-defined.

Now if every embedding of field extensions arise this way, then we have reduced our problem to studying algebraic varieties over $K$ and their rational dominant maps. This is indeed the case.
\begin{thm}
  Let $K$ be a field with $\Char K = 0$. Then there is an equivalence of categories between
  \[
    \left\{ \parbox{4.8cm}{\centering algebraic varieties over $K$\\ rational dominant morphisms}\right\} \longleftrightarrow \left\{\parbox{5.3cm}{\centering infinite f.g.\ field extensions over $K$\\field embeddings}\right\}
  \]
\end{thm}

\begin{proof}
  We define a functor from the left category to the right by sending a variety $X$ to the field of rational functions on $X$, $K(X)$, and morphisms are as above. We have already seen that this is essentially surjective. To show it is fully faithful, suppose $X, Y$ are algebraic varieties over $K$, and $\psi: K(Y) \hookrightarrow K(X)$. We want to show that $\psi = \varphi^*$ for some rational dominant map $X \to Y$.

  We can assume that $X$ and $Y$ are affine, since every variety is birationally equivalent to any Zariski open subset. Then $K(X) = Q(K[X])$ and $K(Y) = Q(K[Y])$. So in particular, we have an embedding $K[Y] \hookrightarrow K(X)$. Since $K[Y]$ is finitely-generated, we can produce a factorization
  \[
    \begin{tikzcd}
      K[Y] \ar[r, hook] \ar[dr, hook] & K(X) \\
      & K[X]_{s_1, \ldots, s_n} \ar[u, hook]
    \end{tikzcd}
  \]
  for some $s_1, \ldots, s_n$. But then $K[X]_{s_1, \ldots, s_n} = K[U]$ for some open $U \subseteq X$. Then the map $K[Y] \to K[U]$ determines a map $U \to Y$, hence a rational map $X \to Y$. The injectivity of the map $K[Y] \to K[U]$ corresponds to the fact that the induced map $U \to Y$ is dominant. So we are done.
\end{proof}

So we have translated our initial problem into a problem about algebraic varieties and rational maps. So we now want to understand how we can construct rational maps.

\begin{defi}[Generating section]\index{generating section}
  Let $X$ be an algebraic variety over $K$, and $\mathcal{F}$ a sheaf of $\mathcal{O}_X$-modules. Let $s_0, \ldots, s_n \in H^0(X, \mathcal{F})$ be sections. We say the sections \emph{generate} $\mathcal{F}$ if the natural map
  \[
    \bigoplus_{i = 0}^{n + 1} \mathcal{O}_X \to \mathcal{F}
  \]
  induced by the $s_i$ is a surjective map of $\mathcal{O}_X$-modules.
\end{defi}

This is a very useful concept, because it lets us construct maps into projective spaces.
\begin{thm}
  Let $X$ be an algebraic variety over $K$.
  \begin{enumerate}
    \item If $\varphi: X \to \P^n$ is a morphism over $K$, then $\varphi^* \mathcal{O}_{\P^n}(1)$ is an invertible sheaf on $X$, generated by the sections $\varphi^* x_0, \ldots, \varphi^* x_n \in H^0(X, \varphi^* \mathcal{O}_{\P^n}(1))$.
    \item If $\mathcal{L}$ is an invertible sheaf on $X$, and if $s_0, \ldots, s_n \in H^0(X, \mathcal{L})$ which generate $\mathcal{L}$, then there exists a unique morphism $\varphi: X \to \P^n$ such that $\varphi^* \mathcal{O}(1) \cong \mathcal{L}$ and $\varphi^* x_i = s_i$.
  \end{enumerate}
\end{thm}
The moral of the story is that to study morphisms of algebraic varieties, we need to understand invertible sheaves and $n$-tuples of generating sections. This is the focus of the course -- to understand invertible sheaves and their sections.

\begin{proof}\leavevmode
  \begin{enumerate}
    \item The pullback of an invertible sheaf is an invertible, and the same for sections. Since $x_0, \ldots, x_n$ generate $H^0(\mathcal{O}_{\P^n}(1))$, they generate, and this is again preserved under pullback.
    \item The idea is that in homogeneous coordinates, we map $x \in X$ to $[s_0(x): \cdots : s_n(x)] \in \P^n$.

      In more detail, define $X_{s_i} = \{p \in X: s_i \in \mathfrak{m}_p \mathcal{L}_p\}$. This is a Zariski open set, and $s_i$ is invertible on $X_{s_i}$. Thus there is a dual section $s_i^\vee \in \mathcal{L}^{\vee}$ such that $s_i \otimes s_i^\vee \in \mathcal{L} \otimes \mathcal{L}^\vee \cong \mathcal{O}_X$ is equal to $1$. Define the map $K_{s_i} \to \A^n$ by the map
      \begin{align*}
        K[\A^n] &\to H^0(X_{s_i}, \mathcal{O}_{s_i})\\
        y_i &\mapsto s_j \otimes s_i^\vee.
      \end{align*}
      Identifying $\A^n$ as the $x_i \not= 0$ chart of $\P^n$, one checks that $X = \bigcup X_{s_i}$, and this map gives the desired map $X \to \P^n$.\qedhere
  \end{enumerate}
\end{proof}

Amongst all maps, some are more equal than others. In particular, embeddings are nice.

\begin{defi}[Very ample sheaf]\index{very ample sheaf}\index{sheaf!very ample}
  Let $X$ be an algebraic variety over $K$, and $\mathcal{L}$ be an invertible sheaf. We say that $\mathcal{L}$ is very ample if there is a closed immersion $\varphi: X \to \P^n$ such that $\varphi^* \mathcal{O}_{\P^n}(1) \cong \mathcal{L}$.
\end{defi}
The natural question is --- is there a criteria to identify when $\mathcal{L}$ is very ample. The answer is yes! If $X \hookrightarrow \P^n$, is injective, then given any two points in $X$, there is a hyperplane in $\P^n$ that passes through one but not the other. We want something slightly more, since we want the differential of the map to be injective as well.

\begin{prop}
  Assume $K = \bar{K}$, and $X$ is a projective variety over $K$. Fix an invertible sheaf $\mathcal{L}$ on $X$, and sections $s_0, \ldots, s_n \in H^0(X, \mathcal{L})$ generating $\mathcal{L}$. Then the associated map $\varphi: X \to \P^n$ is a closed embedding iff
  \begin{enumerate}
    \item For every distinct closed points $p \not= q \in X$, there exists $s_{p, q} \in V = \bra s_0, \ldots, s_n\ket \subseteq H^0(X, \mathcal{L})$ such that $s_{p, q} \in \mathfrak{m}_p \mathcal{L}_p$ but $s_{p, q} \not \in \mathfrak{m}_q \mathcal{L}_q$.
    \item For every $p \in X$, the set $\{s \in V \mid s \in \mathfrak{m}_p \mathcal{L}_p\}$ spans the vector space $\mathfrak{m}_p \mathcal{L}_p /\mathfrak{m}^2_p \mathcal{L}_p$.
  \end{enumerate}
\end{prop}

\begin{defi}[Separate points and tangent vectors]
  With the hypothesis of the proposition, we say that
  \begin{itemize}
    \item Elements of $V$ \term{separate points} if $V$ satisfies (i).
    \item Elements of $V$ \term{separate tangent vectors} if $V$ satisfies (ii).
  \end{itemize}
\end{defi}

\begin{proof}\leavevmode
  \begin{itemize}
    \item[$(\Rightarrow)$] Suppose $\phi$ is a closed immersion. Then it is injective on points. So suppose $p \not= q$ are (closed) points. Then there is some hyperplane $H_{p, q}$ in $\P^n$ passing through $p$ but not $q$, and $H_{p, q}$ is represented by a global section of $\mathcal{O}(1)$. Then $s_{p, q} = \varphi^* H_{p, q} \in V \subseteq H^0(X, \mathcal{L})$ is such that $s_{p, q} \in \mathfrak{m}_p \mathcal{L}_p$ and $s_{p, q} \not \in \mathfrak{m}_q \mathcal{L}_q$. So (i) is satisfied.

      To see (ii), observe that by definition, the map of sheaves $\mathcal{O}_{\P^n} \to \varphi_* \mathcal{O}_X$ is surjective. So if $p \in X$ is a closed point, then in particular $\mathfrak{m}_{p, \P^n} \to \mathfrak{m}_{p, X}$ is surjective. This implies that the quotient
       \[
        \frac{\mathfrak{m}_{p, \P^n}}{\mathfrak{m}_{p, \P^n}^2} \twoheadrightarrow \frac{\mathfrak{m}_{p, X}}{\mathfrak{m}_{p, X}^2}.
      \]
      is also surjective. But on $\P^n$, $x_0, \ldots, x_n$ certainly spans $\mathfrak{m}_{p, \P^n}$. So their image $s_0, \ldots, s_n$ in $\mathfrak{m}_{p, X}$ also span $\mathfrak{m}_{p, X}/\mathfrak{m}_{p, X}^2$.

    \item[$(\Leftarrow)$] We first show that $\varphi$ is injective on closed points. For any $p \not= q \in X$, write the given $s_{p, q}$ as
      \[
        s_{p, q} = \sum \lambda_i s_i = \sum \lambda_i \varphi^* x_i = \varphi^* \sum \lambda_i x_i
      \]
      for some $\lambda_i \in K$. So we can take $H_{p, q}$ to be given by the vanishing set of $\sum \lambda_i x_i$, and so it is injective on closed point. It follows that it is also injective on schematic points.

      To show that $\varphi$ is in fact a closed immersion, we need to show that $\mathcal{O}_{\P^n} \to \varphi_* \mathcal{O}_X$ is surjective. As before, it is enough to prove that it holds at the level of stalks over closed points. To show this, we apply the following lemma:
      \begin{lemma}
        Let $f: A \to B$ be a local morphism of local rings such that
        \begin{itemize}
          \item $A/\mathfrak{m}_A \to B/\mathfrak{m}_B$ is an isomorphism;
          \item $\mathfrak{m}_A \to \mathfrak{m}_B/\mathfrak{m}_B^2$ is surjective; and
          \item $B$ is a finitely-generated $A$-module.
        \end{itemize}
        Then $f$ is surjective.\fakeqed
      \end{lemma}
      To check the first condition, note that we have
      \[
        \frac{\mathcal{O}_{p, \P^n}}{\mathfrak{m}_{p, \P^n}} \cong \frac{\mathcal{O}_{p, X}}{\mathfrak{m}_{p, X}} \cong K.
      \]
      Now since $\mathfrak{m}_{p, \P^n}$ is generated by $x_0, \ldots, x_n$, the second condition is the same as saying
      \[
        \mathfrak{m}_{p, \P^n} \to \frac{\mathfrak{m}_{p, X}}{\mathfrak{m}_{p, X}^2}
      \]
      is surjective. The last part is immediate.\qedhere
  \end{itemize}
\end{proof}
However, this is still not a very pleasant hypothesis to check, since it requires us to really understand the structure of $V$. In general, given an invertible sheaf $\mathcal{L}$, it is unlikely that we can concretely understand the space of sections. Thus, what we want to do in this course is to simplify this criterion.

\begin{problem}
  Find easier ways to decide whether $\mathcal{L}$ is very ample/ample/has some kind of ``positivity''.
\end{problem}

We have the following theorem of Serre:
\begin{thm}[Serre]
  Let $X$ be a projective scheme over a Noetherian ring $A$, and let $\mathcal{L}$ be a very ample invertible sheaf, and $\mathcal{F}$ a coherent $\mathcal{O}_X$-module. Then there exists a positive integer $n_0 = n_0(\mathcal{F}, \mathcal{L})$ such that for all $n \geq n_0$, the twist $\mathcal{F} \otimes \mathcal{L}^n$ is generated by global sections.\fakeqed
\end{thm}

So we can try to weaken the condition of very ampleness to only require the condition of this theorem to hold.
\begin{defi}[Ample sheaf]\index{ample sheaf}
  Let $X$ be a Noetherian scheme over $A$, and $\mathcal{L}$ an invertible sheaf over $X$. We say $\mathcal{L}$ is ample iff for any coherent $\mathcal{O}_X$-module $\mathcal{F}$, there is an $n_0$ such that for all $n \geq n_0$, the sheaf $\mathcal{F} \otimes \mathcal{L}^n$ is generated by global sections.
\end{defi}

While this seems like a rather weak condition, by taking $\mathcal{F}$ to be $\mathcal{O}_X$, we can find some $\mathcal{L}^n$ that is generated by global sections. So at least it gives some map to $\P^n$. Can we make this an embedding? The answer is yes, and this is again a theorem of Serre.

\begin{thm}[Serre]
  Let $X$ be a scheme of finite type over a Noetherian ring $A$, and $\mathcal{L}$ an invertible sheaf on $X$. Then $\mathcal{L}$ is ample iff there exists $m > 0$ such that $\mathcal{L}^m$ is very ample.
\end{thm}

We only prove this when $A = K$ is a field.
\begin{proof}\leavevmode
  \begin{itemize}
    \item[($\Leftarrow$)] Let $\mathcal{L}^n$ be very ample, and $\mathcal{F}$ a coherent sheaf. By Serre's theorem, there exists $n_0$ such that for all $j \geq j_0$, the sheafs
      \[
        \mathcal{F} \otimes \mathcal{L}^{mj}, (\mathcal{F} \otimes \mathcal{L}) \otimes \mathcal{L}^{mj}, \ldots, (\mathcal{F} \otimes \mathcal{L}^{m - 1}) \otimes \mathcal{L}^{mj}
      \]
      are all globally generated. So $\mathcal{F} \otimes \mathcal{L}^n$ is globally generated for $n \geq mj_0$.
    \item[($\Rightarrow$)] Suppose $\mathcal{L}$ is ample. Then $\mathcal{L}^m$ is globally generated for $m$ sufficiently large. We claim that there exists $t_1, \ldots, t_n \in H^0(X, \mathcal{L}^N)$ such that $\mathcal{L}|_{X_{t_i}}$ are all trivial (i.e.\ isomorphic to $\mathcal{O}_{X_{t_i}}$), and $X = \bigcup X_{t_i}$.

      By compactness, it suffices to show that for each $p \in X$, there is some $t \in H^0(X, \mathcal{L}^n)$ (for some $n$) such that $p \in X_{t_i}$ and $\mathcal{L}$ is trivial on $X_{t_i}$. First of all, since $\mathcal{L}$ is locally free by definition, we can find an open affine $U$ containing $p$ such that $\mathcal{L}|_U$ is trivial.

      Thus, it suffices to produce a section $t$ that vanishes on $Y = X - U$ but not at $p$. Then $p \in X_t \subseteq U$ and hence $\mathcal{L}$ is trivial on $X_t$. Vanishing on $Y$ is the same as belonging to the ideal sheaf $\mathcal{I}_Y$. Since $\mathcal{I}_Y$ is coherent, ampleness implies there is some large $n$ such that $\mathcal{I}_Y \otimes \mathcal{L}^n$ is generated by global sections. In particular, since $\mathcal{I}_Y \otimes \mathcal{L}^n$ doesn't vanish at $p$, we can find some $t \in \Gamma(X, \mathcal{I}_Y \otimes \mathcal{L}^N)$ such that $t \not \in \mathfrak{m}_p (\mathcal{I}_Y \otimes \mathcal{L}^n)_p$. Since $\mathcal{I}_Y$ is a subsheaf of $\mathcal{O}_X$, we can view $t$ as a section of $\mathcal{L}^n$, and this $t$ works.

      Fix a point $x \in X$, and take an affine open $U$ containing $x$ such that $\mathcal{L}|_U$ is trivial, i.e.\ $\mathcal{L}|_U \cong \mathcal{O}_U$. Define $Z = X \setminus U$.

      We know there exists $\ell$ such that $I_Z \otimes \mathcal{L}^{\ell}$ is globally generated. So in particular, there exists $s \in H^0(X, I_Z \otimes \mathcal{L}^\ell)$ such that $s_x \not \in \mathfrak{m}_X \mathcal{L}_{x, X}$. So we can define $X_S$ to be the affine Zariski open such that $X_s = \{p \in X : s_p \not \in \mathfrak{m}_p \mathcal{L}_p\}$.

      Now given the $X_{t_i}$, for each fixed $i$, we let $\{b_{ij}\}$ generate $\mathcal{O}_{X_{t_i}}$ as an $A$-algebra. Then for large $n$, $c_{ij} = t_i^n b_{ij}$ extends to a global section $c_{ij} \in \Gamma(X, \mathcal{L}^n)$ (by clearing denominators). We can pick an $n$ large enough to work for all $b_{ij}$. Then we use $\{t_i^n, c_{ij}\}$ as our generating sections to construct a morphism to $\P^N$, and let $\{x_i, x_{ij}\}$ be the corresponding coordinates. Observe that $\bigcup X_{t_i} = X$ implies the $t_i^n$ already generate $\mathcal{L}^n$. Now each $x_{t_i}$ gets mapped to $U_i \subseteq \P^N$, the vanishing set of $x_i$. The map $\mathcal{O}_{U_i} \to \varphi_* \mathcal{O}_{X_{t_i}}$ corresponds to the map
      \[
        A[y_i, y_{ij}] \to B_i,
      \]
      where $y_{ij}$ is mapped to $c_{ij}/t_i^n = b_{ij}$. So by assumption, this is surjective, and so we have a closed embedding.\qedhere
  \end{itemize}
\end{proof}
\begin{ex}
  Let $\mathcal{L}$ be a sheaf over $X$ (which is itself a projective variety over $K$). Then the following are equivalent:
  \begin{enumerate}
    \item $\mathcal{L}$ is ample.
    \item $\mathcal{L}^m$ is ample for all $m > 0$.
    \item $\mathcal{L}^m$ is ample for some $m > 0$.
  \end{enumerate}
\end{ex}

\begin{thm}[Serre]
  Let $X$ be a projective scheme over a Noetherian ring $A$, and $\mathcal{L}$ is very ample on $X$. Let $\mathcal{F}$ be a coherent sheaf. Then
  \begin{enumerate}
    \item For all $i \geq 0$ and $n \in \N$, $H^i(\mathcal{F} \otimes \mathcal{L}^n)$ is a finitely-generated $A$-module.
    \item There exists $n_0 \in \N$ such that for all $n \geq n_0$, $H^i(\mathcal{F} \otimes \mathcal{L}^n) = 0$ for all $i > 0$.\fakeqed
  \end{enumerate}
\end{thm}
The proof is exactly the same as the case of $\mathcal{O}(1)$ on $\P^n$.

It turns out this result characterizes ample sheaves:
\begin{thm}
  Let $X$ be a proper scheme over a Noetherian ring $A$, and $\mathcal{L}$ an invertible sheaf. Then the following are equivalent:
  \begin{enumerate}
    \item $\mathcal{L}$ is ample.
    \item For all coherent $\mathcal{F}$ on $X$, there exists $n_0 \in \N$ such that for all $n \geq n_0$, we have $H^i(\mathcal{F} \otimes \mathcal{L}^n) = 0$.
  \end{enumerate}
\end{thm}

\begin{proof}
  Proving (i) $\Rightarrow$ (ii) is the same as the first part of the theorem last time.

  To prove (ii) $\Rightarrow$ (i), fix a point $x \in X$, and consider the sequence
  \[
    0 \to \mathfrak{m}_x \mathcal{F} \to \mathcal{F} \to \mathcal{F}_x \to 0.
  \]
  We twist by $\mathcal{L}^n$, where $n$ is sufficiently big, and take cohomology. Then we have a long exact sequence
  \[
    0 \to H^0(\mathfrak{m}_x \mathcal{F}(n)) \to H^0(\mathcal{F}(n)) \to H^0(\mathcal{F}_x(n) \to H^1(\mathfrak{m}_x \mathcal{F}(n)) = 0.
  \]
  In particular, the map $H^0(\mathcal{F}(n)) \to H^0(\mathcal{F}_x(n)$ is surjective. This mean at $x$, $\mathcal{F}(n)$ is globally generated. Then by compactness, there is a single $n$ large enough such taht $\mathcal{F}(n)$ is globally generated everywhere.
\end{proof}

\section{Divisors}
\subsection{Weil divisors}
Let $X$ be a Noetherian scheme. If $X$ is regular in codimension $1$ (e.g.\ if $X$ is normal), which one should think of as saying the singular locus of $X$ having codimension $ \geq 2$. Then if $Y \subseteq X$ is a codimension $1$ integral subscheme, then the local ring $\mathcal{O}_{X, Y}$ is a DVR. So there is a valuation
\[
  \val_Y: \mathcal{O}_{X, Y} \to \Z,
\]
which is the order of vanishing/poles along $Y$.

\begin{defi}[Weil divisor]\index{Weil divisor}
  A \emph{Weil divisor} is a formal sum
  \[
    \sum a_i Y_i,
  \]
  where the $a_i \in \Z$ and $Y_i$ are codimension $1$ integral subschemes. We write $\WDiv(X)$ for the group of Weil divisors of $X$.

  For $K$ a field, a \term{Weil $K$-divisor} is the same where we allow $a_i \in K$.
\end{defi}

\begin{defi}[Effective divisor]\index{effective divisor}
  We say a Weil divisor is \emph{effective} if $a_i \geq 0$ for all $i$. We write $D \geq 0$.
\end{defi}

\begin{defi}[Support]\index{support}
  The \emph{support} of $D = \sum a_i Y_i$ is
  \[
    \supp(D) = \bigcup Y_i.
  \]
\end{defi}
Another important kind of divisors is principal divisors.

\begin{defi}[Principal divisor]\index{principal divisor}\index{$\div(f)$}
  If $f \in K(X)$, then we define the \emph{principal divisor}
  \[
    \div(f) = \sum_{Y} \val_Y(f) \cdot Y.
  \]
\end{defi}
One can show that $\val_Y(f)$ is non-zero for only finitely many $Y$'s, so this is a genuine divisor. To see this, there is always a Zariski open $U$ such that $f|_U$ is invertible, So $\val_Y(f) \not= 0$ implies $Y \subseteq X \setminus U$. Since $X$ is Noetherian, $X \setminus U$ can only contain finitely many codimension $1$ subscheme.
Observe that $\div(fg) = \div(f) + \div(g)$. So the principal divisors form a subgroup of the Weil divisors.

\begin{defi}[Class group]\index{class group}
  The \emph{class group} of $X$, $\Cl(X)$, is the group of Weil divisors quotiented out by the principal divisors.
\end{defi}

\begin{eg}
  Take $X = \A^1_K$, and $f = \frac{x^3}{x + 1} \in K(X)$. Then
  \[
    \div(f) = 3[0] - [-1].
  \]
\end{eg}

A useful result for the future is the following:
\begin{thm}[Hartog's lemma]\index{Hartog's lemma}
  Let $X$ be normal, and $f \in \mathcal{O}(X \setminus V)$ for some $V \geq 2$. Then $f \in \mathcal{O}_X$.
\end{thm}

\subsection{Cartier divisors}
Weil divisors do not behave too well on weirder schemes. A better alternative is Cartier divisors. Let $\mathcal{L}$ be a line bundle, and $ss$ a rational section of $\mathcal{L}$, i.e.\ $s$ is a section of $\mathcal{L}|_U$ for some open $U \subseteq X$. Given such an $s$, we can define
\[
  \div(s) = \sum_Y \val_Y(s) \cdot Y.
\]
To make sense of $\val_Y(s)$, for a fixed codimension $1$ subscheme $Y$, pick $W$ such that $W \cap W \not= \emptyset$, and $\mathcal{L}|_W$ is trivial. Then we can make sense of $\val_Y(s)$ using the trivialization. It is clear from Hartog's lemma that
\begin{prop}
  If $X$ is normal, then
  \[
    \div: \{\text{rational sections of }\mathcal{L}\} \to \WDiv(X).
  \]
  is well-defined, and two sections have the same image iff they differ by an element of $\mathcal{O}_X^*$.
\end{prop}

\begin{cor}
  If $X$ is normal and proper, then
  \[
    \div \{\text{rational sections of }\mathcal{L}\}/K^* \to \WDiv(X).
  \]
\end{cor}

\begin{proof}
  Properness implies $\mathcal{O}_X^* = K^*$.
\end{proof}

\begin{eg}
  Take $X = \P^1_K$, and $s = \frac{X^2}{X + Y} \in H^0(\mathcal{O}(1))$, where $X, Y$ are our homogeneous coordinates. Then
  \[
    \div(s) = 2[0:1] - [1:-1].
  \]
\end{eg}

Let $X$ be a normal Noetherian scheme. Fix a Weil divisor $X$. We define the sheaf $\mathcal{O}_X(D)$\index{$\mathcal{O}_X(D)$} by setting, for all $U \subseteq X$ open,
\[
  \mathcal{O}_U(D) = \{f \in K(X) : \div(f) + D |_U \geq 0\}.
\]
\begin{prop}
  $\mathcal{O}_X(D)$ is a rank $1$ quasicoherent $\mathcal{O}_X$-module. \fakeqed.
\end{prop}

\begin{defi}[Locally principal]\index{locally principal}
  Let $D$ be a Weil divisor on $X$. Fix $x \in X$. Then $D$ is locally principal at $x$ if there exists an open set $U \subseteq X$ containing $x$ such that $D|_U = \div(f)|_U$ for some $f \in K(X)$.
\end{defi}

\begin{prop}
  If $D$ is locally principal at every point $x$, then $\mathcal{O}_X(D)$ is an invertible sheaf.
\end{prop}

\begin{proof}
  If $U \subseteq X$ is such that $D|_U = \div(f)|_U$, then there is an isomorphism
  \begin{align*}
    \mathcal{O}_X|_U &\to \mathcal{O}_X(D) |_U\\
    g &\mapsto g/f.\qedhere
  \end{align*}
\end{proof}

\printindex
\end{document}
