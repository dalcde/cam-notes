\documentclass[a4paper]{article}

\def\npart {III}
\def\nterm {Lent}
\def\nyear {2017}
\def\nlecturer {A. G. Kovalev}
\def\ncourse {Riemannian Geometry}

% Imports
\ifx \nextra \undefined
  \usepackage[pdftex,
    hidelinks,
    pdfauthor={Dexter Chua},
    pdfsubject={Cambridge Maths Notes: Part \npart\ - \ncourse},
    pdftitle={Part \npart\ - \ncourse},
  pdfkeywords={Cambridge Mathematics Maths Math \npart\ \nterm\ \nyear\ \ncourse}]{hyperref}
  \title{Part \npart\ - \ncourse}
\else
  \usepackage[pdftex,
    hidelinks,
    pdfauthor={Dexter Chua},
    pdfsubject={Cambridge Maths Notes: Part \npart\ - \ncourse\ (\nextra)},
    pdftitle={Part \npart\ - \ncourse\ (\nextra)},
  pdfkeywords={Cambridge Mathematics Maths Math \npart\ \nterm\ \nyear\ \ncourse\ \nextra}]{hyperref}

  \title{Part \npart\ - \ncourse \\ {\Large \nextra}}
\fi

\author{Lectured by \nlecturer \\\small Notes taken by Dexter Chua}
\date{\nterm\ \nyear}

\usepackage{alltt}
\usepackage{amsfonts}
\usepackage{amsmath}
\usepackage{amssymb}
\usepackage{amsthm}
\usepackage{booktabs}
\usepackage{caption}
\usepackage{enumitem}
\usepackage{fancyhdr}
\usepackage{graphicx}
\usepackage{mathtools}
\usepackage{microtype}
\usepackage{multirow}
\usepackage{pdflscape}
\usepackage{pgfplots}
\usepackage{siunitx}
\usepackage{tabularx}
\usepackage{tikz}
\usepackage{tkz-euclide}
\usepackage[normalem]{ulem}
\usepackage[all]{xy}

\pgfplotsset{compat=1.12}

\pagestyle{fancyplain}
\lhead{\emph{\nouppercase{\leftmark}}}
\ifx \nextra \undefined
  \rhead{
    \ifnum\thepage=1
    \else
      \npart\ \ncourse
    \fi}
\else
  \rhead{
    \ifnum\thepage=1
    \else
      \npart\ \ncourse\ (\nextra)
    \fi}
\fi
\usetikzlibrary{arrows}
\usetikzlibrary{decorations.markings}
\usetikzlibrary{decorations.pathmorphing}
\usetikzlibrary{positioning}
\usetikzlibrary{fadings}
\usetikzlibrary{intersections}
\usetikzlibrary{cd}

\newcommand*{\Cdot}{\raisebox{-0.25ex}{\scalebox{1.5}{$\cdot$}}}
\newcommand {\pd}[2][ ]{
  \ifx #1 { }
    \frac{\partial}{\partial #2}
  \else
    \frac{\partial^{#1}}{\partial #2^{#1}}
  \fi
}

% Theorems
\theoremstyle{definition}
\newtheorem*{aim}{Aim}
\newtheorem*{axiom}{Axiom}
\newtheorem*{claim}{Claim}
\newtheorem*{cor}{Corollary}
\newtheorem*{defi}{Definition}
\newtheorem*{eg}{Example}
\newtheorem*{fact}{Fact}
\newtheorem*{law}{Law}
\newtheorem*{lemma}{Lemma}
\newtheorem*{notation}{Notation}
\newtheorem*{prop}{Proposition}
\newtheorem*{thm}{Theorem}

\renewcommand{\labelitemi}{--}
\renewcommand{\labelitemii}{$\circ$}
\renewcommand{\labelenumi}{(\roman{*})}

\let\stdsection\section
\renewcommand\section{\newpage\stdsection}

% Strike through
\def\st{\bgroup \ULdepth=-.55ex \ULset}

% Maths symbols
\newcommand{\bra}{\langle}
\newcommand{\ket}{\rangle}

\newcommand{\N}{\mathbb{N}}
\newcommand{\Z}{\mathbb{Z}}
\newcommand{\Q}{\mathbb{Q}}
\renewcommand{\H}{\mathbb{H}}
\newcommand{\R}{\mathbb{R}}
\newcommand{\C}{\mathbb{C}}
\newcommand{\Prob}{\mathbb{P}}
\renewcommand{\P}{\mathbb{P}}
\newcommand{\E}{\mathbb{E}}
\newcommand{\F}{\mathbb{F}}
\newcommand{\cU}{\mathcal{U}}
\newcommand{\RP}{\mathbb{RP}}
\newcommand{\CP}{\mathbb{CP}}

\newcommand{\ph}{\,\cdot\,}

\DeclareMathOperator{\sech}{sech}
\DeclareMathOperator{\cosech}{cosech}
\DeclareMathOperator{\cosec}{cosec}

\DeclareMathOperator{\covol}{covol}
\DeclareMathOperator{\vol}{vol}

\let\Im\relax
\let\Re\relax
\DeclareMathOperator{\Im}{Im}
\DeclareMathOperator{\Re}{Re}
\DeclareMathOperator{\im}{im}
\DeclareMathOperator{\image}{image}
\DeclareMathOperator{\Ann}{Ann}

\DeclareMathOperator*{\res}{res}
\DeclareMathOperator{\Res}{Res}
\DeclareMathOperator{\Ind}{Ind}

\DeclareMathOperator{\tr}{tr}
\DeclareMathOperator{\diag}{diag}
\DeclareMathOperator{\rank}{rank}
\DeclareMathOperator{\card}{card}
\DeclareMathOperator{\spn}{span}
\DeclareMathOperator{\adj}{adj}

\DeclareMathOperator{\erf}{erf}
\DeclareMathOperator{\erfc}{erfc}

\DeclareMathOperator{\ord}{ord}
\DeclareMathOperator{\Sym}{Sym}

\DeclareMathOperator{\sgn}{sgn}
\DeclareMathOperator{\orb}{orb}
\DeclareMathOperator{\stab}{stab}
\DeclareMathOperator{\ccl}{ccl}

\DeclareMathOperator{\lcm}{lcm}
\DeclareMathOperator{\hcf}{hcf}

\DeclareMathOperator{\Int}{Int}
\DeclareMathOperator{\id}{id}

\DeclareMathOperator{\betaD}{beta}
\DeclareMathOperator{\gammaD}{gamma}
\DeclareMathOperator{\Poisson}{Poisson}
\DeclareMathOperator{\binomial}{binomial}
\DeclareMathOperator{\multinomial}{multinomial}
\DeclareMathOperator{\Bernoulli}{Bernoulli}
\DeclareMathOperator{\like}{like}

\DeclareMathOperator{\var}{var}
\DeclareMathOperator{\cov}{cov}
\DeclareMathOperator{\bias}{bias}
\DeclareMathOperator{\mse}{mse}
\DeclareMathOperator{\corr}{corr}

\DeclareMathOperator{\otp}{otp}
\DeclareMathOperator{\dom}{dom}

\DeclareMathOperator{\Root}{Root}
\DeclareMathOperator{\supp}{supp}
\DeclareMathOperator{\rel}{rel}
\DeclareMathOperator{\Hom}{Hom}
\DeclareMathOperator{\Aut}{Aut}
\DeclareMathOperator{\Gal}{Gal}
\DeclareMathOperator{\Mat}{Mat}
\DeclareMathOperator{\End}{End}
\DeclareMathOperator{\Char}{char}
\DeclareMathOperator{\ev}{ev}
\DeclareMathOperator{\St}{St}
\DeclareMathOperator{\Lk}{Lk}
\DeclareMathOperator{\disc}{disc}
\DeclareMathOperator{\Isom}{Isom}
\DeclareMathOperator{\length}{length}
\DeclareMathOperator{\energy}{energy}
\DeclareMathOperator{\area}{area}
\DeclareMathOperator{\Syl}{Syl}
\DeclareMathOperator{\cl}{cl}
\DeclareMathOperator{\fix}{fix}

\newcommand{\GL}{\mathrm{GL}}
\newcommand{\SL}{\mathrm{SL}}
\newcommand{\PGL}{\mathrm{PGL}}
\newcommand{\PSL}{\mathrm{PSL}}
\newcommand{\PSU}{\mathrm{PSU}}
\newcommand{\Or}{\mathrm{O}}
\newcommand{\SO}{\mathrm{SO}}
\newcommand{\U}{\mathrm{U}}
\newcommand{\SU}{\mathrm{SU}}

\renewcommand{\d}{\mathrm{d}}
\newcommand{\D}{\mathrm{D}}

\tikzset{->/.style = {decoration={markings,
                                  mark=at position 1 with {\arrow[scale=2]{latex'}}},
                      postaction={decorate}}}
\tikzset{<-/.style = {decoration={markings,
                                  mark=at position 0 with {\arrowreversed[scale=2]{latex'}}},
                      postaction={decorate}}}
\tikzset{<->/.style = {decoration={markings,
                                   mark=at position 0 with {\arrowreversed[scale=2]{latex'}},
                                   mark=at position 1 with {\arrow[scale=2]{latex'}}},
                       postaction={decorate}}}
\tikzset{->-/.style = {decoration={markings,
                                   mark=at position #1 with {\arrow[scale=2]{latex'}}},
                       postaction={decorate}}}
\tikzset{-<-/.style = {decoration={markings,
                                   mark=at position #1 with {\arrowreversed[scale=2]{latex'}}},
                       postaction={decorate}}}

\tikzset{circ/.style = {fill, circle, inner sep = 0, minimum size = 3}}
\tikzset{mstate/.style={circle, draw, blue, text=black, minimum width=0.7cm}}

\definecolor{mblue}{rgb}{0.2, 0.3, 0.8}
\definecolor{morange}{rgb}{1, 0.5, 0}
\definecolor{mgreen}{rgb}{0.1, 0.4, 0.2}
\definecolor{mred}{rgb}{0.5, 0, 0}

\def\drawcirculararc(#1,#2)(#3,#4)(#5,#6){%
    \pgfmathsetmacro\cA{(#1*#1+#2*#2-#3*#3-#4*#4)/2}%
    \pgfmathsetmacro\cB{(#1*#1+#2*#2-#5*#5-#6*#6)/2}%
    \pgfmathsetmacro\cy{(\cB*(#1-#3)-\cA*(#1-#5))/%
                        ((#2-#6)*(#1-#3)-(#2-#4)*(#1-#5))}%
    \pgfmathsetmacro\cx{(\cA-\cy*(#2-#4))/(#1-#3)}%
    \pgfmathsetmacro\cr{sqrt((#1-\cx)*(#1-\cx)+(#2-\cy)*(#2-\cy))}%
    \pgfmathsetmacro\cA{atan2(#2-\cy,#1-\cx)}%
    \pgfmathsetmacro\cB{atan2(#6-\cy,#5-\cx)}%
    \pgfmathparse{\cB<\cA}%
    \ifnum\pgfmathresult=1
        \pgfmathsetmacro\cB{\cB+360}%
    \fi
    \draw (#1,#2) arc (\cA:\cB:\cr);%
}
\newcommand\getCoord[3]{\newdimen{#1}\newdimen{#2}\pgfextractx{#1}{\pgfpointanchor{#3}{center}}\pgfextracty{#2}{\pgfpointanchor{#3}{center}}}

\def\Xint#1{\mathchoice
   {\XXint\displaystyle\textstyle{#1}}%
   {\XXint\textstyle\scriptstyle{#1}}%
   {\XXint\scriptstyle\scriptscriptstyle{#1}}%
   {\XXint\scriptscriptstyle\scriptscriptstyle{#1}}%
   \!\int}
\def\XXint#1#2#3{{\setbox0=\hbox{$#1{#2#3}{\int}$}
     \vcenter{\hbox{$#2#3$}}\kern-.5\wd0}}
\def\ddashint{\Xint=}
\def\dashint{\Xint-}


\begin{document}
\maketitle
{\small
\setlength{\parindent}{0em}
\setlength{\parskip}{1em}
This course is a possible natural sequel of the course Differential Geometry offered in Michaelmas Term. We shall explore various techniques and results revealing intricate and subtle relations between Riemannian metrics, curvature and topology. I hope to cover much of the following:

\emph{A closer look at geodesics and curvature.} Brief review from the Differential Geometry course. Geodesic coordinates and Gauss' lemma. Jacobi fields, completeness and the Hopf--Rinow theorem. Variations of energy, Bonnet--Myers diameter theorem and Synge's theorem.

\emph{Hodge theory and Riemannian holonomy.} The Hodge star and Laplace--Beltrami operator. The Hodge decomposition theorem (with the `geometry part' of the proof). Bochner--Weitzenock formulae. Holonomy groups. Interplays with curvature and de Rham cohomology.

\emph{Ricci curvature.} Fundamental groups and Ricci curvature. The Cheeger--Gromoll splitting theorem.

\subsubsection*{Pre-requisites}
Manifolds, differential forms, vector fields. Basic concepts of Riemannian geometry (curvature, geodesics etc.) and Lie groups. The course Differential Geometry offered in Michaelmas Term is the ideal pre-requisite.
}
\tableofcontents

\section{Basics of Riemannian manifolds}
This will be a very brief summary of some topics in the Michaelmas Differential Geometry course.

Let $M$ be a smooth manifold. A \term{Riemannian metric} $g$ on $M$ is an inner product on the tangent bundle $TM$ varying smoothly with the fibers. Formally, this is a global section of $T^*M \otimes T^*M$ that is fiberwise symmetric and positive definite.

On every coordinate neighbourhood with coordinates $x = (x_1, \cdots, x_n)$, we can write
\[
  g = \sum_{i, j = 1}^n g_{ij}(x)\;\d x_i \d x_j.
\]
These $g_{ij}$ are given by
\[
  g_{ij} = g\left(\frac{\partial}{\partial x_i}, \frac{\partial}{\partial x_j}\right)
\]
and are $C^\infty$ functions. The pair $(M, g)$ is called a \term{Riemannian manifold}.

\begin{eg}
  The manifold $\R^k$ has a canonical metric given by the Euclidean metric. In the usual coordinates, $g$ is given by $g_{ij} = \delta_{ij}$.
\end{eg}

The Whitney embedding theorem says every smooth manifold $M$ admits an embedding into $\R^k$ for some $k$. In other words, $M$ is diffeomorphic to a submanifold of $\R^k$. In fact, we can pick $k$ such that $k \leq 2 \dim M$.

Using such an embedding, we can induce a Riemannian metric on $M$ by restricting the inner product from Euclidean space, since we have inclusions $T_pM \hookrightarrow T_p \R^k \cong \R^k$.

A bit more generally, if $(N, h)$ is a Riemannian manifold, and $F: M \to N$ is an immersion, then the pullback $g = F^*h$ defines a metric on $M$, where the condition of immersion is required for the pullback to be non-degenerate.

In particular, we can consider the case when $F$ is a diffeomorphism $M \to N$.
\begin{defi}[Isometry]\index{isometry}
  Let $(M, g)$ and $(N, h)$ be Riemannian manifolds. We say $f: M \to N$ is an \emph{isometry} if it is a diffeomorphism and $f^*h = g$. In other words, for any $p \in M$ and $u, v \in T_p M$, we need
  \[
    h\big((\d f)_p u, (\d f)_p v\big) = g(u, v).
  \]
\end{defi}
This is a natural equivalence relation between Riemannian manifolds.

\begin{eg}
  Let $G$ be a Lie group. Then for any $x$, we have maps $L_x, R_x: G \to G$ given by
  \begin{align*}
    L_x(y) &= xy\\
    R_x(y) &= yx
  \end{align*}
  These maps are in fact diffeomorphisms of $G$.

  Given such a group $G$, we can define the \term{Lie algebra} $\mathfrak{g} = T_e G$. We say a vector field $X$ on $G$ is said to be \emph{left-invariant} if for any $x \in G$, we have $\d (L_x) X = X$. Any such $X$ arises from picking some $X_e \in \mathfrak{g}$ and then setting
  \[
    X_a = \d (L_a) X_e.
  \]
  The resulting vector field is indeed smooth, as shown in the differential geometry course.

  We say a Riemannian metric $g$ is \emph{left-invariant}\index{left-invariant metric}\index{right-invariant metric} if each $L_x$ is an isometry. Again, to construct these, we can just pick a metric at the identity and the propagating it around using left-translation. More explicitly, given any inner product on $\bra \ph, \ph\ket$ on $T_eG$, we can define $g$ by
  \[
    g(u, v) = \bra (\d L_{x^{-1}})_x u, (\d L_{x^{-1}})_x v\ket
  \]
  for all $x \in G$ and $u, v \in T_x G$. The argument for smoothness is similar to that for vector fields.
\end{eg}
Of course, everything works when we replace ``left'' with ``right''. A Riemannian metric is said to be \emph{bi-invariant}\index{bi-invariant metric} if it is both left- and right-invariant. These are harder to find, but it is a fact that every compact Lie group admits a bi-invariant metric. The basic idea of the proof is to start from a left-invariant metric, then integrate the metric along right translations of all group elements. Here compactness is necessary for the result to be finite.

We will later see that we cannot drop the compactness condition. There are non-compact Lie groups that do not admit bi-invariant metrics, such as $\SL(2, \R)$.

\subsubsection*{Levi-Cevita connection}
Let $(M, g)$ be a Riemannian manifold. Then there is a preferred choice of connection on $M$ (ie. on $TM$), known as the Levi-Cevita connection $\nabla: \Omega^0_M(TM) \to \Omega^1_M(TM)$. This satisfies the compatibility condition
\[
  Z g(X, Y) = g(\nabla_Z X, Y) + g(X, \nabla_Z Y),
\]
and also it is symmetric in the sense that
\[
  \nabla_X Y - \nabla_Y X = [X, Y].
\]
The first property can be written as
\[
  \d (g(X, Y)) = g(\nabla X, Y) + g(X, \nabla Y),
\]
and the second property can be expressed in coordinate representation by
\[
  \Gamma_{jk}^i = \Gamma_{kj}^i,
\]
where $\Gamma_{jk}^i$ is defined by
\[
  \nabla_{\partial/\partial x_j} \frac{\partial}{\partial x_k} = \sum_{i = 1}^n \Gamma_{jk}^i \frac{\partial}{\partial x_i}.
\]
The connection $\nabla$ induces a unique covariant derivative on $T^*M$, also denoted $\nabla$, given by
\[
  X\bra \alpha, Y\ket = \bra \nabla_X \alpha , Y\ket + \bra \alpha, \nabla_X Y\ket
\]
for any $X,Y \in \Vect(M)$ and $\alpha \in \in \Omega^1(M)$. This can be further extended to a unique connection $\nabla$ on tensor bundles $(TM)^{\otimes q} \otimes (T^*M)^{\otimes p}$ for any $p, q \geq 0$.

This is a completely general construction. In general, suppose we have vector bundles $E$ and $F$, and $s_1 \in \Gamma(E)$ and $s_2 \in \Gamma(F)$. If we have connections $\Gamma^E$ and $\Gamma^F$ on $E$ and $F$ respectively, then we can define
\[
  \nabla^{E\otimes F} (s_1 \otimes s_2) = (\nabla^E s_1) \otimes s_2 + s_1 \otimes (\nabla^F s_2).
\]
Recall that the Riemannian metric is formally a section $g \in \Gamma(T^*M \otimes T^*M)$. Then the compatibility with the metric can be written in the following even more compact form:
\[
  \nabla g = 0.
\]

\printindex
\end{document}
