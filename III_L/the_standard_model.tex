\documentclass[a4paper]{article}

\def\npart {III}
\def\nterm {Lent}
\def\nyear {2017}
\def\nlecturer {C. E. Thomas}
\def\ncourse {The Standard Model}
\def\nofficial {https://www.damtp.cam.ac.uk/user/cet34/teaching/SM/}

% Imports
\ifx \nextra \undefined
  \usepackage[pdftex,
    hidelinks,
    pdfauthor={Dexter Chua},
    pdfsubject={Cambridge Maths Notes: Part \npart\ - \ncourse},
    pdftitle={Part \npart\ - \ncourse},
  pdfkeywords={Cambridge Mathematics Maths Math \npart\ \nterm\ \nyear\ \ncourse}]{hyperref}
  \title{Part \npart\ - \ncourse}
\else
  \usepackage[pdftex,
    hidelinks,
    pdfauthor={Dexter Chua},
    pdfsubject={Cambridge Maths Notes: Part \npart\ - \ncourse\ (\nextra)},
    pdftitle={Part \npart\ - \ncourse\ (\nextra)},
  pdfkeywords={Cambridge Mathematics Maths Math \npart\ \nterm\ \nyear\ \ncourse\ \nextra}]{hyperref}

  \title{Part \npart\ - \ncourse \\ {\Large \nextra}}
\fi

\author{Lectured by \nlecturer \\\small Notes taken by Dexter Chua}
\date{\nterm\ \nyear}

\usepackage{alltt}
\usepackage{amsfonts}
\usepackage{amsmath}
\usepackage{amssymb}
\usepackage{amsthm}
\usepackage{booktabs}
\usepackage{caption}
\usepackage{enumitem}
\usepackage{fancyhdr}
\usepackage{graphicx}
\usepackage{mathtools}
\usepackage{microtype}
\usepackage{multirow}
\usepackage{pdflscape}
\usepackage{pgfplots}
\usepackage{siunitx}
\usepackage{tabularx}
\usepackage{tikz}
\usepackage{tkz-euclide}
\usepackage[normalem]{ulem}
\usepackage[all]{xy}

\pgfplotsset{compat=1.12}

\pagestyle{fancyplain}
\lhead{\emph{\nouppercase{\leftmark}}}
\ifx \nextra \undefined
  \rhead{
    \ifnum\thepage=1
    \else
      \npart\ \ncourse
    \fi}
\else
  \rhead{
    \ifnum\thepage=1
    \else
      \npart\ \ncourse\ (\nextra)
    \fi}
\fi
\usetikzlibrary{arrows}
\usetikzlibrary{decorations.markings}
\usetikzlibrary{decorations.pathmorphing}
\usetikzlibrary{positioning}
\usetikzlibrary{fadings}
\usetikzlibrary{intersections}
\usetikzlibrary{cd}

\newcommand*{\Cdot}{\raisebox{-0.25ex}{\scalebox{1.5}{$\cdot$}}}
\newcommand {\pd}[2][ ]{
  \ifx #1 { }
    \frac{\partial}{\partial #2}
  \else
    \frac{\partial^{#1}}{\partial #2^{#1}}
  \fi
}

% Theorems
\theoremstyle{definition}
\newtheorem*{aim}{Aim}
\newtheorem*{axiom}{Axiom}
\newtheorem*{claim}{Claim}
\newtheorem*{cor}{Corollary}
\newtheorem*{defi}{Definition}
\newtheorem*{eg}{Example}
\newtheorem*{fact}{Fact}
\newtheorem*{law}{Law}
\newtheorem*{lemma}{Lemma}
\newtheorem*{notation}{Notation}
\newtheorem*{prop}{Proposition}
\newtheorem*{thm}{Theorem}

\renewcommand{\labelitemi}{--}
\renewcommand{\labelitemii}{$\circ$}
\renewcommand{\labelenumi}{(\roman{*})}

\let\stdsection\section
\renewcommand\section{\newpage\stdsection}

% Strike through
\def\st{\bgroup \ULdepth=-.55ex \ULset}

% Maths symbols
\newcommand{\bra}{\langle}
\newcommand{\ket}{\rangle}

\newcommand{\N}{\mathbb{N}}
\newcommand{\Z}{\mathbb{Z}}
\newcommand{\Q}{\mathbb{Q}}
\renewcommand{\H}{\mathbb{H}}
\newcommand{\R}{\mathbb{R}}
\newcommand{\C}{\mathbb{C}}
\newcommand{\Prob}{\mathbb{P}}
\renewcommand{\P}{\mathbb{P}}
\newcommand{\E}{\mathbb{E}}
\newcommand{\F}{\mathbb{F}}
\newcommand{\cU}{\mathcal{U}}
\newcommand{\RP}{\mathbb{RP}}
\newcommand{\CP}{\mathbb{CP}}

\newcommand{\ph}{\,\cdot\,}

\DeclareMathOperator{\sech}{sech}
\DeclareMathOperator{\cosech}{cosech}
\DeclareMathOperator{\cosec}{cosec}

\DeclareMathOperator{\covol}{covol}
\DeclareMathOperator{\vol}{vol}

\let\Im\relax
\let\Re\relax
\DeclareMathOperator{\Im}{Im}
\DeclareMathOperator{\Re}{Re}
\DeclareMathOperator{\im}{im}
\DeclareMathOperator{\image}{image}
\DeclareMathOperator{\Ann}{Ann}

\DeclareMathOperator*{\res}{res}
\DeclareMathOperator{\Res}{Res}
\DeclareMathOperator{\Ind}{Ind}

\DeclareMathOperator{\tr}{tr}
\DeclareMathOperator{\diag}{diag}
\DeclareMathOperator{\rank}{rank}
\DeclareMathOperator{\card}{card}
\DeclareMathOperator{\spn}{span}
\DeclareMathOperator{\adj}{adj}

\DeclareMathOperator{\erf}{erf}
\DeclareMathOperator{\erfc}{erfc}

\DeclareMathOperator{\ord}{ord}
\DeclareMathOperator{\Sym}{Sym}

\DeclareMathOperator{\sgn}{sgn}
\DeclareMathOperator{\orb}{orb}
\DeclareMathOperator{\stab}{stab}
\DeclareMathOperator{\ccl}{ccl}

\DeclareMathOperator{\lcm}{lcm}
\DeclareMathOperator{\hcf}{hcf}

\DeclareMathOperator{\Int}{Int}
\DeclareMathOperator{\id}{id}

\DeclareMathOperator{\betaD}{beta}
\DeclareMathOperator{\gammaD}{gamma}
\DeclareMathOperator{\Poisson}{Poisson}
\DeclareMathOperator{\binomial}{binomial}
\DeclareMathOperator{\multinomial}{multinomial}
\DeclareMathOperator{\Bernoulli}{Bernoulli}
\DeclareMathOperator{\like}{like}

\DeclareMathOperator{\var}{var}
\DeclareMathOperator{\cov}{cov}
\DeclareMathOperator{\bias}{bias}
\DeclareMathOperator{\mse}{mse}
\DeclareMathOperator{\corr}{corr}

\DeclareMathOperator{\otp}{otp}
\DeclareMathOperator{\dom}{dom}

\DeclareMathOperator{\Root}{Root}
\DeclareMathOperator{\supp}{supp}
\DeclareMathOperator{\rel}{rel}
\DeclareMathOperator{\Hom}{Hom}
\DeclareMathOperator{\Aut}{Aut}
\DeclareMathOperator{\Gal}{Gal}
\DeclareMathOperator{\Mat}{Mat}
\DeclareMathOperator{\End}{End}
\DeclareMathOperator{\Char}{char}
\DeclareMathOperator{\ev}{ev}
\DeclareMathOperator{\St}{St}
\DeclareMathOperator{\Lk}{Lk}
\DeclareMathOperator{\disc}{disc}
\DeclareMathOperator{\Isom}{Isom}
\DeclareMathOperator{\length}{length}
\DeclareMathOperator{\energy}{energy}
\DeclareMathOperator{\area}{area}
\DeclareMathOperator{\Syl}{Syl}
\DeclareMathOperator{\cl}{cl}
\DeclareMathOperator{\fix}{fix}

\newcommand{\GL}{\mathrm{GL}}
\newcommand{\SL}{\mathrm{SL}}
\newcommand{\PGL}{\mathrm{PGL}}
\newcommand{\PSL}{\mathrm{PSL}}
\newcommand{\PSU}{\mathrm{PSU}}
\newcommand{\Or}{\mathrm{O}}
\newcommand{\SO}{\mathrm{SO}}
\newcommand{\U}{\mathrm{U}}
\newcommand{\SU}{\mathrm{SU}}

\renewcommand{\d}{\mathrm{d}}
\newcommand{\D}{\mathrm{D}}

\tikzset{->/.style = {decoration={markings,
                                  mark=at position 1 with {\arrow[scale=2]{latex'}}},
                      postaction={decorate}}}
\tikzset{<-/.style = {decoration={markings,
                                  mark=at position 0 with {\arrowreversed[scale=2]{latex'}}},
                      postaction={decorate}}}
\tikzset{<->/.style = {decoration={markings,
                                   mark=at position 0 with {\arrowreversed[scale=2]{latex'}},
                                   mark=at position 1 with {\arrow[scale=2]{latex'}}},
                       postaction={decorate}}}
\tikzset{->-/.style = {decoration={markings,
                                   mark=at position #1 with {\arrow[scale=2]{latex'}}},
                       postaction={decorate}}}
\tikzset{-<-/.style = {decoration={markings,
                                   mark=at position #1 with {\arrowreversed[scale=2]{latex'}}},
                       postaction={decorate}}}

\tikzset{circ/.style = {fill, circle, inner sep = 0, minimum size = 3}}
\tikzset{mstate/.style={circle, draw, blue, text=black, minimum width=0.7cm}}

\definecolor{mblue}{rgb}{0.2, 0.3, 0.8}
\definecolor{morange}{rgb}{1, 0.5, 0}
\definecolor{mgreen}{rgb}{0.1, 0.4, 0.2}
\definecolor{mred}{rgb}{0.5, 0, 0}

\def\drawcirculararc(#1,#2)(#3,#4)(#5,#6){%
    \pgfmathsetmacro\cA{(#1*#1+#2*#2-#3*#3-#4*#4)/2}%
    \pgfmathsetmacro\cB{(#1*#1+#2*#2-#5*#5-#6*#6)/2}%
    \pgfmathsetmacro\cy{(\cB*(#1-#3)-\cA*(#1-#5))/%
                        ((#2-#6)*(#1-#3)-(#2-#4)*(#1-#5))}%
    \pgfmathsetmacro\cx{(\cA-\cy*(#2-#4))/(#1-#3)}%
    \pgfmathsetmacro\cr{sqrt((#1-\cx)*(#1-\cx)+(#2-\cy)*(#2-\cy))}%
    \pgfmathsetmacro\cA{atan2(#2-\cy,#1-\cx)}%
    \pgfmathsetmacro\cB{atan2(#6-\cy,#5-\cx)}%
    \pgfmathparse{\cB<\cA}%
    \ifnum\pgfmathresult=1
        \pgfmathsetmacro\cB{\cB+360}%
    \fi
    \draw (#1,#2) arc (\cA:\cB:\cr);%
}
\newcommand\getCoord[3]{\newdimen{#1}\newdimen{#2}\pgfextractx{#1}{\pgfpointanchor{#3}{center}}\pgfextracty{#2}{\pgfpointanchor{#3}{center}}}

\def\Xint#1{\mathchoice
   {\XXint\displaystyle\textstyle{#1}}%
   {\XXint\textstyle\scriptstyle{#1}}%
   {\XXint\scriptstyle\scriptscriptstyle{#1}}%
   {\XXint\scriptscriptstyle\scriptscriptstyle{#1}}%
   \!\int}
\def\XXint#1#2#3{{\setbox0=\hbox{$#1{#2#3}{\int}$}
     \vcenter{\hbox{$#2#3$}}\kern-.5\wd0}}
\def\ddashint{\Xint=}
\def\dashint{\Xint-}


\begin{document}
\maketitle
{\small
\setlength{\parindent}{0em}
\setlength{\parskip}{1em}
The Standard Model of particle physics is, by far, the most successful application of quantum field theory (QFT). At the time of writing, it accurately describes all experimental measurements involving strong, weak, and electromagnetic interactions. The course aims to demonstrate how this model, a QFT with gauge group $\SU(3) \times \SU(2) \times \U(1)$ and fermion fields for the leptons and quarks, is realised in nature. It is intended to complement the more general Advanced QFT course.

We begin by defining the Standard Model in terms of its local (gauge) and global symmetries and its elementary particle content (spin-half leptons and quarks, and spin-one gauge bosons). The parity $P$, charge-conjugation $C$ and time-reversal $T$ transformation properties of the theory are investigated. These need not be symmetries manifest in nature; e.g. only left-handed particles feel the weak force and so it violates parity symmetry. We show how $CP$ violation becomes possible when there are three generations of particles and describe its consequences.

Ideas of spontaneous symmetry breaking are applied to discuss the Higgs Mechanism and why the weakness of the weak force is due to the spontaneous breaking of the $\SU(2) \times \U(1)$ gauge symmetry. Recent measurements of what appear to be Higgs boson decays will be presented.

We show how to obtain cross sections and decay rates from the matrix element squared of a process. These can be computed for various scattering and decay processes in the electroweak sector using perturbation theory because the couplings are small. We touch upon the topic of neutrino masses and oscillations, an important window to physics beyond the Standard Model.

The strong interaction is described by quantum chromodynamics (QCD), the non-abelian gauge theory of the (unbroken) $\SU(3)$ gauge symmetry. At low energies quarks are confined and form bound states called hadrons. The coupling constant decreases as the energy scale increases, to the point where perturbation theory can be used. As an example we consider electron- positron annihilation to final state hadrons at high energies. Time permitting, we will discuss nonperturbative approaches to QCD. For example, the framework of effective field theories can be used to make progress in the limits of very small and very large quark masses.

Both very high-energy experiments and very precise experiments are currently striving to observe effects that cannot be described by the Standard Model alone. If time permits, we comment on how the Standard Model is treated as an effective field theory to accommodate (so far hypothetical) effects beyond the Standard Model.

\subsubsection*{Pre-requisites}
It is necessary to have attended the Quantum Field Theory and the Symmetries, Fields and Particles courses, or to be familiar with the material covered in them. It would be advantageous to attend the Advanced QFT course during the same term as this course, or to study renormalisation and non-abelian gauge fixing.
}
\tableofcontents

\section{Introduction}
\emph{It the whole course, we will ignore the existence of gravity.}

In the standard model, there are a few classes of things:
\begin{itemize}
  \item Forces are mediated by spin 1 \term{gauge bosons}. These include
    \begin{itemize}
      \item The \term{electromagnetic field}, which is mediated by the \term{photon}. This is described by \term{quantum electrodynamics} (\term{QED});
      \item The \term{weak interaction}\index{weak nuclear force}, which is mediated by the $W^{\pm}$\index{$W^{\pm}$ boson} and $Z$\index{$Z$ boson} \emph{bosons}; and
      \item The \term{strong interaction}\index{strong force}, which is mediated by \term{gluons} $g$. This is described by the theory of \term{quantum chromodynamics} (\term{QCD}).
    \end{itemize}
    While the electromagnetic field and weak interaction seem very different, we will see that at high energies, they merge together, and can be described by a single gauge group.
  \item Matter is described by spin $\frac{1}{2}$ \term{fermions}. These include
    \begin{itemize}
      \item \emph{Neutrinos}\index{neutrinos}: $\nu_e, \nu_\mu, \nu_\tau$. These interact only via the weak interaction.
      \item \emph{Charged leptons}\index{leptons}\index{charged leptons}: $e, \mu, \tau$. These interact with the electromagnetic field and weak interactions.
      \item \emph{Quarks}\index{quarks}: $\begin{pmatrix}u\\d\end{pmatrix}$, $\begin{pmatrix}c\\s\end{pmatrix}$, $\begin{pmatrix}t\\b\end{pmatrix}$. These have electric charges $\begin{pmatrix}+\frac{2}{3}\\ -\frac{1}{3}\end{pmatrix}$. They interact with all interactions.
    \end{itemize}
    We see that each type of matter comes in three \term{generations}. We do not know why.
  \item There is the \term{Higgs boson}, which has spin 0. This is responsible for giving mass to the $W^{\pm}, Z$ bosons and fermions. This was just discovered in 2012 in CERN, and subsequently more properties have been discovered, eg. its spin.
\end{itemize}

As one would expect from the name, the gauge bosons are manifestations of local gauge symmetries. The gauge group in the Standard Model is
\[
  \SU(3)_C \times \SU(2)_L \times \U(1)_Y.
\]
The subscripts indicate which things the group are responsible for. The $\SU(3)_C$ describes the strong force, and gives us QCD. The $C$ stands for ``colour''. This is complicated.

The remaining bit is what we are going to focus on for most of the course. The $\SU(2)_L$ interaction is chiral, and only couples to left-handed particles. The $Y$ at $\U(1)_Y$ refers to what is known as the hypercharge. The $\SU(2)_L \times \U(1)_Y$ gives us a \emph{unified} description of QED and weak interaction, collectively known as the electroweak force. We will see that there is spontaneous symmetry breaking of this electroweak part, which will give us weak and electromagnetic interactions.

Note that a lot of the standard model was discovered experimentally, but this course focuses on the theoretical parts of standard model. Thus, we will mostly pull these theories out of a hat without giving much motivation for what that is the case.

\section{Chiral and gauge symmetries}
We'll move reasonably quickly through this, as we are just reviewing some concepts from the QFT course. Throughout, we will use ``natural units'' where $c = \hbar = 1$.

\subsection{Chiral symmetry}
Consider a spin-$\frac{1}{2}$ \term{Dirac fermion} $\psi$, which satisfies the Dirac equation
\[
  (i \slashed{\partial} - m) \psi = 0,
\]
where as usual
\[
  \slashed{\partial} = \gamma^\mu \partial_\mu,
\]
and the \term{Dirac matrices}\index{$\gamma^\mu$} $\gamma^\mu$ satisfy the \term{Clifford algebra} relations
\[
  \{\gamma^\mu, \gamma^\nu\} = 2 g^{\mu\nu} I,
\]
where $g^{\mu\nu} = \diag(+1, -1, -1, -1)$ is the Minkowski metric. We usually drop the ``$I$'' from the equation.

We define\index{$\gamma^5$}
\[
  \gamma^\mu = +i \gamma^0 \gamma^1 \gamma^2 \gamma^3,
\]
which satisfies
\[
  (\gamma^5)^2 = I,\quad \{\gamma^5, \gamma^\mu\} = 0.
\]
One can do a lot of stuff without choosing a particular basis/representation for the $\gamma$-matrices, and the physics we get out must be the same regardless of which representation we choose, but sometimes it is convenient to pick some particular representation to work with. We'll generally use the \term{chiral representation} (or \term{Weyl representation}), where
\[
  \gamma^0 =
  \begin{pmatrix}
    0 & 1 \\
    1 & 0
  \end{pmatrix}, \quad
  \gamma^i =
  \begin{pmatrix}
    0 & \sigma^i\\
    -\sigma^i & 0
  \end{pmatrix},\quad
  \gamma^5 =
  \begin{pmatrix}
    -1 & 0\\
    0 & 1
  \end{pmatrix},
\]
where the $\sigma^i$ are the \term{Pauli matrices}.

Consider the massless limit of the Dirac equation. Then we just have
\[
  \slashed{\partial} \psi = 0.
\]
Using the anti-commutator relation of $\gamma^\mu$ with $\gamma^5$, we find that we also have
\[
  \slashed{\partial} \gamma^5 \psi = 0.
\]
We now define the projection operators
\[
  P_{R, L} = \frac{1}{2} (1 \pm \gamma^5),
\]
which satisfy
\[
  (P_{R, L})^2 = P_{R, L},\quad P_L + P_R = I,\quad P_L P_R = P_R P_L = 0.
\]
We define
\[
  \psi_{R, L} = P_{R, L} \psi.
\]
Then we have
\[
  \gamma^5 \psi_{R, L} = \pm \psi_{R, L}.
\]
We say $\psi_{R, L}$ have \emph{definite chirality}\index{chirality}, and we say they are ``\term{right-handed}'' and ``\term{left-handed}''.

In the chiral representation, we have
\[
  P_L =
  \begin{pmatrix}
    1 & 0\\
    0 & 0
  \end{pmatrix},
  P_R =
  \begin{pmatrix}
    0 & 0\\
    0 & 1
  \end{pmatrix}.
\]
So $\psi_{R, L}$ only contains lower (upper resp.) two components. In the case of quantum field theory, $\psi_R$ annihilates right-handed chiral particles, and $\psi_L$ annihilates left-handed chiral particles.

\separator
We can try to write down the Dirac Lagrangian (density) for a general fermion:
\[
  L = \bar\psi (i \slashed{\partial} - m) \psi = \bar\psi_L i \slashed{\partial} \psi_L + \bar\psi_L i \slashed{\partial} \psi_R - m (\bar\psi_L \psi_R + \bar\psi_R \psi_L).
\]
If the fermion is massless, then we have a $\U(1)_L \times \U(1)_R$ global symmetry --- under an element $(\alpha_L, \alpha_R) \in \U(1)_L \times \U(1)_R$, the fermion transforms as
\[
  \begin{pmatrix}
    \psi_L\\
    \psi_R
  \end{pmatrix} \mapsto
  \begin{pmatrix}
    e^{i\alpha_L} \psi_L\\
    e^{i\alpha_R} \psi_R
  \end{pmatrix}.
\]
The adjoint field transforms as
\[
  \begin{pmatrix}
    \bar{\psi}_L\\
    \bar{\psi}_R
  \end{pmatrix} \mapsto
  \begin{pmatrix}
    e^{-i\alpha_L} \bar{\psi}_L\\
    e^{-i\alpha_R} \bar{\psi}_R
  \end{pmatrix}.
\]
In the case of a massive particle, we only have a single $\U(1)$ symmetry, where we have to transform $\psi_L$ and $\psi_R$ in the same way, ie. pick $\alpha_L = \alpha_R$.


\printindex
\end{document}
