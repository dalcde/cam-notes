\documentclass[a4paper]{article}

\def\npart {III}
\def\nterm {Lent}
\def\nyear {2017}
\def\nlecturer {C. E. Thomas}
\def\ncourse {The Standard Model}
\def\nofficial {https://www.damtp.cam.ac.uk/user/cet34/teaching/SM/}
\def\nlectures {MWF.9}

% Imports
\ifx \nextra \undefined
  \usepackage[pdftex,
    hidelinks,
    pdfauthor={Dexter Chua},
    pdfsubject={Cambridge Maths Notes: Part \npart\ - \ncourse},
    pdftitle={Part \npart\ - \ncourse},
  pdfkeywords={Cambridge Mathematics Maths Math \npart\ \nterm\ \nyear\ \ncourse}]{hyperref}
  \title{Part \npart\ - \ncourse}
\else
  \usepackage[pdftex,
    hidelinks,
    pdfauthor={Dexter Chua},
    pdfsubject={Cambridge Maths Notes: Part \npart\ - \ncourse\ (\nextra)},
    pdftitle={Part \npart\ - \ncourse\ (\nextra)},
  pdfkeywords={Cambridge Mathematics Maths Math \npart\ \nterm\ \nyear\ \ncourse\ \nextra}]{hyperref}

  \title{Part \npart\ - \ncourse \\ {\Large \nextra}}
\fi

\author{Lectured by \nlecturer \\\small Notes taken by Dexter Chua}
\date{\nterm\ \nyear}

\usepackage{alltt}
\usepackage{amsfonts}
\usepackage{amsmath}
\usepackage{amssymb}
\usepackage{amsthm}
\usepackage{booktabs}
\usepackage{caption}
\usepackage{enumitem}
\usepackage{fancyhdr}
\usepackage{graphicx}
\usepackage{mathtools}
\usepackage{microtype}
\usepackage{multirow}
\usepackage{pdflscape}
\usepackage{pgfplots}
\usepackage{siunitx}
\usepackage{tabularx}
\usepackage{tikz}
\usepackage{tkz-euclide}
\usepackage[normalem]{ulem}
\usepackage[all]{xy}

\pgfplotsset{compat=1.12}

\pagestyle{fancyplain}
\lhead{\emph{\nouppercase{\leftmark}}}
\ifx \nextra \undefined
  \rhead{
    \ifnum\thepage=1
    \else
      \npart\ \ncourse
    \fi}
\else
  \rhead{
    \ifnum\thepage=1
    \else
      \npart\ \ncourse\ (\nextra)
    \fi}
\fi
\usetikzlibrary{arrows}
\usetikzlibrary{decorations.markings}
\usetikzlibrary{decorations.pathmorphing}
\usetikzlibrary{positioning}
\usetikzlibrary{fadings}
\usetikzlibrary{intersections}
\usetikzlibrary{cd}

\newcommand*{\Cdot}{\raisebox{-0.25ex}{\scalebox{1.5}{$\cdot$}}}
\newcommand {\pd}[2][ ]{
  \ifx #1 { }
    \frac{\partial}{\partial #2}
  \else
    \frac{\partial^{#1}}{\partial #2^{#1}}
  \fi
}

% Theorems
\theoremstyle{definition}
\newtheorem*{aim}{Aim}
\newtheorem*{axiom}{Axiom}
\newtheorem*{claim}{Claim}
\newtheorem*{cor}{Corollary}
\newtheorem*{defi}{Definition}
\newtheorem*{eg}{Example}
\newtheorem*{fact}{Fact}
\newtheorem*{law}{Law}
\newtheorem*{lemma}{Lemma}
\newtheorem*{notation}{Notation}
\newtheorem*{prop}{Proposition}
\newtheorem*{thm}{Theorem}

\renewcommand{\labelitemi}{--}
\renewcommand{\labelitemii}{$\circ$}
\renewcommand{\labelenumi}{(\roman{*})}

\let\stdsection\section
\renewcommand\section{\newpage\stdsection}

% Strike through
\def\st{\bgroup \ULdepth=-.55ex \ULset}

% Maths symbols
\newcommand{\bra}{\langle}
\newcommand{\ket}{\rangle}

\newcommand{\N}{\mathbb{N}}
\newcommand{\Z}{\mathbb{Z}}
\newcommand{\Q}{\mathbb{Q}}
\renewcommand{\H}{\mathbb{H}}
\newcommand{\R}{\mathbb{R}}
\newcommand{\C}{\mathbb{C}}
\newcommand{\Prob}{\mathbb{P}}
\renewcommand{\P}{\mathbb{P}}
\newcommand{\E}{\mathbb{E}}
\newcommand{\F}{\mathbb{F}}
\newcommand{\cU}{\mathcal{U}}
\newcommand{\RP}{\mathbb{RP}}
\newcommand{\CP}{\mathbb{CP}}

\newcommand{\ph}{\,\cdot\,}

\DeclareMathOperator{\sech}{sech}
\DeclareMathOperator{\cosech}{cosech}
\DeclareMathOperator{\cosec}{cosec}

\DeclareMathOperator{\covol}{covol}
\DeclareMathOperator{\vol}{vol}

\let\Im\relax
\let\Re\relax
\DeclareMathOperator{\Im}{Im}
\DeclareMathOperator{\Re}{Re}
\DeclareMathOperator{\im}{im}
\DeclareMathOperator{\image}{image}
\DeclareMathOperator{\Ann}{Ann}

\DeclareMathOperator*{\res}{res}
\DeclareMathOperator{\Res}{Res}
\DeclareMathOperator{\Ind}{Ind}

\DeclareMathOperator{\tr}{tr}
\DeclareMathOperator{\diag}{diag}
\DeclareMathOperator{\rank}{rank}
\DeclareMathOperator{\card}{card}
\DeclareMathOperator{\spn}{span}
\DeclareMathOperator{\adj}{adj}

\DeclareMathOperator{\erf}{erf}
\DeclareMathOperator{\erfc}{erfc}

\DeclareMathOperator{\ord}{ord}
\DeclareMathOperator{\Sym}{Sym}

\DeclareMathOperator{\sgn}{sgn}
\DeclareMathOperator{\orb}{orb}
\DeclareMathOperator{\stab}{stab}
\DeclareMathOperator{\ccl}{ccl}

\DeclareMathOperator{\lcm}{lcm}
\DeclareMathOperator{\hcf}{hcf}

\DeclareMathOperator{\Int}{Int}
\DeclareMathOperator{\id}{id}

\DeclareMathOperator{\betaD}{beta}
\DeclareMathOperator{\gammaD}{gamma}
\DeclareMathOperator{\Poisson}{Poisson}
\DeclareMathOperator{\binomial}{binomial}
\DeclareMathOperator{\multinomial}{multinomial}
\DeclareMathOperator{\Bernoulli}{Bernoulli}
\DeclareMathOperator{\like}{like}

\DeclareMathOperator{\var}{var}
\DeclareMathOperator{\cov}{cov}
\DeclareMathOperator{\bias}{bias}
\DeclareMathOperator{\mse}{mse}
\DeclareMathOperator{\corr}{corr}

\DeclareMathOperator{\otp}{otp}
\DeclareMathOperator{\dom}{dom}

\DeclareMathOperator{\Root}{Root}
\DeclareMathOperator{\supp}{supp}
\DeclareMathOperator{\rel}{rel}
\DeclareMathOperator{\Hom}{Hom}
\DeclareMathOperator{\Aut}{Aut}
\DeclareMathOperator{\Gal}{Gal}
\DeclareMathOperator{\Mat}{Mat}
\DeclareMathOperator{\End}{End}
\DeclareMathOperator{\Char}{char}
\DeclareMathOperator{\ev}{ev}
\DeclareMathOperator{\St}{St}
\DeclareMathOperator{\Lk}{Lk}
\DeclareMathOperator{\disc}{disc}
\DeclareMathOperator{\Isom}{Isom}
\DeclareMathOperator{\length}{length}
\DeclareMathOperator{\energy}{energy}
\DeclareMathOperator{\area}{area}
\DeclareMathOperator{\Syl}{Syl}
\DeclareMathOperator{\cl}{cl}
\DeclareMathOperator{\fix}{fix}

\newcommand{\GL}{\mathrm{GL}}
\newcommand{\SL}{\mathrm{SL}}
\newcommand{\PGL}{\mathrm{PGL}}
\newcommand{\PSL}{\mathrm{PSL}}
\newcommand{\PSU}{\mathrm{PSU}}
\newcommand{\Or}{\mathrm{O}}
\newcommand{\SO}{\mathrm{SO}}
\newcommand{\U}{\mathrm{U}}
\newcommand{\SU}{\mathrm{SU}}

\renewcommand{\d}{\mathrm{d}}
\newcommand{\D}{\mathrm{D}}

\tikzset{->/.style = {decoration={markings,
                                  mark=at position 1 with {\arrow[scale=2]{latex'}}},
                      postaction={decorate}}}
\tikzset{<-/.style = {decoration={markings,
                                  mark=at position 0 with {\arrowreversed[scale=2]{latex'}}},
                      postaction={decorate}}}
\tikzset{<->/.style = {decoration={markings,
                                   mark=at position 0 with {\arrowreversed[scale=2]{latex'}},
                                   mark=at position 1 with {\arrow[scale=2]{latex'}}},
                       postaction={decorate}}}
\tikzset{->-/.style = {decoration={markings,
                                   mark=at position #1 with {\arrow[scale=2]{latex'}}},
                       postaction={decorate}}}
\tikzset{-<-/.style = {decoration={markings,
                                   mark=at position #1 with {\arrowreversed[scale=2]{latex'}}},
                       postaction={decorate}}}

\tikzset{circ/.style = {fill, circle, inner sep = 0, minimum size = 3}}
\tikzset{mstate/.style={circle, draw, blue, text=black, minimum width=0.7cm}}

\definecolor{mblue}{rgb}{0.2, 0.3, 0.8}
\definecolor{morange}{rgb}{1, 0.5, 0}
\definecolor{mgreen}{rgb}{0.1, 0.4, 0.2}
\definecolor{mred}{rgb}{0.5, 0, 0}

\def\drawcirculararc(#1,#2)(#3,#4)(#5,#6){%
    \pgfmathsetmacro\cA{(#1*#1+#2*#2-#3*#3-#4*#4)/2}%
    \pgfmathsetmacro\cB{(#1*#1+#2*#2-#5*#5-#6*#6)/2}%
    \pgfmathsetmacro\cy{(\cB*(#1-#3)-\cA*(#1-#5))/%
                        ((#2-#6)*(#1-#3)-(#2-#4)*(#1-#5))}%
    \pgfmathsetmacro\cx{(\cA-\cy*(#2-#4))/(#1-#3)}%
    \pgfmathsetmacro\cr{sqrt((#1-\cx)*(#1-\cx)+(#2-\cy)*(#2-\cy))}%
    \pgfmathsetmacro\cA{atan2(#2-\cy,#1-\cx)}%
    \pgfmathsetmacro\cB{atan2(#6-\cy,#5-\cx)}%
    \pgfmathparse{\cB<\cA}%
    \ifnum\pgfmathresult=1
        \pgfmathsetmacro\cB{\cB+360}%
    \fi
    \draw (#1,#2) arc (\cA:\cB:\cr);%
}
\newcommand\getCoord[3]{\newdimen{#1}\newdimen{#2}\pgfextractx{#1}{\pgfpointanchor{#3}{center}}\pgfextracty{#2}{\pgfpointanchor{#3}{center}}}

\def\Xint#1{\mathchoice
   {\XXint\displaystyle\textstyle{#1}}%
   {\XXint\textstyle\scriptstyle{#1}}%
   {\XXint\scriptstyle\scriptscriptstyle{#1}}%
   {\XXint\scriptscriptstyle\scriptscriptstyle{#1}}%
   \!\int}
\def\XXint#1#2#3{{\setbox0=\hbox{$#1{#2#3}{\int}$}
     \vcenter{\hbox{$#2#3$}}\kern-.5\wd0}}
\def\ddashint{\Xint=}
\def\dashint{\Xint-}


\begin{document}
\maketitle
{\small
\setlength{\parindent}{0em}
\setlength{\parskip}{1em}
The Standard Model of particle physics is, by far, the most successful application of quantum field theory (QFT). At the time of writing, it accurately describes all experimental measurements involving strong, weak, and electromagnetic interactions. The course aims to demonstrate how this model, a QFT with gauge group $\SU(3) \times \SU(2) \times \U(1)$ and fermion fields for the leptons and quarks, is realised in nature. It is intended to complement the more general Advanced QFT course.

We begin by defining the Standard Model in terms of its local (gauge) and global symmetries and its elementary particle content (spin-half leptons and quarks, and spin-one gauge bosons). The parity $P$, charge-conjugation $C$ and time-reversal $T$ transformation properties of the theory are investigated. These need not be symmetries manifest in nature; eg. only left-handed particles feel the weak force and so it violates parity symmetry. We show how $CP$ violation becomes possible when there are three generations of particles and describe its consequences.

Ideas of spontaneous symmetry breaking are applied to discuss the Higgs Mechanism and why the weakness of the weak force is due to the spontaneous breaking of the $\SU(2) \times \U(1)$ gauge symmetry. Recent measurements of what appear to be Higgs boson decays will be presented.

We show how to obtain cross sections and decay rates from the matrix element squared of a process. These can be computed for various scattering and decay processes in the electroweak sector using perturbation theory because the couplings are small. We touch upon the topic of neutrino masses and oscillations, an important window to physics beyond the Standard Model.

The strong interaction is described by quantum chromodynamics (QCD), the non-abelian gauge theory of the (unbroken) $\SU(3)$ gauge symmetry. At low energies quarks are confined and form bound states called hadrons. The coupling constant decreases as the energy scale increases, to the point where perturbation theory can be used. As an example we consider electron- positron annihilation to final state hadrons at high energies. Time permitting, we will discuss nonperturbative approaches to QCD. For example, the framework of effective field theories can be used to make progress in the limits of very small and very large quark masses.

Both very high-energy experiments and very precise experiments are currently striving to observe effects that cannot be described by the Standard Model alone. If time permits, we comment on how the Standard Model is treated as an effective field theory to accommodate (so far hypothetical) effects beyond the Standard Model.

\subsubsection*{Pre-requisites}
It is necessary to have attended the Quantum Field Theory and the Symmetries, Fields and Particles courses, or to be familiar with the material covered in them. It would be advantageous to attend the Advanced QFT course during the same term as this course, or to study renormalisation and non-abelian gauge fixing.
}
\tableofcontents

\section{Introduction}
\emph{It the whole course, we will ignore the existence of gravity.}

In the standard model, there are a few classes of things:
\begin{itemize}
  \item Forces are mediated by spin 1 \term{gauge bosons}. These include
    \begin{itemize}
      \item The \term{electromagnetic field}, which is mediated by the \term{photon}. This is described by \term{quantum electrodynamics} (\term{QED});
      \item The \term{weak interaction}\index{weak nuclear force}, which is mediated by the $W^{\pm}$\index{$W^{\pm}$ boson} and $Z$\index{$Z$ boson} \emph{bosons}; and
      \item The \term{strong interaction}\index{strong force}, which is mediated by \term{gluons} $g$. This is described by the theory of \term{quantum chromodynamics} (\term{QCD}).
    \end{itemize}
    While the electromagnetic field and weak interaction seem very different, we will see that at high energies, they merge together, and can be described by a single gauge group.
  \item Matter is described by spin $\frac{1}{2}$ \term{fermions}. These include
    \begin{itemize}
      \item \emph{Neutrinos}\index{neutrinos}: $\nu_e, \nu_\mu, \nu_\tau$. These interact only via the weak interaction.
      \item \emph{Charged leptons}\index{leptons}\index{charged leptons}: $e, \mu, \tau$. These interact with the electromagnetic field and weak interactions.
      \item \emph{Quarks}\index{quarks}: $\begin{pmatrix}u\\d\end{pmatrix}$, $\begin{pmatrix}c\\s\end{pmatrix}$, $\begin{pmatrix}t\\b\end{pmatrix}$. These have electric charges $\begin{pmatrix}+\frac{2}{3}\\ -\frac{1}{3}\end{pmatrix}$. They interact with all interactions.
    \end{itemize}
    We see that each type of matter comes in three \term{generations}. We do not know why.
  \item There is the \term{Higgs boson}, which has spin 0. This is responsible for giving mass to the $W^{\pm}, Z$ bosons and fermions. This was just discovered in 2012 in CERN, and subsequently more properties have been discovered, eg. its spin.
\end{itemize}

As one would expect from the name, the gauge bosons are manifestations of local gauge symmetries. The gauge group in the Standard Model is
\[
  \SU(3)_C \times \SU(2)_L \times \U(1)_Y.
\]
The subscripts indicate which things the group are responsible for. The $\SU(3)_C$ describes the strong force, and gives us QCD. The $C$ stands for ``colour''. This is complicated.

The remaining bit is what we are going to focus on for most of the course. The $\SU(2)_L$ interaction is chiral, and only couples to left-handed particles. The $Y$ at $\U(1)_Y$ refers to what is known as the hypercharge. The $\SU(2)_L \times \U(1)_Y$ gives us a \emph{unified} description of QED and weak interaction, collectively known as the electroweak force. We will see that there is spontaneous symmetry breaking of this electroweak part, which will give us weak and electromagnetic interactions.

Note that a lot of the standard model was discovered experimentally, but this course focuses on the theoretical parts of standard model. Thus, we will mostly pull these theories out of a hat without giving much motivation for what that is the case.

\section{Chiral and gauge symmetries}
We'll move reasonably quickly through this, as we are just reviewing some concepts from the QFT course. Throughout, we will use ``natural units'' where $c = \hbar = 1$.

\subsection{Chiral symmetry}
Consider a spin-$\frac{1}{2}$ \term{Dirac fermion} $\psi$, which satisfies the Dirac equation
\[
  (i \slashed{\partial} - m) \psi = 0,
\]
where as usual
\[
  \slashed{\partial} = \gamma^\mu \partial_\mu,
\]
and the \term{Dirac matrices}\index{$\gamma^\mu$} $\gamma^\mu$ satisfy the \term{Clifford algebra} relations
\[
  \{\gamma^\mu, \gamma^\nu\} = 2 g^{\mu\nu} I,
\]
where $g^{\mu\nu} = \diag(+1, -1, -1, -1)$ is the Minkowski metric. We usually drop the ``$I$'' from the equation.

We define\index{$\gamma^5$}
\[
  \gamma^\mu = +i \gamma^0 \gamma^1 \gamma^2 \gamma^3,
\]
which satisfies
\[
  (\gamma^5)^2 = I,\quad \{\gamma^5, \gamma^\mu\} = 0.
\]
One can do a lot of stuff without choosing a particular basis/representation for the $\gamma$-matrices, and the physics we get out must be the same regardless of which representation we choose, but sometimes it is convenient to pick some particular representation to work with. We'll generally use the \term{chiral representation} (or \term{Weyl representation}), where
\[
  \gamma^0 =
  \begin{pmatrix}
    0 & 1 \\
    1 & 0
  \end{pmatrix}, \quad
  \gamma^i =
  \begin{pmatrix}
    0 & \sigma^i\\
    -\sigma^i & 0
  \end{pmatrix},\quad
  \gamma^5 =
  \begin{pmatrix}
    -1 & 0\\
    0 & 1
  \end{pmatrix},
\]
where the $\sigma^i$ are the \term{Pauli matrices}.

Consider the massless limit of the Dirac equation. Then we just have
\[
  \slashed{\partial} \psi = 0.
\]
Using the anti-commutator relation of $\gamma^\mu$ with $\gamma^5$, we find that we also have
\[
  \slashed{\partial} \gamma^5 \psi = 0.
\]
We now define the projection operators
\[
  P_{R, L} = \frac{1}{2} (1 \pm \gamma^5),
\]
which satisfy
\[
  (P_{R, L})^2 = P_{R, L},\quad P_L + P_R = I,\quad P_L P_R = P_R P_L = 0.
\]
We define
\[
  \psi_{R, L} = P_{R, L} \psi.
\]
Then we have
\[
  \gamma^5 \psi_{R, L} = \pm \psi_{R, L}.
\]
We say $\psi_{R, L}$ have \emph{definite chirality}\index{chirality}, and we say they are ``\term{right-handed}'' and ``\term{left-handed}''.

In the chiral representation, we have
\[
  P_L =
  \begin{pmatrix}
    1 & 0\\
    0 & 0
  \end{pmatrix},
  P_R =
  \begin{pmatrix}
    0 & 0\\
    0 & 1
  \end{pmatrix}.
\]
So $\psi_{R, L}$ only contains lower (upper resp.) two components. In the case of quantum field theory, $\psi_R$ annihilates right-handed chiral particles, and $\psi_L$ annihilates left-handed chiral particles.

\separator
We can try to write down the Dirac Lagrangian (density) for a general fermion:
\[
  L = \bar\psi (i \slashed{\partial} - m) \psi = \bar\psi_L i \slashed{\partial} \psi_L + \bar\psi_L i \slashed{\partial} \psi_R - m (\bar\psi_L \psi_R + \bar\psi_R \psi_L).
\]
If the fermion is massless, then we have a $\U(1)_L \times \U(1)_R$ global symmetry --- under an element $(\alpha_L, \alpha_R) \in \U(1)_L \times \U(1)_R$, the fermion transforms as
\[
  \begin{pmatrix}
    \psi_L\\
    \psi_R
  \end{pmatrix} \mapsto
  \begin{pmatrix}
    e^{i\alpha_L} \psi_L\\
    e^{i\alpha_R} \psi_R
  \end{pmatrix}.
\]
The adjoint field transforms as
\[
  \begin{pmatrix}
    \bar{\psi}_L\\
    \bar{\psi}_R
  \end{pmatrix} \mapsto
  \begin{pmatrix}
    e^{-i\alpha_L} \bar{\psi}_L\\
    e^{-i\alpha_R} \bar{\psi}_R
  \end{pmatrix}.
\]
In the case of a massive particle, we only have a single $\U(1)_V$ symmetry, where we have to transform $\psi_L$ and $\psi_R$ in the same way, ie. pick $\alpha_L = \alpha_R$.

\subsubsection*{Review of Dirac field}
Recall that in the case of a quantum field, then
\[
  \psi = \sum_{s, p}\left[b^s (p) u^s(p) e^{-ip\cdot x} + d^{s\dagger}(p) v^s(p) e^{-ip\cdot x}\right],
\]
where $s = \pm \frac{1}{2}$, and
\[
  \sum_p = \int \frac{\d^3 p}{(2\pi)^3 (2E_\mathbf{p})}.
\]
Here $b^\dagger$ and $b^\dagger$ create positive and negative frequency particles. We use relativistic normalization, where the states
\[
  \bket{p} = b^\dagger(p) \bket{0}
\]
satisfy
\[
  \braket{p}{q} = (2\pi)^3 2E_p \delta^{(3)} (\mathbf{p} - \mathbf{q}).
\]
In the chiral representation, we have
\[
  u^s(p) =
  \begin{pmatrix}
    \sqrt{p \cdot \sigma} \xi^s\\
    \sqrt{p \cdot \bar\sigma} \xi^s
  \end{pmatrix},\quad
  v^s(p) =
  \begin{pmatrix}
    \sqrt{p \cdot \sigma} \eta^s\\
    -\sqrt{p \cdot \bar\sigma} \eta^s
  \end{pmatrix},
\]
where as usual
\[
  \sigma^\mu = (I, \sigma^i),\quad \bar{\sigma}^\mu = (I, - \sigma^i).
\]
We define the \term{helicity} to be the projection of the angular momentum onto the direction of the linear momentum:
\[
  h = \mathbf{J} \cdot \hat{\mathbf{p}} = \mathbf{S} \cdot \hat{\mathbf{p}},
\]
where
\[
  \mathbf{J} = -i \mathbf{r} \times \nabla + \mathbf{S}
\]
is the total angular momentum, and $\mathbf{S}$ is the spin operator given by
\[
  S_i = \frac{i}{4} \varepsilon_{ijk} \gamma^j \gamma^k = \frac{1}{2}
  \begin{pmatrix}
    \sigma^i & 0\\
    0 & \sigma^i
  \end{pmatrix}.
\]
The massless spinor satisfies
\[
  \slashed{p}u = 0.
\]
From this, we can show that
\[
  hu^{s}(p) = \frac{\gamma^5}{2} u^s(p).
\]
To show this, we start with $\slashed p u = 0$, multiply by $\gamma^5 \gamma^0/p^0$, and then use the fact that $\gamma^5 \gamma^0 \gamma^i = 2 S^i$.

The whole point about this is that for a massless spinor, helicity is the same as chirality. In particular, we have
\[
  h u_{L, R} = \frac{\gamma^5}{2} u_{L, R} = \mp \frac{1}{2} u_{L, R},
\]
where again $u_{L, R} = P_{L, R} u$, and $P_{L, R} = \frac{1}{2} (1 \mp \gamma^5)$.

So $u_{L, R}$ has helicity $\mp\frac{1}{2}$. Note that chiral states are only eigenstates of the Dirac equation when $m = 0$. Also, while helicity can be defined for any $m$, it is not Lorentz invariant when $m \not= 0$.

\subsection{Gauge symmetry}
If we promote the $\alpha$ to $\alpha(x)$, a function of $x$, the kinetic term in the Dirac Lagrangian is no longer invariant. In particular, it transforms as
\[
  \bar\psi i \slashed{\psi} \mapsto \bar\psi i \slashed \psi - \bar\psi \gamma^\mu \psi \partial_\mu \alpha (x).
\]
To fix this problem, we introduce a gauge covariant derivative $\D_\mu$ such that we have the transformation law
\[
  \D_\mu \psi(x) \mapsto \exp(i\alpha (x)) \D_\mu \psi(x).
\]
To do this, we introduce a gauge field $A_\mu(x)$, and then define
\[
  \D_\mu \psi(x) = (\partial_\mu + i g A_\mu) \psi(x).
\]
We then claim that $A_\mu$ transforms as
\[
  A_\mu \mapsto A_\mu - \frac{1}{g} \partial_\mu \alpha(x).
\]
Using these, it is straightforward to check that
\[
  \bar\psi i \slashed D \psi \mapsto \bar\psi i \slashed D \psi.
\]
We can introduce a kinetic term for $A_\mu$:
\[
  \mathcal{L}_G = -\frac{1}{4} F_{\mu\nu} F^{\mu\nu},
\]
where
\[
  F_{\mu\nu} = \partial_\mu A_\nu - \partial_\nu A_\mu = \frac{1}{ig} [D_\mu, D_\nu].
\]
So far, this assumes we work with the whole field $\psi$ itself. In reality, we find the weak field couples only with left-handed fields, and we have to modify these accordingly.

In real life, we have more complicated gauge group, and we have to deal with \emph{non-abelian gauge symmetries}, where we deal with non-abelian gauge groups, eg. $\SU(2)$. However, we will not deal with these more complicated cases just yet.

\subsection{Types of symmetry}
Symmetries can manifest themselves in a number of ways.
\begin{enumerate}
  \item We can have an \term{intact symmetry}, or \term{exact symmetry}. In other words, this is an actual symmetry. For example, $\U(1)_{EM}$ and $\SU(3)_C$ are exact symmetries in the standard model.
  \item Symmetries can be broken by an \term{anomaly}. This is a symmetry that exists in the classical theory, but goes away when we quantize. For example, global axial symmetry for massless fields in the standard model.
  \item Symmetry is explicitly broken by some terms in the Lagrangian. This is not a symmetry, but if those annoying terms are small (intentionally left vague), then we have an \term{approximate symmetry}, and it may also be useful to consider these.

    For example, in the standard model, the up and down quarks are very close in mass, but not exactly the same. This gives to the (global) isospin symmetry.
  \item The symmetry is respected by the Lagrangian $\mathcal{L}$, but not by the vacuum. This is a ``hidden symmetry''.
    \begin{enumerate}
      \item We can have a \term{spontaneously broken symmetry}: we have a vacuum expectation value for one or more scalar fields for one or more scalar fields, eg. the breaking of $\SU(2)_L \times \U(1)_Y$ into $\U(1)_{EM}$.
      \item Even without scalar fields, we can get \term{dynamical symmetry breaking} from quantum effects. An example of this in the standard model is the $\SU(2)_L \times \SU(2)_R$ global symmetry in the strong interaction.
    \end{enumerate}
\end{enumerate}
One can argue that (i) is the only case where we actually have a symmetry, but the others are useful to consider as well.

\section{Discrete symmetries}
We have three discrete symmetries:
\begin{itemize}
  \item Parity (\term{P}): $(t, \mathbf{x}) \mapsto (t, -\mathbf{x})$
  \item Time-reversal (\term{T}): $(t, \mathbf{x}) \mapsto (-t, \mathbf{x})$
  \item Charge conjugation (\term{C}): This sends particles to anti-particles and vice versa.
\end{itemize}
Gauge theories with vector-like couplings (as opposed to chiral couplings) to Fermions are invariant under these symmetries. However, theories that only involve left-handed particles are not. For example, the weak interaction is not symmetric. Even worse, the weak interaction is not invariant even under the combination CP. From the CPT theorem, which we will discuss later, we know our theory is invariant under the combination CPT. This implies that the weak interaction must break T symmetry as well. This CP violation has important consequences, and it's one of the Sakhator conditions required for a matter-antimatter asymmetry.

We have just been saying words, but we haven't defined these concepts formally. To understand these statements, we first consider theories which \emph{are} C, P, T invariant.

\subsection{Symmetry operators}
Consider a general Poincar\'e transformation that can be written
\[
  x^\mu \mapsto x'^\mu = \Lambda^\mu\!_\nu x^\nu + a^\mu.
\]
A \term{proper Lorentz transform} has $\det \Lambda = +1$.
\begin{defi}[Parity transform]\index{parity transform}
  The \emph{parity transform} is
  \[
    \Lambda^\mu\!_\nu = \Prob^\mu\!_\nu =
    \begin{pmatrix}
      1 & 0 & 0 & 0\\
      0 & -1 & 0 & 0\\
      0 & 0 & -1 & 0\\
      0 & 0 & 0 & -1
    \end{pmatrix}.
  \]
\end{defi}

\begin{defi}[Time reversal transform]\index{time reversal transform}
  The \emph{time reversal transform} is given by
  \[
    \mathbb{T}^\mu\!_\nu =
    \begin{pmatrix}
      -1 & 0 & 0 & 0\\
      0 & 1 & 0 & 0\\
      0 & 0 & 1 & 0\\
      0 & 0 & 0 & 1
    \end{pmatrix}
  \]
\end{defi}
These are \emph{improper} Lorentz transforms.

So far we have been talking about symmetries of our universe $\R^{1, 3}$. What happens when we pass on to the quantum state space?

We have the following theorem of Wigner:
\begin{thm}[Wigner]
  If physics is invariant under $\Psi \mapsto \Psi'$, where $\Psi, \Psi'$ are some vectors in a Hilbert space, then there's an operator $W$ such that
  \[
    \Psi' = W \Psi,
  \]
  where $W$ is either unitary and linear, ie.
  \[
    \bra W \Phi, W \Psi\ket = \bar \Phi, \Psi,\quad W(\alpha \Phi + \beta \Psi) = \alpha W(\Phi) + \beta W(\Psi).
  \]
  or $W$ is a anti-unitary and anti-linear, ie.
  \[
    \bra W \Phi, W \Psi\ket = \bra \Phi, \Psi\ket^*,\quad W(\alpha \Phi + \beta \Psi) = \alpha^* W(\Phi) + \beta^* W(\Psi).
  \]
\end{thm}

In general, given a transformation $x^\mu \mapsto \Lambda^\mu\!_\nu x^\nu + a^\mu$ as before, we write the corresponding transformation on the quantum state space as $W(\Lambda, a)$.

Consider an infinitesimal transformation
\[
  \Lambda^\mu\!_\nu = \delta^\mu\!_\nu + \omega^\mu\!_\nu,\quad a^\mu = \varepsilon^\mu,
\]
where $\omega$ and $\varepsilon$ are small parameters. Then the corresponding operator $W$ can be expanded as
\[
  W(\Lambda, a) = W(I + \omega, \varepsilon) = 1 + \frac{i}{2} \omega_{\mu\nu} J^{\mu\nu} + i \varepsilon_\mu P^\mu,
\]
where $J$ are the generators of rotations and boosts, and $P$ generates translations (ie. $P^0$ is the Hamiltonian and the $P^i$ are the linear momenta).

Of course, we cannot write parity and time reversal in this form, because they are discrete symmetries, but we can look at what happens when we combine these transformations with infinitesimal ones.

Now, we write
\[
  \hat{P} = W(\Prob, 0),\quad \hat{T} = W(\mathbb{T}, 0),
\]
then from the general composition rule, we have
\begin{align*}
  \hat{P} W \hat{P}^{-1} &= W(\Prob \Lambda \Prob^{-1}, \Prob a)\\
  \hat{T} W \hat{T}^{-1} &= W(\mathbb{T} \Lambda \mathbb{T}^{-1}, \mathbb{T} a)
\end{align*}
Inserting expansions for $W$ in terms of $\omega$ and $\varepsilon$ on both sides, and comparing coefficients of $- \varepsilon_0$, we find
\begin{align*}
  \hat{P} i H \hat{P}^{-1} &= iH\\
  \hat{T} iH \hat{T}^{-1} &= -iH.
\end{align*}
So $iH$ and $\hat{P}$ commute, but $iH$ and $\hat{T}$ \emph{anti-commute}.

Now consider a normalized energy eigenstate with energy $E$, we have
\[
  \bra \Psi, i H \Psi\ket = iE.
\]
Assuming that $\hat{P}$ and $\hat{T}$ are symmetries of the theory, then $\hat{P} \Psi$ and $\hat{T} \Psi$ should be eigenstates with energy $E$. We have no problems with $\hat{P}$: we need
\[
  \bra \hat{P} \Psi, iH \hat{P} \Psi\ket = \bra \hat{P} \Psi, \hat{P} iH \Psi \ket.
\]
We want this to be equal to $\bra \Psi, iH \Psi\ket = iE$. So we see what we need is that $\hat{P}$ has to be unitary (and hence linear as well).

But things go wrong if we look at $\hat{T}$. We have
\[
  \bra \hat{T} \Psi, iH \hat{T} \Psi\ket = \bra \hat{T} \Psi, -\hat{T} iH \Psi \ket.
\]
We now want this to be $iE = \bra \Psi, iH \Psi\ket$. So what we need is that
\[
  \bra \hat{T} \Psi, \hat{T} iH \Psi \ket^* = \bra \Psi, iH \Psi\ket = iE,
\]
since the complex conjugate of $iE$ is $-iE$. So we know $\hat{T}$ must instead be anti-unitary, and hence anti-linear.

We now consider each of the individual symmetries in turn.
\subsection{Parity}
Consider the parity transformation
\begin{align*}
  x^\mu &\mapsto x^\mu_P = (x^0, -\mathbf{x})\\
  p^\mu &\mapsto p^\mu_P = (p^0, -\mathbf{p}).
\end{align*}
We start by looking at scalar fields. Consider a complex scalar field
\[
  \phi(x) = \sum_p \left(a(p) e^{-ip\cdot x} + c(p)^\dagger e^{+i p\cdot x}\right),
\]
where $a(p)$ is an annihilation operator for a particle and $c(p)^\dagger$ is a creation operator for the anti-particles.

We want to see what happens when $P$ acts on this. We would expect $\hat{P}$ maps a particle
\[
  \bket{p} \mapsto \eta^*_a\bket{p_P},
\]
where $\eta^*_a$ is a complex phase.

We can alternatively write this as
\[
  \bket{p}a^\dagger(p) \bket{0} = \eta_a^* a^\dagger(p_P)\bket{0}.
\]
We assume the vacuum is parity-invariant, ie. $\hat{P}\bket{0} = \bket{0}$. So we can write this as
\[
  \hat{p}a^\dagger \bket{p}^{-1}\bket{0} = \eta_a^* a^\dagger(p_P) \bket{0}.
\]
We now postulate that our operator $a^\dagger$ transforms as
\[
  \hat{p}a^\dagger \bket{p}^{-1} = \eta_a^* a^\dagger(p_P).
\]
To conserve normalizations, we know we also have
\[
  \hat{P} a(p) \hat{P}^{-1} = \eta_a a(p_P).
\]
Similarly, the $c$ operators transform as
\[
  \hat{P} c^\dagger(p) \hat{P}^{-1} = \eta_c^* c^\dagger(p_P).
\]
Thus in general, we have
\begin{align*}
  \hat{P} \phi(x) \hat{P}^{-1} &= \sum_p \left(\hat{P}a(p)\hat{P}^{-1} e^{-ip\cdot x} + \hat{P}c^\dagger (p)\hat{P}^{-1} e^{+ip\cdot x}\right)\\
  &= \sum_p \left(\eta_a^* a(p_P) e^{-ip\cdot x} + \eta_c^* c^\dagger (p_P) e^{+ip\cdot x}\right)\\
  \intertext{We re-label $p_P \leftrightarrow p$, and then get}
  &= \sum_p \left(\eta_a a(p) e^{-i p_P \cdot x} + \eta_c^* c^\dagger(p) e^{+ip_P \cdot x}\right)\\
  \intertext{We now note that $x \cdot p_P = x_P \cdot p$ by inspection. So we have}
  &= \sum_p\left(\eta_a a(p) e^{-ip\cdot x_P} + \eta_c^* c^\dagger(p) e^{ip\cdot x_P}\right)
\end{align*}
This looks almost like $\phi(x_P)$, but we have the factors of $\eta_a$ and $\eta_c^*$. If we had $\eta_a = \eta_c^* \equiv \eta_p$, then we just get
\[
  \hat{P} \phi(x) \hat{P}^{-1} = \eta_P \phi(x_P).
\]
We can argue that this must be the case. We can either think that $\hat{P} \phi(x)\hat{P}^{-1}$ must ``look like'' $\phi(x)$, or we can argue that otherwise, $[\phi(x), \hat{P} \phi^\dagger(y) \hat{P}^{-1}]$ would not in general vanish for spacelike $x - y_P$.

We call $\eta_P$ the \term{intrinsic parity}.

For real scalar fields, we have $a = c$, and so $\eta_a = \eta_c$, and so $\eta_a = \eta_p = \eta_p^*$. So $\eta_p = \pm 1$. We call it a \term{scalar field} if we have $\eta_p = +1$, and a \term{pseudoscalar field} otherwise.

For a complex scalar field, this $\eta_P$ may be complex, but if there is an associated conserved charge $Q$, then $\eta_P$ can be related to $Q$.

\separator

If we have a vector field $V^\mu (x)$, then we have
\[
  V^\mu(x) = \sum_{p, \lambda} \mathcal{E}^\mu a^\lambda(p) e^{-ip\cdot x} + \mathcal{E}^{\mu *} (\lambda, p) c^{\dagger \lambda} (p) e^{ip\cdot x}.
\]
where $\mathcal{E}^\mu(\lambda, p)$ are the polarization vectors and $\lambda$ ranges over $\lambda = -1, 0, 1$.

Using similar computations, we find
\[
  \hat{P} V^\mu \hat{P}^{-1} = \sum_{p, \lambda} \left(\mathcal{E}^\mu (\lambda, p_P)a^\lambda(p) e^{-ip\cdot x_P} \eta_a + \mathcal{E}^{\mu*}(\lambda, p_P) c^{\dagger\lambda} (p) e^{+ip\cdot x_P} \eta_c^*\right).
\]
Note that $\lambda$ doesn't change under parity transforms, as it is about angular momentum, which is things that look like $\mathbf{x} \times \mathbf{p}$.

This time we have the annoying problem of the polarization vectors $\mathcal{E}^\mu(\lambda, p_P)$, and we need to deal with the $p_P$. We use the result that
\[
  \mathcal{E}^\mu(\lambda, p_P) = - \Prob^\mu\!_\nu \mathcal{E}^\nu (\lambda, P),
\]
which we can show using an explicit form for $\mathcal{E}^\mu$ and Lorentz transformations. We then see that
\[
  \hat{P} V^\mu \hat{P}^{-1} = - P^\mu\!_\nu \eta_P V^\nu(x_P),
\]
where for the same reasons as before, we have
\[
  \eta_P = \eta_a = \eta_c^*.
\]
Then \term{vector fields} have $\eta_P = -1$, and \term{axial vector fields} have $\eta_P = +1$.

\separator

For a Dirac field, the creation and annihilation operators behave like scalar fields, and the spin component $s$ is unchanged, as for vectors. We have
\[
  \hat{P} b^s(p) \hat{P}^{-1} = \eta_P b^s(p_P),\quad \hat{P} d^{s\dagger}(p) \hat{P}^{-1} = \eta_d^* d^{s\dagger}(p_P).
\]
Then we have
\[
  \hat{P} \psi(x) \hat{P}^{-1} = \sum_{p, s} \left(\eta_b b^s(p_P) u^s(p) e^{-ip\cdot s} \eta_d^* d^{s\dagger} (p_P) v^s(p) e^{+ip\cdot x}\right).
\]
Using the same steps as before, we get
\[
  \hat{P}\psi(x) \hat{P}^{-1} = \sum_{p, s}\left( \eta_b b^s(p) u(p_P) e^{-p\cdot x_P} + \eta_d^* d^{s\dagger}(p) v^s(p_P) e^{+ip\cdot x_P}\right).
\]
We use that
\[
  u^s(p_P) = \gamma^0 u^s(p),\quad v^s(p_P) = - \gamma^0 v^s(p),
\]
which we can verify using Lorentz boosts. Then we find
\[
  \hat{P} \psi(x) \hat{P}^{-1} = \gamma^0 \sum_{p, s}\left( \eta_b b^s(p) u(p) e^{-p\cdot x_P} - \eta_d^* d^{s\dagger}(p) v^s(p) e^{+ip\cdot x_P}\right).
\]
So again, we require that
\[
  \eta_b = - \eta_d^*,
\]
and we end up with
\[
  \hat{P} \psi(x) \hat{P}^{-1} = \eta_p \gamma^0 \psi(x_p).
\]
Similarly, we have
\[
  \hat{P} \bar\psi(x) \hat{P}^{-1} = \eta_P^* \bar\psi(x_P) \gamma^0.
\]
We can think of the minus sign as saying the particles and anti-particles have opposite intrinsic parity.

Note that this gives
\[
  \hat{P} \psi_L \hat{P}^{-1} = \gamma^0 \psi_R \eta_P.
\]
We can also check that if $\psi(x)$ satisfies the Dirac equation, then so does $\hat{P} \psi(x) \hat{P}^{-1}$.

We can now determine how various Fermions bilinears transform. For example, we have
\[
  \bar\psi(x) \psi(x) \mapsto \bar\psi(x_P) \psi(x_P).
\]
So it transforms as a scalar. On the other hand, we have
\[
  \bar\psi(x) \gamma^5 \psi(x) \mapsto - \bar{\psi}(x_P) \gamma^5 \psi(x_P),
\]
and so this transforms as a \emph{pseudoscalar}. We also have
\[
  \bar\psi \gamma^\mu \psi(x) \mapsto \Prob^\mu\!_0 \bar\psi(x_P) \gamma^\mu \psi(x),
\]
and so this transforms as a \emph{scalar}. Finally, we have
\[
  \bar\psi(x) \gamma^5 \gamma^\mu \psi(x_P) \mapsto - \Prob^\mu\!_0 \bar\psi(x_P) \gamma^\mu \psi(x_P).
\]
\subsection{Charge conjugation}
Unlike parity, this is not a spacetime symmetry. It transforms particles to anti-particles, and vice versa. This is again a unitary operator $\hat{C}$.

\subsubsection*{Scalar fields}
We can go through the derivations we did for parity, and find that for scalar fields, we have
\begin{align*}
  \hat{C} a(p)\hat{C}^{-1} &= \eta_c c(p)\\
  \hat{C} c(p) \hat{C}^{-1} &= \eta_c^* a(p)
\end{align*}
Again, $\hat{C}$ acts trivially on the vacuum, so if we have a single particle $\bket{p}$ of momentum $p$, then we have
\[
  \hat{C}\bket{p} = \hat{C} a^\dagger(p) \bket{0} = \eta_c^* c^\dagger(p) \hat{C} \bket{0} = \eta_c^* \bket{\bar{p}},
\]
where $\bket{\bar{p}}$ is the anti-particle of momentum $p$.

From the decomposition, we find that
\begin{align*}
  \hat{C} \phi(x) \hat{C}^{-1} &= \eta_c \phi^\dagger (x)\\
  \hat{C} \phi^\dagger (x) \hat{C}^{-1} &= \eta_c^* \phi (x).
\end{align*}
If $\phi$ is a real field, then $\phi^\dagger = \phi$. Then $\eta_c = \pm 1$. This is the \emph{intrinsic $c$-parity} of the field.
\begin{eg}
  We'll see later that the photon field transforms like
  \[
    \hat{C}A_\mu(x) \hat{C}^{-1} = - A_\mu(x).
  \]
  Experimentally, we see that $\pi^0$ only decays to $2$ photons, but not $1$ or $3$. Therefore, assuming that $c$-parity is conserved, we infer that
  \[
    \eta_c^{\pi_0} = (-1)^2 = +1.
  \]
\end{eg}

For a complex scalar field, $\eta_c$ is arbitrary. However, we can do a global $\U(1)$ rotation
\[
  \phi \mapsto \phi' = e^{i \beta} \phi
\]
such that $\eta_c' = 1$. So we can always redefine our field $\phi$ so that $\eta_c = 1$. So it is in fact less interesting.

\subsubsection*{Dirac fields}
With Dirac fields, first we define the $4 \times 4$ matrix $C$ (in spinor space) such that
\[
  (C \gamma^\mu)^T = C \gamma^\mu.
\]
There is no such unique choice of $C$, and we can just pick one. In the chiral representation, where $\gamma^0$ and $\gamma^2$ are symmetric while $\gamma^1$ and $\gamma^3$ are anti-symmetric, a suitable choice of $C$ is
\[
  C = -i \gamma^0 \gamma^2 =
  \begin{pmatrix}
    i \sigma^2 & 0\\
    0 & i \sigma^2
  \end{pmatrix}.
\]
One can check that this indeed satisfies the required properties, and that
\[
  C = -C^T = - C^\dagger =  -C^{-1}.
\]
Also, we have
\[
  (\gamma^\mu)^T = - C \gamma^\mu C^{-1},\quad (\gamma^5)^T = + C \gamma^5 C^{-1}.
\]
Similarly to bosons,
\begin{align*}
  \hat{C} b^s(p) \hat{C}^{-1} &= \eta_c d^s(p)\\
  \hat{C} d^{s\dagger}(p) \hat{C}^{-1} &= \eta_c b^{s\dagger} (p).
\end{align*}
Note that the spin is unchanged under charge conjugation again. We now compare
\[
  \hat{C} \psi(x) \hat{C}^{-1} = \eta_c \sum_{p, s}\left(d^s(p) u^s(p)e^{-ip\cdot x} + b^{s\dagger} (p) v^s(p) e^{+ip\cdot x}\right).
\]
with
\[
  \bar\psi^T(x) = \sum_{p, s} \left( b^{s\dagger}(p) \bar{u}^{sT}(p) e^{+ip\cdot x} + d^s (p) \bar{v}^{sT}(p) e^{-ip\cdot x}\right).
\]
Considering these spinors, if we take 
\[
  \eta^s = i \sigma^2 \xi^{s*},
\]
then we can write
\[
  v^s(p) = C \bar{u}^{sT},\quad u^s(p) = c \bar{v}^{sT}(p).
\]
So we find that
\[
  \psi^c(x) \equiv \hat{C} \psi(x) \hat{C}^{-1} = \eta_c C \bar\psi^T (x).
\]
In a similar way, we have
\[
  \bar\psi^c(x) \equiv \hat{C} \bar\psi(x) \hat{C}^{-1} = \eta_c^* \psi^T(x) C = - \eta_c^* \psi^T(x)  C^{-1}.
\]
Note that if $\psi(x)$ satisfies the Dirac equation, then so does $\psi^c(x)$.

Apart from Dirac fermions, there are also things called \term{Majorana fermions}. They have $b^s(p) = d^s(p)$. This means that the particle is its own antiparticle, and for these, we have
\[
  \psi^c(x) = \psi(x).
\]
These fermions have to be neutral, and it is not known whether the only neutral fermions in the standard model (the neutrinos) are Dirac fermions or Majorana fermions. Experimentally, if they are indeed Majorana fermions, then we would be able to observe neutrino-less double $\beta$ decay, but current experiments can neither observe nor rule out this possibility.

\subsubsection*{Fermion bilinears}
We can look at how Fermion bilinears change. Let
\[
  j^\mu = \bar\psi(x) = \gamma^\mu \psi(x).
\]
Then we have
\begin{align*}
  \hat{C} j^\mu(x) \hat{C}^{-1} &= \hat{C} \bar\psi \hat{C}^{-1} \gamma^\mu \hat{C} \psi \hat{C}^{-1}\\
  &= -\eta_c^* \psi^T C^{-1} \gamma^\mu C \bar\psi \eta_c
\end{align*}
\printindex
\end{document}
