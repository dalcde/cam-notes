\documentclass[a4paper]{article}

\def\npart {III}
\def\nterm {Lent}
\def\nyear {2018}
\def\nlecturer {J.\ Miller}
\def\ncourse {Schramm--Loewner Evolutions}
\def\nofficial {https://statslab.cam.ac.uk/~jpm205/teaching/lent2018/}

% Imports
\ifx \nextra \undefined
  \usepackage[pdftex,
    hidelinks,
    pdfauthor={Dexter Chua},
    pdfsubject={Cambridge Maths Notes: Part \npart\ - \ncourse},
    pdftitle={Part \npart\ - \ncourse},
  pdfkeywords={Cambridge Mathematics Maths Math \npart\ \nterm\ \nyear\ \ncourse}]{hyperref}
  \title{Part \npart\ - \ncourse}
\else
  \usepackage[pdftex,
    hidelinks,
    pdfauthor={Dexter Chua},
    pdfsubject={Cambridge Maths Notes: Part \npart\ - \ncourse\ (\nextra)},
    pdftitle={Part \npart\ - \ncourse\ (\nextra)},
  pdfkeywords={Cambridge Mathematics Maths Math \npart\ \nterm\ \nyear\ \ncourse\ \nextra}]{hyperref}

  \title{Part \npart\ - \ncourse \\ {\Large \nextra}}
\fi

\author{Lectured by \nlecturer \\\small Notes taken by Dexter Chua}
\date{\nterm\ \nyear}

\usepackage{alltt}
\usepackage{amsfonts}
\usepackage{amsmath}
\usepackage{amssymb}
\usepackage{amsthm}
\usepackage{booktabs}
\usepackage{caption}
\usepackage{enumitem}
\usepackage{fancyhdr}
\usepackage{graphicx}
\usepackage{mathtools}
\usepackage{microtype}
\usepackage{multirow}
\usepackage{pdflscape}
\usepackage{pgfplots}
\usepackage{siunitx}
\usepackage{tabularx}
\usepackage{tikz}
\usepackage{tkz-euclide}
\usepackage[normalem]{ulem}
\usepackage[all]{xy}

\pgfplotsset{compat=1.12}

\pagestyle{fancyplain}
\lhead{\emph{\nouppercase{\leftmark}}}
\ifx \nextra \undefined
  \rhead{
    \ifnum\thepage=1
    \else
      \npart\ \ncourse
    \fi}
\else
  \rhead{
    \ifnum\thepage=1
    \else
      \npart\ \ncourse\ (\nextra)
    \fi}
\fi
\usetikzlibrary{arrows}
\usetikzlibrary{decorations.markings}
\usetikzlibrary{decorations.pathmorphing}
\usetikzlibrary{positioning}
\usetikzlibrary{fadings}
\usetikzlibrary{intersections}
\usetikzlibrary{cd}

\newcommand*{\Cdot}{\raisebox{-0.25ex}{\scalebox{1.5}{$\cdot$}}}
\newcommand {\pd}[2][ ]{
  \ifx #1 { }
    \frac{\partial}{\partial #2}
  \else
    \frac{\partial^{#1}}{\partial #2^{#1}}
  \fi
}

% Theorems
\theoremstyle{definition}
\newtheorem*{aim}{Aim}
\newtheorem*{axiom}{Axiom}
\newtheorem*{claim}{Claim}
\newtheorem*{cor}{Corollary}
\newtheorem*{defi}{Definition}
\newtheorem*{eg}{Example}
\newtheorem*{fact}{Fact}
\newtheorem*{law}{Law}
\newtheorem*{lemma}{Lemma}
\newtheorem*{notation}{Notation}
\newtheorem*{prop}{Proposition}
\newtheorem*{thm}{Theorem}

\renewcommand{\labelitemi}{--}
\renewcommand{\labelitemii}{$\circ$}
\renewcommand{\labelenumi}{(\roman{*})}

\let\stdsection\section
\renewcommand\section{\newpage\stdsection}

% Strike through
\def\st{\bgroup \ULdepth=-.55ex \ULset}

% Maths symbols
\newcommand{\bra}{\langle}
\newcommand{\ket}{\rangle}

\newcommand{\N}{\mathbb{N}}
\newcommand{\Z}{\mathbb{Z}}
\newcommand{\Q}{\mathbb{Q}}
\renewcommand{\H}{\mathbb{H}}
\newcommand{\R}{\mathbb{R}}
\newcommand{\C}{\mathbb{C}}
\newcommand{\Prob}{\mathbb{P}}
\renewcommand{\P}{\mathbb{P}}
\newcommand{\E}{\mathbb{E}}
\newcommand{\F}{\mathbb{F}}
\newcommand{\cU}{\mathcal{U}}
\newcommand{\RP}{\mathbb{RP}}
\newcommand{\CP}{\mathbb{CP}}

\newcommand{\ph}{\,\cdot\,}

\DeclareMathOperator{\sech}{sech}
\DeclareMathOperator{\cosech}{cosech}
\DeclareMathOperator{\cosec}{cosec}

\DeclareMathOperator{\covol}{covol}
\DeclareMathOperator{\vol}{vol}

\let\Im\relax
\let\Re\relax
\DeclareMathOperator{\Im}{Im}
\DeclareMathOperator{\Re}{Re}
\DeclareMathOperator{\im}{im}
\DeclareMathOperator{\image}{image}
\DeclareMathOperator{\Ann}{Ann}

\DeclareMathOperator*{\res}{res}
\DeclareMathOperator{\Res}{Res}
\DeclareMathOperator{\Ind}{Ind}

\DeclareMathOperator{\tr}{tr}
\DeclareMathOperator{\diag}{diag}
\DeclareMathOperator{\rank}{rank}
\DeclareMathOperator{\card}{card}
\DeclareMathOperator{\spn}{span}
\DeclareMathOperator{\adj}{adj}

\DeclareMathOperator{\erf}{erf}
\DeclareMathOperator{\erfc}{erfc}

\DeclareMathOperator{\ord}{ord}
\DeclareMathOperator{\Sym}{Sym}

\DeclareMathOperator{\sgn}{sgn}
\DeclareMathOperator{\orb}{orb}
\DeclareMathOperator{\stab}{stab}
\DeclareMathOperator{\ccl}{ccl}

\DeclareMathOperator{\lcm}{lcm}
\DeclareMathOperator{\hcf}{hcf}

\DeclareMathOperator{\Int}{Int}
\DeclareMathOperator{\id}{id}

\DeclareMathOperator{\betaD}{beta}
\DeclareMathOperator{\gammaD}{gamma}
\DeclareMathOperator{\Poisson}{Poisson}
\DeclareMathOperator{\binomial}{binomial}
\DeclareMathOperator{\multinomial}{multinomial}
\DeclareMathOperator{\Bernoulli}{Bernoulli}
\DeclareMathOperator{\like}{like}

\DeclareMathOperator{\var}{var}
\DeclareMathOperator{\cov}{cov}
\DeclareMathOperator{\bias}{bias}
\DeclareMathOperator{\mse}{mse}
\DeclareMathOperator{\corr}{corr}

\DeclareMathOperator{\otp}{otp}
\DeclareMathOperator{\dom}{dom}

\DeclareMathOperator{\Root}{Root}
\DeclareMathOperator{\supp}{supp}
\DeclareMathOperator{\rel}{rel}
\DeclareMathOperator{\Hom}{Hom}
\DeclareMathOperator{\Aut}{Aut}
\DeclareMathOperator{\Gal}{Gal}
\DeclareMathOperator{\Mat}{Mat}
\DeclareMathOperator{\End}{End}
\DeclareMathOperator{\Char}{char}
\DeclareMathOperator{\ev}{ev}
\DeclareMathOperator{\St}{St}
\DeclareMathOperator{\Lk}{Lk}
\DeclareMathOperator{\disc}{disc}
\DeclareMathOperator{\Isom}{Isom}
\DeclareMathOperator{\length}{length}
\DeclareMathOperator{\energy}{energy}
\DeclareMathOperator{\area}{area}
\DeclareMathOperator{\Syl}{Syl}
\DeclareMathOperator{\cl}{cl}
\DeclareMathOperator{\fix}{fix}

\newcommand{\GL}{\mathrm{GL}}
\newcommand{\SL}{\mathrm{SL}}
\newcommand{\PGL}{\mathrm{PGL}}
\newcommand{\PSL}{\mathrm{PSL}}
\newcommand{\PSU}{\mathrm{PSU}}
\newcommand{\Or}{\mathrm{O}}
\newcommand{\SO}{\mathrm{SO}}
\newcommand{\U}{\mathrm{U}}
\newcommand{\SU}{\mathrm{SU}}

\renewcommand{\d}{\mathrm{d}}
\newcommand{\D}{\mathrm{D}}

\tikzset{->/.style = {decoration={markings,
                                  mark=at position 1 with {\arrow[scale=2]{latex'}}},
                      postaction={decorate}}}
\tikzset{<-/.style = {decoration={markings,
                                  mark=at position 0 with {\arrowreversed[scale=2]{latex'}}},
                      postaction={decorate}}}
\tikzset{<->/.style = {decoration={markings,
                                   mark=at position 0 with {\arrowreversed[scale=2]{latex'}},
                                   mark=at position 1 with {\arrow[scale=2]{latex'}}},
                       postaction={decorate}}}
\tikzset{->-/.style = {decoration={markings,
                                   mark=at position #1 with {\arrow[scale=2]{latex'}}},
                       postaction={decorate}}}
\tikzset{-<-/.style = {decoration={markings,
                                   mark=at position #1 with {\arrowreversed[scale=2]{latex'}}},
                       postaction={decorate}}}

\tikzset{circ/.style = {fill, circle, inner sep = 0, minimum size = 3}}
\tikzset{mstate/.style={circle, draw, blue, text=black, minimum width=0.7cm}}

\definecolor{mblue}{rgb}{0.2, 0.3, 0.8}
\definecolor{morange}{rgb}{1, 0.5, 0}
\definecolor{mgreen}{rgb}{0.1, 0.4, 0.2}
\definecolor{mred}{rgb}{0.5, 0, 0}

\def\drawcirculararc(#1,#2)(#3,#4)(#5,#6){%
    \pgfmathsetmacro\cA{(#1*#1+#2*#2-#3*#3-#4*#4)/2}%
    \pgfmathsetmacro\cB{(#1*#1+#2*#2-#5*#5-#6*#6)/2}%
    \pgfmathsetmacro\cy{(\cB*(#1-#3)-\cA*(#1-#5))/%
                        ((#2-#6)*(#1-#3)-(#2-#4)*(#1-#5))}%
    \pgfmathsetmacro\cx{(\cA-\cy*(#2-#4))/(#1-#3)}%
    \pgfmathsetmacro\cr{sqrt((#1-\cx)*(#1-\cx)+(#2-\cy)*(#2-\cy))}%
    \pgfmathsetmacro\cA{atan2(#2-\cy,#1-\cx)}%
    \pgfmathsetmacro\cB{atan2(#6-\cy,#5-\cx)}%
    \pgfmathparse{\cB<\cA}%
    \ifnum\pgfmathresult=1
        \pgfmathsetmacro\cB{\cB+360}%
    \fi
    \draw (#1,#2) arc (\cA:\cB:\cr);%
}
\newcommand\getCoord[3]{\newdimen{#1}\newdimen{#2}\pgfextractx{#1}{\pgfpointanchor{#3}{center}}\pgfextracty{#2}{\pgfpointanchor{#3}{center}}}

\def\Xint#1{\mathchoice
   {\XXint\displaystyle\textstyle{#1}}%
   {\XXint\textstyle\scriptstyle{#1}}%
   {\XXint\scriptstyle\scriptscriptstyle{#1}}%
   {\XXint\scriptscriptstyle\scriptscriptstyle{#1}}%
   \!\int}
\def\XXint#1#2#3{{\setbox0=\hbox{$#1{#2#3}{\int}$}
     \vcenter{\hbox{$#2#3$}}\kern-.5\wd0}}
\def\ddashint{\Xint=}
\def\dashint{\Xint-}

\renewcommand\D{\mathbb{D}}
\DeclareMathOperator\hcap{hcap}
\newcommand\SLE{\mathrm{SLE}}

\begin{document}
\maketitle
{\small
\setlength{\parindent}{0em}
\setlength{\parskip}{1em}
Schramm--Loewner Evolution (SLE) is a family of random curves in the plane, indexed by a parameter $\kappa \geq 0$. These non-crossing curves are the fundamental tool used to describe the scaling limits of a host of natural probabilistic processes in two dimensions, such as critical percolation interfaces and random spanning trees. Their introduction by Oded Schramm in 1999 was a milestone of modern probability theory.

The course will focus on the definition and basic properties of SLE. The key ideas are conformal invariance and a certain spatial Markov property, which make it possible to use It\^o calculus for the analysis. In particular we will show that, almost surely, for $\kappa \leq 4$ the curves are simple, for $4 \leq \kappa < 8$ they have double points but are non-crossing, and for $\kappa \geq 8$ they are space-filling. We will then explore the properties of the curves for a number of special values of $\kappa$ (locality, restriction properties) which will allow us to relate the curves to other conformally invariant structures.

The fundamentals of conformal mapping will be needed, though most of this will be developed as required. A basic familiarity with Brownian motion and It\^o calculus will be assumed but recalled.
}
\tableofcontents

\section{Introduction}
\section{Conformal transformations}
\subsection{Conformal transformations}
\begin{defi}[Conformal map]\index{conformal map}
  Let $U, V$ be domains in $\C$. We say a map $f: U \to V$ is \emph{conformal} if it is a bijection.
\end{defi}

We will write \term{$\D$} for the open unit disk, and \term{$\H$} for the upper half plane. An important theorem about conformal maps is the following:
\begin{thm}[Riemann mapping theorem]
  Let $U$ be a simply connected domain with $U \not= \C$ and $z \in U$ be any point. Then there exists a unique conformal transformation $f: \D \to U$ such that $f(0) = z$, and $f'(0)$ is real and positive.
\end{thm}
We shall not prove this theorem, as it is a standard result. An immediate corollary is that any two simply connected domains that are distinct from $\C$ are conformally equivalent.

\begin{eg}
  Take $U = \D$. Then for $z \in \D$, the map promised by the Riemann mapping theorem is
  \[
    f(w) = \frac{w + z}{1 + \bar{z} w}.
  \]
  In general, every conformal transformation $f: \D \to \D$ is of the form
  \[
    f(w) = \lambda \frac{w + z}{1 + \bar{z}w}
  \]
  for some $|\lambda| = 1$ and $z \in \D$.
\end{eg}

\begin{eg}
  The map $f: \H \to \D$ given by
  \[
    f(z) = \frac{z - i}{z + i},
  \]
  is a conformal transformation, with inverse
  \[
    g(w) = \frac{i(1 + w)}{1 - w}.
  \]
  In general, the conformal transformations $\H \to \H$ consist of maps of the form
  \[
    f(z) = \frac{az + b}{c z + d}
  \]
  with $a, b, c, d \in \R$ and $ad - bc \not= 0$.
\end{eg}

\begin{eg}
  For $t \geq 0$, we let $H_t = \H \setminus [0, 2\sqrt{t} i]$. The map $H_t \to \H$ given by
  \[
    z \mapsto \sqrt{z^2 + 4t}
  \]
  is a conformal transformation. Observe that this map satisfies
  \[
    |g_t(z) - z| = |\sqrt{z^2 + 4t} - z| \to 0\text{ as } z \to \infty.
  \]
  So $g_t(z) \sim z$ for large $z$.

  Observe also that the family $g_t(z)$ satisfies the ODE
  \[
    \frac{\partial g}{\partial t} = \frac{2}{g_t(z)},\quad g_0(z) = z.
  \]
  We can think of these functions $g_t$ as being generated by the curve $\gamma(t) = 2 \sqrt{t}i$, where for each $t$, the function $g_t$ is the conformal transformation that sends $\H \setminus \gamma([0, t])$ to $\H$ (satisfying $|g_t(z) - z| \to \infty$). Given the set of functions $g_t$, we can recover the curve $\gamma$ as follows --- for each $z \in \H$, we can ask what is the minimum $t$ such that $g_t$ is not defined on $z$. By ODE theorems, there is a solution up till the point when $g_t(z) = 0$, in which case the denominator of the right hand side blows up. Call this time $\tau(z)$. We then see that for each $t$, there is a unique $z$ such that $\tau(z) = t$, and we have $\gamma(t) = z$.

  More generally, suppose $\gamma$ is any simple (i.e.\ non-self-intersecting) curve in $\H$ starting from $0$. Then for each $t$, we can let $g_t$ be the unique conformal transformation that maps $\H \setminus \gamma([0, rt])$ to $\H$ with $|g_t(z) - z| \to \infty$ as above (we will later see such a map exists). Then Loewner's theorem says there is a continuous, real-valued function $W$ such that
  \[
    \frac{\partial g_t}{\partial t} = \frac{2}{g_t(z) - W_t},\quad g_0(z) = z.
  \]
  This is the \term{chordal Loewner equation}. We can turn this around --- given a function $W_t$, what is the corresponding curve $\gamma(t)$? If $W = 0$, then $\gamma(t) = 2\sqrt{t}i$. More excitingly, if we take $W = \sqrt{\kappa} B$, where $B$ is a standard Brownian motion, and then interpret this equation in stochastic calculus, then we obtain $\SLE_\kappa$.
\end{eg}
\subsection{Brownian motion and harmonic functions}
Recall that a function $f = u + iv$ is holomorphic iff it satisfies the \term{Cauchy--Riemann equations}
\[
  \frac{\partial u}{\partial x} = \frac{\partial v}{\partial y},\quad \frac{\partial u}{\partial y} = - \frac{\partial v}{\partial x}.
\]
This in particular implies that both $u$ and $v$ are \term{harmonic}.
\begin{defi}[Harmonic function]\index{harmonic function}
  A function $f: \R^k \to \R$ is \emph{harmonic} if it is $C^2$ and
  \[
    \Delta f = \left(\frac{\partial^2}{\partial x_1^2} + \cdots + \frac{\partial^2}{\partial x_k^2}\right) f = 0.
  \]
\end{defi}
Indeed, we can calculate
\[
  \frac{\partial^2 u}{\partial x^2} + \frac{\partial^2 u}{\partial y^2} = \frac{\partial}{\partial x} \frac{\partial v}{\partial y} + \frac{\partial}{\partial y} \left(- \frac{\partial v}{\partial x}\right) = 0.
\]
In Advanced Probability, we saw that Brownian motion is very closely related to harmonic functions. We say a process $B = B^1 + i B^2$ is a \term{complex Brownian motion} if $(B^1, B^2)$ is a standard Brownian motion in $\R^2$.

Recall some results from Advanced Probability:
\begin{thm}
  Let $u$ be a harmonic function on a bounded domain $D$ which is continuous on $\bar{D}$. For $z \in D$, let $\P_z$ be the law of a complex Brownian motion starting from $z$, and let $\tau$ be the first hitting time of $D$. Then
  \[
    u(z) = \E_z[u(B_\tau)].\fakeqed
  \]
\end{thm}

\begin{cor}[Mean value property]\index{mean value property}
  If $u$ is a harmonic function, then, whenever it makes sense, we have
  \[
    u(z) = \frac{1}{2\pi} \int_0^{2\pi} u(z + re^{i\theta})\;\d \theta.\qedhere
  \]
\end{cor}
\begin{cor}[Maximum principle]\index{maximum principle}
  Let $u$ be harmonic in a domain $D$. If $u$ attains its maximum at an interior point in $D$, then $u$ is constant.
\end{cor}

\begin{cor}[Maximum modulus principle]\index{maximum modulus principle}
  Let $D$ be a domain and let $f: D \to \C$ be holomorphic. If $|f|$ attains its maximum in the interior of $D$, then $f$ is constant.
\end{cor}

\begin{proof}
  Observe that if $f$ is holomorphic, then $\log |f|$ is harmonic and attains its maximum in the interior of $D$. It then follows that $|f|$ is constant (if $|f|$ vanishes somewhere, then consider $\log |f + M|$ for some large $M$, and do some patching if necessary). It is then a standard result that a holomorphic function of constant modulus is constant.
\end{proof}

The following lemma should also be familiar from the old complex analysis days as well:
\begin{lemma}[Schwarz lemma]\index{Schwarz lemma}
  Let $f: \D \to \D$ be a holomorphic map with $f(0) = 0$. Then $|f(z)| \leq |z|$ for all $z \in \D$. If $|f(z)| = |z|$ for some non-zero $z \in \D$, then $f(w) = \lambda w$ for some $\lambda \in \C$ with $|\lambda| = 1$.
\end{lemma}

\begin{proof}
  Consider the map
  \[
   g(z) =
    \begin{cases}
      f(z)/z & z \not =0\\
      f'(0) & z = 0
    \end{cases}.
  \]
  Then one sees that $g$ is holomorphic and $|g(z)| \leq 1$ for all $z \in \partial\D$, hence for all $z \in \D$ by the maximum modulus principle. If $|f(z_0)| = |z_0|$ for some $z_0 \in \D \setminus \{0\}$, then $g$ must be constant, so $f$ is linear.
\end{proof}

\subsection{Distortion estimates for conformal maps}

We write $\mathcal{U}$\index{$\mathcal{U}$} for the collection of conformal transformations $f: \D \to D$, where $D$ is any simply connected domain with $0 \in D$ and $D \not= \C$, with $f(0) = 0$ and $f'(0) = 1$. Thus, it must be of the form
\[
  f(z) = z + \sum_{n = 2}^\infty a_n z^n.
\]
\begin{thm}[Koebe-1/4 theorem]\index{Koebe-1/4 theorem}
  If $F \in \mathcal{U}$ and $0 < r \leq 1$, then $B(0, r/4) \subseteq f(r, \D)$.
\end{thm}
By scaling it suffices to prove this for the case $r = 1$. This follows from the following result:
\begin{prop}
  If $f \in \mathcal{U}$, then $|a_2| \leq 2$.
\end{prop}
The proof of this proposition will involve quite some work. So let us just conclude the theorem from this.
\begin{proof}[Proof of theorem]
  Suppose $f: \D to D$ is in $\mathcal{U}$, and $z_0 \not \in D$. We shall show that $|z_0| \geq \frac{1}{4}$. Consider the function
  \[
    \tilde{f}(z) = \frac{z_0 f(z)}{z_0 - f(z)}.
  \]
  Since $\tilde{f}$ is composition of conformal transformations, it is itself conformal, and a direct computation shows $\tilde{f} \in \mathcal{U}$. Moreover, if
  \[
    f(z) = z + a_2 z^2 + \cdots,
  \]
  then
  \[
    \tilde{f}(z) = z + \left(a_2 + \frac{1}{z_0}\right)z^2 + \cdots.
  \]
  So we obtain the bounds
  \[
    |a_2|, \left|a_2 + \frac{1}{z_0}\right| \leq 2.
  \]
  By the triangle inequality, we must have $|z_0^{-1}| \leq 4$, hence $|z_0| \geq \frac{1}{4}$.
\end{proof}
\subsection{Half-plane capacity}
\begin{defi}[Compact $\H$-hull]\index{compact $\H$-hull}
  A set $A \subseteq \H$ is called a \emph{compact $\H$-hull} if $A$ is compact, $A = \H \cap \bar{A}$ and $\H \setminus A$ is simply connected. We write $\mathcal{Q}$\index{$\mathcal{Q}$} for the collection of compact $\H$-hulls.
\end{defi}

\begin{prop}
  For each $A \in \mathcal{Q}$, there exists a unique conformal transformation $g_A: \H \setminus A \to \H$ with $|g_A(z) - a| \to 0$ as $z \to \infty$.
\end{prop}

The proof of this requires
\begin{thm}[Schwarz reflection principle]\index{Schwarz reflection principle}
  Let $D \subseteq \H$ be a simply connected domain, and let $\phi: D \to \H$ be a conformal transformation which is bounded on bounded sets and sends $\R \cap D$ to $\R$. Then $\phi$ extends by reflection to a conformal transformation on
  \[
    D^* = D \cup \{\bar{z} : z \in D\} = D \cup \bar{D}
  \]
  by setting $\phi(\bar{z}) = \overline{\phi(z)}$.\fakeqed
\end{thm}

\begin{proof}[Proof of proposition]
  The Riemann mapping theorem implies that there exists a conformal transformation $g: \H \setminus A \to \H$. Then $g(z) \to \infty$ as $z \to \infty$. By the Schwarz reflection principle, extend $g$ to a conformal transformation defined on $\C \setminus (A \cup \bar{A})$.

  By Laurent expanding $g$ at $\infty$, we can write
  \[
    g(z) = \sum_{n = -\infty}^N b_{-N} z^N.
  \]
  Since $g$ maps the real line to the real line, all $b_i$ must be real. Moreover, by injectivity, considering large $z$ shows that $N = 1$. In other words, we can write
  \[
    g(z) = b_{-1} z + b_0 + \sum_{n = 1}^\infty \frac{b_n}{z^n},
  \]
  with $b_{-1} \not= 0$. We can then define
  \[
    g_A(z) = \frac{g(z) - b_0}{b_{-1}}.
  \]
  Since $b_0$ and $b_{-1}$ are both real, this is still a conformal transformation, and $|g_A(z) - z| \to 0$ as $z \to \infty$.

  To show uniqueness, suppose $g_A, g_A'$ are two such functions. Then $g_A' \circ g_A^{-1}: \H \to \H$ is such a function for $A = \emptyset$. Thus, it suffices to show that if $g: \H \to \H$ is a conformal mapping such that $g(z) - z \to 0$ as $z \to \infty$, then in fact $g = z$. But we can expand $g(z) - z$ as
  \[
    g(z) - z = \sum_{n = 1}^\infty \frac{c_n}{z^n},
  \]
  and this has to be holomorphic at $0$. So $c_n = 0$ for all $n$, and we are done.
\end{proof}

\begin{defi}[Half-plane capacity]\index{half-plane capacity}
  Let $A \in \mathcal{Q}$. Then the \emph{half-plane capacity} of $A$ is defined to be
  \[
    \hcap(A) = \lim_{z \to \infty} z(g_A(z) - z).
  \]
  Thus, we have
  \[
    g_A(z) = z + \frac{\hcap(A)}{z} + \sum_{n = 2}^\infty \frac{b_n}{z^n}.
  \]
\end{defi}
We shall show that $\hcap(A)$ is in fact non-negative and increasing.

\begin{eg}
  $z \mapsto \sqrt{z^2 + 4t}$ is the unique conformal transformation $\H \setminus [0, \sqrt{t}i] \to \H$ with $|\sqrt{z^2 + 4z} - z| \to 0$ as $z \to \infty$. We can expand
  \[
    \sqrt{z^2 + 4t} = z + \frac{2t}{z} + \cdots.
  \]
  Thus, we know that $\hcap([0, 2\sqrt{t} i]) = 2t$.
\end{eg}

\begin{eg}
  The map $z \mapsto z + \frac{1}{z}$ maps $\H \setminus \bar{\D} \to \H$. Again, we have $|z + \frac{1}{z} - z| \to 0$ as $z \to \infty$, and so $\hcap(\H \cap \bar{D}) = 1$.
\end{eg}
We note some immediate properties:
\begin{prop}\leavevmode
  \begin{enumerate}
    \item Scaling: If $r > 0$ and $A \in \mathcal{Q}$, then $\hcap(rA) = r^2 \hcap(A)$.
    \item Translation invariance: If $x \in \R$ and $a \in \mathcal{Q}$, then $\hcap(A + x) = \hcap(A)$.
    \item Monotonicity: If $A, \tilde{A} \in \mathcal{Q}$ are such that $A \subseteq \tilde{A}$. Then $\hcap(A) \leq \hcap(\tilde{A})$.
  \end{enumerate}
\end{prop}

\begin{proof}\leavevmode
  \begin{enumerate}
    \item We have $g_{rA}(z) = r g_A(z/r)$.
    \item Observe $g_{A + x}(z) = g_A(z - x) + x$.
    \item We can write
      \[
        g_{\tilde{A}} = g_{g_A(\tilde{A} \setminus A)} \circ g_A.
      \]
      Thus, expanding out tells us
      \[
        \hcap(\tilde{A}) = \hcap(A) + \hcap(g_A(\tilde{A}\setminus A)).
      \]
      So the desired result follows if we know that the half-plane capacity is non-negative, which we will prove next.\qedhere
  \end{enumerate}
\end{proof}
Observe that these results together imply that if $A \in \mathcal{Q}$ and $A \subseteq r(\bar{\D} \cap \H)$, then
\[
  \hcap(A) \leq \hcap(r(\bar{\D} \cap \H)) \leq r^2 \hcap (\bar{\D} \cap \H) = r^2.
\]
So we know that $\hcap(A) \leq 4 \diam(A)^2$

How can we prove non-negativity? After a long time, probability theory is going to re-enter the course, because we are going to use Brownian motion to prove this!

\begin{prop}
  Suppose $A \in \mathcal{Q}$ and $B_t$ be complex Brownian motion, and $\tau = \inf\{t \geq 0: B_t \not \in \H \setminus A\}$. Then
  \begin{enumerate}
    \item For all $z \in \H \setminus A$, we have $\im(z - g_A(z)) = \E_z[\im(B_\tau)]$
    \item
      \[
        \hcap(A) = \lim_{y \to \infty} y \E_y[\im(B_\tau)].
      \]
    \item If $A \subseteq \bar{\D} \cap \H$, then
      \[
        \hcap(A) = \frac{2}{\pi} \int_0^\pi \E_{e^{i\theta}}[\im(B_\tau)]\sin \theta \;\d \theta.
      \]
  \end{enumerate}
\end{prop}
It is immediate from (ii) that
\begin{cor}
  $\hcap(A) \geq 0$.
\end{cor}

\begin{proof}\leavevmode
  \begin{enumerate}
    \item Let $\phi(z) = \im(z - g_A(z))$. Then $\phi$ is harmonic on $\H \setminus A$, since $z - g_A(z)$ is holomorphic. As
      \[
        g_A(Z) = z + \frac{\hcap (A)}{z} + \cdots
      \]
      and $\im(g_A(z)) = 0$ when $z \in \partial(\H \setminus A)$, we know $\varphi$ is harmonic, continuous and bounded. These are exactly the conditions needed to solve the Dirichlet problem using Brownian motion. So the result follows.

    \item We have
      \begin{align*}
        \hcap(A) &= \lim_{z \to \infty} z (g_A(z) - z) \\
        &= \lim_{y \to \infty} (iy) (g_A(iy) - iy)\\
        &= \lim_{y \to \infty} y \im (iy - g_A(iy))\\
        &= \lim_{y \to \infty} y \E_{iy} [\im(B_\tau)]
      \end{align*}
      where we use the fact that $\hcap(A)$ is real, so we can take the limit of the real part instead.
    \item See example sheet.\qedhere
  \end{enumerate}
\end{proof}

In some sense, what we used above is that Brownian motion is conformally invariant, where we used a conformal mapping to transfer between our knowledge of Brownian motion on $\H$ to that on $\H \setminus \D$. Informally, this says the conformal image of a Brownian motion is a Brownian motion, up to a (random) change of time, and in particular, the image of the path is unchanged.

We have in fact seen this before. We have seen that rotations preserve Brownian motion, and so does scaling, except when we scale we need to scale our time as well. Since conformal mappings are locally rotations and scalings, the result should not be surprising. The random time change happens because we are performing different transformations all over the domain.

\begin{thm}
  Let $D, \tilde{D} \subseteq \C$ be domains, and $f: D \to \tilde{D}$ a conformal transformation. Let $B, \tilde{B}$ be Brownian motions starting from $z \in D, \tilde{z} \in \tilde{D}$ respectively, with $f(z) =\ tilde{z}$. Let
  \begin{align*}
    \tau = \inf\{t \geq 0: B_t \not \in D\}\\
    \tilde{\tau} = \inf\{t \geq 0: \tilde{B}_t \not \in \tilde{D}\}
  \end{align*}
  Set
  \begin{align*}
    \tau' &= \int_0^\tau |f'(B_s)|^2\;\d s\\
    \sigma(t) &= \inf\left\{s \geq 0 : \int_0^s |f'(B_r)|^2\;\d r = t\right\}\\
    B_t' = f(B_{\sigma(t)}).
  \end{align*}
  Then $(B_t' : t < \tilde{\tau}')$ has the same distribution as $(\tilde{B}_t: t < \tilde{\tau})$.
\end{thm}

\begin{proof}
  See Stochastic Calculus.
\end{proof}

\begin{eg}
  We can use conformal invariance to deduce the first exist distribution of a Brownian motion from different domains. First of all, observe that from $\D$, starting from the origin, the first exit distribution is uniform. Thus, applying a conformal transformation, we see that the exit distribution starting from $z$ is
  \[
    f(e^{i\theta}) = \frac{1}{2\pi}\left(\frac{1 - |z|^2}{ |e^{i\theta} - z|}\right).
  \]
  Similarly, on $\H$, starting from $z = x + iy$, the exit distribution is
  \[
    f(u) = \frac{1}{\pi} \frac{y}{(x - u)^2 + y^2}.
  \]
  Note that if $x = 0, y = 1$, then this is just
  \[
    f(u) \frac{1}{\pi}\left(\frac{1}{u^2 + 1}\right).
  \]
  This is the Cauchy distribution!
\end{eg}

\printindex
\end{document}
