\documentclass[a4paper]{article}

\def\npart {IB}
\def\nterm {Lent}
\def\nyear {2016}
\def\nlecturer {O. Randal-Williams}
\def\ncourse {Groups, Rings and Modules}
\def\nlectures {MWF.12}
\def\nnotready {}

% Imports
\ifx \nextra \undefined
  \usepackage[pdftex,
    hidelinks,
    pdfauthor={Dexter Chua},
    pdfsubject={Cambridge Maths Notes: Part \npart\ - \ncourse},
    pdftitle={Part \npart\ - \ncourse},
  pdfkeywords={Cambridge Mathematics Maths Math \npart\ \nterm\ \nyear\ \ncourse}]{hyperref}
  \title{Part \npart\ - \ncourse}
\else
  \usepackage[pdftex,
    hidelinks,
    pdfauthor={Dexter Chua},
    pdfsubject={Cambridge Maths Notes: Part \npart\ - \ncourse\ (\nextra)},
    pdftitle={Part \npart\ - \ncourse\ (\nextra)},
  pdfkeywords={Cambridge Mathematics Maths Math \npart\ \nterm\ \nyear\ \ncourse\ \nextra}]{hyperref}

  \title{Part \npart\ - \ncourse \\ {\Large \nextra}}
\fi

\author{Lectured by \nlecturer \\\small Notes taken by Dexter Chua}
\date{\nterm\ \nyear}

\usepackage{alltt}
\usepackage{amsfonts}
\usepackage{amsmath}
\usepackage{amssymb}
\usepackage{amsthm}
\usepackage{booktabs}
\usepackage{caption}
\usepackage{enumitem}
\usepackage{fancyhdr}
\usepackage{graphicx}
\usepackage{mathtools}
\usepackage{microtype}
\usepackage{multirow}
\usepackage{pdflscape}
\usepackage{pgfplots}
\usepackage{siunitx}
\usepackage{tabularx}
\usepackage{tikz}
\usepackage{tkz-euclide}
\usepackage[normalem]{ulem}
\usepackage[all]{xy}

\pgfplotsset{compat=1.12}

\pagestyle{fancyplain}
\lhead{\emph{\nouppercase{\leftmark}}}
\ifx \nextra \undefined
  \rhead{
    \ifnum\thepage=1
    \else
      \npart\ \ncourse
    \fi}
\else
  \rhead{
    \ifnum\thepage=1
    \else
      \npart\ \ncourse\ (\nextra)
    \fi}
\fi
\usetikzlibrary{arrows}
\usetikzlibrary{decorations.markings}
\usetikzlibrary{decorations.pathmorphing}
\usetikzlibrary{positioning}
\usetikzlibrary{fadings}
\usetikzlibrary{intersections}
\usetikzlibrary{cd}

\newcommand*{\Cdot}{\raisebox{-0.25ex}{\scalebox{1.5}{$\cdot$}}}
\newcommand {\pd}[2][ ]{
  \ifx #1 { }
    \frac{\partial}{\partial #2}
  \else
    \frac{\partial^{#1}}{\partial #2^{#1}}
  \fi
}

% Theorems
\theoremstyle{definition}
\newtheorem*{aim}{Aim}
\newtheorem*{axiom}{Axiom}
\newtheorem*{claim}{Claim}
\newtheorem*{cor}{Corollary}
\newtheorem*{defi}{Definition}
\newtheorem*{eg}{Example}
\newtheorem*{fact}{Fact}
\newtheorem*{law}{Law}
\newtheorem*{lemma}{Lemma}
\newtheorem*{notation}{Notation}
\newtheorem*{prop}{Proposition}
\newtheorem*{thm}{Theorem}

\renewcommand{\labelitemi}{--}
\renewcommand{\labelitemii}{$\circ$}
\renewcommand{\labelenumi}{(\roman{*})}

\let\stdsection\section
\renewcommand\section{\newpage\stdsection}

% Strike through
\def\st{\bgroup \ULdepth=-.55ex \ULset}

% Maths symbols
\newcommand{\bra}{\langle}
\newcommand{\ket}{\rangle}

\newcommand{\N}{\mathbb{N}}
\newcommand{\Z}{\mathbb{Z}}
\newcommand{\Q}{\mathbb{Q}}
\renewcommand{\H}{\mathbb{H}}
\newcommand{\R}{\mathbb{R}}
\newcommand{\C}{\mathbb{C}}
\newcommand{\Prob}{\mathbb{P}}
\renewcommand{\P}{\mathbb{P}}
\newcommand{\E}{\mathbb{E}}
\newcommand{\F}{\mathbb{F}}
\newcommand{\cU}{\mathcal{U}}
\newcommand{\RP}{\mathbb{RP}}
\newcommand{\CP}{\mathbb{CP}}

\newcommand{\ph}{\,\cdot\,}

\DeclareMathOperator{\sech}{sech}
\DeclareMathOperator{\cosech}{cosech}
\DeclareMathOperator{\cosec}{cosec}

\DeclareMathOperator{\covol}{covol}
\DeclareMathOperator{\vol}{vol}

\let\Im\relax
\let\Re\relax
\DeclareMathOperator{\Im}{Im}
\DeclareMathOperator{\Re}{Re}
\DeclareMathOperator{\im}{im}
\DeclareMathOperator{\image}{image}
\DeclareMathOperator{\Ann}{Ann}

\DeclareMathOperator*{\res}{res}
\DeclareMathOperator{\Res}{Res}
\DeclareMathOperator{\Ind}{Ind}

\DeclareMathOperator{\tr}{tr}
\DeclareMathOperator{\diag}{diag}
\DeclareMathOperator{\rank}{rank}
\DeclareMathOperator{\card}{card}
\DeclareMathOperator{\spn}{span}
\DeclareMathOperator{\adj}{adj}

\DeclareMathOperator{\erf}{erf}
\DeclareMathOperator{\erfc}{erfc}

\DeclareMathOperator{\ord}{ord}
\DeclareMathOperator{\Sym}{Sym}

\DeclareMathOperator{\sgn}{sgn}
\DeclareMathOperator{\orb}{orb}
\DeclareMathOperator{\stab}{stab}
\DeclareMathOperator{\ccl}{ccl}

\DeclareMathOperator{\lcm}{lcm}
\DeclareMathOperator{\hcf}{hcf}

\DeclareMathOperator{\Int}{Int}
\DeclareMathOperator{\id}{id}

\DeclareMathOperator{\betaD}{beta}
\DeclareMathOperator{\gammaD}{gamma}
\DeclareMathOperator{\Poisson}{Poisson}
\DeclareMathOperator{\binomial}{binomial}
\DeclareMathOperator{\multinomial}{multinomial}
\DeclareMathOperator{\Bernoulli}{Bernoulli}
\DeclareMathOperator{\like}{like}

\DeclareMathOperator{\var}{var}
\DeclareMathOperator{\cov}{cov}
\DeclareMathOperator{\bias}{bias}
\DeclareMathOperator{\mse}{mse}
\DeclareMathOperator{\corr}{corr}

\DeclareMathOperator{\otp}{otp}
\DeclareMathOperator{\dom}{dom}

\DeclareMathOperator{\Root}{Root}
\DeclareMathOperator{\supp}{supp}
\DeclareMathOperator{\rel}{rel}
\DeclareMathOperator{\Hom}{Hom}
\DeclareMathOperator{\Aut}{Aut}
\DeclareMathOperator{\Gal}{Gal}
\DeclareMathOperator{\Mat}{Mat}
\DeclareMathOperator{\End}{End}
\DeclareMathOperator{\Char}{char}
\DeclareMathOperator{\ev}{ev}
\DeclareMathOperator{\St}{St}
\DeclareMathOperator{\Lk}{Lk}
\DeclareMathOperator{\disc}{disc}
\DeclareMathOperator{\Isom}{Isom}
\DeclareMathOperator{\length}{length}
\DeclareMathOperator{\energy}{energy}
\DeclareMathOperator{\area}{area}
\DeclareMathOperator{\Syl}{Syl}
\DeclareMathOperator{\cl}{cl}
\DeclareMathOperator{\fix}{fix}

\newcommand{\GL}{\mathrm{GL}}
\newcommand{\SL}{\mathrm{SL}}
\newcommand{\PGL}{\mathrm{PGL}}
\newcommand{\PSL}{\mathrm{PSL}}
\newcommand{\PSU}{\mathrm{PSU}}
\newcommand{\Or}{\mathrm{O}}
\newcommand{\SO}{\mathrm{SO}}
\newcommand{\U}{\mathrm{U}}
\newcommand{\SU}{\mathrm{SU}}

\renewcommand{\d}{\mathrm{d}}
\newcommand{\D}{\mathrm{D}}

\tikzset{->/.style = {decoration={markings,
                                  mark=at position 1 with {\arrow[scale=2]{latex'}}},
                      postaction={decorate}}}
\tikzset{<-/.style = {decoration={markings,
                                  mark=at position 0 with {\arrowreversed[scale=2]{latex'}}},
                      postaction={decorate}}}
\tikzset{<->/.style = {decoration={markings,
                                   mark=at position 0 with {\arrowreversed[scale=2]{latex'}},
                                   mark=at position 1 with {\arrow[scale=2]{latex'}}},
                       postaction={decorate}}}
\tikzset{->-/.style = {decoration={markings,
                                   mark=at position #1 with {\arrow[scale=2]{latex'}}},
                       postaction={decorate}}}
\tikzset{-<-/.style = {decoration={markings,
                                   mark=at position #1 with {\arrowreversed[scale=2]{latex'}}},
                       postaction={decorate}}}

\tikzset{circ/.style = {fill, circle, inner sep = 0, minimum size = 3}}
\tikzset{mstate/.style={circle, draw, blue, text=black, minimum width=0.7cm}}

\definecolor{mblue}{rgb}{0.2, 0.3, 0.8}
\definecolor{morange}{rgb}{1, 0.5, 0}
\definecolor{mgreen}{rgb}{0.1, 0.4, 0.2}
\definecolor{mred}{rgb}{0.5, 0, 0}

\def\drawcirculararc(#1,#2)(#3,#4)(#5,#6){%
    \pgfmathsetmacro\cA{(#1*#1+#2*#2-#3*#3-#4*#4)/2}%
    \pgfmathsetmacro\cB{(#1*#1+#2*#2-#5*#5-#6*#6)/2}%
    \pgfmathsetmacro\cy{(\cB*(#1-#3)-\cA*(#1-#5))/%
                        ((#2-#6)*(#1-#3)-(#2-#4)*(#1-#5))}%
    \pgfmathsetmacro\cx{(\cA-\cy*(#2-#4))/(#1-#3)}%
    \pgfmathsetmacro\cr{sqrt((#1-\cx)*(#1-\cx)+(#2-\cy)*(#2-\cy))}%
    \pgfmathsetmacro\cA{atan2(#2-\cy,#1-\cx)}%
    \pgfmathsetmacro\cB{atan2(#6-\cy,#5-\cx)}%
    \pgfmathparse{\cB<\cA}%
    \ifnum\pgfmathresult=1
        \pgfmathsetmacro\cB{\cB+360}%
    \fi
    \draw (#1,#2) arc (\cA:\cB:\cr);%
}
\newcommand\getCoord[3]{\newdimen{#1}\newdimen{#2}\pgfextractx{#1}{\pgfpointanchor{#3}{center}}\pgfextracty{#2}{\pgfpointanchor{#3}{center}}}

\def\Xint#1{\mathchoice
   {\XXint\displaystyle\textstyle{#1}}%
   {\XXint\textstyle\scriptstyle{#1}}%
   {\XXint\scriptstyle\scriptscriptstyle{#1}}%
   {\XXint\scriptscriptstyle\scriptscriptstyle{#1}}%
   \!\int}
\def\XXint#1#2#3{{\setbox0=\hbox{$#1{#2#3}{\int}$}
     \vcenter{\hbox{$#2#3$}}\kern-.5\wd0}}
\def\ddashint{\Xint=}
\def\dashint{\Xint-}


\begin{document}
\maketitle
{\small
\noindent\textbf{Groups}\\
Basic concepts of group theory recalled from Part IA Groups. Normal subgroups, quotient groups and isomorphism theorems. Permutation groups. Groups acting on sets, permutation representations. Conjugacy classes, centralizers and normalizers. The centre of a group. Elementary properties of finite $p$-groups. Examples of finite linear groups and groups arising from geometry. Simplicity of $A_n$.

\vspace{5pt}
\noindent Sylow subgroups and Sylow theorems. Applications, groups of small order.\hspace*{\fill} [8]

\vspace{10pt}
\noindent\textbf{Rings}\\
Definition and examples of rings (commutative, with 1). Ideals, homomorphisms, quotient rings, isomorphism theorems. Prime and maximal ideals. Fields. The characteristic of a field. Field of fractions of an integral domain.

\vspace{5pt}
\noindent Factorization in rings; units, primes and irreducibles. Unique factorization in principal ideal domains, and in polynomial rings. Gauss' Lemma and Eisenstein's irreducibility criterion.

\vspace{5pt}
\noindent Rings $\Z[\alpha]$ of algebraic integers as subsets of $\C$ and quotients of $\Z[x]$. Examples of Euclidean domains and uniqueness and non-uniqueness of factorization. Factorization in the ring of Gaussian integers; representation of integers as sums of two squares.

\vspace{5pt}
\noindent Ideals in polynomial rings. Hilbert basis theorem.\hspace*{\fill} [10]

\vspace{10pt}
\noindent\textbf{Modules}\\
Definitions, examples of vector spaces, abelian groups and vector spaces with an endomorphism. Sub-modules, homomorphisms, quotient modules and direct sums. Equivalence of matrices, canonical form. Structure of finitely generated modules over Euclidean domains, applications to abelian groups and Jordan normal form.\hspace*{\fill} [6]}

\tableofcontents

\section{Groups}
\subsection{Basic concepts}
\begin{defi}[Group]
  A \emph{group} is a triple $(G, \ph, e)$, where $G$ is a set, $\ph: G\times G \to G$ is a function and $e \in G$ such that
  \begin{enumerate}
    \item For all $a, b, c \in G$, $(a \cdot b) \cdot c = a \cdot (b \cdot c)$.\hfill (associativity)
    \item For all $a \in G$, $a \cdot e = e \cdot a = a$.\hfill (identity)
    \item For all $a \in G$, there exists $a^{-1} \in G$ such that $a \cdot a^{-1} = a^{-1} \cdot a = e$.\hfill (inverse)
  \end{enumerate}
\end{defi}
Some people add a stupid axiom that $g \cdot h \in G$ for all $g, h \in G$, but this is already implied by saying $\cdot$ is a function to $G$. You can write that down as well, and no one will say you are stupid. But they might \emph{think} so.

\begin{lemma}
  The inverse of an element is unique.
\end{lemma}

\begin{proof}
  Let $a^{-1}, b$ be inverses of $a$. Then
  \[
    b = b \cdot e = b \cdot a \cdot a^{-1} = e \cdot a^{-1} = a^{-1}.
  \]
\end{proof}

\begin{defi}[Subgroup]
  If $(G, \ph)$ is a group and $H \subseteq G$ is a subset, it is a \emph{subgroup} if
  \begin{enumerate}
    \item $e \in H$,
    \item $a, b \in H$ implies $a\cdot b\in H$,
    \item $\ph: H \times H \to H$ makes $(H, \ph, e)$ a group.
  \end{enumerate}
  We write $H \leq G$ if $H$ is a subgroup of $G$.
\end{defi}
Note that the last condition in some sense encompasses the first two, but we need the first two conditions to hold before that statement makes sense at all.

\begin{lemma}
  $H \subseteq G$ is a subgroup if $H$ is non-empty and for any $h_1, h_2 \in H$, we have $h_1h_2^{-1} \in H$.
\end{lemma}

\begin{defi}[Abelian group]
  A group $G$ is \emph{abelian} if $a\cdot b = b\cdot a$ for all $a, b\in G$.
\end{defi}

\begin{eg}\leavevmode
  \begin{enumerate}
    \item The following are groups: $(\Z, +, 0)$, $(\Q, +, 0)$, $(\R, +, 0)$, $(\C, +, 0)$.
    \item Groups of symmetries
      \begin{enumerate}
        \item The symmetric group $S_n$ of permutations of $\{1, 2, \cdots, n\}$.
        \item The dihedral group $D_{2n}$ is the symmetries of a regular $n$-gon.
        \item The group $\GL_n(\R)$ is the group of invertible $n\times n$ real matrices, which also is the group of invertible $\R$-linear maps from the vector space $\R^n$ to itself.
      \end{enumerate}
    \item The alternating group $A_n \leq S_n$.
    \item The cyclic group $C_n \leq D_{2n}$.
    \item The special linear group $\SL_n(\R) \leq \GL_n(\R)$, the subgroup of matrices of determinant $1$.
    \item The Klein-four group $C_2 \times C_2$.
    \item The quaternions $Q_8 = \{\pm 1, \pm i, \pm j, \pm k\}$ with $ij = k, ji = -k$, $i^2 = j^2 = k^2 = -1$, $(-1)^2 = 1$.
  \end{enumerate}
\end{eg}

With groups and subgroups, we can talk about cosets.
\begin{defi}[Coset]
  If $H \leq G$, $g \in G$, the \emph{left coset} $gH$ is the set
  \[
    gH = \{x \in G: x = g\cdot h\text{ for some }h \in H\}.
  \]
\end{defi}
For example, since $H$ is a subgroup, we know $e \in H$. So for any $g \in G$, we must have $g \in gH$.

The collection of $H$-cosets in $G$ forms a partition of $G$, and furthermore, all $H$ cosets $gH$ are in bijection with $H$ itself, via $h \mapsto gh$. An immediate consequence is
\begin{thm}[Lagrange's theorem]
  Let $G$ be a finite group, and $H \leq G$. Then
  \[
    |G| = |H| |G:H|,
  \]
  where $|G:H|$ is the number of $H$ cosets in $G$.
\end{thm}
We can do exactly the same thing with right cosets and get the same conclusion.

We have implicitly used the following notation:
\begin{defi}[Order of group]
  The \emph{order} of a group is the number of elements in $G$, written $|G|$.
\end{defi}

Instead of order of the group, we can ask what the order of an element is.
\begin{defi}[Order of element]
  The \emph{order} of an element $g \in G$ is the smallest positive $n$ such that $g^n = e$. If there is no such $n$, we say $g$ has infinite order.

  We write $\ord(g) = n$.
\end{defi}

A basic lemma is as follows:
\begin{lemma}
  If $G$ is a finite group and $g \in G$ has order $n$, then $n \mid G$.
\end{lemma}

\begin{proof}
  Consider the following subset:
  \[
    H= \{e, g, g^2, \cdots, g^{n - 1}\}.
  \]
  This is a subgroup of $G$, because it is non-empty and $g^rg^{-s} = g^{r - s}$ is on the list (we might have to add $n$ to the power of $g$ to make it positive, but this is fine since $g^n = e$). Moreover, there are no repeats in the list: if $g^i = g^j$, with wlog $i \geq j$, then $g^{i - j} = e$. So $i - j < n$. By definition of $n$, we must have $i - j = 0$, ie. $i = j$.

  Hence Lagrange's theorem tells us $n = |H| \mid |G|$.
\end{proof}

\subsection{Normal subgroups, quotients, homomorphisms, isomorphisms}
We all know what the definition of a normal subgroup is. But if we were not given the definition, how could we have come up with it?

Let $H \leq G$ be a subgroup. Notice that if $gH = g'H$, then $g \in g'H$. So $g = g'\cdot h$ for some $h$. In other words, $(g')^{-1} \cdot g = h \in H$. So if two elements represent the same coset, their difference is in $H$. The argument is also reversible. Hence two elements $g, g'$ represent the same $H$-coset if and only if $(g')^{-1} g \in H$.

Suppose we try to make the set $G/H = \{gH: g \in G\}$ into a group, by the obvious formula
\[
  (g_1 H) \cdot (g_2 H) = g_1 g_2 H.
\]
However, this doesn't necessarily make sense. If we take a different representative for the same coset, we want to make sure it gives the same answer. If $g_2 H = g_2' H$, then $g_2' = g_2 \cdot h$ for some $h \in H$. So
\[
  (g_1 H) \cdot (g_2' H) = g_1 g_2' H = g_1 g_2 h H = g_1g_2 H = (g_1 H) \cdot (g_2 H).
\]
So all is good.

What if we change $g_1$? If $g_1H = g_1' H$, then $g_1' = g_1 \cdot h$ for some $h \in H$. So
\[
  (g_1' H) \cdot (g_2 H) = g_1' g_2 H = g_1 h g_2 H.
\]
Now we are stuck. We would really want the equality
\[
  g_1 h g_2 H = g_1 g_2 H
\]
to hold. This requires
\[
  (g_1g_2)^{-1} g_1 h g_2 \in H.
\]
This is equivalent to
\[
  g_2^{-1} h g_2 \in H.
\]
So for $G/H$ to actually be a group under this operation, we must have, for any $h \in H$ and $g \in G$, the property $g^{-1} h g \in H$ to hold.

This is not necessarily true for an arbitrary $H$. Those nice ones that satisfy this property are known as \emph{normal subgroups}.
\begin{defi}[Normal subgroup]
  A subgroup $H \leq G$ is \emph{normal} if for any $h \in H$ and $g \in G$, we have $g^{-1}h g \in H$. We write $H \lhd G$.
\end{defi}

This allows us to make the following definition:
\begin{defi}[Quotient group]
  If $H \lhd G$ is a normal subgroup, then the set $G/H$ of left $H$-cosets forms a group with multiplication
  \[
    (g_1 H) \cdot (g_2 H) = g_1 g_2 H.
  \]
  with identity $eH = H$. This is known as the \emph{quotient group}.
\end{defi}
This is indeed a group. Normality was defined such that this is well-defined. Multiplication is associative since multiplication in $G$ is associative. The inverse of $gH$ is $g^{-1}H$, and $eH$ is easily seen to be the identity.

So far, we've just been looking at groups themselves. We would also like to know how groups interact with each other. In other words, we want to study functions between groups. However, we don't allow \emph{arbitrary} functions, since groups have some structure, and we would like the functions to respect the group structures. These nice functions are known as \emph{homomorphisms}.

\begin{defi}[Homomorphism]
  If $(G, \ph, e_G)$ and $(H, *, e_H)$ are groups, a function $\phi: G\to H$ is a \emph{homomorphism} if $\phi(e_G) = e_H$, and for $g, g' \in G$, we have
  \[
    \phi(g \cdot g') = \phi(g) * \phi(g').
  \]
\end{defi}
If we think carefully, $\phi(e_G) = e_H$ can be derived from the second condition, but we could as well put it in as well. It is easy to show the following:
\begin{lemma}
  If $\phi: G \to H$ is a homomorphism, then
  \[
    \phi(g^{-1}) = \phi(g)^{-1}.
  \]
\end{lemma}

\begin{proof}
  We compute $\phi(g\cdot g^{-1})$ in two ways. On the one hand, we have
  \[
    \phi(g\cdot g^{-1}) = \phi(e) = e.
  \]
  On the other hand, we have
  \[
    \phi(g\cdot g^{-1}) = \phi(g) * \phi(g^{-1}).
  \]
  By the uniqueness of inverse, we must have
  \[
    \phi(g^{-1}) = \phi(g)^{-1}.
  \]
\end{proof}

Given any homomorphism, we can build two groups out of it:
\begin{defi}[Kernel]
  The \emph{kernel} of a homomorphism $\phi: G \to H$ is
  \[
    \ker(\phi) = \{g \in G: \phi(g) = 0\}.
  \]
\end{defi}

\begin{defi}[Image]
  The \emph{image} of a homomorphism $\phi: G \to H$ is
  \[
    \im (\phi) = \{h\in H: h = \phi(g) \text{ for some }g \in G\}.
  \]
\end{defi}

There are some basic properties about these:
\begin{lemma}
  For a homomorphism $\phi: G\to H$, the kernel $\ker (\phi)$ is a \emph{normal subgroup}, and the image $\im (\phi)$ is a subgroup of $H$.
\end{lemma}

There is only one possible way we can prove this.
\begin{proof}
  To see $\ker(\phi)$ is a subgroup, let $g, h \in \ker \phi$. Then
  \[
    \phi(g\cdot h^{-1}) = \phi(g) * \phi(h)^{-1} = e * e^{-1} = e.
  \]
  So $gh^{-1} \in \ker \phi$. Also, $\phi(e) = e$. So $\ker(\phi)$ is non-empty. So it is a subgroup.

  To show it is normal, let $g \in \ker(\phi)$. Let $x \in G$. we want to show $x^{-1}gx \in \ker(\phi)$. We have
  \[
    \phi(x^{-1} xg) = \phi(x^{-1}) * \phi(g) * \phi(x) = \phi(x^{-1}) * \phi(x) = \phi(x^{-1}x) = \phi(e) = e.
  \]
  So $x^{-1}gx \in \ker(\phi)$. So $\ker(\phi)$ is normal.

  Also, if $\phi(g), \phi(h) \in \im (\phi)$, then
  \[
    \phi(g) * \phi(h)^{-1} = \phi(gh^{-1}) \in \im (\phi).
  \]
  Also, $e \in \im(\phi)$. So $\im(\phi)$ is a subgroup.
\end{proof}

Homomorphisms are nice. But those that are bijections are even nicer.
\begin{defi}[Isomorphism]
  An \emph{isomorphism} is a homomorphism that is also a bijection.
\end{defi}

\begin{defi}[Isomorphic group]
  Two groups $G$ and $H$ are \emph{isomorphic} if there is an isomorphism between them. We write $G \cong H$.
\end{defi}
Usually, we identify two isomorphic groups as being ``the same''.

It is an exercise to show the following:
\begin{lemma}
  If $\phi$ is an isomorphism, then the inverse $\phi^{-1}$ is also an isomorphism.
\end{lemma}

When studying groups, it is often helpful to break the group apart into smaller groups, which are hopefully easier to study. We will have three isomorphism theorems to do so. These allow us to break our groups apart into smaller groups, and then if a miracle happens, we can patch these together to get information about the big group. Even if miracles do not happen, these are useful tools to have.

% complete

An important theorem relating the image and the kernel of a homomorphism is the following:
\begin{thm}[First isomorphism theorem]
  Let $\phi: G \to H$ is a homomorphism, then $\ker(\phi) \lhd G$ and
  \[
    \frac{G}{\ker (\phi)} \cong \im (\phi).
  \]
\end{thm}

\begin{proof}
  We have already proved that $\ker (\phi)$ is a normal subgroup. We now have to construct a homomorphism $\phi: G/\ker(\phi) \to \im (\phi)$, and prove it is an isomorphism.

  Define our function as follows:
  \begin{align*}
    f: \frac{G}{\ker (\phi)} &\to \im (\phi)\\
    g\ker (\phi) &\mapsto \phi(g).
  \end{align*}
  We first tackle the obvious problem that this might not be well-defined, since we are picking a representative for the coset. If $g \ker(\phi) = g' \ker(\phi)$, then we know $g^{-1} \cdot g' \in \ker(\phi)$. So $\phi(g^{-1} \cdot g') = e$. So we know
  \[
    e = \phi(g^{-1} \cdot g') = \phi(g)^{-1} * \phi(g').
  \]
  Multiplying the whole thing by $\phi(g)$ gives $\phi(g) = \phi(g')$. Hence this function is well-defined.

  Next we show it is a homomorphism. To see $f$ is a homomorphism, we have
  \begin{align*}
    f(g\ker (\phi) \cdot g'\ker(\phi)) &= f(gg'\ker(\phi)) \\
    &= \phi(gg') \\
    &= \phi(g) * \phi(g') \\
    &= f(g\ker(\phi)) * f(g'\ker(\phi)).
  \end{align*}
  So $f$ is a homomorphism. Finally, we show it is a bijection.

  To show it is surjective, let $h \in \im (\phi)$. Then $h = \phi(g)$ for some $g$. So $h = f(g\ker (\phi))$ is in the image of $f$.

  To show injectivity, suppose $f(g\ker (\phi)) = f(g'\ker(\phi))$. So $\phi(g) = \phi(g')$. So $\phi(g^{-1} \cdot g') = e$. Hence $g^{-1} \cdot g' \in \ker(\phi)$, and hence $g \ker(\phi) = g'\ker(\phi)$. So done.
\end{proof}

Before we move on to further isomorphism theorems, we see how we can use these to identify two groups which are not obvious the same.

\begin{eg}
  Consider a homomorphism $\phi: \C \to \C\setminus \{0\}$ given by $z \mapsto e^z$. We also know that
  \[
    e^{z + w} = e^z e^w.
  \]
  This means $\phi$ is a homomorphism if we think of it as $\phi: (\C, +) \to (\C\setminus \{0\}, \times)$.

  What is the image of this homomorphism? The existence of $\log$ shows that $\phi$ is surjective. So $\im \phi = \C\setminus \{0\}$. What about the kernel? It is given by
  \[
    \ker(\phi) = \{z \in \C: e^z = 1\} = 2\pi i \Z,
  \]
  ie. the set of all integer multiples of $2\pi i$. The conclusion is that
  \[
    (\C / (2\pi i \Z), +) \cong (\C\setminus \{0\}, \times).
  \]
\end{eg}

The second isomorphism is a slightly more complicated theorem.
\begin{thm}[Second isomorphism theorem]
  Let $H \leq G$ and $K \lhd G$. Then $HK = \{h\cdot k: h \in H, k \in K\}$ is a subgroup of $G$, and $H\cap K \lhd H$. Moreover,
  \[
    \frac{HK}{K} \cong \frac{H}{H \cap K}.
  \]
\end{thm}

\begin{proof}
  Let $hk, h'k' \in HK$. Then
  \[
    h'k'(hk)^{-1} = h'k' k^{-1}h^{-1} = (h'h^{-1})(hk'k^{-1}h^{-1}).
  \]
  The first term is in $H$, while the second term is $k'k^{-1} \in K$ conjugated by $h$, which also has to be in $K$ be normality. So this is something in $H$ times something in $K$, and hence in $HK$. $HK$ also contains $e$, and is hence a group.

  To show $H \cap K \lhd H$, consider $x \in H\cap K$ and $h \in H$. Consider $h^{-1} x h$. Since $x \in K$, the normality of $K$ implies $h^{-1}xh \in K$. Also, since $x, h \in H$, closure implies $h^{-1}xh \in H$. So $h^{-1} x h \in H \cap K$. So $H \cap K \lhd H$.

  Now we can start to prove the second isomorphism theorem. To do so, we apply the first isomorphism theorem to it. Define
  \begin{align*}
    \phi: H &\to G/K\\
    &h \mapsto hK
  \end{align*}
  This is easily seen to be a homomorphism. We apply the first isomorphism theorem to this homomorphism. The image is all $K$ cosets represented by something in $H$, ie.
  \[
    \im (\phi) = \frac{HK}{K}.
  \]
  Then the kernel of $\phi$ is
  \[
    \ker(\phi) = \{h: hK = eK\} = \{h: h \in K\} = H \cap K.
  \]
  So the first isomorphism theorem says
  \[
    \frac{H}{H \cap K} = \frac{HK}{K}.
  \]
\end{proof}
Notice we did more work than we really had to. We could have started by writing down $\phi$ and checked it is a homomorphism. Then since $H \cap K$ is its kernel, it has to be a normal subgroup.

Before we move on to the third isomorphism theorem, we notice that if $K \lhd G$, then there is a bijection
\begin{align*}
  \{\text{subgroups of }G/K\} &\longleftrightarrow\{\text{subgroups of }G\text{ which contain }K\}\\
  X \leq \frac{G}{K} &\longrightarrow \{g \in G: gK \in X\}\\
  \frac{L}{K} \leq \frac{G}{K} &\longleftarrow K \lhd L \leq G.
\end{align*}
It is an elementary exercise to show that these are inverses of each other.

This specializes to the bijection of normal subgorups:
\[
  \{\text{normal subgroups of }G/K\} \longleftrightarrow\{\text{normal subgroups of }G\text{ which contain }K\}\\
\]
using the same bijection.

\begin{thm}[Third isomorphism theorem]
  Let $K \leq L \leq G$ be normal subgroups of $G$. Then
  \[
    \frac{G}{K}\big/ \frac{L}{K} \cong \frac{G}{L}.
  \]
\end{thm}

\begin{proof}
  Define the homomorphism
  \begin{align*}
    \phi: G/K &\to G/L\\
    gK &\mapsto gL
  \end{align*}
  As always, we have to check this is well-defined. If $gK = g'K$, then $g^{-1}g' \in K \subseteq L$. So $gL = g'L$. This is also a homomorphism since
  \[
    \phi(gK \cdot g'K) = \phi(gg'K) = gg'L = (gL) \cdot (g'L) = \phi(gK) \cdot \phi(g'K).
  \]
  This clearly is surjective, since any coset $gL$ is the image $\phi(gK)$. So the image is $G/L$. The kernel is then
  \[
    \ker(\phi) = \{gK: gL = L\} = \{gK: g \in L\} = \frac{L}{K}.
  \]
  So the conclusion follows by the first isomorphism theorem.
\end{proof}
\end{document}
