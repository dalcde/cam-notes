\documentclass[a4paper]{article}

\def\npart {II}
\def\nterm {Michaelmas}
\def\nyear {2016}
\def\nlecturer {J. Miller}
\def\ncourse {Probability and Measure}
\def\nlecture {MWF.9}

% Imports
\ifx \nextra \undefined
  \usepackage[pdftex,
    hidelinks,
    pdfauthor={Dexter Chua},
    pdfsubject={Cambridge Maths Notes: Part \npart\ - \ncourse},
    pdftitle={Part \npart\ - \ncourse},
  pdfkeywords={Cambridge Mathematics Maths Math \npart\ \nterm\ \nyear\ \ncourse}]{hyperref}
  \title{Part \npart\ - \ncourse}
\else
  \usepackage[pdftex,
    hidelinks,
    pdfauthor={Dexter Chua},
    pdfsubject={Cambridge Maths Notes: Part \npart\ - \ncourse\ (\nextra)},
    pdftitle={Part \npart\ - \ncourse\ (\nextra)},
  pdfkeywords={Cambridge Mathematics Maths Math \npart\ \nterm\ \nyear\ \ncourse\ \nextra}]{hyperref}

  \title{Part \npart\ - \ncourse \\ {\Large \nextra}}
\fi

\author{Lectured by \nlecturer \\\small Notes taken by Dexter Chua}
\date{\nterm\ \nyear}

\usepackage{alltt}
\usepackage{amsfonts}
\usepackage{amsmath}
\usepackage{amssymb}
\usepackage{amsthm}
\usepackage{booktabs}
\usepackage{caption}
\usepackage{enumitem}
\usepackage{fancyhdr}
\usepackage{graphicx}
\usepackage{mathtools}
\usepackage{microtype}
\usepackage{multirow}
\usepackage{pdflscape}
\usepackage{pgfplots}
\usepackage{siunitx}
\usepackage{tabularx}
\usepackage{tikz}
\usepackage{tkz-euclide}
\usepackage[normalem]{ulem}
\usepackage[all]{xy}

\pgfplotsset{compat=1.12}

\pagestyle{fancyplain}
\lhead{\emph{\nouppercase{\leftmark}}}
\ifx \nextra \undefined
  \rhead{
    \ifnum\thepage=1
    \else
      \npart\ \ncourse
    \fi}
\else
  \rhead{
    \ifnum\thepage=1
    \else
      \npart\ \ncourse\ (\nextra)
    \fi}
\fi
\usetikzlibrary{arrows}
\usetikzlibrary{decorations.markings}
\usetikzlibrary{decorations.pathmorphing}
\usetikzlibrary{positioning}
\usetikzlibrary{fadings}
\usetikzlibrary{intersections}
\usetikzlibrary{cd}

\newcommand*{\Cdot}{\raisebox{-0.25ex}{\scalebox{1.5}{$\cdot$}}}
\newcommand {\pd}[2][ ]{
  \ifx #1 { }
    \frac{\partial}{\partial #2}
  \else
    \frac{\partial^{#1}}{\partial #2^{#1}}
  \fi
}

% Theorems
\theoremstyle{definition}
\newtheorem*{aim}{Aim}
\newtheorem*{axiom}{Axiom}
\newtheorem*{claim}{Claim}
\newtheorem*{cor}{Corollary}
\newtheorem*{defi}{Definition}
\newtheorem*{eg}{Example}
\newtheorem*{fact}{Fact}
\newtheorem*{law}{Law}
\newtheorem*{lemma}{Lemma}
\newtheorem*{notation}{Notation}
\newtheorem*{prop}{Proposition}
\newtheorem*{thm}{Theorem}

\renewcommand{\labelitemi}{--}
\renewcommand{\labelitemii}{$\circ$}
\renewcommand{\labelenumi}{(\roman{*})}

\let\stdsection\section
\renewcommand\section{\newpage\stdsection}

% Strike through
\def\st{\bgroup \ULdepth=-.55ex \ULset}

% Maths symbols
\newcommand{\bra}{\langle}
\newcommand{\ket}{\rangle}

\newcommand{\N}{\mathbb{N}}
\newcommand{\Z}{\mathbb{Z}}
\newcommand{\Q}{\mathbb{Q}}
\renewcommand{\H}{\mathbb{H}}
\newcommand{\R}{\mathbb{R}}
\newcommand{\C}{\mathbb{C}}
\newcommand{\Prob}{\mathbb{P}}
\renewcommand{\P}{\mathbb{P}}
\newcommand{\E}{\mathbb{E}}
\newcommand{\F}{\mathbb{F}}
\newcommand{\cU}{\mathcal{U}}
\newcommand{\RP}{\mathbb{RP}}
\newcommand{\CP}{\mathbb{CP}}

\newcommand{\ph}{\,\cdot\,}

\DeclareMathOperator{\sech}{sech}
\DeclareMathOperator{\cosech}{cosech}
\DeclareMathOperator{\cosec}{cosec}

\DeclareMathOperator{\covol}{covol}
\DeclareMathOperator{\vol}{vol}

\let\Im\relax
\let\Re\relax
\DeclareMathOperator{\Im}{Im}
\DeclareMathOperator{\Re}{Re}
\DeclareMathOperator{\im}{im}
\DeclareMathOperator{\image}{image}
\DeclareMathOperator{\Ann}{Ann}

\DeclareMathOperator*{\res}{res}
\DeclareMathOperator{\Res}{Res}
\DeclareMathOperator{\Ind}{Ind}

\DeclareMathOperator{\tr}{tr}
\DeclareMathOperator{\diag}{diag}
\DeclareMathOperator{\rank}{rank}
\DeclareMathOperator{\card}{card}
\DeclareMathOperator{\spn}{span}
\DeclareMathOperator{\adj}{adj}

\DeclareMathOperator{\erf}{erf}
\DeclareMathOperator{\erfc}{erfc}

\DeclareMathOperator{\ord}{ord}
\DeclareMathOperator{\Sym}{Sym}

\DeclareMathOperator{\sgn}{sgn}
\DeclareMathOperator{\orb}{orb}
\DeclareMathOperator{\stab}{stab}
\DeclareMathOperator{\ccl}{ccl}

\DeclareMathOperator{\lcm}{lcm}
\DeclareMathOperator{\hcf}{hcf}

\DeclareMathOperator{\Int}{Int}
\DeclareMathOperator{\id}{id}

\DeclareMathOperator{\betaD}{beta}
\DeclareMathOperator{\gammaD}{gamma}
\DeclareMathOperator{\Poisson}{Poisson}
\DeclareMathOperator{\binomial}{binomial}
\DeclareMathOperator{\multinomial}{multinomial}
\DeclareMathOperator{\Bernoulli}{Bernoulli}
\DeclareMathOperator{\like}{like}

\DeclareMathOperator{\var}{var}
\DeclareMathOperator{\cov}{cov}
\DeclareMathOperator{\bias}{bias}
\DeclareMathOperator{\mse}{mse}
\DeclareMathOperator{\corr}{corr}

\DeclareMathOperator{\otp}{otp}
\DeclareMathOperator{\dom}{dom}

\DeclareMathOperator{\Root}{Root}
\DeclareMathOperator{\supp}{supp}
\DeclareMathOperator{\rel}{rel}
\DeclareMathOperator{\Hom}{Hom}
\DeclareMathOperator{\Aut}{Aut}
\DeclareMathOperator{\Gal}{Gal}
\DeclareMathOperator{\Mat}{Mat}
\DeclareMathOperator{\End}{End}
\DeclareMathOperator{\Char}{char}
\DeclareMathOperator{\ev}{ev}
\DeclareMathOperator{\St}{St}
\DeclareMathOperator{\Lk}{Lk}
\DeclareMathOperator{\disc}{disc}
\DeclareMathOperator{\Isom}{Isom}
\DeclareMathOperator{\length}{length}
\DeclareMathOperator{\energy}{energy}
\DeclareMathOperator{\area}{area}
\DeclareMathOperator{\Syl}{Syl}
\DeclareMathOperator{\cl}{cl}
\DeclareMathOperator{\fix}{fix}

\newcommand{\GL}{\mathrm{GL}}
\newcommand{\SL}{\mathrm{SL}}
\newcommand{\PGL}{\mathrm{PGL}}
\newcommand{\PSL}{\mathrm{PSL}}
\newcommand{\PSU}{\mathrm{PSU}}
\newcommand{\Or}{\mathrm{O}}
\newcommand{\SO}{\mathrm{SO}}
\newcommand{\U}{\mathrm{U}}
\newcommand{\SU}{\mathrm{SU}}

\renewcommand{\d}{\mathrm{d}}
\newcommand{\D}{\mathrm{D}}

\tikzset{->/.style = {decoration={markings,
                                  mark=at position 1 with {\arrow[scale=2]{latex'}}},
                      postaction={decorate}}}
\tikzset{<-/.style = {decoration={markings,
                                  mark=at position 0 with {\arrowreversed[scale=2]{latex'}}},
                      postaction={decorate}}}
\tikzset{<->/.style = {decoration={markings,
                                   mark=at position 0 with {\arrowreversed[scale=2]{latex'}},
                                   mark=at position 1 with {\arrow[scale=2]{latex'}}},
                       postaction={decorate}}}
\tikzset{->-/.style = {decoration={markings,
                                   mark=at position #1 with {\arrow[scale=2]{latex'}}},
                       postaction={decorate}}}
\tikzset{-<-/.style = {decoration={markings,
                                   mark=at position #1 with {\arrowreversed[scale=2]{latex'}}},
                       postaction={decorate}}}

\tikzset{circ/.style = {fill, circle, inner sep = 0, minimum size = 3}}
\tikzset{mstate/.style={circle, draw, blue, text=black, minimum width=0.7cm}}

\definecolor{mblue}{rgb}{0.2, 0.3, 0.8}
\definecolor{morange}{rgb}{1, 0.5, 0}
\definecolor{mgreen}{rgb}{0.1, 0.4, 0.2}
\definecolor{mred}{rgb}{0.5, 0, 0}

\def\drawcirculararc(#1,#2)(#3,#4)(#5,#6){%
    \pgfmathsetmacro\cA{(#1*#1+#2*#2-#3*#3-#4*#4)/2}%
    \pgfmathsetmacro\cB{(#1*#1+#2*#2-#5*#5-#6*#6)/2}%
    \pgfmathsetmacro\cy{(\cB*(#1-#3)-\cA*(#1-#5))/%
                        ((#2-#6)*(#1-#3)-(#2-#4)*(#1-#5))}%
    \pgfmathsetmacro\cx{(\cA-\cy*(#2-#4))/(#1-#3)}%
    \pgfmathsetmacro\cr{sqrt((#1-\cx)*(#1-\cx)+(#2-\cy)*(#2-\cy))}%
    \pgfmathsetmacro\cA{atan2(#2-\cy,#1-\cx)}%
    \pgfmathsetmacro\cB{atan2(#6-\cy,#5-\cx)}%
    \pgfmathparse{\cB<\cA}%
    \ifnum\pgfmathresult=1
        \pgfmathsetmacro\cB{\cB+360}%
    \fi
    \draw (#1,#2) arc (\cA:\cB:\cr);%
}
\newcommand\getCoord[3]{\newdimen{#1}\newdimen{#2}\pgfextractx{#1}{\pgfpointanchor{#3}{center}}\pgfextracty{#2}{\pgfpointanchor{#3}{center}}}

\def\Xint#1{\mathchoice
   {\XXint\displaystyle\textstyle{#1}}%
   {\XXint\textstyle\scriptstyle{#1}}%
   {\XXint\scriptstyle\scriptscriptstyle{#1}}%
   {\XXint\scriptscriptstyle\scriptscriptstyle{#1}}%
   \!\int}
\def\XXint#1#2#3{{\setbox0=\hbox{$#1{#2#3}{\int}$}
     \vcenter{\hbox{$#2#3$}}\kern-.5\wd0}}
\def\ddashint{\Xint=}
\def\dashint{\Xint-}


\begin{document}
\maketitle
{\small
\noindent\emph{Analysis II is essential}
\vspace{10pt}

\noindent Measure spaces, $\sigma$-algebras, $\pi$-systems and uniqueness of extension, statement *and proof* of Carath\'eodory's extension theorem. Construction of Lebesgue measure on $\R$. The Borel $\sigma$-algebra of $\R$. Existence of non-measurable subsets of $\R$. Lebesgue-Stieltjes measures and probability distribution functions. Independence of events, independence of $\sigma$-algebras. The Borel--Cantelli lemmas. Kolmogorov's zero-one law.\hspace*{\fill}[6]

\vspace{5pt}
\noindent Measurable functions, random variables, independence of random variables. Construction of the integral, expectation. Convergence in measure and convergence almost everywhere. Fatou's lemma, monotone and dominated convergence, differentiation under the integral sign. Discussion of product measure and statement of Fubini's theorem.\hspace*{\fill}[6]

\vspace{5pt}
\noindent Chebyshev's inequality, tail estimates. Jensen's inequality. Completeness of $L^p$ for $1 \leq p \leq \infty$. The H\"older and Minkowski inequalities, uniform integrability.\hspace*{\fill}[4]

\vspace{5pt}
\noindent $L^2$ as a Hilbert space. Orthogonal projection, relation with elementary conditional probability. Variance and covariance. Gaussian random variables, the multivariate normal distribution.\hspace*{\fill}[2]

\vspace{5pt}
\noindent The strong law of large numbers, proof for independent random variables with bounded fourth moments. Measure preserving transformations, Bernoulli shifts. Statements *and proofs* of maximal ergodic theorem and Birkhoff's almost everywhere ergodic theorem, proof of the strong law.\hspace*{\fill}[4]

\vspace{5pt}
\noindent The Fourier transform of a finite measure, characteristic functions, uniqueness and inversion. Weak convergence, statement of L\'vy's convergence theorem for characteristic functions. The central limit theorem.\hspace*{\fill}[2]%
}

\tableofcontents
\setcounter{section}{-1}
\section{Introduction}
Recall that if $f: [0, 1] \to \R$ is continuous, then the Riemann integral of $f$ is defined as follows:
\begin{enumerate}
  \item Take a partition $0 = t_0 < t_1 < \cdots < t_n = 1$ of $[0, 1]$.
  \item Consider the Riemann sum
    \[
      \sum_{j = 1}^n f(t_j) (t_j - t_{j - 1})
    \]
  \item The Riemann integral is
    \[
      \int f = \text{Limit of Riemann sums as the mesh size of the partition }\to 0.
    \]
\end{enumerate}
% insert diagram

In this course, we are going to learn a different way of calculating integrals. The idea is very simple, but it is going to be very powerful mathematically.

The idea of measure theory is to use a different approximation scheme. Instead of partitioning the domain, we partition the range of the function. We fix some numbers $r_0 < r_1 < r_2 < \cdots < r_n$.

% insert analogous diagram

We then approximate the integral of $f$ by
\[
  \sum_{j = 1}^n r_j \cdot (\text{``size of }f^{-1}([r_{j - 1}, r_j])\text{''}).
\]
We then define the integral as the limit of approximations of this type as the mesh size of the partition $\to 0$.

We can make an analogy with bankers --- If a Riemann banker is given a stack of money, they would just add the values of the money in order. A measure-theoretic banker will sort the bank notes according to the type, and then find the total value by multiplying the number of each type by the value, and adding up.

Why would we want to do so? It turns out this leads to a much more general theory of interaction on a much more general spaces. In the context of $\R$, this theory of integration is much much more powerful than the Riemann sum, and can integrate a much wider class of functions. Theorems for interchanging limits and integration also become much stronger.

\section{Measures}
The starting point of all these is to come up with a function that determines the ``size'' of a given set, known as a \emph{measure}. It turns out we cannot sensibly define a size for \emph{all} subsets of $[0, 1]$. Thus, we need to restrict our attention to a collection of ``nice'' subsets.

\begin{defi}[$\sigma$-algebra]\index{$\sigma$-algebra}\index{sigma-algebra}
  Let $E$ be a set. A \emph{$\sigma$-algebra} $\mathcal{E}$ on $E$ is a collection of subsets of $E$ such that
  \begin{enumerate}
    \item $\emptyset \in \mathcal{E}$.
    \item $A \in \mathcal{E}$ implies that $A^C = X \setminus A \in \mathcal{E}$.
    \item For any sequence $(A_n)$ in $\mathcal{E}$, we have that
      \[
        \bigcup_n A_n \in \mathcal{E}.
      \]
  \end{enumerate}
  The pair $(E, \mathcal{E})$ is called a \emph{measurable space}.\index{measurable space}
\end{defi}
Note that the axioms imply that $\sigma$-algebras are also closed under intersections.

\begin{defi}[Measure]\index{measure}
  A \emph{measure} on a measurable space $(E, \mathcal{E})$ is a function $\mu: \mathcal{E} \to [0, \infty]$ such that
  \begin{enumerate}
    \item $\mu(\emptyset) = 0$
    \item Countable additivity: For any disjoint sequence $(A_n)$ in $\mathcal{E}$, then
      \[
        \mu\left(\bigcup_n A_n\right) = \sum_{n = 1}^\infty \mu(A_n).
      \]
  \end{enumerate}
\end{defi}

\begin{eg}
  Let $E$ be any countable set, and $\mathcal{E} = P(E)$ be the set of all subsets of $E$. A \emph{mass function}\index{mass function} is any function $m: E \to [0, \infty]$. We can then define a measure by setting
  \[
    \mu(A) = \sum_{x \in A} m(x).
  \]
  In particular, if we put $m(x) = 1$ for all $x \in E$, then we obtain the \emph{counting measure}\index{counting measure}.
\end{eg}

Countable spaces are nice, because we can always take $\mathcal{E} = P(E)$, and the measure can be defined on all possible subsets. However, for ``bigger'' spaces, we have to be more careful. The set of all subsets is often ``too large''. We will see a concrete and also important example of this later.

In general, $\sigma$-algebras are often described on large spaces in terms of a smaller set, known as the \emph{generating sets}\index{generating set}.
\begin{defi}[Generator of $\sigma$-algebra]\index{generator of $\sigma$-algebra}
  Let $E$ be a set, and that $\mathcal{A} \subseteq P(E)$ be a collection of subsets of $E$. We define
  \[
    \sigma(\mathcal{A}) = \{A \subseteq E: A \in \mathcal{E}\text{ for all $\sigma$-algebras $\mathcal{E}$ that contain $\mathcal{A}$}\}..
  \]
  In other words $\sigma(\mathcal{A})$ is the smallest sigma algebra that contains $\mathcal{A}$. This is known as the sigma algebra \emph{generated by} $\mathcal{A}$.
\end{defi}

\begin{eg}
  Take $E = \Z$, and $\mathcal{A} = \{\{x\}: x \in \Z\}$. Then $\sigma(\mathcal{A})$ is just $P(E)$, since every subset of $E$ can be written as a countable union of singletons.
\end{eg}

\begin{eg}
  Take $E = \Z$, and let $\mathcal{A} = \{ \{x, x + 1, x + 2, x + 3, \cdots\}: x \in E\}$. Then again $\sigma(E)$ is the set of all subsets of $E$.
\end{eg}

The following is the most important $\sigma$-algebra in the course:
\begin{defi}[Borel $\sigma$-algebra]\index{Borel $\sigma$-algebra}
  Let $E = \R$, and $\mathcal{A} = \{U \subseteq \R: U \text{ is open}\}$. Then $\sigma(\mathcal{A})$ is known as the \emph{Borel $\sigma$-algebra}, which is \emph{not} the set of all subsets of $\R$.

  We can equivalently define this by $\tilde{\mathcal{A}} = \{(a, b): a < b, a, b \in \Q\}$. Then $\sigma(\tilde{\mathcal{A}})$ is also the Borel $\sigma$-algebra.
\end{defi}

When proving things about $\sigma$-algebras, we typically check a given property on a generating set. Often, it us useful to focus on special generating sets which are called $\pi$-systems.

\begin{defi}[$\pi$-system]\index{pi-system}\index{$\pi$-system}
  Let $\mathcal{A}$ be a collection of subsets of $E$. Then $\mathcal{A}$ is called a \emph{$\pi$-system} if
  \begin{enumerate}
    \item $\emptyset \in A$
    \item If $A, B \in \mathcal{A}$, then $A \cap B \in A$.
  \end{enumerate}
\end{defi}

\begin{defi}[d-system]\index{d-system}
  Let $\mathcal{A}$ be a collection of subsets of $E$. Then $\mathcal{A}$ is called a \emph{d-system} if
  \begin{enumerate}
    \item $E \in \mathcal{A}$
    \item If $A, B \in \mathcal{A}$ and $A \subseteq B$, then $B \setminus A \in \mathcal{A}$
    \item For all increasing sequences $(A_n)$ in $\mathcal{A}$, we have that $\bigcup_n A_n \in \mathcal{A}$.
  \end{enumerate}
\end{defi}
The point of d-systems and $\pi$-systems is that they separate the axioms of a $\sigma$-algebra into two parts. More precisely, we have
\begin{prop}
  A collection $\mathcal{A}$ is a $\sigma$-algebra if and only if it is both a $\pi$-system and a $d$-system.
\end{prop}
This follows rather straightforwardly from the definitions.

\begin{lemma}[Dynkin's $\pi$-system lemma]\index{Dynkin's $\pi$-system lemma}
  Let $\mathcal{A}$ be a $\pi$-system. Then any d-system which contains $\mathcal{A}$ contains $\sigma(A)$.
\end{lemma}
While this seems like a simple statement with an easy proof, this is very useful and will be used a lot.

\begin{proof}
  Let $\mathcal{D}$ be the intersection of all d-systems containing $\mathcal{A}$, ie. the smallest d-system containing $\mathcal{A}$. We show that $\mathcal{D}$ contains $\sigma(\mathcal{A})$. To do so, we will show that $\mathcal{D}$ is a $\pi$-system, hence a $\sigma$-algebra.

  We let
  \[
    \mathcal{D}' = \{ B \in \mathcal{D}: B \cap A \in \mathcal{D}\text{ for all }A \in \mathcal{A}\}.
  \]
  We note that $\mathcal{D}' \supseteq \mathcal{A}$ because $\mathcal{A}$ is a $\pi$-system, and is hence closed under intersections. We check that $\mathcal{D}'$ is a d-system. It is clear that $E \in \mathcal{D}'$. If we have $B_1, B_2 \in \mathcal{D}'$, where $B_1 \subseteq B_2$, then for any $A \in \mathcal{A}$, we have
  \[
    (B_2 \setminus B_1) \cap A = (B_2 \cap A) \setminus (B_1 \cap A).
  \]
  By definition of $\mathcal{D}'$, we know $B_2 \cap A$ and $B_1 \cap A$ are elements of $\mathcal{D}$. Since $\mathcal{D}$ is a d-system, we know this intersection is in $\mathcal{D}$. So $B_2 \setminus B_1 \in \mathcal{D}'$.

  Finally, suppose that $(B_n)$ is an increasing sequence in $\mathcal{D}'$, with $B = \bigcup B_n$. Then for every $A \in \mathcal{A}$, we have that
  \[
    \left(\bigcup B_n\right) \cap A = \bigcup (B_n \cap A) = B \cap A \in \mathcal{D}.
  \]
  Therefore $B \in \mathcal{D}'$.

  Therefore $\mathcal{D}'$ is a d-system contained in $\mathcal{D}$, which also contains $\mathcal{A}$. By our choice of $\mathcal{D}$, we know $\mathcal{D}' = \mathcal{D}$.

  We now let
  \[
    \mathcal{D}'' = \{B \in \mathcal{D}: B \cap A \in \mathcal{D}\text{ for all }A \in \mathcal{D}\}.
  \]
  Since $\mathcal{D}' = \mathcal{D}$, we again have $\mathcal{A} \subseteq \mathcal{D}''$, and the same argument as above implies that $\mathcal{D}''$ is a d-system which is between $\mathcal{A}$ and $\mathcal{D}$. But the only way that can happen is if $\mathcal{D}'' = \mathcal{D}$, and this implies that $\mathcal{D}$ is a $\pi$-system.
\end{proof}

Now let's starting constructing an actual, interesting measure. To do so, we will prove a useful theorem that says to specify a measure, it will suffice to specify on a special generating subset of the $\sigma$-algebra.

\begin{defi}[Set function]
  Let $\mathcal{A}$ be a collection of subsets of $E$ with $\emptyset \in \mathcal{A}$. A \term{set function} function $\mu: \mathcal{A} \to [0, \infty]$ such that $\mu(\emptyset) = 0$.
\end{defi}

\begin{defi}[Increasing set function]\index{Increasing set function}\index{set function, increasing}
  A set function is \emph{increasing} if it has the proper that for all $A, B \in \mathcal{A}$ with $A \subseteq B$, we have $\mu(A) \leq \mu(B)$.
\end{defi}
\begin{defi}[Additive set function]\index{Additive set function}\index{set function, additive}
  A set function is \emph{additive} if whenever $A, B \in \mathcal{A}$ and $A \cup B \in \mathcal{A}$, $A \cup B = \emptyset$, then $\mu(A \cup B) = \mu(A) + \mu(B)$.
\end{defi}

\begin{defi}[Countably additive set function]\index{countably additive set function}\index{set function, countably additive}
  A set function is \emph{countably additive} if whenever $A_n$ is a seqquence of disjoint sets in $\mathcal{A}$ with $\cup A_n \in \mathcal{A}$, then
  \[
    \mu\left(\bigcup_n A_n \right) = \sum_n \mu(A_n).
  \]
\end{defi}

Under these definitions, a measure is just a countable additive set function defined on a $\sigma$-algebra.

\begin{defi}[Countably subadditive set function]\index{countably subadditive set function}\index{set function, countably additive}
  A set function is \emph{countably subadditive} if whenever $(A_n)$ is a sequence of sets in $\mathcal{A}$ with $\bigcup_n A_n \in \mathcal{A}$, then
  \[
    \mu\left(\bigcup_n A_n\right) \leq \sum_n \mu(A_n).
  \]
\end{defi}

\begin{defi}[Ring]\index{ring}
  A collection of subsets $\mathcal{A}$ is a \emph{ring} on $E$ if $\emptyset \in A$ and for all $A, B \in \mathcal{A}$, we have $B \setminus A \in \mathcal{A}$ and $A \cup B \in \mathcal{A}$.
\end{defi}

\begin{defi}[Algebra]\index{algebra}
  A collection of subsets $\mathcal{A}$ is an \emph{algebra} on $E$ if $\emptyset \in A$, and for all $A, B \in \mathcal{A}$, we have $A^C \in \mathcal{A}$ and $A \cup B \in \mathcal{A}$.
\end{defi}
So an algebra is like a $\sigma$-algebra, but it is just closed under finite unions only, rather than countable unions.

The big theorem that allows us to construct measures is the Caratheodory extension theorem.
\begin{thm}[Caratheodory extension theorem]\index{Caratheodory extension theorem}
  Let $\mathcal{A}$ be a ring on $E$, and $\mu$ a countably additive set function on $\mathcal{A}$. Then $\mu$ extends to a measure on the $\sigma$-algebra generated by $\mathcal{A}$.
\end{thm}

\begin{proof}(non-examinable)
  We start by defining what we want our measure to be. For $B \subseteq E$, we set
  \[
    \mu^*(B) = \inf\left\{\sum_n\mu(A_n): (A_n) \in \mathcal{A}\text{ and } B\subseteq \bigcup A_n\right\}.
  \]
  If it happens that there is no such sequence, we set this to be $\infty$. This measure is known as the \term{outer measure}. It is clear that $\mu^*(\phi) = 0$, and that $\mu^*$ is increasing.

  We say a set $A \subseteq E$ is $\mu^*$-measurable if
  \[
    \mu^*(B) = \mu^*(B \cap A) + \mu^*(A^C)
  \]
  for all $B \subseteq E$. We let
  \[
    \mathcal{M} = \{\text{$\mu^*$-measurable sets}\}.
  \]
  We will show the following:
  \begin{enumerate}
    \item $\mathcal{M}$ is a $\sigma$-algebra containing $\mathcal{A}$.
    \item $\mu^*$ is a measure on $\mathcal{M}$ with $\mu^*|_{\mathcal{A}} = \mu$.
  \end{enumerate}
  Note that it is not true in general that $\mathcal{M} = \sigma(A)$. However, we will always have $M \supseteq \sigma(\mathcal{A})$.

  We are going to break this up into five nice bitesize chunks.

  \begin{claim}
    $\mu^*$ is countably subadditive.
  \end{claim}
  Suppose $B\subseteq \bigcup_n B_n$. We need to show that $\mu^*(B) \leq \sum_n \mu^*(B_n)$. We can wlog assume that $\mu^*(B_n)$ is finite for all $n$, or else the inequality is trivial. Let $\varepsilon > 0$. Then by definition of the outer measure, for each $n$, we can find a sequence $(B_{n, m})_{m = 1}^\infty$ in $\mathcal{A}$ with the property that
  \[
    B_n \subseteq \bigcup_m B_{n, m}
  \]
  and
  \[
    \mu^*(B_n) + \frac{\varepsilon}{2^n} \geq \sum_m \mu(B_{n, m}).
  \]
  Then we have
  \[
    B \subseteq \bigcup_n b_n \subseteq \bigcup_{n, m}B_{n, m}.
  \]
  Thus, by definition, we have
  \[
    \mu^*(B) \leq \sum_{n, m}\mu^*(B_{n, m}) \leq \sum_n \left(\mu^*(B_n) + \frac{\varepsilon}{2^n}\right) = \varepsilon + \sum_n \mu^*(B_n).
  \]
  Since $\varepsilon$ was arbitrary, we are done.
  \begin{claim}
    $\mu^*$ agrees with $\mu$ on $\mathcal{A}$.
  \end{claim}
  In the first example sheet, we will show that if $\mathcal{A}$ is a ring and $\mu$ is a countably addiive set function on $\mu$, then $\mu$ is in fact countably subadditive and increasing.

  Assuming this, suppose that $A, (A_n)$ are in $\mathcal{A}$ and $A \subseteq \bigcup_n A_n$. Then by subadditivity, we have
  \[
    \mu(A) \leq \sum_n \mu(A \cap A_n) \leq \sum_n \mu(A_n),
  \]
  using that $\mu$ is countably subadditivity and increasing. Note that we have to do this in two steps, rather than just applying countable subadditivity, since we did not assume that $\bigcup_n A_n \in \mathcal{A}$. Taking the infimum over all sequences, we have
  \[
    \mu(A) \leq \mu^*(A).
  \]
  Also, we see by definition that $\mu(A) \geq \mu^*(A)$, since $A$ covers $A$. So we get that $\mu(A) = \mu^*(A)$ for all $A \in \mathcal{A}$.

  \begin{claim}
    $\mathcal{M}$ contains $\mathcal{A}$.
  \end{claim}
  Suppose that $A \in \mathcal{A}$ and $B \subseteq E$. We need to show that
  \[
    \mu^*(B) = \mu^*(B \cap A) + \mu^*(B \cap A^C).
  \]
  Since $\mu^*$ is countably subadditive, we immediately have $\mu^*(B) \leq \mu^*(B \cap A) + \mu^*(B \cap A^C)$. For the other inequality, we first observe that it is trivial if $\mu^*(B)$ is infinite. If it is finite, then by definition, given $\varepsilon > 0$, we can find some $(B_n)$ in $\mathcal{A}$ such that $B \subseteq \bigcup_n B_n$ and
  \[
    \mu^*(B) + \varepsilon \geq \sum_n \mu(B_n).
  \]
  Then we have
  \begin{align*}
    B \cap A &\subseteq \bigcup_n (B_n \cap A)\\
    B \cap A^C &\subseteq \bigcup_n (B_n \cap A^C)
  \end{align*}
  We notice that $B_n \cap A^C = B_n \setminus A \in \mathcal{A}$. Thus, by definition of $\mu^*$, we have
  \begin{align*}
    \mu^* (B \cap A) + \mu^*(B \cap A^c) &\leq \sum_n \mu(B_n \cap A) + \sum_n \mu(B_n \cap A^C)\\
    &= \sum_n (\mu(B_n \cap A) + \mu(B_n \cap A^C))\\
    &= \sum_n \mu(B_n) &\\
    &\leq \mu^*(B_n) + \varepsilon.
  \end{align*}
  Since $\varepsilon$ was arbitrary, the result follows.

  \begin{claim}
    We show that $\mathcal{M}$ is an algebra.
  \end{claim}
  We first show that $E \in \mathcal{M}$. This is true since iwe obviously have
  \[
    \mu^*(B) = \mu^*(B \cap E) + \mu^*(B \cap E^C)
  \]
  for all $B \subseteq E$.

  Next, note that if $A \in \mathcal{M}$, then by definition we have
  \[
    \mu^*(B) = \mu^*(B \cap A) + \mu^*(B \cap A^C).
  \]
  Now note that this definition is symmetric in $A$ and $A^C$. So we also have $A^C \in M$.

  Finally, we have to show that $\mathcal{M}$ is closed under intersection (which is equivalent to being closed under union when we have complements). Suppose $A_1, A_2 \in \mathcal{M}$ and $B \subseteq E$. Then we have
  \begin{align*}
    \mu^*(B) ={}& \mu^*(B \cap A_1) + \mu^*(B \cap A_1^C)\\
    ={}& \mu^*(B \cap A_1 \cap A_2) + \mu^*(B \cap A_1 \cap A_2^C) + \mu^*(B \cap A_1^C)\\
    ={}& \mu^*(B \cap (A_1 \cap A_2)) + \mu^*(B \cap (A_1\cap A_2)^C \cap A_1) \\
    &+ \mu^*(B \cap (A_1 \cap A_2)^C \cap A_1^C)\\
    ={}& \mu^*(B \cap (A_1 \cap A_2)) + \mu^*(B \cap (A_1 \cap A_2)^C).
  \end{align*}
  So we have $A_1 \cap A_2 \in \mathcal{M}$. So $\mathcal{M}$ is an algebra.
  \begin{claim}
    $\mathcal{M}$ is a $\sigma$-algebra, and $\mu^*$ is a measure on $\mathcal{M}$.
  \end{claim}
  To show that $\mathcal{M}$ is a $\sigma$-algebra, we need to show that it is closed under countable unions. We let $(A_n)$ be a disjoint collection of sets in $\mathcal{M}$, then we want to show that $A = \bigcup_n A_n \in \mathcal{M}$ and $\mu^*(A) = \sum_n \mu^*(A_n)$.

  Suppose that $B \subseteq E$. Then we have
  \begin{align*}
    \mu^*(B) &= \mu^*(B \cap A_1) + \mu^*(B \cap A_1^C)\\
    \intertext{Using the fact that $A_2 \in \mathcal{M}$ and $A_1 \cap A_2  =\emptyset$, we have}
    &= \mu^*(B \cap A_1) + \mu^*(B \cap A_2) + \mu^*(B \cap A_1^C \cap A_2^C)\\
    &= \cdots\\
    &= \sum_{i = 1}^n \mu^*(B\cap A_i) + \mu^*(B \cap A_1^C \cap \cdots \cap A_n^C)\\
    &\geq \sum_{i = 1}^n \mu^*(B \cap A_i) + \mu^*(B \cap A^C).
  \end{align*}
  Taking the limit as $n \to \infty$, we have
  \[
    \mu^*(B) \geq \sum_{i = 1}^\infty \mu^*(B \cap A_i) + \mu^*(B \cap A^C).
  \]
  By the countable-subadditivity of $\mu^*$, we have
  \[
    \mu^*(B \cap A) \leq \sum_{i = 1}^\infty \mu^*(B \cap A_i).
  \]
  Thus we obtain
  \[
    \mu^*(B) \geq \mu^*(B \cap A) + \mu^*(B \cap A^C).
  \]
  By countable subadditivity, we also have inequality in the other direction.  So equality holds. So $A \in \mathcal{M}$. So $\mathcal{M}$ is a $\sigma$-algebra.

  To see that $\mu^*$ is a measure on $\mathcal{M}$, noet that the above implies that
  \[
    \mu^*(B) = \sum_{i = 1}^\infty (B \cap A_i) + \mu^*(B \cap A^C).
  \]
  Taking $B = A$, this gives
  \[
    \mu^*(A) = \sum_{i = 1}^\infty (A \cap A_i) + \mu^*(A \cap A^C) = \sum_{i = 1}^\infty \mu^*(A_i).
  \]
\end{proof}
\subsection{Uniqueness of measure}
\begin{thm}
  Suppose that $\mu_1, \mu_2$ are measures on $(E, \mathcal{E})$ with $\mu_1(E) = \mu_2(E) + \infty$. If $\mathcal{A}$ is a $\pi$-system with $\sigma(A) = \mathcal{E}$, and $\mu_1$ agrees with $\mu_2$ on $\mathcal{A}$, then $\mu_1 = \mu_2$.
\end{thm}

\begin{proof}
  Let
  \[
    \mathcal{D} = \{A \in \mathcal{E}: \mu_1(A) = \mu_2(A)\}
  \]
  We know that $\mathcal{D} \supseteq \mathcal{A}$. By Dynkin's lemma, it suffices to show that $\mathcal{D}$ is a d-system. The things to check are:
\begin{enumerate}
  \item $E \in \mathcal{D}$ --- this is obvious.
  \item If $A, B \in \mathcal{D}$ with $A \subseteq B$, then $B \setminus A \in \mathcal{D}$. Indeed, we have the equations
    \begin{align*}
      \mu_1(B) &= \mu_1(A) + \mu_1(B \setminus A) < \infty\\
      \mu_2(B) &= \mu_2(A) + \mu_2(B \setminus A) < \infty.
    \end{align*}
    Since $\mu_1(B) = \mu_2(B)$ and $\mu_1(A) = \mu_2(A)$, we must have $\mu_1(B \setminus A) = \mu_2(B \setminus A)$.
  \item Let $(A_n) \in \mathcal{D}$ be an increasing sequence with $\bigcup A_n = A$. Then
    \[
      \mu_1(A) = \lim_{n \to \infty}\mu_1(A_n) = \lim_{n \to \infty} \mu_2(A_n) = \mu_2(A).
    \]
    So $A \in\mathcal{D}$.
\end{enumerate}
\end{proof}
The assumption that $\mu_1(E) = \mu_2(E) < \infty$ is necessary. The theorem does not necessarily hold without it. We can see this from a simple counterexample:

\begin{eg}
  Let $E = \Z$, and let $\mathcal{E} = P(E)$. We let
  \[
    \mathcal{A} = \{\{x, x+1, x+2, \cdots\}: x \in E\} \cup \{\emptyset\}.
  \]
  This is a $\pi$-system with $\sigma(A) = \mathcal{E}$. We let $\mu_1(A)$ be the number of elements in $A$, and $\mu_2 = 2\mu_1(A)$. Then obviously $\mu_1 \not= \mu_2$, but $\mu_1(A) = \infty = \mu_2(A)$ for $A \in \mathcal{A}$.
\end{eg}

\begin{defi}[Borel $\sigma$-algebra]\index{Borel $\sigma$-algebra}\index{$\mathcal{B}(E)$}
  Let $E$ be a topological space. We define the \emph{Borel $\sigma$-algebra} as
  \[
    \mathcal{B}(E) = \sigma(\{U \subseteq \mathcal{E}: U \text{ is open}\}).
  \]
  We write \term{$\mathcal{B}$} for $\mathcal{B}(\R)$.
\end{defi}

\begin{defi}[Borel measure and Radon measure]\index{Borel measure}
  A measure $\mu$ on $(E, \mathcal{B}(E))$  is called a \emph{Borel measure}. If $\mu(K) < \infty$ for all $K \subseteq E$ compact, then $\mu$ is a \term{Radon measure}.
\end{defi}
The most important example of a Lebesgue measure we will consider is the \emph{Lebesgue measure}.

\begin{thm}
  There exists a unique Borel measure $\mu$ on $\R$ with $\mu([a, b]) = b - a$.
\end{thm}

\begin{proof}
  We first show uniqueness. Suppose $\tilde{\mu}$ is another measure on $\mathcal{B}$ satisfying the above property. We want to apply the previous uniqueness theorem, but our measure is not finite. So we need to carefully get around that problem.
  
  For each $n \in \Z$, we set
  \begin{align*}
    \mu_n(A) &= \mu(A \cap (n, n + 1]))\\
    \tilde{\mu}_n(A) &= \tilde{\mu}(A \cap (n, n + 1]))
  \end{align*}
  Then $\mu_n$ and $\tilde{\mu}_n$ are finite measures on $\mathcal{B}$ which agree on the $\pi$-system of intervals of the form $(a, b]$ with $a, b \in \R$, $a < b$. Therefore we have $\mu_n = \tilde{\mu}_n$ for all $n \in \Z$. Now we have
  \[
    \mu(A) = \sum_{n \in \Z} \mu(A \cap (n, n + 1]) = \sum_{n\in \Z}\mu_n(A) = \sum_{n \in \Z}\tilde{\mu}_n(A) = \tilde{\mu}(A)
  \]
  for all Borel sets $A$.

  To show existence, we want to use the Caratheodory extension theorem. We let $\mathcal{A}$ be the collection of finite, disjoint unions of the form
  \[
    A = (a_1, b_1] \cup (a_2, b_2] \cup \cdots \cup (a_n, b_n].
  \]
  Then $\mathcal{A}$ is a ring  of subsets of $R$, and $\sigma(A) = \mathcal{B}$ (details are to be checked on the first example sheet).

  We set
  \[
    \mu(A) = \sum_{i = 1}^n (b_i - a_i).
  \]
  We note that $\mu$ is well-defined, since if
  \[
    A = (a_1, b_1] \cup \cdots \cup (a_n, b_n] = (\tilde{a}_1, \tilde{b}_1] \cup \cdots \cup (\tilde{a}_n, \tilde{b}_n],
  \]
  then
  \[
    \sum_{i = 1}^n (b_i - a_i) = \sum_{i = 1}^n (\tilde{b}_i - \tilde{a}_i).
  \]
  Also, if $\mu$ is additive, $A, B \in \mathcal{A}$, $A \cap B = \emptyset$ and $A \cup B \in \mathcal{A}$, we obviously have $\mu(A \cup B) = \mu(A) + \mu(B)$. So $\mu$ is additive.

  Finally, we have to show that $\mu$ is in fact countably additive. Let $(A_n)$ be a disjoint sequence in $\mathcal{A}$, and let $\bigcup_{i = 1}^\infty A_n \in \mathcal{A}$. Then we need to show that $\mu(A) = \sum_{n = 1}^\infty \mu(A_n)$.

  Since $\mu$ is additive, we have
  \begin{align*}
    \mu(A) &= \mu(A_1) + \mu(A \setminus A_1) \\
    &= \mu(A_1) + \mu(A_2) + \mu(A \setminus A_1 \cup A_2)\\
    &= \sum_{i = 1}^n \mu(A_i) + \mu\left(A \setminus \bigcup_{i = 1}^n A_i\right)
  \end{align*}
  To finish the proof, we show that 
  \[
    \mu\left(A \setminus \bigcup_{i = 1}^n A_i\right) \to 0\text{ as }n \to \infty.
  \]
\end{proof}

\printindex
\end{document}
