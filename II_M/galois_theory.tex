\documentclass[a4paper]{article}

\def\npart {II}
\def\nterm {Michaelmas}
\def\nyear {2015}
\def\nlecturer {C. Birkar}
\def\ncourse {Galois Theory}
\def\nnotready {}

% Imports
\ifx \nextra \undefined
  \usepackage[pdftex,
    hidelinks,
    pdfauthor={Dexter Chua},
    pdfsubject={Cambridge Maths Notes: Part \npart\ - \ncourse},
    pdftitle={Part \npart\ - \ncourse},
  pdfkeywords={Cambridge Mathematics Maths Math \npart\ \nterm\ \nyear\ \ncourse}]{hyperref}
  \title{Part \npart\ - \ncourse}
\else
  \usepackage[pdftex,
    hidelinks,
    pdfauthor={Dexter Chua},
    pdfsubject={Cambridge Maths Notes: Part \npart\ - \ncourse\ (\nextra)},
    pdftitle={Part \npart\ - \ncourse\ (\nextra)},
  pdfkeywords={Cambridge Mathematics Maths Math \npart\ \nterm\ \nyear\ \ncourse\ \nextra}]{hyperref}

  \title{Part \npart\ - \ncourse \\ {\Large \nextra}}
\fi

\author{Lectured by \nlecturer \\\small Notes taken by Dexter Chua}
\date{\nterm\ \nyear}

\usepackage{alltt}
\usepackage{amsfonts}
\usepackage{amsmath}
\usepackage{amssymb}
\usepackage{amsthm}
\usepackage{booktabs}
\usepackage{caption}
\usepackage{enumitem}
\usepackage{fancyhdr}
\usepackage{graphicx}
\usepackage{mathtools}
\usepackage{microtype}
\usepackage{multirow}
\usepackage{pdflscape}
\usepackage{pgfplots}
\usepackage{siunitx}
\usepackage{tabularx}
\usepackage{tikz}
\usepackage{tkz-euclide}
\usepackage[normalem]{ulem}
\usepackage[all]{xy}

\pgfplotsset{compat=1.12}

\pagestyle{fancyplain}
\lhead{\emph{\nouppercase{\leftmark}}}
\ifx \nextra \undefined
  \rhead{
    \ifnum\thepage=1
    \else
      \npart\ \ncourse
    \fi}
\else
  \rhead{
    \ifnum\thepage=1
    \else
      \npart\ \ncourse\ (\nextra)
    \fi}
\fi
\usetikzlibrary{arrows}
\usetikzlibrary{decorations.markings}
\usetikzlibrary{decorations.pathmorphing}
\usetikzlibrary{positioning}
\usetikzlibrary{fadings}
\usetikzlibrary{intersections}
\usetikzlibrary{cd}

\newcommand*{\Cdot}{\raisebox{-0.25ex}{\scalebox{1.5}{$\cdot$}}}
\newcommand {\pd}[2][ ]{
  \ifx #1 { }
    \frac{\partial}{\partial #2}
  \else
    \frac{\partial^{#1}}{\partial #2^{#1}}
  \fi
}

% Theorems
\theoremstyle{definition}
\newtheorem*{aim}{Aim}
\newtheorem*{axiom}{Axiom}
\newtheorem*{claim}{Claim}
\newtheorem*{cor}{Corollary}
\newtheorem*{defi}{Definition}
\newtheorem*{eg}{Example}
\newtheorem*{fact}{Fact}
\newtheorem*{law}{Law}
\newtheorem*{lemma}{Lemma}
\newtheorem*{notation}{Notation}
\newtheorem*{prop}{Proposition}
\newtheorem*{thm}{Theorem}

\renewcommand{\labelitemi}{--}
\renewcommand{\labelitemii}{$\circ$}
\renewcommand{\labelenumi}{(\roman{*})}

\let\stdsection\section
\renewcommand\section{\newpage\stdsection}

% Strike through
\def\st{\bgroup \ULdepth=-.55ex \ULset}

% Maths symbols
\newcommand{\bra}{\langle}
\newcommand{\ket}{\rangle}

\newcommand{\N}{\mathbb{N}}
\newcommand{\Z}{\mathbb{Z}}
\newcommand{\Q}{\mathbb{Q}}
\renewcommand{\H}{\mathbb{H}}
\newcommand{\R}{\mathbb{R}}
\newcommand{\C}{\mathbb{C}}
\newcommand{\Prob}{\mathbb{P}}
\renewcommand{\P}{\mathbb{P}}
\newcommand{\E}{\mathbb{E}}
\newcommand{\F}{\mathbb{F}}
\newcommand{\cU}{\mathcal{U}}
\newcommand{\RP}{\mathbb{RP}}
\newcommand{\CP}{\mathbb{CP}}

\newcommand{\ph}{\,\cdot\,}

\DeclareMathOperator{\sech}{sech}
\DeclareMathOperator{\cosech}{cosech}
\DeclareMathOperator{\cosec}{cosec}

\DeclareMathOperator{\covol}{covol}
\DeclareMathOperator{\vol}{vol}

\let\Im\relax
\let\Re\relax
\DeclareMathOperator{\Im}{Im}
\DeclareMathOperator{\Re}{Re}
\DeclareMathOperator{\im}{im}
\DeclareMathOperator{\image}{image}
\DeclareMathOperator{\Ann}{Ann}

\DeclareMathOperator*{\res}{res}
\DeclareMathOperator{\Res}{Res}
\DeclareMathOperator{\Ind}{Ind}

\DeclareMathOperator{\tr}{tr}
\DeclareMathOperator{\diag}{diag}
\DeclareMathOperator{\rank}{rank}
\DeclareMathOperator{\card}{card}
\DeclareMathOperator{\spn}{span}
\DeclareMathOperator{\adj}{adj}

\DeclareMathOperator{\erf}{erf}
\DeclareMathOperator{\erfc}{erfc}

\DeclareMathOperator{\ord}{ord}
\DeclareMathOperator{\Sym}{Sym}

\DeclareMathOperator{\sgn}{sgn}
\DeclareMathOperator{\orb}{orb}
\DeclareMathOperator{\stab}{stab}
\DeclareMathOperator{\ccl}{ccl}

\DeclareMathOperator{\lcm}{lcm}
\DeclareMathOperator{\hcf}{hcf}

\DeclareMathOperator{\Int}{Int}
\DeclareMathOperator{\id}{id}

\DeclareMathOperator{\betaD}{beta}
\DeclareMathOperator{\gammaD}{gamma}
\DeclareMathOperator{\Poisson}{Poisson}
\DeclareMathOperator{\binomial}{binomial}
\DeclareMathOperator{\multinomial}{multinomial}
\DeclareMathOperator{\Bernoulli}{Bernoulli}
\DeclareMathOperator{\like}{like}

\DeclareMathOperator{\var}{var}
\DeclareMathOperator{\cov}{cov}
\DeclareMathOperator{\bias}{bias}
\DeclareMathOperator{\mse}{mse}
\DeclareMathOperator{\corr}{corr}

\DeclareMathOperator{\otp}{otp}
\DeclareMathOperator{\dom}{dom}

\DeclareMathOperator{\Root}{Root}
\DeclareMathOperator{\supp}{supp}
\DeclareMathOperator{\rel}{rel}
\DeclareMathOperator{\Hom}{Hom}
\DeclareMathOperator{\Aut}{Aut}
\DeclareMathOperator{\Gal}{Gal}
\DeclareMathOperator{\Mat}{Mat}
\DeclareMathOperator{\End}{End}
\DeclareMathOperator{\Char}{char}
\DeclareMathOperator{\ev}{ev}
\DeclareMathOperator{\St}{St}
\DeclareMathOperator{\Lk}{Lk}
\DeclareMathOperator{\disc}{disc}
\DeclareMathOperator{\Isom}{Isom}
\DeclareMathOperator{\length}{length}
\DeclareMathOperator{\energy}{energy}
\DeclareMathOperator{\area}{area}
\DeclareMathOperator{\Syl}{Syl}
\DeclareMathOperator{\cl}{cl}
\DeclareMathOperator{\fix}{fix}

\newcommand{\GL}{\mathrm{GL}}
\newcommand{\SL}{\mathrm{SL}}
\newcommand{\PGL}{\mathrm{PGL}}
\newcommand{\PSL}{\mathrm{PSL}}
\newcommand{\PSU}{\mathrm{PSU}}
\newcommand{\Or}{\mathrm{O}}
\newcommand{\SO}{\mathrm{SO}}
\newcommand{\U}{\mathrm{U}}
\newcommand{\SU}{\mathrm{SU}}

\renewcommand{\d}{\mathrm{d}}
\newcommand{\D}{\mathrm{D}}

\tikzset{->/.style = {decoration={markings,
                                  mark=at position 1 with {\arrow[scale=2]{latex'}}},
                      postaction={decorate}}}
\tikzset{<-/.style = {decoration={markings,
                                  mark=at position 0 with {\arrowreversed[scale=2]{latex'}}},
                      postaction={decorate}}}
\tikzset{<->/.style = {decoration={markings,
                                   mark=at position 0 with {\arrowreversed[scale=2]{latex'}},
                                   mark=at position 1 with {\arrow[scale=2]{latex'}}},
                       postaction={decorate}}}
\tikzset{->-/.style = {decoration={markings,
                                   mark=at position #1 with {\arrow[scale=2]{latex'}}},
                       postaction={decorate}}}
\tikzset{-<-/.style = {decoration={markings,
                                   mark=at position #1 with {\arrowreversed[scale=2]{latex'}}},
                       postaction={decorate}}}

\tikzset{circ/.style = {fill, circle, inner sep = 0, minimum size = 3}}
\tikzset{mstate/.style={circle, draw, blue, text=black, minimum width=0.7cm}}

\definecolor{mblue}{rgb}{0.2, 0.3, 0.8}
\definecolor{morange}{rgb}{1, 0.5, 0}
\definecolor{mgreen}{rgb}{0.1, 0.4, 0.2}
\definecolor{mred}{rgb}{0.5, 0, 0}

\def\drawcirculararc(#1,#2)(#3,#4)(#5,#6){%
    \pgfmathsetmacro\cA{(#1*#1+#2*#2-#3*#3-#4*#4)/2}%
    \pgfmathsetmacro\cB{(#1*#1+#2*#2-#5*#5-#6*#6)/2}%
    \pgfmathsetmacro\cy{(\cB*(#1-#3)-\cA*(#1-#5))/%
                        ((#2-#6)*(#1-#3)-(#2-#4)*(#1-#5))}%
    \pgfmathsetmacro\cx{(\cA-\cy*(#2-#4))/(#1-#3)}%
    \pgfmathsetmacro\cr{sqrt((#1-\cx)*(#1-\cx)+(#2-\cy)*(#2-\cy))}%
    \pgfmathsetmacro\cA{atan2(#2-\cy,#1-\cx)}%
    \pgfmathsetmacro\cB{atan2(#6-\cy,#5-\cx)}%
    \pgfmathparse{\cB<\cA}%
    \ifnum\pgfmathresult=1
        \pgfmathsetmacro\cB{\cB+360}%
    \fi
    \draw (#1,#2) arc (\cA:\cB:\cr);%
}
\newcommand\getCoord[3]{\newdimen{#1}\newdimen{#2}\pgfextractx{#1}{\pgfpointanchor{#3}{center}}\pgfextracty{#2}{\pgfpointanchor{#3}{center}}}

\def\Xint#1{\mathchoice
   {\XXint\displaystyle\textstyle{#1}}%
   {\XXint\textstyle\scriptstyle{#1}}%
   {\XXint\scriptstyle\scriptscriptstyle{#1}}%
   {\XXint\scriptscriptstyle\scriptscriptstyle{#1}}%
   \!\int}
\def\XXint#1#2#3{{\setbox0=\hbox{$#1{#2#3}{\int}$}
     \vcenter{\hbox{$#2#3$}}\kern-.5\wd0}}
\def\ddashint{\Xint=}
\def\dashint{\Xint-}


\begin{document}
\maketitle
{\small
\noindent Field extensions, tower law, algebraic extensions; irreducible polynomials and relation with simple algebraic extensions. Finite multiplicative subgroups of a field are cyclic. Existence and uniqueness of splitting fields.\hspace*{\fill} [6]

\vspace{5pt}
\noindent Existence and uniquness of algebraic closure.\hspace*{\fill} [1]

\vspace{5pt}
\noindent Separability. Theorem of primitive element. Trace and norm.\hspace*{\fill} [3]

\vspace{5pt}
\noindent Normal and Galois extensions, automorphic groups. Fundamental theorem of Galois theory.\hspace*{\fill} [3]

\vspace{5pt}
\noindent Galois theory of finite fields. Reduction mod $p$.\hspace*{\fill} [2]

\vspace{5pt}
\noindent Cyclotomic polynomials, Kummer theory, cyclic extensions. Symmetric functions. Galois theory of cubics and quartics.\hspace*{\fill} [4]

\vspace{5pt}
\noindent Solubility by radicals. Insolubility of general quintic equations and other classical problems.\hspace*{\fill} [3]

\vspace{5pt}
\noindent Artin's theorem on the subfield fixed by a finite group of automorphisms. Polynomial invariants of a finite group; examples.\hspace*{\fill}  [2]}

\tableofcontents

\section{Solving equations}
Here we adopt the notation that if $R$ is a ring, then $R[t]$ is the polynomial ring over $R$ in the variable $t$. Usually, we take $R = \Q$ and consider polynomials $f(t) \in \Q[t]$. The objective is then to find roots to the equation $f(t) = 0$. Often, we want to restrict our search domain. For example, we might ask if there is a root in $\Q$. We will use $\Root_f(X)$ to denote the set of all roots of $f$ in $X$.

We will start from the simple cases, and build up to general solutions of Quartic equations.
\subsection{Linear equations}
Suppose that $f = t + a \in \Q[t]$ (with $a\in \Q$). We have $\Root_f(\Q) = \{-a\}$.

\subsection{Quadratic equations}
Consider a simple quadratic $f = t^2 + 1 \in \Q[t]$. Then $\Root_f(\Q) = \emptyset$ since the square of all rationals are positive. However, in the complex plane, we have $\Root_f(\C) = \{\sqrt{-1}, -\sqrt{-1}\}$.

In general, let $f = t^2 + at + b\in \Q[t]$. Then the roots are given by
\[
  \Root_f(\C) = \left\{\frac{-a \pm \sqrt{a^2 - 4b}}{2}\right\}
\]
\subsection{Cubic equations}
Let $f = t^3 + c\in \Q[t]$. The roots are then
\[
  \Root_f(\C) = \{\sqrt[3]{-c}, \mu\sqrt[3]{-c}, \mu^2 \sqrt[3]{-c}\},
\]
where $\mu = \frac{-1 + \sqrt{-3}}{2}$ is the 3rd root of unity. Note that $\mu$ is defined by the equation $\mu^3 - 1 = 0$, and satisfies $\mu^2 + \mu + 1 = 0$.

In general, let $f = t^3 + at^2 + bt + c \in \Q[t]$, and let $\Root_f(\C) = \{\alpha_1, \alpha_2, \alpha_3\}$, not necessarily distinct.

Our objective is to solve $f = 0$. Before doing so, we have to make it explicit what we mean by ``solving'' the equation. As in solving the quadratic, we want to express the roots $\alpha_1, \alpha_2$ and $\alpha_3$ in terms of ``radicals'' involving $a, b$ and $c$.

Unlike the quadratic case, there is no straightforward means of coming up with a general formula. The result we currently have is the result of many many years of hard work, and the substitutions we make seemingly come out of nowhere. However, after a lot of magic, we will indeed come up with a general formula for it.

We first simplify our polynomial by assuming $a = 0$. Given any polynomial $f = t^3 + at^2 + bt + c$, we can perform the change of variables $t\mapsto t - \frac{a}{3}$, and get rid of the coefficient of $a^2$. So we can assume $a = 0$.

Let $\mu$ be as above. Define
\begin{align*}
  \beta &= \alpha_1 + \mu \alpha_2 + \mu^2 \alpha_3\\
  \gamma &= \alpha_1 + \mu^2 \alpha_2 + \mu \alpha_3
\end{align*}
These are the \emph{Lagrange resolvers}. We obtain
\begin{align*}
  \beta\gamma &= \alpha_1^2 + \alpha_2^2 +\alpha_3^2 + (\mu + \mu^2)(\alpha_1\alpha_2 + \alpha_2\alpha_3 + \alpha_1\alpha_3)\\
  \intertext{Since $\mu^2 + \mu + 1 = 0$, we have $\mu^2 + \mu = -1$. So we can simplify to obtain}
  &= (\alpha_1 + \alpha_2 + \alpha_3)^2 - 3(\alpha_1\alpha_2 + \alpha_2\alpha_3 + \alpha_1\alpha_3)\\
  \intertext{We have $\alpha_1 + \alpha_2 + \alpha_3 = -a = 0$, while $b = \alpha_1\alpha_2 + \alpha_2\alpha_3 + \alpha_1\alpha_3$. So}
  &= -3b\\
  \intertext{Cubing, we obtain}
  \beta^3\gamma^3 &= -27b^3.
\end{align*}
On the other hand, recalling again that $\alpha_1 + \alpha_2 + \alpha_3 = 0$, we have
\begin{align*}
  \beta^3 + \gamma^3 &= (\alpha_1 + \mu \alpha_2 + \mu^2 \alpha_3)^3 + (\alpha_1 + \mu^2\alpha_2 + \mu \alpha_3)^3 + (\alpha_1 + \alpha_2 + \alpha_3)^3\\
  &= 3(\alpha_1^2 + \alpha_2^3 + \alpha_3^2) + 18\alpha_1\alpha_2\alpha_3\\
  \intertext{We have $\alpha_1\alpha_2\alpha_3 = -c$, and since $\alpha_i^3 + b\alpha_i + c = 0$ for all $i$, summing gives $\alpha_1^3 +  \alpha_2^3 + \alpha_3^3 + 3c = 0$. So}
  &= -27c
\end{align*}
Hence, we obtain
\[
  (t - \beta^3)(t - \gamma^3) = t^2 + 27ct - 27b^3.
\]
We already know how to solve this equation using the quadratic formula. We obtain
\[
  \{\beta^3, \gamma^3\} = \left\{\frac{-27 c \pm \sqrt{(27c)^3 + 4\times 27b^3}}{2}\right\}
\]
We now have $\beta^3$ and $\gamma^3$ in terms of radicals. So we can find $\beta$ and $\gamma$ in terms of radicals. Finally, we can solve for $\alpha_i$ using
\begin{align*}
  0 &= \alpha_1 + \alpha_2 + \alpha_3\\
  \beta &= \alpha_1 + \mu \alpha_2 + \mu^2 \alpha_3\\
  \gamma &= \alpha_1 + \mu^2 \alpha_2 + \mu \alpha_3
\end{align*}
In particular, we obtain
\begin{align*}
  \alpha_1 &= \frac{1}{3}(\beta + \gamma)\\
  \alpha_2 &= \frac{1}{3}(\mu^2 \beta + \mu \gamma)\\
  \alpha_3 &= \frac{1}{3}(\mu \beta + \mu^2 \gamma)
\end{align*}
So we can solve a cubic in terms of radicals.

This was a lot of magic involved, and indeed this was discovered through a lot of hard work throughout many many years. This is also not a very helpful result since we have no idea where these substitutions came from and why they intuitively work.

\subsection{Quartic equations}
Assume $f = t^4 + at^3 + bt^2 + ct + d\in \Q[t]$. Let $\Root_f(\C) = \{\alpha_1, \alpha_2, \alpha_3, \alpha_4\}$. Can we express all these in terms of radicals? Again the answer is yes, but the procedure is much more complicated.

We can perform a similar change of variable to assume $a = 0$. So $\alpha_1 + \alpha_2 + \alpha_3 + \alpha_4 = 0$.

This time, define
\begin{align*}
  \beta &= \alpha_1 + \alpha_2\\
  \gamma &= \alpha_1 + \alpha_3\\
  \lambda &= \alpha_1 + \alpha_4
\end{align*}
Doing some calculations, we see that
\begin{align*}
  \beta^2 &= -(\alpha_1 + \alpha_2)(\alpha_3 + \alpha_4)\\
  \gamma^2 &= -(\alpha_1 + \alpha_3)(\alpha_2 + \alpha_4)\\
  \lambda^2 &= -(\alpha_1 + \alpha_4)(\alpha_2 + \alpha_3)
\end{align*}
Now consider
\begin{align*}
  g &= (t - \beta^2)(t - \gamma^2)(t - \lambda^2)\\
  &= t^3 + 2bt^2 + (b^2 - 4d)t - c^2
\end{align*}
This we know how to solve, and so we are done.

\subsection{Quintics and above}
So far so good. But how about polynomials of higher degrees? In general, let $f \in \Q[t]$. Can we write down all the roots of $f$ in terms of radicals? We know that the answer is yes if $\deg f \leq 4$.

Unfortunately, for $\deg f \geq 5$, the answer is no. Of course, this ``no'' means no \emph{in general}. For example, $f = (t - 1)(t - 2) \cdots (t - 5)\in \Q[t]$ has the obvious roots in terms of radicals.

There isn't an easy proof of this result. We will first have to associate a \emph{field extension} $F \supseteq \Q$ for our polynomial $f$. Then we associate a \emph{Galois group} $G$ to this field extension. We will show that $f$ has a solution in terms of radicals iff the Galois group is ``soluble'', where ``soluble'' has a specific algebraic definition in group theory we will explore later. This process is what we will study in Galois theory.

In a nutshell, Galois theory is the study of field extensions and the associated Galois groups. Nowadays, Galois theory finds its applications in number theory, algebraic geometry and even cryptography.

\section{Field extensions}
After all that (hopefully) fun introduction and motivation, we will now start Galois theory in a more abstract way. Everything starts from field extensions.

\begin{defi}[Field extension]
  A \emph{field extension} is an inclusion of a field $E\subseteq F$, where $E$ inherits the algebraic operations from $F$. Alternatively, we can define this by a injective homomorphism $E\to F$. We say $F$ is an \emph{extension} of $E$, and $E$ is a \emph{subfield} of $F$.
\end{defi}

\begin{eg}\leavevmode
  \begin{enumerate}
    \item $\Q\subseteq \R$ is a field extension.
    \item $\Q \subseteq \C$ is a field extension.
    \item $\Q\subseteq \Q(\sqrt{2}) = \{a + b\sqrt{2}: a, b\in \Q\} \subseteq \R$ is a field extension.
  \end{enumerate}
\end{eg}

Now suppose $K\subseteq L$ is a field extension. We can consider $L$ as a vector space over $K$. We know that $L$ already comes with an additive abelian group structure, and we can define scalar multiplication by simply multiplying: if $a\in K, \alpha\in L$, then $a\cdot \alpha$ is simply defined as multiplication in $L$.

\begin{defi}[Degree of field extension]
  The \emph{degree} of $L$ over $K$ is $[L:K]$ is the dimension of $L$ as a vector space over $K$. The extension is \emph{finite} if the degree is finite.
\end{defi}
In this course, we are mostly concerned with finite extensions.

\begin{eg}\leavevmode
  \begin{enumerate}
    \item Consider $\R\subseteq \C$. This is a finite extension with degree $[\C:\R] = 2$ since we have a basis of $\{1, i\}$.
    \item The extension $\Q\subseteq \Q(\sqrt{2})$ has degree $2$ since we have a basis of $\{1, \sqrt{2}\}$.
    \item The extension $\Q\subseteq \R$ is not finite.
  \end{enumerate}
\end{eg}

\begin{thm}[Tower Law]
  Let $K\subseteq L \subseteq F$ be field extensions. Then
  \[
    [F:K] = [F:L][L:K]
  \]
\end{thm}

\begin{proof}
  Assume $[F:L]$ and $[L:K]$ are finite. Let $\{\alpha_1, \cdots, \alpha_m\}$ be a basis for $L$ over $K$, and $\{\beta_1, \cdots, \beta_n\}$ be a basis for $F$ over $L$. Pick $\gamma \in F$. Then we can write
  \[
    \gamma = \sum_i b_i \beta_i,\quad b_i\in L.
  \]
  For each $b_i$, we can write as
  \[
    b_i = \sum_j a_{ij}\alpha_{j},\quad a_{ij}\in K.
  \]
  So we can write
  \[
    \gamma = \sum_i \left(\sum_j a_{ij}\alpha_j\right)\beta_i = \sum_{i, j} a_{ij}\alpha_j \beta_i.
  \]
  So $T = \{\alpha_j\beta_i\}_{i, j}$ spans $F$ over $K$. To show that this is a basis, we have to show that they are linearly independent. Consider the case where $\gamma = 0$. Then we must have $b_i = 0$ since $\{\beta_i\}$ is a basis of $F$ over $L$. Hence each $a_{ij} = 0$ since $\{\alpha_j\}$ is a basis of $L$ over $K$.

  This implies that $T$ is a basis of $F$ over $K$. So
  \[
    [F:K] = |T| = nm = [F:L][L:K].
  \]
  Finally, if $[F:L] = \infty$ or $[L:K] = \infty$, then clearly $[F:K] = \infty$ as well. So equality holds as well.
\end{proof}

\begin{defi}[Algebraic number]
  Let $K\subseteq L$ be a field extension, $\alpha\in L$. We define
  \[
    I_\alpha = \{f\in K[t] : f(\alpha) = 0\}\subseteq K[t]
  \]
  This is the set of polynomials for which $\alpha$ is a root. It is easy to show that $I_\alpha$ is an ideal, since it is the kernel of the ring homomorphism $K[t] \to L$ by $g \mapsto g(\alpha)$.

  We say $\alpha$ is \emph{algebraic} over $K$ if $I_\alpha \not= 0$. Otherwise, $\alpha$ is \emph{transcendental} over $K$.

  We say $L$ is \emph{algebraic} over $K$ if every element of $L$ is algebraic.
\end{defi}

\begin{eg}\leavevmode
  \begin{enumerate}
    \item $\sqrt[9]{7}$ is algebraic over $\Q$ because $f(\sqrt[9]{7}) = 0$, where $f = t^9 - 7$. In general, any number written with radicals is algebraic over $\Q$.
    \item $\pi$ is not algebraic over $\Q$.
  \end{enumerate}
\end{eg}
These are rather simple examples, and the following lemma will provide us a way of generating much more examples.

\begin{lemma}
  Let $K\subseteq L$ be a finite extension. Then $L$ is algebraic over $K$.
\end{lemma}

\begin{proof}
  Let $n = [L:K]$, and let $\alpha\in L$. Then $1, \alpha, \alpha^2, \cdots, \alpha^n$ are linearly dependent over $K$ (since there are $n + 1$ elements). So there exists some $a_i \in K$ (not all zero) such that
  \[
    a_n \alpha^n + a_{n - 1}\alpha^{n - 1} + \cdots + a_1 \alpha + a_0 = 0.
  \]
  So we have a non-trivial polynomial that vanishes at $\alpha$. So $\alpha$ is algebraic over $K$.

  Since $\alpha$ was arbitrary, $L$ itself is algebraic.
\end{proof}

Let $K\subseteq L$ be a field extension, $\alpha\in L$ algebraic. Since $K$ is a field, $K[t]$ is a PID (principal ideal domain). So $I_\alpha = \bra P_\alpha\ket$ for some monic $P_\alpha \in K[t]$. So every element of $I_\alpha$ is just a multiple of $P_\alpha$.
\begin{defi}[Minimal polynomial]
  Let $K\subseteq L$ be a field extension, $\alpha \in L$. The \emph{minimal polynomial} of $\alpha$ over $K$ is a monic polynomial $P_\alpha$ such that $I_\alpha = \bra P_\alpha\ket$.
\end{defi}

\begin{eg}\leavevmode
  \begin{enumerate}
    \item Consider $\Q\subseteq\R$, $\alpha = \sqrt[3]{2}$. Then the minimal polynomial is $P_\alpha = t^3 - 2$.
    \item Consider $\R\subseteq\C$, $\alpha = \sqrt[3]{2}$. Then the minimal polynomial is $P_\alpha = t - \sqrt[3]{2}$.
  \end{enumerate}
\end{eg}
\begin{prop}
  Let $K\subseteq L$ be a field extension, $\alpha\in L$ algebraic over $K$, and $P_\alpha$ the minimal polynomial. Then $P_\alpha$ is irreducible in $K[t]$.
\end{prop}

\begin{proof}
  Assume that $P_\alpha = QR$ in $K[t]$. So $0 = P_\alpha(\alpha) = Q(\alpha) R(\alpha)$. So $Q(\alpha) = 0$ or $R(\alpha) = 0$. Say $Q(\alpha) = 0$. So $Q\in I_\alpha$. So $Q$ is a multiple of $P_\alpha$. However, we also know that $P_\alpha$ is a multiple of $Q_\alpha$. This is possible only if $R$ is a unit in $K[t]$, ie. $R\in K$. So $P_\alpha$ is irreducible.
\end{proof}
\end{document}
