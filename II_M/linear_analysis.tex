\documentclass[a4paper]{article}

\def\npart {II}
\def\nterm {Michaelmas}
\def\nyear {2015}
\def\nlecturer {J. W. Luk}
\def\ncourse {Linear Analysis}
\def\nnotready {}

% Imports
\ifx \nextra \undefined
  \usepackage[pdftex,
    hidelinks,
    pdfauthor={Dexter Chua},
    pdfsubject={Cambridge Maths Notes: Part \npart\ - \ncourse},
    pdftitle={Part \npart\ - \ncourse},
  pdfkeywords={Cambridge Mathematics Maths Math \npart\ \nterm\ \nyear\ \ncourse}]{hyperref}
  \title{Part \npart\ - \ncourse}
\else
  \usepackage[pdftex,
    hidelinks,
    pdfauthor={Dexter Chua},
    pdfsubject={Cambridge Maths Notes: Part \npart\ - \ncourse\ (\nextra)},
    pdftitle={Part \npart\ - \ncourse\ (\nextra)},
  pdfkeywords={Cambridge Mathematics Maths Math \npart\ \nterm\ \nyear\ \ncourse\ \nextra}]{hyperref}

  \title{Part \npart\ - \ncourse \\ {\Large \nextra}}
\fi

\author{Lectured by \nlecturer \\\small Notes taken by Dexter Chua}
\date{\nterm\ \nyear}

\usepackage{alltt}
\usepackage{amsfonts}
\usepackage{amsmath}
\usepackage{amssymb}
\usepackage{amsthm}
\usepackage{booktabs}
\usepackage{caption}
\usepackage{enumitem}
\usepackage{fancyhdr}
\usepackage{graphicx}
\usepackage{mathtools}
\usepackage{microtype}
\usepackage{multirow}
\usepackage{pdflscape}
\usepackage{pgfplots}
\usepackage{siunitx}
\usepackage{tabularx}
\usepackage{tikz}
\usepackage{tkz-euclide}
\usepackage[normalem]{ulem}
\usepackage[all]{xy}

\pgfplotsset{compat=1.12}

\pagestyle{fancyplain}
\lhead{\emph{\nouppercase{\leftmark}}}
\ifx \nextra \undefined
  \rhead{
    \ifnum\thepage=1
    \else
      \npart\ \ncourse
    \fi}
\else
  \rhead{
    \ifnum\thepage=1
    \else
      \npart\ \ncourse\ (\nextra)
    \fi}
\fi
\usetikzlibrary{arrows}
\usetikzlibrary{decorations.markings}
\usetikzlibrary{decorations.pathmorphing}
\usetikzlibrary{positioning}
\usetikzlibrary{fadings}
\usetikzlibrary{intersections}
\usetikzlibrary{cd}

\newcommand*{\Cdot}{\raisebox{-0.25ex}{\scalebox{1.5}{$\cdot$}}}
\newcommand {\pd}[2][ ]{
  \ifx #1 { }
    \frac{\partial}{\partial #2}
  \else
    \frac{\partial^{#1}}{\partial #2^{#1}}
  \fi
}

% Theorems
\theoremstyle{definition}
\newtheorem*{aim}{Aim}
\newtheorem*{axiom}{Axiom}
\newtheorem*{claim}{Claim}
\newtheorem*{cor}{Corollary}
\newtheorem*{defi}{Definition}
\newtheorem*{eg}{Example}
\newtheorem*{fact}{Fact}
\newtheorem*{law}{Law}
\newtheorem*{lemma}{Lemma}
\newtheorem*{notation}{Notation}
\newtheorem*{prop}{Proposition}
\newtheorem*{thm}{Theorem}

\renewcommand{\labelitemi}{--}
\renewcommand{\labelitemii}{$\circ$}
\renewcommand{\labelenumi}{(\roman{*})}

\let\stdsection\section
\renewcommand\section{\newpage\stdsection}

% Strike through
\def\st{\bgroup \ULdepth=-.55ex \ULset}

% Maths symbols
\newcommand{\bra}{\langle}
\newcommand{\ket}{\rangle}

\newcommand{\N}{\mathbb{N}}
\newcommand{\Z}{\mathbb{Z}}
\newcommand{\Q}{\mathbb{Q}}
\renewcommand{\H}{\mathbb{H}}
\newcommand{\R}{\mathbb{R}}
\newcommand{\C}{\mathbb{C}}
\newcommand{\Prob}{\mathbb{P}}
\renewcommand{\P}{\mathbb{P}}
\newcommand{\E}{\mathbb{E}}
\newcommand{\F}{\mathbb{F}}
\newcommand{\cU}{\mathcal{U}}
\newcommand{\RP}{\mathbb{RP}}
\newcommand{\CP}{\mathbb{CP}}

\newcommand{\ph}{\,\cdot\,}

\DeclareMathOperator{\sech}{sech}
\DeclareMathOperator{\cosech}{cosech}
\DeclareMathOperator{\cosec}{cosec}

\DeclareMathOperator{\covol}{covol}
\DeclareMathOperator{\vol}{vol}

\let\Im\relax
\let\Re\relax
\DeclareMathOperator{\Im}{Im}
\DeclareMathOperator{\Re}{Re}
\DeclareMathOperator{\im}{im}
\DeclareMathOperator{\image}{image}
\DeclareMathOperator{\Ann}{Ann}

\DeclareMathOperator*{\res}{res}
\DeclareMathOperator{\Res}{Res}
\DeclareMathOperator{\Ind}{Ind}

\DeclareMathOperator{\tr}{tr}
\DeclareMathOperator{\diag}{diag}
\DeclareMathOperator{\rank}{rank}
\DeclareMathOperator{\card}{card}
\DeclareMathOperator{\spn}{span}
\DeclareMathOperator{\adj}{adj}

\DeclareMathOperator{\erf}{erf}
\DeclareMathOperator{\erfc}{erfc}

\DeclareMathOperator{\ord}{ord}
\DeclareMathOperator{\Sym}{Sym}

\DeclareMathOperator{\sgn}{sgn}
\DeclareMathOperator{\orb}{orb}
\DeclareMathOperator{\stab}{stab}
\DeclareMathOperator{\ccl}{ccl}

\DeclareMathOperator{\lcm}{lcm}
\DeclareMathOperator{\hcf}{hcf}

\DeclareMathOperator{\Int}{Int}
\DeclareMathOperator{\id}{id}

\DeclareMathOperator{\betaD}{beta}
\DeclareMathOperator{\gammaD}{gamma}
\DeclareMathOperator{\Poisson}{Poisson}
\DeclareMathOperator{\binomial}{binomial}
\DeclareMathOperator{\multinomial}{multinomial}
\DeclareMathOperator{\Bernoulli}{Bernoulli}
\DeclareMathOperator{\like}{like}

\DeclareMathOperator{\var}{var}
\DeclareMathOperator{\cov}{cov}
\DeclareMathOperator{\bias}{bias}
\DeclareMathOperator{\mse}{mse}
\DeclareMathOperator{\corr}{corr}

\DeclareMathOperator{\otp}{otp}
\DeclareMathOperator{\dom}{dom}

\DeclareMathOperator{\Root}{Root}
\DeclareMathOperator{\supp}{supp}
\DeclareMathOperator{\rel}{rel}
\DeclareMathOperator{\Hom}{Hom}
\DeclareMathOperator{\Aut}{Aut}
\DeclareMathOperator{\Gal}{Gal}
\DeclareMathOperator{\Mat}{Mat}
\DeclareMathOperator{\End}{End}
\DeclareMathOperator{\Char}{char}
\DeclareMathOperator{\ev}{ev}
\DeclareMathOperator{\St}{St}
\DeclareMathOperator{\Lk}{Lk}
\DeclareMathOperator{\disc}{disc}
\DeclareMathOperator{\Isom}{Isom}
\DeclareMathOperator{\length}{length}
\DeclareMathOperator{\energy}{energy}
\DeclareMathOperator{\area}{area}
\DeclareMathOperator{\Syl}{Syl}
\DeclareMathOperator{\cl}{cl}
\DeclareMathOperator{\fix}{fix}

\newcommand{\GL}{\mathrm{GL}}
\newcommand{\SL}{\mathrm{SL}}
\newcommand{\PGL}{\mathrm{PGL}}
\newcommand{\PSL}{\mathrm{PSL}}
\newcommand{\PSU}{\mathrm{PSU}}
\newcommand{\Or}{\mathrm{O}}
\newcommand{\SO}{\mathrm{SO}}
\newcommand{\U}{\mathrm{U}}
\newcommand{\SU}{\mathrm{SU}}

\renewcommand{\d}{\mathrm{d}}
\newcommand{\D}{\mathrm{D}}

\tikzset{->/.style = {decoration={markings,
                                  mark=at position 1 with {\arrow[scale=2]{latex'}}},
                      postaction={decorate}}}
\tikzset{<-/.style = {decoration={markings,
                                  mark=at position 0 with {\arrowreversed[scale=2]{latex'}}},
                      postaction={decorate}}}
\tikzset{<->/.style = {decoration={markings,
                                   mark=at position 0 with {\arrowreversed[scale=2]{latex'}},
                                   mark=at position 1 with {\arrow[scale=2]{latex'}}},
                       postaction={decorate}}}
\tikzset{->-/.style = {decoration={markings,
                                   mark=at position #1 with {\arrow[scale=2]{latex'}}},
                       postaction={decorate}}}
\tikzset{-<-/.style = {decoration={markings,
                                   mark=at position #1 with {\arrowreversed[scale=2]{latex'}}},
                       postaction={decorate}}}

\tikzset{circ/.style = {fill, circle, inner sep = 0, minimum size = 3}}
\tikzset{mstate/.style={circle, draw, blue, text=black, minimum width=0.7cm}}

\definecolor{mblue}{rgb}{0.2, 0.3, 0.8}
\definecolor{morange}{rgb}{1, 0.5, 0}
\definecolor{mgreen}{rgb}{0.1, 0.4, 0.2}
\definecolor{mred}{rgb}{0.5, 0, 0}

\def\drawcirculararc(#1,#2)(#3,#4)(#5,#6){%
    \pgfmathsetmacro\cA{(#1*#1+#2*#2-#3*#3-#4*#4)/2}%
    \pgfmathsetmacro\cB{(#1*#1+#2*#2-#5*#5-#6*#6)/2}%
    \pgfmathsetmacro\cy{(\cB*(#1-#3)-\cA*(#1-#5))/%
                        ((#2-#6)*(#1-#3)-(#2-#4)*(#1-#5))}%
    \pgfmathsetmacro\cx{(\cA-\cy*(#2-#4))/(#1-#3)}%
    \pgfmathsetmacro\cr{sqrt((#1-\cx)*(#1-\cx)+(#2-\cy)*(#2-\cy))}%
    \pgfmathsetmacro\cA{atan2(#2-\cy,#1-\cx)}%
    \pgfmathsetmacro\cB{atan2(#6-\cy,#5-\cx)}%
    \pgfmathparse{\cB<\cA}%
    \ifnum\pgfmathresult=1
        \pgfmathsetmacro\cB{\cB+360}%
    \fi
    \draw (#1,#2) arc (\cA:\cB:\cr);%
}
\newcommand\getCoord[3]{\newdimen{#1}\newdimen{#2}\pgfextractx{#1}{\pgfpointanchor{#3}{center}}\pgfextracty{#2}{\pgfpointanchor{#3}{center}}}

\def\Xint#1{\mathchoice
   {\XXint\displaystyle\textstyle{#1}}%
   {\XXint\textstyle\scriptstyle{#1}}%
   {\XXint\scriptstyle\scriptscriptstyle{#1}}%
   {\XXint\scriptscriptstyle\scriptscriptstyle{#1}}%
   \!\int}
\def\XXint#1#2#3{{\setbox0=\hbox{$#1{#2#3}{\int}$}
     \vcenter{\hbox{$#2#3$}}\kern-.5\wd0}}
\def\ddashint{\Xint=}
\def\dashint{\Xint-}


\begin{document}
\maketitle
{\small
\noindent Normed and Banach spaces. Linear mappings, continuity, boundedness, and norms. Finite-dimensional normed spaces.\hspace*{\fill} [4]

\vspace{5pt}
\noindent The Baire category theorem. The principle of uniform boundedness, the closed graph theorem and the inversion theorem; other applications.\hspace*{\fill} [5]

\vspace{5pt}
\noindent The normality of compact Hausdorff spaces. Urysohn's lemma and Tiezte's extension theorem. Spaces of continuous functions. The Stone-Weierstrass theorem and applications. Equicontinuity: the Ascoli-Arzel\`a theorem.\hspace*{\fill} [5]

\vspace{5pt}
\noindent Inner product spaces and Hilbert spaces; examples and elementary properties. Orthonormal systems, and the orthogonalization process. Bessel's inequality, the Parseval equation, and the Riesz-Fischer theorem. Duality; the self duality of Hilbert space.\hspace*{\fill} [5]

\vspace{5pt}
\noindent Bounded linear operations, invariant subspaces, eigenvectors; the spectrum and resolvent set. Compact operators on Hilbert space; discreteness of spectrum. Spectral theorem for compact Hermitian operators.\hspace*{\fill} [5]}

\tableofcontents

\setcounter{section}{-1}
\section{Introduction}
Linear analysis is the study of (infinite dimensional) vector spaces with extra structure (eg. a norm or inner product) that allows us to do analysis. We will be interested also in linear operators on spaces of functions.

\section{Normed vector spaces}
\begin{defi}[Normed vector space]
  A \emph{normed vector space} is a pair $(V, \|\cdot \|)$ such that $V$ is a vector space over a field $\F$ and $\|\cdot \|$ is a function $\|\cdot \|: V \mapsto \R$, known as the \emph{norm}, satisfying
  \begin{enumerate}
    \item $\|\mathbf{v}\| \geq 0$ for all $v\in V$, with equality iff $\mathbf{v} = \mathbf{0}$.
    \item $\| \lambda \mathbf{v}\| = |\lambda| \|\mathbf{v}\|$ for all $\lambda \in \F, \mathbf{v}\in V$.
    \item $\|\mathbf{v} + \mathbf{w}\| \leq \|\mathbf{v}\| + \|\mathbf{w}\|$ for all $\mathbf{v}, \mathbf{w} \in V$.
  \end{enumerate}
\end{defi}

\begin{eg}
  Let $V$ be a finite dimensional vector space, and $\{\mathbf{e}_1, \cdots, \mathbf{e}_n\}$ a basis. Then, for any $\mathbf{v} = \sum_{i = 1}^n v_i \mathbf{e}_i$, we can define a norm as
  \[
    \|\mathbf{v}\| = \sqrt{\sum_{i = 1}^n v_i^2}.
  \]
\end{eg}
Recall from IB Metric and Topological Spaces that $(V, d)$ is a metric space if the metric $d: V\times V \to \R$ satisfies
\begin{enumerate}
  \item $d(x, x) = 0$ for all $x\in V$.
  \item $d(x, y) = d(y, x)$ for all $x, y\in V$.
  \item $d(x, y) \leq d(x, z) + d(z, y)$ for all $x, y, z\in V$.
\end{enumerate}
Recall also that a topological spaces is a set $V$ together with a topology (a collection of open subsets) such that
\begin{enumerate}
  \item $\emptyset$ and $V$ are open subsets.
  \item The union of open subsets is open.
  \item The finite intersection of open subsets is open.
\end{enumerate}
As we have seen in Metric and Topological Spaces, a norm on a vector space induces a metric by $d(\mathbf{v}, \mathbf{w}) = \|\mathbf{v} - \mathbf{w}\|$. This metric in terms defines a topology on $V$ where the open sets are given by ``$U\subseteq V$ is open iff for any $x\in U$, $\exists \varepsilon$ such that $B(x, \varepsilon) = \{y\in V: d(x, y) < \varepsilon\}\subseteq U$.

This induced topology is not just a random topology on the vector space. They have the nice property that the vector space operators behave well under this topology.
\begin{prop}
  Addition $+: V\times V \to V$, and scalar multiplication $\cdot: \F \times V \to V$ are continuous with respect to the topology induced by the norm (and the usual product topology).
\end{prop}

\begin{proof}
  Let $U$ be open in $V$. We want to show that $(+)^{-1} (U)$ is open. Let $(v_1, v_2) \in (+)^{-1}(U)$, ie. $v_1 + v_2 \in U$. Since $v_1 + v_2 \in U$, there exists $\varepsilon$ such that $B(v_1 + v_2, \varepsilon) \subseteq U$. We want to show that $B(v_1, \frac{\varepsilon}{2}) + B(v_2, \frac{\varepsilon}{2}) \subseteq U$. This is nothing but the triangle inequality. So done.

  Scalar multiplication can be done in a very similar way.
\end{proof}
This motivates the following definition - we can do without the norm, and just require a topology in which addition and scalar multiplication are continuous.
\begin{defi}[Topological vector space]
  A \emph{topological vector space} $(V, \mathcal{U})$ is a vector space $V$ together with a topology $\mathcal{U}$ such that addition and scalar multiplication are continuous maps, and moreover singleton points $\{\mathbf{v}\}$ are closed sets.
\end{defi}
The requirement that points are closed is just a rather technical requirement that we should not pay much attention to.

A natural question to ask is: when is a topological vector space \emph{normable}? ie. Given a topological vector space, can we find a norm that induces the topology?

To answer this question, we will first need a few definitions.

\begin{defi}[Absolute convexity]
  Let $V$ be a vector space. Then $C\subseteq V$ is \emph{absolutely convex} (or \emph{balanced convex}) if for any $\lambda, \mu \in \F$ such that $|\lambda| + |\mu| \leq 1$, we have $\lambda C + \mu C \subseteq C$. In other words, if $c_1, c_2 \in C$, we have $\lambda c_1 + \mu c_2 \in C$.
\end{defi}

\begin{prop}
  If $(V, \|\cdot \|)$ is a normed vector space, then $B(t) = B(\mathbf{0}, t) = \{\mathbf{v}: \|\mathbf{v}\| < t\}$ is absolutely convex.
\end{prop}

\begin{proof}
  By triangle inequality.
\end{proof}

\begin{defi}[Bounded subset]
  Let $V$ be a topological vector space. Then $B\subseteq V$ is \emph{bounded} if for every open neighbourhood $U\subseteq V$ of $\mathbf{0}$, there is some $s > 0$ such that $B\subseteq t U$ for all $t > s$.
\end{defi}
At first sight, this might seem like a rather weird definition. Intuitively, this just means that $B$ is bounded if, whenever we take any open set $U$, by enlarging it by a scalar multiple, we can make it fully contain $B$.

\begin{eg}
  $B(t)$ in a normed vector space is bounded.
\end{eg}

\begin{prop}
  A topological vector space $(V, \mathcal{U})$ is normable if and only if there exists an absolutely convex, bounded open neighbourhood of $\mathbf{0}$.
\end{prop}

\begin{proof}
  One direction is obvious - if $V$ is normable, then $B(t)$ is an absolutely convex, bounded open neighbourhood of $\mathbf{0}$.

  The other direction is not too difficult as well. We define the Minkowski functional $\mu: V \to \R$ by
  \[
    \mu_C(\mathbf{v}) = \inf \{ t: v\in tC\},
  \]
  where $C$ is our absolutely convex, bounded open neighbourhood. We can show that this is a norm on $V$ (exercise - hint: you will need to use everything, such as the definition of a topological vector space and absolute convexity).
\end{proof}
In fact, the condition of absolute convexity can be replaced ``convex'', where ``convex'' means for every $t\in [0, 1]$, $tC + (1 - t)C \subseteq C$.

Among all normed spaces, some are particularly nice, known as Banach spaces.
\begin{defi}[Banach spaces]
  A normed vector space is a \emph{Banach space} if it is complete as a metric space, ie. every Cauchy sequence converges.
\end{defi}

\begin{eg}\leavevmode
  \begin{enumerate}
    \item A finite dimensional vector space (which is isomorphic to $\F^n$ for some $n$) is Banach.
    \item Let $X$ be a compact Hausdorff space. Then let $B(X) = \{f: X\to \R$ such that $f$ is bounded$\}$. This is obviously a vector space, and we can define the norm be $\|f\| = \sup_{x\in X}f(x)$. It is easy to show that this is a norm. It is less trivial to show that this is a Banach space.

      Let $\{f_n\}\subseteq B(X)$ be a Cauchy sequence. Then for any $x$, $\{f_n(x)\}\subseteq \R$ is also Cauchy. So we can define $f(x) = \lim\limits_{n \to \infty}f_n(x)$. We can show that $f_n \to f$ in the norm given (exercise).

    \item Define $X$ as before, and let $C(X) = \{f: X\to \R$ such that $f$ is continuous$\}$. Since any continuous $f$ is bounded, so $C(X) \subseteq B(X)$. We define the norm as before.

      Since we know that $C(X)\subseteq B(X)$, to show that $C(X)$ is Banach, it suffices to show that $C(X) \subseteq B(X)$ is closed, ie. if $f_n \to f$, $f_n \in C(X)$, then $f\in C(X)$, ie. the uniform limit of a continuous function is continuous.
    \item For $1 \leq p < \infty$, define
      \[
        \hat{L}_p ([0, 1]) = \left\{f: [0, 1] \to \R \text{ such that }f\text{ is continuous}, \int_0^1  \|f\|^p \;\d x < \infty\right\}.
      \]
      We define the norm $\|\cdot \|_{\hat{L}_p}$ by
      \[
        \|f\|_{\hat{L}_p} = \left(\int_0^1 \|f\|^p \;\d x\right)^{1/p}.
      \]
      It is easy to show that $\hat{L}_p$ is indeed a vector space, and we now check that this is a norm.
      \begin{enumerate}
        \item $\|f\|_{\hat{L}_p} \geq 0$ is obvious. Also, suppose that $\|f\|_{\hat{L}_p} = 0$. Then we must have $f = 0$. Otherwise, if $f\not = 0$, say $f(x) = \varepsilon$ for some $x$. Then there is some $\delta$ such that for any $y\in (x - \delta, x + \delta)$, we have $\|f(y)\| \geq \frac{\varepsilon}{2}$. Hence
          \[
            \|f\|_{\hat{L}_p} = \left(\int_0^1 |f|^p \;\d x\right)^{1/p} \geq \left[2\delta\left(\frac{\varepsilon}{2}\right)^p\right]^{1/p} > 0.
          \]
        \item $\|\lambda f\| = |\lambda|\|f\|$ is obvious
        \item The triangle inequality is the exactly what the Minkowski inequality says, which is in the example sheet.
      \end{enumerate}
      It turns out that $\hat{L}_p$ is \emph{not} a Banach space. We can brute-force a hard proof here, but we will later develop some tools that allow us to prove this much more easily.

      Hence, we define $L_p([0, 1])$ to be the completion of $\hat{L}_p ([0, 1])$. In II Probability and Measure, we will show that $L_p([0, 1])$ is in fact the space
      \[
        L_p([0, 1]) = \left\{f: [0, 1] \to \R\text{ such that } \int_0^1 \|f\|^p \;\d x < \infty\right\}/{\sim},
      \]
      where the integral is the Lebesgue integral, and we are quotienting by the relation $f\sim g$ if $f = g$ Lebesgue almost everywhere. You will understand what these terms mean in the II Probability and Measure course.

    \item $\ell_p$ spaces: for $p\in [1, \infty)$, define
        \[
          \ell_p (\F) = \left\{(x_1, x_2, \cdots): x_i \in \F, \sum_{i = 1}^\infty \|x_i\|^p < \infty\right\},
        \]
        with the norm
        \[
          \|x\|_{\ell_p} = \left(\sum_{i = 1}^\infty \|x_i\|^p\right)^{1/p}.
        \]
        It should be easy to check that this is a normed vector space. Moreover, this is a Banach space. Proof is in example sheet.
      \item $\ell_\infty$ space: we define
        \[
          \ell_\infty = \left\{(x_1, x_2, \cdots): x_i\in \F, \sup_{i\in \N} |x_i| < \infty\right\}
        \]
        with norm
        \[
          \|x\|_{\ell_\infty} = \sup_{i\in \N}\|x_i\|.
        \]
        Again, this is a Banach space.
      \item Let $B = B(1)$ be the unit open ball in $\R^n$. Define $C(B)$ to be the set of continuous functions $f: B\to \R$. Note that unlike in our previous example, these functions need not be bounded. Define a basis for the topology as follows:

        Let $\{K_i\}_{i = 1}^\infty$ be a sequence of compact subsets of $B$ such that $K_i \subseteq K_{i + 1}$ and $\bigcup_{i = 1}^\infty = B$. We define the basis to include
        \[
          \left\{f \in C(B): \sup_{x \in K_i} |f(x)| < \frac{1}{m}\right\}
        \]
        for each $m, i = 1, 2, \cdots$, as well as the translations of these sets.

        This weird basis is chosen such that $f_n \to f$ in this topology iff $f_n \to f$ uniformly in every compact set.
  \end{enumerate}
\end{eg}

\subsection{Bounded linear maps}
Now let $X, Y$ be normed vector spaces, and $\mathcal{L}(X, Y)$ be the set of linear maps from $X$ to $Y$.

\begin{defi}[Bounded linear map]
  $T: X\to Y$ is a \emph{bounded linear map} if there is a constant $C > 0$ such that $\|Tx\|_Y \leq C\|x\|_X$ for all $x\in X$. We write $\mathcal{B}(X, Y)$ for the set of bounded linear maps from $X$ to $Y$.
\end{defi}
This is equivalent to saying $T(B_X(1)) \subseteq B_Y(C)$ for some $C > 0$. This also equivalent to saying that $T(B)$ is bounded for every bounded subset $B$ of $X$. Note that this final characterization is also valid when we just have a topological vector space.

How does boundedness relate to the topological structure of the vector spaces? It turns out that boundedness is the same as continuity.

\begin{prop}
  Let $X$, $Y$ be normed vector spaces, $T: X\to Y$ a linear map. Then the following are equivalent:
  \begin{enumerate}
    \item $T$ is continuous.
    \item $T$ is continuous at 0.
    \item $T$ is bounded.
  \end{enumerate}
\end{prop}

\begin{proof}
  (i) $\Rightarrow$ (ii) is obvious.

  (ii) $\Rightarrow $ (iii): Consider $B_Y(1) \subseteq Y$, the unit open ball. Since $T$ is continuous at $0$, $T^{-1}(B_Y(1))\subseteq X$ is open. Hence there exists $\varepsilon > 0$ such that $B_X(\varepsilon) \subseteq T^{-1}(B_Y(1))$. So $T(B_x(\varepsilon)) \subseteq B_Y(1)$. So $B_X(1) \subseteq B_Y\left(\frac{1}{\varepsilon}\right)$. So $T$ is bounded.

  (iii) $\Rightarrow$ (i): Let $\varepsilon > 0$. Then $\|T x_1 - T x_2\|_Y = \|T(x_1 - x_2)\|_Y \leq C\|x_1 - x_2\|_X$. This is less than $\varepsilon$ if $\|x_1 - x_2\| < C^{-1}\varepsilon$. So done.
\end{proof}

Suppose again that $X$ and $Y$ are normed vector spaces. We can view $\mathcal{B}(X, Y)$ as vector space. We now define a norm to make it a normed vector space.

\begin{defi}[Norm on $\mathcal{B}(X, Y)$]
  Let $T: X\to Y$ be a bounded linear map. Define $\|T\|_{\mathcal{B}(X, Y)}$ by
  \[
    \|T\|_{\mathcal{B}(X, Y)} = \sup_{\|x\| \leq 1} \|T x\|_Y.
  \]
\end{defi}
Alternatively, this is the minimum $C$ such that $\|Tx\|_Y \leq C\|x\|_X$ for all $x$. In particular, we have
\[
  \|Tx\|_Y \leq \|T\|_{\mathcal{B}(X, Y)}\|x\|_X
\]
\subsection{Dual spaces}
We will frequently be interested in one particular case of $\mathcal{B}(X, Y)$.
\begin{defi}[Dual space]
  Let $V$ be a normed vector space. The \emph{dual space} is
  \[
    V^* = \mathcal{B}(V, \F)
  \]
  We call the elements of $V^*$ \emph{functionals}.
\end{defi}
\end{document}
