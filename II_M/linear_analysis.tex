\documentclass[a4paper]{article}

\def\npart {II}
\def\nterm {Michaelmas}
\def\nyear {2015}
\def\nlecturer {J. W. Luk}
\def\ncourse {Linear Analysis}
\def\nlectures {TTS.9}
\def\nnotready {}

% Imports
\ifx \nextra \undefined
  \usepackage[pdftex,
    hidelinks,
    pdfauthor={Dexter Chua},
    pdfsubject={Cambridge Maths Notes: Part \npart\ - \ncourse},
    pdftitle={Part \npart\ - \ncourse},
  pdfkeywords={Cambridge Mathematics Maths Math \npart\ \nterm\ \nyear\ \ncourse}]{hyperref}
  \title{Part \npart\ - \ncourse}
\else
  \usepackage[pdftex,
    hidelinks,
    pdfauthor={Dexter Chua},
    pdfsubject={Cambridge Maths Notes: Part \npart\ - \ncourse\ (\nextra)},
    pdftitle={Part \npart\ - \ncourse\ (\nextra)},
  pdfkeywords={Cambridge Mathematics Maths Math \npart\ \nterm\ \nyear\ \ncourse\ \nextra}]{hyperref}

  \title{Part \npart\ - \ncourse \\ {\Large \nextra}}
\fi

\author{Lectured by \nlecturer \\\small Notes taken by Dexter Chua}
\date{\nterm\ \nyear}

\usepackage{alltt}
\usepackage{amsfonts}
\usepackage{amsmath}
\usepackage{amssymb}
\usepackage{amsthm}
\usepackage{booktabs}
\usepackage{caption}
\usepackage{enumitem}
\usepackage{fancyhdr}
\usepackage{graphicx}
\usepackage{mathtools}
\usepackage{microtype}
\usepackage{multirow}
\usepackage{pdflscape}
\usepackage{pgfplots}
\usepackage{siunitx}
\usepackage{tabularx}
\usepackage{tikz}
\usepackage{tkz-euclide}
\usepackage[normalem]{ulem}
\usepackage[all]{xy}

\pgfplotsset{compat=1.12}

\pagestyle{fancyplain}
\lhead{\emph{\nouppercase{\leftmark}}}
\ifx \nextra \undefined
  \rhead{
    \ifnum\thepage=1
    \else
      \npart\ \ncourse
    \fi}
\else
  \rhead{
    \ifnum\thepage=1
    \else
      \npart\ \ncourse\ (\nextra)
    \fi}
\fi
\usetikzlibrary{arrows}
\usetikzlibrary{decorations.markings}
\usetikzlibrary{decorations.pathmorphing}
\usetikzlibrary{positioning}
\usetikzlibrary{fadings}
\usetikzlibrary{intersections}
\usetikzlibrary{cd}

\newcommand*{\Cdot}{\raisebox{-0.25ex}{\scalebox{1.5}{$\cdot$}}}
\newcommand {\pd}[2][ ]{
  \ifx #1 { }
    \frac{\partial}{\partial #2}
  \else
    \frac{\partial^{#1}}{\partial #2^{#1}}
  \fi
}

% Theorems
\theoremstyle{definition}
\newtheorem*{aim}{Aim}
\newtheorem*{axiom}{Axiom}
\newtheorem*{claim}{Claim}
\newtheorem*{cor}{Corollary}
\newtheorem*{defi}{Definition}
\newtheorem*{eg}{Example}
\newtheorem*{fact}{Fact}
\newtheorem*{law}{Law}
\newtheorem*{lemma}{Lemma}
\newtheorem*{notation}{Notation}
\newtheorem*{prop}{Proposition}
\newtheorem*{thm}{Theorem}

\renewcommand{\labelitemi}{--}
\renewcommand{\labelitemii}{$\circ$}
\renewcommand{\labelenumi}{(\roman{*})}

\let\stdsection\section
\renewcommand\section{\newpage\stdsection}

% Strike through
\def\st{\bgroup \ULdepth=-.55ex \ULset}

% Maths symbols
\newcommand{\bra}{\langle}
\newcommand{\ket}{\rangle}

\newcommand{\N}{\mathbb{N}}
\newcommand{\Z}{\mathbb{Z}}
\newcommand{\Q}{\mathbb{Q}}
\renewcommand{\H}{\mathbb{H}}
\newcommand{\R}{\mathbb{R}}
\newcommand{\C}{\mathbb{C}}
\newcommand{\Prob}{\mathbb{P}}
\renewcommand{\P}{\mathbb{P}}
\newcommand{\E}{\mathbb{E}}
\newcommand{\F}{\mathbb{F}}
\newcommand{\cU}{\mathcal{U}}
\newcommand{\RP}{\mathbb{RP}}
\newcommand{\CP}{\mathbb{CP}}

\newcommand{\ph}{\,\cdot\,}

\DeclareMathOperator{\sech}{sech}
\DeclareMathOperator{\cosech}{cosech}
\DeclareMathOperator{\cosec}{cosec}

\DeclareMathOperator{\covol}{covol}
\DeclareMathOperator{\vol}{vol}

\let\Im\relax
\let\Re\relax
\DeclareMathOperator{\Im}{Im}
\DeclareMathOperator{\Re}{Re}
\DeclareMathOperator{\im}{im}
\DeclareMathOperator{\image}{image}
\DeclareMathOperator{\Ann}{Ann}

\DeclareMathOperator*{\res}{res}
\DeclareMathOperator{\Res}{Res}
\DeclareMathOperator{\Ind}{Ind}

\DeclareMathOperator{\tr}{tr}
\DeclareMathOperator{\diag}{diag}
\DeclareMathOperator{\rank}{rank}
\DeclareMathOperator{\card}{card}
\DeclareMathOperator{\spn}{span}
\DeclareMathOperator{\adj}{adj}

\DeclareMathOperator{\erf}{erf}
\DeclareMathOperator{\erfc}{erfc}

\DeclareMathOperator{\ord}{ord}
\DeclareMathOperator{\Sym}{Sym}

\DeclareMathOperator{\sgn}{sgn}
\DeclareMathOperator{\orb}{orb}
\DeclareMathOperator{\stab}{stab}
\DeclareMathOperator{\ccl}{ccl}

\DeclareMathOperator{\lcm}{lcm}
\DeclareMathOperator{\hcf}{hcf}

\DeclareMathOperator{\Int}{Int}
\DeclareMathOperator{\id}{id}

\DeclareMathOperator{\betaD}{beta}
\DeclareMathOperator{\gammaD}{gamma}
\DeclareMathOperator{\Poisson}{Poisson}
\DeclareMathOperator{\binomial}{binomial}
\DeclareMathOperator{\multinomial}{multinomial}
\DeclareMathOperator{\Bernoulli}{Bernoulli}
\DeclareMathOperator{\like}{like}

\DeclareMathOperator{\var}{var}
\DeclareMathOperator{\cov}{cov}
\DeclareMathOperator{\bias}{bias}
\DeclareMathOperator{\mse}{mse}
\DeclareMathOperator{\corr}{corr}

\DeclareMathOperator{\otp}{otp}
\DeclareMathOperator{\dom}{dom}

\DeclareMathOperator{\Root}{Root}
\DeclareMathOperator{\supp}{supp}
\DeclareMathOperator{\rel}{rel}
\DeclareMathOperator{\Hom}{Hom}
\DeclareMathOperator{\Aut}{Aut}
\DeclareMathOperator{\Gal}{Gal}
\DeclareMathOperator{\Mat}{Mat}
\DeclareMathOperator{\End}{End}
\DeclareMathOperator{\Char}{char}
\DeclareMathOperator{\ev}{ev}
\DeclareMathOperator{\St}{St}
\DeclareMathOperator{\Lk}{Lk}
\DeclareMathOperator{\disc}{disc}
\DeclareMathOperator{\Isom}{Isom}
\DeclareMathOperator{\length}{length}
\DeclareMathOperator{\energy}{energy}
\DeclareMathOperator{\area}{area}
\DeclareMathOperator{\Syl}{Syl}
\DeclareMathOperator{\cl}{cl}
\DeclareMathOperator{\fix}{fix}

\newcommand{\GL}{\mathrm{GL}}
\newcommand{\SL}{\mathrm{SL}}
\newcommand{\PGL}{\mathrm{PGL}}
\newcommand{\PSL}{\mathrm{PSL}}
\newcommand{\PSU}{\mathrm{PSU}}
\newcommand{\Or}{\mathrm{O}}
\newcommand{\SO}{\mathrm{SO}}
\newcommand{\U}{\mathrm{U}}
\newcommand{\SU}{\mathrm{SU}}

\renewcommand{\d}{\mathrm{d}}
\newcommand{\D}{\mathrm{D}}

\tikzset{->/.style = {decoration={markings,
                                  mark=at position 1 with {\arrow[scale=2]{latex'}}},
                      postaction={decorate}}}
\tikzset{<-/.style = {decoration={markings,
                                  mark=at position 0 with {\arrowreversed[scale=2]{latex'}}},
                      postaction={decorate}}}
\tikzset{<->/.style = {decoration={markings,
                                   mark=at position 0 with {\arrowreversed[scale=2]{latex'}},
                                   mark=at position 1 with {\arrow[scale=2]{latex'}}},
                       postaction={decorate}}}
\tikzset{->-/.style = {decoration={markings,
                                   mark=at position #1 with {\arrow[scale=2]{latex'}}},
                       postaction={decorate}}}
\tikzset{-<-/.style = {decoration={markings,
                                   mark=at position #1 with {\arrowreversed[scale=2]{latex'}}},
                       postaction={decorate}}}

\tikzset{circ/.style = {fill, circle, inner sep = 0, minimum size = 3}}
\tikzset{mstate/.style={circle, draw, blue, text=black, minimum width=0.7cm}}

\definecolor{mblue}{rgb}{0.2, 0.3, 0.8}
\definecolor{morange}{rgb}{1, 0.5, 0}
\definecolor{mgreen}{rgb}{0.1, 0.4, 0.2}
\definecolor{mred}{rgb}{0.5, 0, 0}

\def\drawcirculararc(#1,#2)(#3,#4)(#5,#6){%
    \pgfmathsetmacro\cA{(#1*#1+#2*#2-#3*#3-#4*#4)/2}%
    \pgfmathsetmacro\cB{(#1*#1+#2*#2-#5*#5-#6*#6)/2}%
    \pgfmathsetmacro\cy{(\cB*(#1-#3)-\cA*(#1-#5))/%
                        ((#2-#6)*(#1-#3)-(#2-#4)*(#1-#5))}%
    \pgfmathsetmacro\cx{(\cA-\cy*(#2-#4))/(#1-#3)}%
    \pgfmathsetmacro\cr{sqrt((#1-\cx)*(#1-\cx)+(#2-\cy)*(#2-\cy))}%
    \pgfmathsetmacro\cA{atan2(#2-\cy,#1-\cx)}%
    \pgfmathsetmacro\cB{atan2(#6-\cy,#5-\cx)}%
    \pgfmathparse{\cB<\cA}%
    \ifnum\pgfmathresult=1
        \pgfmathsetmacro\cB{\cB+360}%
    \fi
    \draw (#1,#2) arc (\cA:\cB:\cr);%
}
\newcommand\getCoord[3]{\newdimen{#1}\newdimen{#2}\pgfextractx{#1}{\pgfpointanchor{#3}{center}}\pgfextracty{#2}{\pgfpointanchor{#3}{center}}}

\def\Xint#1{\mathchoice
   {\XXint\displaystyle\textstyle{#1}}%
   {\XXint\textstyle\scriptstyle{#1}}%
   {\XXint\scriptstyle\scriptscriptstyle{#1}}%
   {\XXint\scriptscriptstyle\scriptscriptstyle{#1}}%
   \!\int}
\def\XXint#1#2#3{{\setbox0=\hbox{$#1{#2#3}{\int}$}
     \vcenter{\hbox{$#2#3$}}\kern-.5\wd0}}
\def\ddashint{\Xint=}
\def\dashint{\Xint-}


\begin{document}
\maketitle
{\small
\noindent Normed and Banach spaces. Linear mappings, continuity, boundedness, and norms. Finite-dimensional normed spaces.\hspace*{\fill} [4]

\vspace{5pt}
\noindent The Baire category theorem. The principle of uniform boundedness, the closed graph theorem and the inversion theorem; other applications.\hspace*{\fill} [5]

\vspace{5pt}
\noindent The normality of compact Hausdorff spaces. Urysohn's lemma and Tiezte's extension theorem. Spaces of continuous functions. The Stone-Weierstrass theorem and applications. Equicontinuity: the Ascoli-Arzel\`a theorem.\hspace*{\fill} [5]

\vspace{5pt}
\noindent Inner product spaces and Hilbert spaces; examples and elementary properties. Orthonormal systems, and the orthogonalization process. Bessel's inequality, the Parseval equation, and the Riesz-Fischer theorem. Duality; the self duality of Hilbert space.\hspace*{\fill} [5]

\vspace{5pt}
\noindent Bounded linear operations, invariant subspaces, eigenvectors; the spectrum and resolvent set. Compact operators on Hilbert space; discreteness of spectrum. Spectral theorem for compact Hermitian operators.\hspace*{\fill} [5]}

\tableofcontents

\setcounter{section}{-1}
\section{Introduction}
Linear analysis is the study of (infinite dimensional) vector spaces with extra structure (eg. a norm or inner product) that allows us to do analysis. We will be interested also in linear operators on spaces of functions.

\section{Normed vector spaces}
In IB Linear Analysis, we have studied vector spaces in quite a lot of detail. However, just knowing that something is a vector space usually isn't too helpful. Often, we would want the vector space to have some additional structure. The first structure we will study is a \emph{norm}.

\begin{defi}[Normed vector space]
  A \emph{normed vector space} is a pair $(V, \|\cdot \|)$, where $V$ is a vector space over a field $\F$ and $\|\cdot \|$ is a function $\|\cdot \|: V \mapsto \R$, known as the \emph{norm}, satisfying
  \begin{enumerate}
    \item $\|\mathbf{v}\| \geq 0$ for all $\mathbf{v}\in V$, with equality iff $\mathbf{v} = \mathbf{0}$.
    \item $\| \lambda \mathbf{v}\| = |\lambda| \|\mathbf{v}\|$ for all $\lambda \in \F, \mathbf{v}\in V$.
    \item $\|\mathbf{v} + \mathbf{w}\| \leq \|\mathbf{v}\| + \|\mathbf{w}\|$ for all $\mathbf{v}, \mathbf{w} \in V$.
  \end{enumerate}
\end{defi}
Intuitively, we think of $\|\mathbf{v}\|$ as the ``length'' or ``magnitude'' of the vector.

\begin{eg}
  Let $V$ be a finite dimensional vector space, and $\{e_1, \cdots, e_n\}$ a basis. Then, for any $\mathbf{v} = \sum_{i = 1}^n v_i \mathbf{e}_i$, we can define a norm as
  \[
    \|\mathbf{v}\| = \sqrt{\sum_{i = 1}^n v_i^2}.
  \]
\end{eg}

If we are given a norm of a vector space $V$, we immediately obtain two more structures on $V$ for free, namely a metric and a topology.

Recall from IB Metric and Topological Spaces that $(V, d)$ is a metric space if the metric $d: V\times V \to \R$ satisfies
\begin{enumerate}
  \item $d(x, x) = 0$ for all $x\in V$.
  \item $d(x, y) = d(y, x)$ for all $x, y\in V$.
  \item $d(x, y) \leq d(x, z) + d(z, y)$ for all $x, y, z\in V$.
\end{enumerate}
Recall also that a topological spaces is a set $V$ together with a topology (a collection of open subsets) such that
\begin{enumerate}
  \item $\emptyset$ and $V$ are open subsets.
  \item The union of open subsets is open.
  \item The finite intersection of open subsets is open.
\end{enumerate}
As we have seen in IB Metric and Topological Spaces, a norm on a vector space induces a metric by $d(v, w) = \|v - w\|$. This metric in terms defines a topology on $V$ where the open sets are given by ``$U\subseteq V$ is open iff for any $x\in U$, $\exists \varepsilon$ such that $B(x, \varepsilon) = \{y\in V: d(x, y) < \varepsilon\}\subseteq U$''.

This induced topology is not just a random topology on the vector space. They have the nice property that the vector space operators behave well under this topology.
\begin{prop}
  Addition $+: V\times V \to V$, and scalar multiplication $\cdot: \F \times V \to V$ are continuous with respect to the topology induced by the norm (and the usual product topology).
\end{prop}

\begin{proof}
  Let $U$ be open in $V$. We want to show that $(+)^{-1} (U)$ is open. Let $(\mathbf{v}_1, \mathbf{v}_2) \in (+)^{-1}(U)$, ie. $\mathbf{v}_1 + \mathbf{v}_2 \in U$. Since $\mathbf{v}_1 + \mathbf{v}_2 \in U$, there exists $\varepsilon$ such that $B(\mathbf{v}_1 + \mathbf{v}_2, \varepsilon) \subseteq U$. By the triangle inequality, we know that $B(\mathbf{v}_1, \frac{\varepsilon}{2}) + B(\mathbf{v}_2, \frac{\varepsilon}{2}) \subseteq U$. Hence we have $(\mathbf{v}_1, \mathbf{v}_2) \in B\left((\mathbf{v}_1, \mathbf{v}_2), \frac{\varepsilon}{2}\right) \subseteq (+)^{-1}(U)$. So $(+)^{-1}(U)$ is open.

  Scalar multiplication can be done in a very similar way.
\end{proof}
This motivates the following definition - we can do without the norm, and just require a topology in which addition and scalar multiplication are continuous.
\begin{defi}[Topological vector space]
  A \emph{topological vector space} $(V, \mathcal{U})$ is a vector space $V$ together with a topology $\mathcal{U}$ such that addition and scalar multiplication are continuous maps, and moreover singleton points $\{\mathbf{v}\}$ are closed sets.
\end{defi}
The requirement that points are closed is just a rather technical requirement needed in certain proofs. We should, however, not pay too much attention to this when trying to understand it intuitively.

A natural question to ask is: when is a topological vector space \emph{normable}? ie. Given a topological vector space, can we find a norm that induces the topology?

To answer this question, we will first need a few definitions.

\begin{defi}[Absolute convexity]
  Let $V$ be a vector space. Then $C\subseteq V$ is \emph{absolutely convex} (or \emph{balanced convex}) if for any $\lambda, \mu \in \F$ such that $|\lambda| + |\mu| \leq 1$, we have $\lambda C + \mu C \subseteq C$. In other words, if $\mathbf{c}_1, \mathbf{c}_2 \in C$, we have $\lambda \mathbf{c}_1 + \mu \mathbf{c}_2 \in C$.
\end{defi}

\begin{prop}
  If $(V, \|\cdot \|)$ is a normed vector space, then $B(t) = B(\mathbf{0}, t) = \{\mathbf{v}: \|\mathbf{v}\| < t\}$ is absolutely convex.
\end{prop}

\begin{proof}
  By triangle inequality.
\end{proof}

\begin{defi}[Bounded subset]
  Let $V$ be a topological vector space. Then $B\subseteq V$ is \emph{bounded} if for every open neighbourhood $U\subseteq V$ of $\mathbf{0}$, there is some $s > 0$ such that $B\subseteq t U$ for all $t > s$.
\end{defi}
At first sight, this might seem like a rather weird definition. Intuitively, this just means that $B$ is bounded if, whenever we take any open set $U$, by enlarging it by a scalar multiple, we can make it fully contain $B$.

\begin{eg}
  $B(t)$ in a normed vector space is bounded.
\end{eg}

\begin{prop}
  A topological vector space $(V, \mathcal{U})$ is normable if and only if there exists an absolutely convex, bounded open neighbourhood of $\mathbf{0}$.
\end{prop}

\begin{proof}
  One direction is obvious - if $V$ is normable, then $B(t)$ is an absolutely convex, bounded open neighbourhood of $\mathbf{0}$.

  The other direction is not too difficult as well. We define the Minkowski functional $\mu: V \to \R$ by
  \[
    \mu_C(\mathbf{v}) = \inf \{t > 0: \mathbf{v}\in tC\},
  \]
  where $C$ is our absolutely convex, bounded open neighbourhood.

  Note that by definition, for any $t < \mu_C(\mathbf{v})$, $\mathbf{v}\not\in tC$. On the other hand, by absolute convexity, for any $t > \mu_C(\mathbf{v})$, we have $\mathbf{v}\in tC$.

  We now show that this is a norm on $V$:
  \begin{enumerate}
    \item If $\mathbf{v} = \mathbf{0}$, then $\mathbf{v}\in 0C$. So $\mu_C(\mathbf{0}) = 0$. On the other hand, suppose $\mathbf{v} \not= \mathbf{0}$. Since a singleton point is closed, $U = V\setminus \{\mathbf{v}\}$ is an open neighbourhood of $0$. Hence there is some $t$ such that $C\subseteq tU$. Alternatively, $\frac{1}{t}C \subseteq U$. Hence, $\mathbf{v}\not\in \frac{1}{t}C$. So $\mu_C (\mathbf{v}) \geq \frac{1}{t} > 0$. So $\mu_C (\mathbf{v}) = \mathbf{0}$ iff $\mathbf{v} = \mathbf{0}$.

    \item We have
      \[
        \mu_C(\lambda \mathbf{v}) = \inf \{t> 0: \lambda \mathbf{v}\in tC\} = \lambda \inf \{t > 0: \mathbf{v}\in tC\} = \lambda \mu_C(\mathbf{v}).
      \]
    \item We want to show that
      \[
        \mu_C (\mathbf{v} + \mathbf{w}) \leq \mu_C(\mathbf{v}) + \mu_C(\mathbf{w}).
      \]
      This is equivalent to showing that
      \[
        \inf\{t > 0: \mathbf{v} + \mathbf{w} \in tC\} \leq \inf\{t > 0: \mathbf{v}\in tC\} + \inf\{r > 0: \mathbf{w}\in :C\}.
      \]
      This is, in turn equivalent to proving that if $\mathbf{v}\in tC$ and $\mathbf{w}\in rC$, then $(\mathbf{v} + \mathbf{w})\in (t + r)C$.

      Let $\mathbf{v}' = \mathbf{v}/t, \mathbf{w}' = \mathbf{w}/r$. Then we want to show that if $\mathbf{v}' \in C$ and $\mathbf{w}' \in C$, then $\frac{1}{(t + r)}(t \mathbf{v}' + r \mathbf{w}') \in C$. This is exactly what is required by convexity. So done.
  \end{enumerate}
\end{proof}
In fact, the condition of absolute convexity can be replaced ``convex'', where ``convex'' means for every $t\in [0, 1]$, $tC + (1 - t)C \subseteq C$. This is since for every convex bounded $C$, we can find always find a absolutely convex bounded $\tilde{C} \subseteq C$, which is something not hard to prove.

Among all normed spaces, some are particularly nice, known as Banach spaces.
\begin{defi}[Banach spaces]
  A normed vector space is a \emph{Banach space} if it is complete as a metric space, ie. every Cauchy sequence converges.
\end{defi}

\begin{eg}\leavevmode
  \begin{enumerate}
    \item A finite dimensional vector space (which is isomorphic to $\F^n$ for some $n$) is Banach.
    \item Let $X$ be a compact Hausdorff space. Then let
      \[
        B(X) = \{f: X\to \R\text{ such that }f\text{ is bounded}\}.
      \]
      This is obviously a vector space, and we can define the norm be $\|f\| = \sup_{x\in X}f(x)$. It is easy to show that this is a norm. It is less trivial to show that this is a Banach space.

      Let $\{f_n\}\subseteq B(X)$ be a Cauchy sequence. Then for any $x$, $\{f_n(x)\}\subseteq \R$ is also Cauchy. So we can define $f(x) = \lim\limits_{n \to \infty}f_n(x)$.

      To show that $f_n \to f$, let $\varepsilon > 0$. By definition of $f_n$ being Cauchy, there is some $N$ such that for any $n, m > N$ and any fixed $x$, we have $|f_n(x) - f_m(x)| < \varepsilon$. Take the limit as $m \to \infty$. Then $f_m(x) \to f(x)$. So  $|f_n(x) - f(x)| \leq \varepsilon$. Since this is true for all $x$, for any $n > N$, we must have $\|f_n - f\| \leq \varepsilon$. So $f_n \to f$.

    \item Define $X$ as before, and let
      \[
        C(X) = \{f: X\to \R\text{ such that }f\text{ is continuous}\}.
      \]
      Since any continuous $f$ is bounded, so $C(X) \subseteq B(X)$. We define the norm as before.

      Since we know that $C(X)\subseteq B(X)$, to show that $C(X)$ is Banach, it suffices to show that $C(X) \subseteq B(X)$ is closed, ie. if $f_n \to f$, $f_n \in C(X)$, then $f\in C(X)$, ie. the uniform limit of a continuous function is continuous. Proof can be found in IB Analysis II.
    \item For $1 \leq p < \infty$, define
      \[
        \hat{L}_p ([0, 1]) = \{f: [0, 1] \to \R \text{ such that }f\text{ is continuous}\}.
      \]
      We define the norm $\|\cdot \|_{\hat{L}_p}$ by
      \[
        \|f\|_{\hat{L}_p} = \left(\int_0^1 |f|^p \;\d x\right)^{1/p}.
      \]
      It is easy to show that $\hat{L}_p$ is indeed a vector space, and we now check that this is a norm.
      \begin{enumerate}
        \item $\|f\|_{\hat{L}_p} \geq 0$ is obvious. Also, suppose that $\|f\|_{\hat{L}_p} = 0$. Then we must have $f = 0$. Otherwise, if $f\not = 0$, say $f(x) = \varepsilon$ for some $x$. Then there is some $\delta$ such that for any $y\in (x - \delta, x + \delta)$, we have $\|f(y)\| \geq \frac{\varepsilon}{2}$. Hence
          \[
            \|f\|_{\hat{L}_p} = \left(\int_0^1 |f|^p \;\d x\right)^{1/p} \geq \left[2\delta\left(\frac{\varepsilon}{2}\right)^p\right]^{1/p} > 0.
          \]
        \item $\|\lambda f\| = |\lambda|\|f\|$ is obvious
        \item The triangle inequality is the exactly what the Minkowski inequality says, which is in the example sheet.
      \end{enumerate}
      It turns out that $\hat{L}_p$ is \emph{not} a Banach space. We can brute-force a hard proof here, but we will later develop some tools that allow us to prove this much more easily.

      Hence, we define $L_p([0, 1])$ to be the completion of $\hat{L}_p ([0, 1])$. In IID Probability and Measure, we will show that $L_p([0, 1])$ is in fact the space
      \[
        L_p([0, 1]) = \left\{f: [0, 1] \to \R\text{ such that } \int_0^1 |f|^p \;\d x < \infty\right\}/{\sim},
      \]
      where the integral is the Lebesgue integral, and we are quotienting by the relation $f\sim g$ if $f = g$ Lebesgue almost everywhere. You will understand what these terms mean in the IID Probability and Measure course.

    \item $\ell_p$ spaces: for $p\in [1, \infty)$, define
        \[
          \ell_p (\F) = \left\{(x_1, x_2, \cdots): x_i \in \F, \sum_{i = 1}^\infty |x_i|^p < \infty\right\},
        \]
        with the norm
        \[
          \|\mathbf{x}\|_{\ell_p} = \left(\sum_{i = 1}^\infty |x_i|^p\right)^{1/p}.
        \]
        It should be easy to check that this is a normed vector space. Moreover, this is a Banach space. Proof is in example sheet.
      \item $\ell_\infty$ space: we define
        \[
          \ell_\infty = \left\{(x_1, x_2, \cdots): x_i\in \F, \sup_{i\in \N} |x_i| < \infty\right\}
        \]
        with norm
        \[
          \|\mathbf{x}\|_{\ell_\infty} = \sup_{i\in \N}|x_i|.
        \]
        Again, this is a Banach space.
      \item Let $B = B(1)$ be the unit open ball in $\R^n$. Define $C(B)$ to be the set of continuous functions $f: B\to \R$. Note that unlike in our previous example, these functions need not be bounded. So our previous norm cannot be applied. However, we can still define a topology as follows:

        Let $\{K_i\}_{i = 1}^\infty$ be a sequence of compact subsets of $B$ such that $K_i \subseteq K_{i + 1}$ and $\bigcup_{i = 1}^\infty = B$. We define the basis to include
        \[
          \left\{f \in C(B): \sup_{x \in K_i} |f(x)| < \frac{1}{m}\right\}
        \]
        for each $m, i = 1, 2, \cdots$, as well as the translations of these sets.

        This weird basis is chosen such that $f_n \to f$ in this topology iff $f_n \to f$ uniformly in every compact set. It can be showed that this is not normable.
  \end{enumerate}
\end{eg}

\subsection{Bounded linear maps}
Now let $X, Y$ be normed vector spaces, and $\mathcal{L}(X, Y)$ be the set of linear maps from $X$ to $Y$.

\begin{defi}[Bounded linear map]
  $T: X\to Y$ is a \emph{bounded linear map} if there is a constant $C > 0$ such that $\|Tx\|_Y \leq C\|x\|_X$ for all $x\in X$. We write $\mathcal{B}(X, Y)$ for the set of bounded linear maps from $X$ to $Y$.
\end{defi}
This is equivalent to saying $T(B_X(1)) \subseteq B_Y(C)$ for some $C > 0$. This also equivalent to saying that $T(B)$ is bounded for every bounded subset $B$ of $X$. Note that this final characterization is also valid when we just have a topological vector space.

How does boundedness relate to the topological structure of the vector spaces? It turns out that boundedness is the same as continuity.

\begin{prop}
  Let $X$, $Y$ be normed vector spaces, $T: X\to Y$ a linear map. Then the following are equivalent:
  \begin{enumerate}
    \item $T$ is continuous.
    \item $T$ is continuous at 0.
    \item $T$ is bounded.
  \end{enumerate}
\end{prop}

\begin{proof}
  (i) $\Rightarrow$ (ii) is obvious.

  (ii) $\Rightarrow $ (iii): Consider $B_Y(1) \subseteq Y$, the unit open ball. Since $T$ is continuous at $0$, $T^{-1}(B_Y(1))\subseteq X$ is open. Hence there exists $\varepsilon > 0$ such that $B_X(\varepsilon) \subseteq T^{-1}(B_Y(1))$. So $T(B_x(\varepsilon)) \subseteq B_Y(1)$. So $B_X(1) \subseteq B_Y\left(\frac{1}{\varepsilon}\right)$. So $T$ is bounded.

  (iii) $\Rightarrow$ (i): Let $\varepsilon > 0$. Then $\|T \mathbf{x}_1 - T \mathbf{x}_2\|_Y = \|T(\mathbf{x}_1 - \mathbf{x}_2)\|_Y \leq C\|\mathbf{x}_1 - \mathbf{x}_2\|_X$. This is less than $\varepsilon$ if $\|\mathbf{x}_1 - \mathbf{x}_2\| < C^{-1}\varepsilon$. So done.
\end{proof}

Using the obvious operations, $\mathcal{B}(X, Y)$ can be made a vector space. What about a norm?

\begin{defi}[Norm on $\mathcal{B}(X, Y)$]
  Let $T: X\to Y$ be a bounded linear map. Define $\|T\|_{\mathcal{B}(X, Y)}$ by
  \[
    \|T\|_{\mathcal{B}(X, Y)} = \sup_{\|x\| \leq 1} \|T \mathbf{x}\|_Y.
  \]
\end{defi}
Alternatively, this is the minimum $C$ such that $\|T\mathbf{x}\|_Y \leq C\|\mathbf{x}\|_X$ for all $\mathbf{x}$. In particular, we have
\[
  \|T\mathbf{x}\|_Y \leq \|T\|_{\mathcal{B}(X, Y)}\|\mathbf{x}\|_X.
\]
\subsection{Dual spaces}
We will frequently be interested in one particular case of $\mathcal{B}(X, Y)$.
\begin{defi}[Dual space]
  Let $V$ be a normed vector space. The \emph{dual space} is
  \[
    V^* = \mathcal{B}(V, \F).
  \]
  We call the elements of $V^*$ \emph{functionals}. The \emph{algebraic dual} of $V$ is
  \[
    V' = \mathcal{L}(V, \F),
  \]
  where we do not require boundedness.
\end{defi}

One particularly nice property of the dual is that $V^*$ is always a Banach space.

\begin{prop}
  Let $V$ be a normed vector space. Then $V^*$ is a Banach space.
\end{prop}

\begin{proof}
  Suppose $\{T_i\} \in V^*$ is a Cauchy sequence. We define $T$ as follows: for any $\mathbf{v}\in V$, $\{T_i(\mathbf{v})\}\subseteq \F$ is Cauchy sequence. Since $\F$ is complete (it is either $\R$ or $\C$), we can define $T: V\to \R$ by
  \[
    T(\mathbf{v}) = \lim_{n \to \infty}T_n (\mathbf{v}).
  \]
  Our objective is to show that $T_i \to T$. The first step is to show that we indeed have $T \in V^*$, ie. $T$ is a bounded map.

  Let $\|\mathbf{v}\| \leq 1$. Pick $\varepsilon = 1$. Then there is some $N$ such that for all $i > N$, we have
  \[
    |T_i(\mathbf{v}) - T(\mathbf{v})| < 1.
  \]
  Then we have
  \begin{align*}
    |T(\mathbf{v})| &\leq |T_i(\mathbf{v}) - T(\mathbf{v})| + |T_i(\mathbf{v})| \\
    &< 1 + \|T_i\|_{V^*}\|\mathbf{v}\|_V\\
    & \leq 1 + \|T_i\|_{V^*}\\
    &\leq 1 + \sup_i \|T_i\|_{V^*}
  \end{align*}
  Since $T_i$ is Cauchy, $\sup \|T_i\|_{V^*}$ is bounded. Since this bound does not depend on $\mathbf{v}$ (and $N$), we get that $T$ is bounded.

  Now we want to show that $\|T_i - T\|_{V^*} \to 0$ as $n\to \infty$.

  For arbitrary $\varepsilon > 0$, there is some $N$ such that for all $i, j > N$, we have
  \[
    \|T_i - T_j\|_{V^*} < \varepsilon.
  \]
  In particular, for any $\mathbf{v}$, we have
  \[
    |T_i(\mathbf{v}) - T_j(\mathbf{v})| < \varepsilon.
  \]
  Taking the limit as $j\to \infty$, we obtain
  \[
    |T_i(\mathbf{v}) - T(\mathbf{v})| \leq \varepsilon.
  \]
  Since this is true for any $\mathbf{v}$, we have
  \[
    \|T_i - T\|_{V^*} \leq \varepsilon.
  \]
  for all $i > N$. So $T_i \to T$.
\end{proof}
Exercise: in general, for $X, Y$ normed vector spaces, what condition on $X$ and $Y$ guarantees that $\mathcal{B}(X, Y)$ is a Banach space?

\subsection{Adjoint}
The idea of the adjoint is given a $T\in \mathcal{B}(X, Y)$, produce an \emph{adjoint} $T^*\in \mathcal{B}(Y^*, X^*)$.

There is really only one (non-trivial) natural way of doing this. First we can think about what $T^*$ should do. It takes in something from $Y^*$ and produces something in $X^*$. By the definition of the dual space, this is equivalent to taking in a function $g: Y \to \F$ and returning a function $T^*(g): X\to \F$.

To produce this $T^*(g)$, the only things we are allowed to use are $T: X\to Y$ and $g: Y\to \F$. Thus the only option we have is to define $T^*(g)$ as the composition $g\circ T$, or $T^*(g)(\mathbf{x}) = g(T(\mathbf{x}))$ (we also have a silly option of producing the zero map regardless of input, but this is silly). Indeed, this is the definition of the adjoint.

\begin{defi}[Adjoint]
  Let $X, Y$ be normal vector spaces. Given $T\in \mathcal{B}(X, Y)$, we define the \emph{adjoint} of $T$, denoted $T^*$, as a map $T^*\in \mathcal{B}(Y^*, X^*)$ given by
  \[
    T^*(g)(\mathbf{x}) = g(T(\mathbf{x}))
  \]
  for $\mathbf{x} \in X$, $y\in Y^*$. Alternatively, we can write
  \[
    T^*(g) = g\circ T.
  \]
\end{defi}
It is easy to show that our $T^*$ is indeed linear. We now show it is bounded.

\begin{prop}
  $T^*$ is bounded.
\end{prop}

\begin{proof}
  We want to show that $\|T^*\|_{\mathcal{B}(Y^*, X^*)}$ is finite. For simplicity, the supremum is assumed to be taken over non-zero elements of the space. We have
  \begin{align*}
    \|T^*\|_{\mathcal{B}(Y^*, X^*)} &= \sup_{g\in Y^*}\frac{\|T^*(g)\|_{X^*}}{\|g\|_{Y^*}}\\
    &= \sup_{g\in Y^*}\sup_{\mathbf{x}\in X}\frac{|T^*(g)(\mathbf{x})|/\|\mathbf{x}\|_X}{\|g\|_{Y^*}}\\
    &= \sup_{g\in Y^*}\sup_{\mathbf{x}\in X} \frac{|g(T\mathbf{x})|}{\|g\|_{Y^*}\|\mathbf{x}\|_X}\\
    &\leq \sup_{g\in Y^*}\sup_{\mathbf{x}\in X} \frac{\|g\|_{Y^*}\|T\mathbf{x}\|_Y}{\|g\|_{Y^*}\|\mathbf{x}\|_X}\\
    &\leq \sup_{\mathbf{x}\in X} \frac{\|T\|_{\mathcal{B}(X, Y)}\|\mathbf{x}\|_X}{\|\mathbf{x}\|_X}\\
    &= \|T\|_{\mathcal{B}(X, Y)}
  \end{align*}
  So it is finite.
\end{proof}

\subsection{The double dual}
\begin{defi}[Double dual]
  Let $V$ be a normed vector space. Define $V^{**} = (V^*)^*$.
\end{defi}

We want to define a map $\phi: V\to V^{**}$. Again, we can reason about what we expect this function to do. It takes in a $\mathbf{v}\in V$, and produces a $\phi(\mathbf{v}) \in V^{**}$. Expanding the definition, this gives a $\phi(\mathbf{v}): V^* \to \F$. Hence this $\phi(\mathbf{v})$ takes in a $g\in V^*$, and returns a $\phi(\mathbf{v})(g)\in \F$.

This is easy. Since $g \in V^*$, we know that $g$ is a function $g: V\to \F$. Given this function $g$ and a $\mathbf{v}\in V$, it is easy to produce a $\phi(\mathbf{v})(g)\in \F$. Just apply $g$ on $\mathbf{v}$:
\[
  \phi(v)(g) = g(v).
\]
\begin{prop}
  Let $\phi: V\to V^{**}$ be defined by $\phi(\mathbf{v})(g) = g(\mathbf{v})$. Then $\phi$ is a bounded linear map and $\|\phi\|_{\mathcal{B}(V, V^*)} \leq 1$
\end{prop}

\begin{proof}
  Again, we are taking supremum over non-zero elements. We have
  \begin{align*}
    \|\phi\|_{\mathcal{B}(V, V^*)} &= \sup_{\mathbf{v}\in V} \frac{\|\phi(\mathbf{v})\|_{V^{**}}}{\|\mathbf{v}\|_V}\\
    &= \sup_{\mathbf{v}\in V} \sup_{g\in V^*}\frac{|\phi(\mathbf{v})(g)|}{\|\mathbf{v}\|_V\|g\|_{V^*}}\\
    &= \sup_{\mathbf{v}\in V}\sup_{g\in V^*}\frac{|g(\mathbf{v})|}{\|\mathbf{v}\|_V\|g\|_{V^*}}\\
    &\leq 1.
  \end{align*}
\end{proof}
In fact, we will later show that $\|\phi\|_{\mathcal{B}(V, V^*)} = 1$.

\subsection{Isomorphism}
So far, we have discussed a lot about bounded linear maps, which are ``morphisms'' between normed vector spaces. It is thus natural to come up with the notion of isomorphism.

\begin{defi}[Isomorphism]
  Let $X, Y$ be normed vector spaces. Then $T: X\to Y$ is an \emph{isomorphism} if it is a bounded linear map with a bounded linear inverse (ie. it is a homeomorphism).

  We say $X$ and $Y$ are \emph{isomorphic} if there is an isomorphism $T: X\to Y$.

  We say that $T: X\to Y$ is an \emph{isometric} isomorphism if $T$ is an isomorphism and $\|T\mathbf{x}\|_Y = \|\mathbf{x}\|_X$ for all $\mathbf{x}\in X$.

  $X$ and $Y$ are \emph{isometrically isomorphic} if there is an isometric isomorphism between them.
\end{defi}

\begin{eg}
  Consider a finite-dimensional space $\F^n$ with standard basis $\{\mathbf{e}_1, \cdots, \mathbf{e}_n\}$. For any $\mathbf{v} = \sum v_i \mathbf{e}_i$, the norm is defined by
  \[
    \|\mathbf{v}\| = \left(\sum v_i^2\right)^{1/2}.
  \]
  Then any $g\in V^*$ is determined by $g(\mathbf{e}_i)$ for $i = 1, \cdots, n$. We want to show that there are no restrictions on what $g(\mathbf{e}_i)$ can be, ie. whatever values I assign to them, $g$ will still be bounded. We have
  \begin{align*}
    \|g\|_{V^*} &= \sup_{\mathbf{v}\in V}\frac{|g(\mathbf{v})|}{\|\mathbf{v}\|}\\
    &\leq \sup_{\mathbf{v}\in V}\frac{\sum |v_i||g(\mathbf{e}_i)|}{(\sum |v_i|^2)^{1/2}}\\
    &\leq C\sup_{\mathbf{v}\in V}\frac{(\sum |v_i|^2)^{\frac{1}{2}}}{(\sum|v_i|^2)^{\frac{1}{2}}}\left(\sup_i |g(\mathbf{e}_i)|\right)\\
    &= C\sup_i |g(\mathbf{e}_i)|
  \end{align*}
  for some $C$, where the second-to-last line is due to the Cauchy-Schwarz inequality.

  The supremum is finite since $\F^n$ is finite dimensional.

  So $g$ is uniquely determined by a list of values $(g(\mathbf{e}_1), g(\mathbf{e}_2), \cdots, g(\mathbf{e}_n))$. So it has dimension $n$. Therefore, $V^*$ is isomorphic to $\F^n$. By the same lines of argument, $V^{**}$ is isomorphic to $\F^n$.

  In fact, we can show that $\phi: V\to V^{**}$ by $\phi(\mathbf{v})(g) = g(\mathbf{v})$ is an isometric isomorphism (this is not true for general normed vector spaces. Just pick $V$ to be incomplete, then $V$ and $V^{**}$ cannot be isomorphic since $V^{**}$ is complete).
\end{eg}

\begin{eg}
  Consider $\ell_p$ for $p\in [1, \infty)$. What is $\ell_p^*$?

    Suppose $q$ is the \emph{conjugate exponent} of $p$, ie.
    \[
      \frac{1}{q} + \frac{1}{p} = 1.
    \]
    (if $p = 1$, define $q = \infty$) It is easy to see that $\ell_q \subseteq \ell_p^*$ by the following:

    Suppose $(x_1, x_2, \cdots) \in \ell_p$, and $(y_1, y_2, \cdots)\in \ell_q$. Define $y(\mathbf{x}) = \sum_{i = 1}^\infty x_i y_i$. We will show that $y$ defined this way is a bounded linear map. Linearity is easy to see, and boundedness comes from the fact that
    \[
      \|y\|_{\ell_p^*} = \sup_{\mathbf{x}}\frac{\|y(\mathbf{x})\|}{\|\mathbf{x}\|_{\ell_p}} = \sup_{\mathbf{x}} \frac{\sum x_i y_i}{\|\mathbf{x}\|_{\ell_p}} \leq \sup \frac{\|\mathbf{x}\|_{\ell_p}\|\mathbf{y}\|_{\ell_q}}{\|\mathbf{x}\|_{\ell_p}} = \|\mathbf{y}\|_{\ell_p},
    \]
    by the H\"older's inequality. So every $(y_i) \in \ell_q$ determines a bounded linear map. In fact, we can show $\ell_p$ is isomorphic to $\ell_q$.
\end{eg}

\subsection{Finite-dimensional normed vector spaces}
We are now going to look at a special case of normed vector spaces, where the vector space is finite dimensional.

It turns out that finite-dimensional vector spaces have some rather special properties:
\begin{enumerate}
  \item All norms are equivalent.
  \item The closed unit ball is compact.
  \item They are Banach spaces.
  \item All linear maps whose domain is finite dimensional are bounded.
\end{enumerate}
These are what we are going to show in this section.

First of all, we need to say what we mean when we say all norms are ``equivalent''
\begin{defi}[Equivalent norms]
  Let $V$ be a vector space, and $\|\cdot \|_1$, $\|\cdot \|_2$ be norms on $V$. We say that these are equivalent if there exists a constant $C > 0$ such that for any $\mathbf{v}\in V$, we have
  \[
    C^{-1}\|\mathbf{v}\|_2 \leq \|\mathbf{v}\|_1 \leq C\|\mathbf{v}\|_2.
  \]
\end{defi}
It is an exercise to show that equivalent norms induce the same topology, and hence agree on continuity and convergence.

Now let $V$ be an $n$-dimensional vector space with basis $\{\mathbf{e}_1, \cdots, \mathbf{e}_n\}$. We can define the $\ell_p^n$ norm by
\[
  \|\mathbf{v}\|_{\ell_p^n} = \left(\sum_{i = 1}^n |v_i|^p \right)^{1/p},
\]
where
\[
  \mathbf{v} = \sum_{i = 1}^n v_i \mathbf{e}_i.
\]
\begin{prop}
  Let $V$ be an $n$-dimensional vector space. Then all norms on $V$ are equivalent to the norm $\|\cdot\|_{\ell_1^n}$.
\end{prop}

\begin{cor}
  All norms on a finite-dimensional vector space are equivalent.
\end{cor}

\begin{proof}
  Let $\|\cdot \|$ be a norm on $V$.

  Let $\mathbf{v} = (v_1, \cdots, v_n) = \sum v_i \mathbf{e}_i \in V$. Then we have
  \begin{align*}
    \|\mathbf{v}\| &= \left\|\sum v_i \mathbf{e}_i\right\|\\
    &\leq \sum_{i = 1}^n |v_i|\|\mathbf{e}_i\|\\
    &\leq \left(\sup_i \|\mathbf{e}_i\|\right) \sum_{i = 1}^n |v_i|\\
    &\leq C\|\mathbf{v}\|_{\ell_1^n},
  \end{align*}
  where $c = \sup \|\mathbf{e}_i\| < \infty$ since we are taking a finite supremum.

  For the other way round, let $S_1 = \{\mathbf{v}\in v: \|\mathbf{v}\|_{\ell_1^n} = 1\}$. We will show the two following results:
  \begin{enumerate}
    \item $\|\cdot \|: (S_1, \|\cdot \|_{\ell_1^*}) \to \R$
    \item $S_1$ is a compact set.
  \end{enumerate}
  We first see why this gives what we want. We know that for any continuous map from a compact set to $\R$, the image is bounded and the infimum is achieved. So there is some $\mathbf{v}_* \in S_1$ such that
  \[
    \|\mathbf{v}_*\| = \inf_{\mathbf{v}\in S_1} \|\mathbf{v}\|.
  \]
  Since $\mathbf{v}_*\not= 0$, there is some $c'$ such that $\|\mathbf{v}\| \geq c'$ for all $\mathbf{v} \in S_1$.

  Now take an arbitrary non-zero $\mathbf{v} \in V$, since $\frac{\mathbf{v}}{\|\mathbf{v}\|_{\ell_1^n}} \in S_1$, we know that
  \[
    \left\|\frac{\mathbf{v}}{\|\mathbf{v}\|_{\ell_1^n}}\right\| \geq c',
  \]
  which is to say that
  \[
    \|\mathbf{v}\| \geq c' \|\mathbf{v}\|_{\ell_1^n}.
  \]
  Since we have found $c, c' > 0$ such that
  \[
    c' \|\mathbf{v}\|_{\ell_1^n} \leq \|\mathbf{v}\|\leq c \| \mathbf{v}\|_{\ell_1^n},
  \]
  now let $C = \max\left\{c, \frac{1}{c'}\right\} > 0$. Then
  \[
    C^{-1}\|\mathbf{v}\|_2 \leq \|\mathbf{v}\|_1 \leq C\|\mathbf{v}\|_2.
  \]
  So the norms are equivalent. Now we can start to prove (i) and (ii).

  First, let $\mathbf{v}, \mathbf{w}\in V$. We have
  \[
    \big|\|\mathbf{v}\| - \|\mathbf{w}\|\big| \leq \|\mathbf{v} - \mathbf{w}\| \leq C\|\mathbf{v} - \mathbf{w}\|_{\ell_1^n}.
  \]
  Hence when $\mathbf{v}$ is close to $\mathbf{w}$ under $\ell_1^n$, then $\|\mathbf{v}\|$ is close to $\|\mathbf{w}\|$. So it is continuous.

  To show (ii), it suffices to show that the unit ball $B = \{\mathbf{v} \in V: \|\mathbf{v}\|_{\ell_1^n}\leq 1\}$ is compact, since $S_1$ is a closed subset of $B$. We will do so by showing it is sequentially compact.

  Let $\{\mathbf{v}^{(k)}\}_{k = 1}^\infty$ be a sequence in $B$. Write
  \[
    \mathbf{v}^{(k)} = \sum_{i = 1}^n \lambda_i^{(k)} \mathbf{e}_i.
  \]
  Since $\mathbf{v}^{(k)} \in B$, we have
  \[
    \sum_{i = 1}^n |\lambda_i^{(k)}| \leq 1.
  \]
  Consider the sequence $\lambda_1^{(k)}$, which is a sequence in $\F$.

  We know that $|\lambda_1^{(k)}| \leq 1$. So by Bolzano-Weierstrass, there is a convergent subsequence $\lambda_1^{(k_{j_1})}$.

  Now look at $\lambda_2^{(k_{j_1})}$. Since this is bounded, there is a convergent subsequence $\lambda_2^{(k_{j_2})}$.

  Iterate this for all $n$ to obtain a sequence $k_{j_n}$ such that $\lambda_i^{(k_{j_n})}$ is convergent for all $i$. So $\mathbf{v}^{(k_{j_n})}$ is a convergent subsequence.
\end{proof}

\begin{prop}
  Let $V$ be a finite-dimensional normed vector space. Then the closed unit ball
  \[
    \bar{B}(1) = \{\mathbf{v} \in V: \|\mathbf{v}\| \leq 1\}
  \]
  is compact.
\end{prop}

\begin{proof}
  This follows from the proof above.
\end{proof}

\begin{prop}
  Let $V$ be a finite-dimensional normed vector space. Then $V$ is a Banach space.
\end{prop}

\begin{proof}
  Let $\{\mathbf{v}_i\} \in V$ be a Cauchy sequence. Since $\{\mathbf{v}_i\}$ is Cauchy, it is bounded, ie. $\{\mathbf{v}_i\} \subseteq \bar{B}(R)$ for some $R > 0$. By above, $\bar{B}(R)$ is compact. So $\{\mathbf{v}_i\}$ has a convergent subsequence $\mathbf{v}_{i_k} \to \mathbf{v}$. Since $\{\mathbf{v}_i\}$ is Cauchy, we must have $\mathbf{v}_i \to \mathbf{v}$. So $\mathbf{v}_i$ converges.
\end{proof}

\begin{prop}
  Let $V, W$ be normed vector spaces, $V$ be finite-dimensional. Also, let $T: V\to W$ be a linear map. Then $T$ is bounded.
\end{prop}

\begin{proof}
  Recall discussions last time about regarding $V^*$ for finite-dimensional $V$. We will do a similar proof.

  Note that since $V$ is finite-dimensional, $\im T$ finite dimensional. So wlog $W$ is finite-dimensional. Since all norms are equivalent, it suffices to consider the case where the vector spaces have $\ell_1^n$ and $\ell_1^m$ norm. This can be represented by a matrix
  \[
    T(x_1, \cdots, x_n) = \left(\sum T_{1i}x_i, \cdots, \sum T_{mi}x_i\right).
  \]
  We can bound this by
  \[
    \|T(x_1, \cdots, x_n)\| \leq \sum_{j = 1}^m \sum_{i = 1}^n |T_{ji}||x_i|\\ \leq m \left(\sup_{i, j}|T_{ij}|\right) \sum_{i = 1}^n |x_i| \leq C \|\mathbf{x}\|_{\ell_1^n}
  \]
  for some $C > 0$, since we are taking the supremum over a finite set. This implies that $\|T\|_{\mathcal{B}(\ell_1^n, \ell_1^m)} \leq C$.
\end{proof}
\end{document}
