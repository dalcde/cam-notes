\documentclass[a4paper]{article}

\def\npart {IV}
\def\nterm {Easter}
\def\nyear {2017}
\def\nlecturer {M.\ Burger}
\def\ncourse {Bounded Cohomology}

% Imports
\ifx \nextra \undefined
  \usepackage[pdftex,
    hidelinks,
    pdfauthor={Dexter Chua},
    pdfsubject={Cambridge Maths Notes: Part \npart\ - \ncourse},
    pdftitle={Part \npart\ - \ncourse},
  pdfkeywords={Cambridge Mathematics Maths Math \npart\ \nterm\ \nyear\ \ncourse}]{hyperref}
  \title{Part \npart\ - \ncourse}
\else
  \usepackage[pdftex,
    hidelinks,
    pdfauthor={Dexter Chua},
    pdfsubject={Cambridge Maths Notes: Part \npart\ - \ncourse\ (\nextra)},
    pdftitle={Part \npart\ - \ncourse\ (\nextra)},
  pdfkeywords={Cambridge Mathematics Maths Math \npart\ \nterm\ \nyear\ \ncourse\ \nextra}]{hyperref}

  \title{Part \npart\ - \ncourse \\ {\Large \nextra}}
\fi

\author{Lectured by \nlecturer \\\small Notes taken by Dexter Chua}
\date{\nterm\ \nyear}

\usepackage{alltt}
\usepackage{amsfonts}
\usepackage{amsmath}
\usepackage{amssymb}
\usepackage{amsthm}
\usepackage{booktabs}
\usepackage{caption}
\usepackage{enumitem}
\usepackage{fancyhdr}
\usepackage{graphicx}
\usepackage{mathtools}
\usepackage{microtype}
\usepackage{multirow}
\usepackage{pdflscape}
\usepackage{pgfplots}
\usepackage{siunitx}
\usepackage{tabularx}
\usepackage{tikz}
\usepackage{tkz-euclide}
\usepackage[normalem]{ulem}
\usepackage[all]{xy}

\pgfplotsset{compat=1.12}

\pagestyle{fancyplain}
\lhead{\emph{\nouppercase{\leftmark}}}
\ifx \nextra \undefined
  \rhead{
    \ifnum\thepage=1
    \else
      \npart\ \ncourse
    \fi}
\else
  \rhead{
    \ifnum\thepage=1
    \else
      \npart\ \ncourse\ (\nextra)
    \fi}
\fi
\usetikzlibrary{arrows}
\usetikzlibrary{decorations.markings}
\usetikzlibrary{decorations.pathmorphing}
\usetikzlibrary{positioning}
\usetikzlibrary{fadings}
\usetikzlibrary{intersections}
\usetikzlibrary{cd}

\newcommand*{\Cdot}{\raisebox{-0.25ex}{\scalebox{1.5}{$\cdot$}}}
\newcommand {\pd}[2][ ]{
  \ifx #1 { }
    \frac{\partial}{\partial #2}
  \else
    \frac{\partial^{#1}}{\partial #2^{#1}}
  \fi
}

% Theorems
\theoremstyle{definition}
\newtheorem*{aim}{Aim}
\newtheorem*{axiom}{Axiom}
\newtheorem*{claim}{Claim}
\newtheorem*{cor}{Corollary}
\newtheorem*{defi}{Definition}
\newtheorem*{eg}{Example}
\newtheorem*{fact}{Fact}
\newtheorem*{law}{Law}
\newtheorem*{lemma}{Lemma}
\newtheorem*{notation}{Notation}
\newtheorem*{prop}{Proposition}
\newtheorem*{thm}{Theorem}

\renewcommand{\labelitemi}{--}
\renewcommand{\labelitemii}{$\circ$}
\renewcommand{\labelenumi}{(\roman{*})}

\let\stdsection\section
\renewcommand\section{\newpage\stdsection}

% Strike through
\def\st{\bgroup \ULdepth=-.55ex \ULset}

% Maths symbols
\newcommand{\bra}{\langle}
\newcommand{\ket}{\rangle}

\newcommand{\N}{\mathbb{N}}
\newcommand{\Z}{\mathbb{Z}}
\newcommand{\Q}{\mathbb{Q}}
\renewcommand{\H}{\mathbb{H}}
\newcommand{\R}{\mathbb{R}}
\newcommand{\C}{\mathbb{C}}
\newcommand{\Prob}{\mathbb{P}}
\renewcommand{\P}{\mathbb{P}}
\newcommand{\E}{\mathbb{E}}
\newcommand{\F}{\mathbb{F}}
\newcommand{\cU}{\mathcal{U}}
\newcommand{\RP}{\mathbb{RP}}
\newcommand{\CP}{\mathbb{CP}}

\newcommand{\ph}{\,\cdot\,}

\DeclareMathOperator{\sech}{sech}
\DeclareMathOperator{\cosech}{cosech}
\DeclareMathOperator{\cosec}{cosec}

\DeclareMathOperator{\covol}{covol}
\DeclareMathOperator{\vol}{vol}

\let\Im\relax
\let\Re\relax
\DeclareMathOperator{\Im}{Im}
\DeclareMathOperator{\Re}{Re}
\DeclareMathOperator{\im}{im}
\DeclareMathOperator{\image}{image}
\DeclareMathOperator{\Ann}{Ann}

\DeclareMathOperator*{\res}{res}
\DeclareMathOperator{\Res}{Res}
\DeclareMathOperator{\Ind}{Ind}

\DeclareMathOperator{\tr}{tr}
\DeclareMathOperator{\diag}{diag}
\DeclareMathOperator{\rank}{rank}
\DeclareMathOperator{\card}{card}
\DeclareMathOperator{\spn}{span}
\DeclareMathOperator{\adj}{adj}

\DeclareMathOperator{\erf}{erf}
\DeclareMathOperator{\erfc}{erfc}

\DeclareMathOperator{\ord}{ord}
\DeclareMathOperator{\Sym}{Sym}

\DeclareMathOperator{\sgn}{sgn}
\DeclareMathOperator{\orb}{orb}
\DeclareMathOperator{\stab}{stab}
\DeclareMathOperator{\ccl}{ccl}

\DeclareMathOperator{\lcm}{lcm}
\DeclareMathOperator{\hcf}{hcf}

\DeclareMathOperator{\Int}{Int}
\DeclareMathOperator{\id}{id}

\DeclareMathOperator{\betaD}{beta}
\DeclareMathOperator{\gammaD}{gamma}
\DeclareMathOperator{\Poisson}{Poisson}
\DeclareMathOperator{\binomial}{binomial}
\DeclareMathOperator{\multinomial}{multinomial}
\DeclareMathOperator{\Bernoulli}{Bernoulli}
\DeclareMathOperator{\like}{like}

\DeclareMathOperator{\var}{var}
\DeclareMathOperator{\cov}{cov}
\DeclareMathOperator{\bias}{bias}
\DeclareMathOperator{\mse}{mse}
\DeclareMathOperator{\corr}{corr}

\DeclareMathOperator{\otp}{otp}
\DeclareMathOperator{\dom}{dom}

\DeclareMathOperator{\Root}{Root}
\DeclareMathOperator{\supp}{supp}
\DeclareMathOperator{\rel}{rel}
\DeclareMathOperator{\Hom}{Hom}
\DeclareMathOperator{\Aut}{Aut}
\DeclareMathOperator{\Gal}{Gal}
\DeclareMathOperator{\Mat}{Mat}
\DeclareMathOperator{\End}{End}
\DeclareMathOperator{\Char}{char}
\DeclareMathOperator{\ev}{ev}
\DeclareMathOperator{\St}{St}
\DeclareMathOperator{\Lk}{Lk}
\DeclareMathOperator{\disc}{disc}
\DeclareMathOperator{\Isom}{Isom}
\DeclareMathOperator{\length}{length}
\DeclareMathOperator{\energy}{energy}
\DeclareMathOperator{\area}{area}
\DeclareMathOperator{\Syl}{Syl}
\DeclareMathOperator{\cl}{cl}
\DeclareMathOperator{\fix}{fix}

\newcommand{\GL}{\mathrm{GL}}
\newcommand{\SL}{\mathrm{SL}}
\newcommand{\PGL}{\mathrm{PGL}}
\newcommand{\PSL}{\mathrm{PSL}}
\newcommand{\PSU}{\mathrm{PSU}}
\newcommand{\Or}{\mathrm{O}}
\newcommand{\SO}{\mathrm{SO}}
\newcommand{\U}{\mathrm{U}}
\newcommand{\SU}{\mathrm{SU}}

\renewcommand{\d}{\mathrm{d}}
\newcommand{\D}{\mathrm{D}}

\tikzset{->/.style = {decoration={markings,
                                  mark=at position 1 with {\arrow[scale=2]{latex'}}},
                      postaction={decorate}}}
\tikzset{<-/.style = {decoration={markings,
                                  mark=at position 0 with {\arrowreversed[scale=2]{latex'}}},
                      postaction={decorate}}}
\tikzset{<->/.style = {decoration={markings,
                                   mark=at position 0 with {\arrowreversed[scale=2]{latex'}},
                                   mark=at position 1 with {\arrow[scale=2]{latex'}}},
                       postaction={decorate}}}
\tikzset{->-/.style = {decoration={markings,
                                   mark=at position #1 with {\arrow[scale=2]{latex'}}},
                       postaction={decorate}}}
\tikzset{-<-/.style = {decoration={markings,
                                   mark=at position #1 with {\arrowreversed[scale=2]{latex'}}},
                       postaction={decorate}}}

\tikzset{circ/.style = {fill, circle, inner sep = 0, minimum size = 3}}
\tikzset{mstate/.style={circle, draw, blue, text=black, minimum width=0.7cm}}

\definecolor{mblue}{rgb}{0.2, 0.3, 0.8}
\definecolor{morange}{rgb}{1, 0.5, 0}
\definecolor{mgreen}{rgb}{0.1, 0.4, 0.2}
\definecolor{mred}{rgb}{0.5, 0, 0}

\def\drawcirculararc(#1,#2)(#3,#4)(#5,#6){%
    \pgfmathsetmacro\cA{(#1*#1+#2*#2-#3*#3-#4*#4)/2}%
    \pgfmathsetmacro\cB{(#1*#1+#2*#2-#5*#5-#6*#6)/2}%
    \pgfmathsetmacro\cy{(\cB*(#1-#3)-\cA*(#1-#5))/%
                        ((#2-#6)*(#1-#3)-(#2-#4)*(#1-#5))}%
    \pgfmathsetmacro\cx{(\cA-\cy*(#2-#4))/(#1-#3)}%
    \pgfmathsetmacro\cr{sqrt((#1-\cx)*(#1-\cx)+(#2-\cy)*(#2-\cy))}%
    \pgfmathsetmacro\cA{atan2(#2-\cy,#1-\cx)}%
    \pgfmathsetmacro\cB{atan2(#6-\cy,#5-\cx)}%
    \pgfmathparse{\cB<\cA}%
    \ifnum\pgfmathresult=1
        \pgfmathsetmacro\cB{\cB+360}%
    \fi
    \draw (#1,#2) arc (\cA:\cB:\cr);%
}
\newcommand\getCoord[3]{\newdimen{#1}\newdimen{#2}\pgfextractx{#1}{\pgfpointanchor{#3}{center}}\pgfextracty{#2}{\pgfpointanchor{#3}{center}}}

\def\Xint#1{\mathchoice
   {\XXint\displaystyle\textstyle{#1}}%
   {\XXint\textstyle\scriptstyle{#1}}%
   {\XXint\scriptstyle\scriptscriptstyle{#1}}%
   {\XXint\scriptscriptstyle\scriptscriptstyle{#1}}%
   \!\int}
\def\XXint#1#2#3{{\setbox0=\hbox{$#1{#2#3}{\int}$}
     \vcenter{\hbox{$#2#3$}}\kern-.5\wd0}}
\def\ddashint{\Xint=}
\def\dashint{\Xint-}


\newcommand\QH{\mathcal{QH}}

\begin{document}
\maketitle
{\small
\setlength{\parindent}{0em}
\setlength{\parskip}{1em}

The cohomology of a group or a topological space in degree $k$ is a real vector space which describes the ``holes'' bounded by k dimensional cycles and encodes their relations. Bounded cohomology is a refinement which provides these vector spaces with a (semi) norm and hence topological objects acquire mysterious numerical invariants. This theory, introduced in the beginning of the 80's by M.\ Gromov, has deep connections with the geometry of hyperbolic groups and negatively curved manifolds. For instance, hyperbolic groups can be completely characterized by the ``size'' of their bounded cohomology.

The aim of this course is to give an introduction to the bounded cohomology of groups, and treat more in detail one of its important applications to the study of groups acting by homeomorphisms on the circle. More precisely we will treat the following topics:
\begin{enumerate}
  \item Ordinary and bounded cohomology of groups: meaning of these objects in low degrees, that is, zero, one and two; relations with quasimorphisms. Proof that the bounded cohomology in degree two of a non abelian free group contains an isometric copy of the Banach space of bounded sequences of reals. Examples and meaning of bounded cohomology classes of geometric origin with non trivial coefficients.

  \item  Actions on the circle, the bounded Euler class: for a group acting by orientation preserving homeomorphisms of the circle, Ghys has introduced an invariant, the bounded Euler class of the action, and shown that it characterizes (minimal) actions up to conjugation. We will treat in some detail this work as it leads to important applications of bounded cohomology to the question of which groups can act non trivially on the circle: for instance $\SL(2,\Z)$ can, while lattices in ``higher rank Lie groups'' , like $\SL(n,\Z)$ for $n$ at least $3$, can't.

  \item Amenability and resolutions: we will set up the abstract machinery of resolutions and the notions of injective modules in ordinary as well as bounded cohomology; this will provide a powerful way to compute these objects in important cases. A fundamental role in this theory is played by various notions of amenability; the classical notion of amenability for a group, and amenability of a group action on a measure space, due to R.\ Zimmer. The goal is then to describe applications of this machinery to various rigidity questions, and in particular to the theorem due, independently to Ghys, and Burger--Monod, that lattices in higher rank groups don't act on the circle.
\end{enumerate}

\subsubsection*{Pre-requisites}
Prerequisites for this course are minimal: no prior knowledge of group cohomology of any form is needed; we'll develop everything we need from scratch. It is however an advantage to have a ``zoo'' of examples of infinite groups at one's disposal: for example free groups and surface groups. In the third part, we'll need basic measure theory; amenability and ergodic actions will play a role, but there again everything will be built up on elementary measure theory.

The basic reference for this course is R.\ Frigerio, ``Bounded cohomology of discrete groups'', \href{https://arxiv.org/abs/1611.08339}{arXiv:1611.08339}, and for part 3, M.\ Burger \& A.\ Iozzi, ``A useful formula from bounded cohomology'', available at: \url{https://people.math.ethz.ch/~iozzi/publications.html}.%
}
\tableofcontents

\section{Quasi-homomorphisms}
In this chapter, $A$ will denote $\Z$ or $\R$. The idea is as follows. Let $G$ be a group. The usual definition of a group homomorphism $f: G \to A$ requires that for all $x, y \in G$, we have
\[
  f(xy) = f(x) + f(y).
\]
In a quasi-homomorphism, we replace the equality with a weaker notion, and allow for some ``errors''.

\begin{defi}[Quasi-homomorphism]\index{quasi-homomorphism}
  Let $G$ be a group. A function $f: G \to A$ is a \emph{quasi-homomorphism} if the function
  \begin{align*}
    \d f: G \times G &\to A\\
    (x, y) &\mapsto f(x) + f(y) - f(xy)
  \end{align*}
  is \emph{bounded}. We define the \term{defect}\index{quasi-homomorphism!defect} of $f$ to be
  \[
    D(f) = \sup_{x, y \in G} |\d f(x, y)|.
  \]
  We write \term{$\QH(G, A)$} for the $A$-module of quasi-homomorphisms.
\end{defi}

\begin{eg}
  Every homomorphism is a quasi-homomorphism with $D(f) = 0$. Conversely, a quasi-homomorphism with $D(f) = 0$ is a homomorphism.
\end{eg}

We can obtain some ``trivial'' quasi-homomorphisms as follows --- we take any homomorphism, and then edit finitely many values of the homomorphism finitely. Then this is a quasi-homomorphism. More generally, we can add any bounded function to a quasi-homomorphism and still get a quasi-homomorphism.

\begin{notation}
  We write\index{$\ell^\infty(G, A)$}
  \[
    \ell^\infty(G, A) = \{f: G \to A: \text{$f$ is bounded}\}.
  \]
\end{notation}

Thus, we are largely interested in the quasi-homomorphisms modulo $\ell^\infty(G, A)$. Often, we also want to quotient out by the homomorphisms, and obtain
\[
  \frac{\QH(G, A)}{\ell^\infty(G, A) + \Hom(G, A)}.
\]
This contains subtle algebraic and geometric information about $G$, and is related to the second bounded cohomology $H_b^2(G, A)$.

We first prove a few elementary facts about quasi-homomorphisms. The first task is to find canonical representatives of the classes in the quotient $\QH(G, \R)/\ell^\infty (G, \R)$.

\begin{defi}[Homogeneous function]\index{homegeneous function}\index{function!homogeneous}
  A function $f: G \to \R$ is \emph{homogeneous} if for all $n \in \Z$ and $g \in G$, we have $f(g^n) = n f(g)$.
\end{defi}

\begin{lemma}
  Let $f \in \QH(G, A)$. Then for every $g \in G$, the limit
  \[
    Hf(g) = \lim_{n \to \infty} \frac{f(g^n)}{n}
  \]
  exists in $\R$. Moreover,
  \begin{enumerate}
    \item $Hf: G \to \R$ is a homogeneous quasi-homomorphism.
    \item $f - Hf \in \ell^\infty(G, \R)$.
  \end{enumerate}
\end{lemma}
% So this $Hf$ gives us a ``preferred'' representative for each class in$\QH(G, \R) / \ell^\infty(G, \R)$.

\begin{proof}
  We iterate the quasi-homomorphism property
  \[
    |f(ab) - f(a) - f(b)| \leq D(f).
  \]
  Then, viewing $g^{mn} = g^m \cdots g^m$, we obtain
  \[
    |f(g^{mn}) - n f(g^m)| \leq (n - 1) D(f).
  \]
  Similarly, we also have
  \[
    |f(g^{mn}) -m f(g^n)| \leq (m - 1) D(f).
  \]
  Thus, dividing by $nm$, we find
  \begin{align*}
    \left|\frac{f(g^{mn})}{nm}  - \frac{f(g^m)}{m}\right| &\leq \frac{1}{m} D(f)\\
    \left|\frac{f(g^{mn})}{nm}  - \frac{f(g^n)}{n}\right| &\leq \frac{1}{n} D(f).
  \end{align*}
  So we find that
  \[
    \left|\frac{f(g^n)}{n} - \frac{f(g^m)}{m}\right| \leq \left(\frac{1}{m} + \frac{1}{n} \right) D(f).\tag{$*$}
  \]
  Hence the sequence $\frac{f(g^n)}{n}$ is Cauchy, and the limit exists.

  The fact that $Hf$ is a quasi-homomorphism follows from the second assertion. To prove the second assertion, we can just take $n = 1$ in $(*)$ and take $m \to \infty$. Then we find
  \[
    |f(g) - Hf(g)| \leq D(f).
  \]
  So this shows that $f - Hf$ is bounded, hence $Hf$ is a quasi-homomorphism.

  The homogeneity is left as an easy exercise.
\end{proof}

\begin{notation}
  We write \term{$\QH_h(G, \R)$} for the vector space of homogeneous quasi-homomorphisms $G \to \R$.
\end{notation}

Then the above theorem gives
\begin{cor}
  We have
  \[
    \QH(G, \R) = \QH_h(G, \R) \oplus \ell^\infty(G, \R)
  \]
\end{cor}

\begin{proof}
  Indeed, observe that a bounded homogeneous quasi-homomorphism must be identically zero.
\end{proof}

Thus, if we want to study $\QH(G, \R)$, it suffices to just look at the homogeneous quasi-homomorphisms. It turns out these have some very nice perhaps-unexpected properties.
\begin{lemma}
  Let $f: G \to \R$ be a homogeneous quasi-homomorphism.
  \begin{enumerate}
    \item We have $f(xyx^{-1}) = f(y)$ for all $x, y \in G$.
    \item If $G$ is abelian, then $f$ is in fact a homomorphism. Thus
      \[
        \QH_h(G, \R) = \Hom(G, \R).
      \]
  \end{enumerate}
\end{lemma}
Thus, quasi-homomorphisms are only interesting for non-abelian groups.

\begin{proof}\leavevmode
  \begin{enumerate}
    \item Note that for any $x$, the function
      \[
        y \mapsto f(xyx^{-1})
      \]
      is a homogeneous quasi-homomorphism. By the previous corollary, it suffices to show that the function
      \[
        y \mapsto f(xyx^{-1}) - f(y)
      \]
      is a bounded homogeneous quasi-homomorphism, hence zero. This is just an iteration of the quasi-homomorphism property. We have
      \[
        |f(xyx^{-1}) - f(y)| \leq |f(x) + f(y) + f(x^{-1}) - f(y)| + 2D(f) = 2D(f),
      \]
      using the fact that $f(x^{-1}) = -f(x)$ by homogeneity.
    \item If $x$ and $y$ commute, then $(xy)^n = x^n y^n$. So we can use homogeneity to write
      \begin{align*}
        |f(xy) - f(x) - f(y)| &= \frac{1}{n} |f((xy)^n) - f(x^n) - f(y^n)|\\
        &= \frac{1}{n} | f(x^n y^n) - f(x^n) - f(y^n)|\\
        &\leq \frac{1}{n} D(f).
      \end{align*}
      Since $n$ is arbitrary, the difference must vanish.
  \end{enumerate}
\end{proof}

The case of $\QH(G, \Z)/\ell^\infty(G, \Z)$ is more complicated. For example, we have the following nice result?
\begin{ex}
  Show that we have a natural isomorphism
  \[
    \frac{\QH(\Z, \Z)}{\ell^\infty(\Z, \Z)} \cong \R
  \]
  as rings, where the multiplication on the left is composition. In particular, multiplication is commutative.
\end{ex}
%This is strange! We constructed $\R$ with the additive and multiplicative structures, just using $\Z$ and quasi-homomorphisms! This leads to some interesting phenomena.

We now focus on a particular example, and try to work out explicitly some non-trivial elements of $\QH(G, \R)$. In most constructions, the quasi-homomorphisms we construct are \emph{not} homogeneous. So when we construct several quasi-homomorphisms, it takes some non-trivial work to show that they are distinct

We pick $G = \F_2$, the free group on generators $a$, $b$. We will show that $\QH(\F_2, \R)$ is huge. This involves considering the following vector space:
\[
  \ell_{\mathrm{odd}}^\infty (\Z) = \{\alpha: \Z \to \R : \alpha \text{ bounded and } \alpha(-n) = -\alpha (n)\}.
\]
Note that in particular, we have $\alpha(0) = 0$.

Given $\alpha, \beta \in \ell_{\mathrm{odd}}^\infty(\Z)$, we define a quasi-homomorphisms $f_{\alpha, \beta} : \F_2 \to \R$ as follows --- given a reduced word $w = a^{n_1} b^{m_1} \cdots a^{n_k}b^{m_k}$, we let
\[
  f_{\alpha, \beta}(w) = \sum_{i=  1}^k \alpha(n_i) + \sum_{j = 1}^k \beta(m_j).
\]
Allowing for $n_1 = 0$ or $m_k = 0$, this gives a well-defined function $f_{\alpha,\beta}$ defined on all of $\F_2$.

Let's see what this does on some special sequences.
\begin{eg}
  We have
  \[
    f_{\alpha, \beta}(a^n) = \alpha(n),\quad f_{\alpha, \beta}(b^m) = \beta(m),
  \]
  and these are bounded functions on $n, m$.
\end{eg}
So we see that $f_{\alpha, \beta}$ is never homogeneous unless $\alpha = \beta = 0$.

\begin{eg}
  Pick $k_1, k_2, n \not= 0$, and set
  \[
    w = a^{nk_1} b^{nk_2} (b^{k_2}a^{k_1})^{-n} = a^{nk_1} b^{nk_2} \underbrace{a^{-k_1} b^{-k_2} \cdots a^{-k_1} b^{-k_2}}_{n\text{ times}}.
  \]
  This is now in reduced form. So we have
  \[
    f_{\alpha, \beta}(w) = \alpha(n k_1) + \beta (n k_2) - n \alpha(k_1) - n \beta(k_2).
  \]
\end{eg}
This example is important. If $\alpha(k_1) + \beta(k_2) \not= 0$, then this is an unbounded function as $n \to \infty$. However, we know any genuine homomoprhisms $f: \F_2 \to \R$ must factor through the abelianization, but $w$ vanishes in the abelianization. So this suggests our $f_{\alpha, \beta}$ is in some sense very far away from being a homomorphism.

\begin{thm}[P.\ Rolli, 2009]
  The function $f_{\alpha, \beta}$ is a quasi-homomorphism, and the map 
  \[
    \ell^\infty_{\mathrm{odd}} (\Z) \oplus \ell^\infty_{\mathrm{odd}}(\Z) \to \frac{\QH(\F_2, \R)}{\ell^\infty(\F_2, \R) + \Hom(\F_2, \R)}
  \]
  is injective.
\end{thm}
This tells us there are a lot of non-trivial elements in $\QH(\F_2, \R)$.

The advantage of this construction is that the map above is a \emph{linear} map. So to see it is injective, it suffices to see that it has trivial kernel.:
\printindex
\end{document}
