\documentclass[a4paper]{article}

\usepackage[pdftex,
  hidelinks,
  pdfauthor={Dexter Chua},
  pdfsubject={Cambridge Maths Notes: Part IB - Electromagnetism},
  pdftitle={Part IB - Electromagnetism},
pdfkeywords={Cambridge Mathematics Maths Math IB Lent Electromagnetism}]{hyperref}

% Imports
\ifx \nextra \undefined
  \usepackage[pdftex,
    hidelinks,
    pdfauthor={Dexter Chua},
    pdfsubject={Cambridge Maths Notes: Part \npart\ - \ncourse},
    pdftitle={Part \npart\ - \ncourse},
  pdfkeywords={Cambridge Mathematics Maths Math \npart\ \nterm\ \nyear\ \ncourse}]{hyperref}
  \title{Part \npart\ - \ncourse}
\else
  \usepackage[pdftex,
    hidelinks,
    pdfauthor={Dexter Chua},
    pdfsubject={Cambridge Maths Notes: Part \npart\ - \ncourse\ (\nextra)},
    pdftitle={Part \npart\ - \ncourse\ (\nextra)},
  pdfkeywords={Cambridge Mathematics Maths Math \npart\ \nterm\ \nyear\ \ncourse\ \nextra}]{hyperref}

  \title{Part \npart\ - \ncourse \\ {\Large \nextra}}
\fi

\author{Lectured by \nlecturer \\\small Notes taken by Dexter Chua}
\date{\nterm\ \nyear}

\usepackage{alltt}
\usepackage{amsfonts}
\usepackage{amsmath}
\usepackage{amssymb}
\usepackage{amsthm}
\usepackage{booktabs}
\usepackage{caption}
\usepackage{enumitem}
\usepackage{fancyhdr}
\usepackage{graphicx}
\usepackage{mathtools}
\usepackage{microtype}
\usepackage{multirow}
\usepackage{pdflscape}
\usepackage{pgfplots}
\usepackage{siunitx}
\usepackage{tabularx}
\usepackage{tikz}
\usepackage{tkz-euclide}
\usepackage[normalem]{ulem}
\usepackage[all]{xy}

\pgfplotsset{compat=1.12}

\pagestyle{fancyplain}
\lhead{\emph{\nouppercase{\leftmark}}}
\ifx \nextra \undefined
  \rhead{
    \ifnum\thepage=1
    \else
      \npart\ \ncourse
    \fi}
\else
  \rhead{
    \ifnum\thepage=1
    \else
      \npart\ \ncourse\ (\nextra)
    \fi}
\fi
\usetikzlibrary{arrows}
\usetikzlibrary{decorations.markings}
\usetikzlibrary{decorations.pathmorphing}
\usetikzlibrary{positioning}
\usetikzlibrary{fadings}
\usetikzlibrary{intersections}
\usetikzlibrary{cd}

\newcommand*{\Cdot}{\raisebox{-0.25ex}{\scalebox{1.5}{$\cdot$}}}
\newcommand {\pd}[2][ ]{
  \ifx #1 { }
    \frac{\partial}{\partial #2}
  \else
    \frac{\partial^{#1}}{\partial #2^{#1}}
  \fi
}

% Theorems
\theoremstyle{definition}
\newtheorem*{aim}{Aim}
\newtheorem*{axiom}{Axiom}
\newtheorem*{claim}{Claim}
\newtheorem*{cor}{Corollary}
\newtheorem*{defi}{Definition}
\newtheorem*{eg}{Example}
\newtheorem*{fact}{Fact}
\newtheorem*{law}{Law}
\newtheorem*{lemma}{Lemma}
\newtheorem*{notation}{Notation}
\newtheorem*{prop}{Proposition}
\newtheorem*{thm}{Theorem}

\renewcommand{\labelitemi}{--}
\renewcommand{\labelitemii}{$\circ$}
\renewcommand{\labelenumi}{(\roman{*})}

\let\stdsection\section
\renewcommand\section{\newpage\stdsection}

% Strike through
\def\st{\bgroup \ULdepth=-.55ex \ULset}

% Maths symbols
\newcommand{\bra}{\langle}
\newcommand{\ket}{\rangle}

\newcommand{\N}{\mathbb{N}}
\newcommand{\Z}{\mathbb{Z}}
\newcommand{\Q}{\mathbb{Q}}
\renewcommand{\H}{\mathbb{H}}
\newcommand{\R}{\mathbb{R}}
\newcommand{\C}{\mathbb{C}}
\newcommand{\Prob}{\mathbb{P}}
\renewcommand{\P}{\mathbb{P}}
\newcommand{\E}{\mathbb{E}}
\newcommand{\F}{\mathbb{F}}
\newcommand{\cU}{\mathcal{U}}
\newcommand{\RP}{\mathbb{RP}}
\newcommand{\CP}{\mathbb{CP}}

\newcommand{\ph}{\,\cdot\,}

\DeclareMathOperator{\sech}{sech}
\DeclareMathOperator{\cosech}{cosech}
\DeclareMathOperator{\cosec}{cosec}

\DeclareMathOperator{\covol}{covol}
\DeclareMathOperator{\vol}{vol}

\let\Im\relax
\let\Re\relax
\DeclareMathOperator{\Im}{Im}
\DeclareMathOperator{\Re}{Re}
\DeclareMathOperator{\im}{im}
\DeclareMathOperator{\image}{image}
\DeclareMathOperator{\Ann}{Ann}

\DeclareMathOperator*{\res}{res}
\DeclareMathOperator{\Res}{Res}
\DeclareMathOperator{\Ind}{Ind}

\DeclareMathOperator{\tr}{tr}
\DeclareMathOperator{\diag}{diag}
\DeclareMathOperator{\rank}{rank}
\DeclareMathOperator{\card}{card}
\DeclareMathOperator{\spn}{span}
\DeclareMathOperator{\adj}{adj}

\DeclareMathOperator{\erf}{erf}
\DeclareMathOperator{\erfc}{erfc}

\DeclareMathOperator{\ord}{ord}
\DeclareMathOperator{\Sym}{Sym}

\DeclareMathOperator{\sgn}{sgn}
\DeclareMathOperator{\orb}{orb}
\DeclareMathOperator{\stab}{stab}
\DeclareMathOperator{\ccl}{ccl}

\DeclareMathOperator{\lcm}{lcm}
\DeclareMathOperator{\hcf}{hcf}

\DeclareMathOperator{\Int}{Int}
\DeclareMathOperator{\id}{id}

\DeclareMathOperator{\betaD}{beta}
\DeclareMathOperator{\gammaD}{gamma}
\DeclareMathOperator{\Poisson}{Poisson}
\DeclareMathOperator{\binomial}{binomial}
\DeclareMathOperator{\multinomial}{multinomial}
\DeclareMathOperator{\Bernoulli}{Bernoulli}
\DeclareMathOperator{\like}{like}

\DeclareMathOperator{\var}{var}
\DeclareMathOperator{\cov}{cov}
\DeclareMathOperator{\bias}{bias}
\DeclareMathOperator{\mse}{mse}
\DeclareMathOperator{\corr}{corr}

\DeclareMathOperator{\otp}{otp}
\DeclareMathOperator{\dom}{dom}

\DeclareMathOperator{\Root}{Root}
\DeclareMathOperator{\supp}{supp}
\DeclareMathOperator{\rel}{rel}
\DeclareMathOperator{\Hom}{Hom}
\DeclareMathOperator{\Aut}{Aut}
\DeclareMathOperator{\Gal}{Gal}
\DeclareMathOperator{\Mat}{Mat}
\DeclareMathOperator{\End}{End}
\DeclareMathOperator{\Char}{char}
\DeclareMathOperator{\ev}{ev}
\DeclareMathOperator{\St}{St}
\DeclareMathOperator{\Lk}{Lk}
\DeclareMathOperator{\disc}{disc}
\DeclareMathOperator{\Isom}{Isom}
\DeclareMathOperator{\length}{length}
\DeclareMathOperator{\energy}{energy}
\DeclareMathOperator{\area}{area}
\DeclareMathOperator{\Syl}{Syl}
\DeclareMathOperator{\cl}{cl}
\DeclareMathOperator{\fix}{fix}

\newcommand{\GL}{\mathrm{GL}}
\newcommand{\SL}{\mathrm{SL}}
\newcommand{\PGL}{\mathrm{PGL}}
\newcommand{\PSL}{\mathrm{PSL}}
\newcommand{\PSU}{\mathrm{PSU}}
\newcommand{\Or}{\mathrm{O}}
\newcommand{\SO}{\mathrm{SO}}
\newcommand{\U}{\mathrm{U}}
\newcommand{\SU}{\mathrm{SU}}

\renewcommand{\d}{\mathrm{d}}
\newcommand{\D}{\mathrm{D}}

\tikzset{->/.style = {decoration={markings,
                                  mark=at position 1 with {\arrow[scale=2]{latex'}}},
                      postaction={decorate}}}
\tikzset{<-/.style = {decoration={markings,
                                  mark=at position 0 with {\arrowreversed[scale=2]{latex'}}},
                      postaction={decorate}}}
\tikzset{<->/.style = {decoration={markings,
                                   mark=at position 0 with {\arrowreversed[scale=2]{latex'}},
                                   mark=at position 1 with {\arrow[scale=2]{latex'}}},
                       postaction={decorate}}}
\tikzset{->-/.style = {decoration={markings,
                                   mark=at position #1 with {\arrow[scale=2]{latex'}}},
                       postaction={decorate}}}
\tikzset{-<-/.style = {decoration={markings,
                                   mark=at position #1 with {\arrowreversed[scale=2]{latex'}}},
                       postaction={decorate}}}

\tikzset{circ/.style = {fill, circle, inner sep = 0, minimum size = 3}}
\tikzset{mstate/.style={circle, draw, blue, text=black, minimum width=0.7cm}}

\definecolor{mblue}{rgb}{0.2, 0.3, 0.8}
\definecolor{morange}{rgb}{1, 0.5, 0}
\definecolor{mgreen}{rgb}{0.1, 0.4, 0.2}
\definecolor{mred}{rgb}{0.5, 0, 0}

\def\drawcirculararc(#1,#2)(#3,#4)(#5,#6){%
    \pgfmathsetmacro\cA{(#1*#1+#2*#2-#3*#3-#4*#4)/2}%
    \pgfmathsetmacro\cB{(#1*#1+#2*#2-#5*#5-#6*#6)/2}%
    \pgfmathsetmacro\cy{(\cB*(#1-#3)-\cA*(#1-#5))/%
                        ((#2-#6)*(#1-#3)-(#2-#4)*(#1-#5))}%
    \pgfmathsetmacro\cx{(\cA-\cy*(#2-#4))/(#1-#3)}%
    \pgfmathsetmacro\cr{sqrt((#1-\cx)*(#1-\cx)+(#2-\cy)*(#2-\cy))}%
    \pgfmathsetmacro\cA{atan2(#2-\cy,#1-\cx)}%
    \pgfmathsetmacro\cB{atan2(#6-\cy,#5-\cx)}%
    \pgfmathparse{\cB<\cA}%
    \ifnum\pgfmathresult=1
        \pgfmathsetmacro\cB{\cB+360}%
    \fi
    \draw (#1,#2) arc (\cA:\cB:\cr);%
}
\newcommand\getCoord[3]{\newdimen{#1}\newdimen{#2}\pgfextractx{#1}{\pgfpointanchor{#3}{center}}\pgfextracty{#2}{\pgfpointanchor{#3}{center}}}

\def\Xint#1{\mathchoice
   {\XXint\displaystyle\textstyle{#1}}%
   {\XXint\textstyle\scriptstyle{#1}}%
   {\XXint\scriptstyle\scriptscriptstyle{#1}}%
   {\XXint\scriptscriptstyle\scriptscriptstyle{#1}}%
   \!\int}
\def\XXint#1#2#3{{\setbox0=\hbox{$#1{#2#3}{\int}$}
     \vcenter{\hbox{$#2#3$}}\kern-.5\wd0}}
\def\ddashint{\Xint=}
\def\dashint{\Xint-}


\title{Part IB - Electromagnetism}
\author{Lectured by David Tong}
\date{Lent 2015}

\begin{document}
\maketitle
{\small
\noindent\textbf{Electromagnetism and Relativity}\\
Review of Special Relativity; tensors and index notation. Lorentz force law. Electromagnetic tensor. Lorentz transformations of electric and magnetic fields. Currents and the conservation of charge. Maxwell equations in relativistic and non-relativistic forms.\hspace*{\fill} [5]

\vspace{10pt}
\noindent\textbf{Electrostatics}\\
Gauss's law. Application to spherically symmetric and cylindrically symmetric charge distributions.  Point, line and surface charges. Electrostatic potentials; general charge distributions, dipoles. Electrostatic energy. Conductors.\hspace*{\fill} [3]

\vspace{10pt}
\noindent\textbf{Magnetostatics}\\
Magnetic fields due to steady currents. Ampre's law. Simple examples. Vector potentials and the Biot-Savart law for general current distributions. Magnetic dipoles. Lorentz force on current distributions and force between current-carrying wires. Ohm's law.\hspace*{\fill} [3]

\vspace{10pt}
\noindent\textbf{Electrodynamics}\\
Faraday's law of induction for fixed and moving circuits. Electromagnetic energy and Poynting vector.  4-vector potential, gauge transformations. Plane electromagnetic waves in vacuum, polarization.\hspace*{\fill} [5]}

\tableofcontents
\section{Introduction}
Electromagnetism is important.
\subsection{Charge and Current}
The strength of the electromagnetic force experienced by a particle is determined by its \emph{(electric) charge}. The SI unit of charge is the \emph{Columb}. In this course, we assume that the charge can be any real number. However, at the fundamental level, charge is quantised. All particles carry charge $q = ne$ with $n\in \Z$, \footnote{Actually quarks have $n = \pm1/3$ or $\pm2/3$, but when we study quarks, electromagnetism becomes insignificant compared to the strong force. So for all practical purposes we can have $n\in \Z$.} and the basic unit $e \approx 1.6 \times 10^{-19} $C. For example, the electron has $n = -1$, proton has $n = +1$, neutron = $n = 0$.

In this course, it will be more useful to talk about \emph{charge density} $\rho(\mathbf{x}, t)$. This is the charge per unit volume. The total charge in a region $V$ is
\[
  Q = \int_V \rho(\mathbf{x}, t)\; \d^3 x
\]

The motion of charge is described by the \emph{current} density $\mathbf{J}(\mathbf{x}, t)$. For any surface $S$, the integral
\[
  I = \int_S \mathbf{J}\cdot d\mathbf{S}
\]
counts the charge per unit time passing through $S$. $I$ is called the current, and $\mathbf{J}$ is the ``current per unit area''.

Intuitively, if the charge distribution $\rho (\mathbf{x}, t)$ has velocity $\mathbf{v}(x, t)$, then (neglecting relativistic effects), we have
\[
  \mathbf{J} = \rho \mathbf{v}
\]

\begin{eg}
  A wire is a cylinder of cross-sectional area $A$. Suppose there are $n$ electrons per unit volume. Then
  \begin{align*}
    \rho &= nq = -ne\\
    \mathbf{J} &= nq\mathbf{v}\\
    I &= nqvA
  \end{align*}
\end{eg}

It is well know that charge is conserved. However, we can have a stronger statement- charge is conserved locally: it is not possible that a charge in a box disappears and instantaneously appears on the moon. If it disappears in the box, it must have moved to somewhere nearby.

This is captured by the \emph{continuity equation}
\[
  \frac{\partial\rho}{\partial t} + \nabla\cdot \mathbf{J} = 0
\]
The charge $Q$ in some region $V$ is
\[
  Q = \int_V \rho \;\d^3 x
\]
So
\[
  \frac{\d Q}{\d t} = \int_V \frac{\partial\rho}{\partial t}\; d^3 x = -\int_V \nabla\cdot \mathbf{J}\; \d^3 x = -\int_S \mathbf{J}\cdot \d S
\]
where the last equality is by the divergence theorem. i.e. the rate of change in charge is the rate of charge flowing out of the point.

We can take $V = \R^3$, the whole of space. If there are no currents at infinity, then
\[
  \frac{\d Q}{\d t} = 0
\]
So the continuity equation implies the conservation of charge.

\subsection{Forces and Fields}
All forces are mediated by \emph{fields}. In physics a \emph{field} is a dynamical quantity which takes values at every point in space and time. The electromagnetic force is mediated by two fields:
\begin{itemize}
  \item electric field $\mathbf{E}(\mathbf{x}, t)$
  \item magnetic field $\mathbf{B}(\mathbf{x}, t)$
\end{itemize}
Each of these fields is itself a 3-vector. There are two aspects to the force:
\begin{itemize}
  \item Particles create fields
  \item Fields move particles
\end{itemize}
The second aspect is governed by the Lorentz force law:
\[
  \mathbf{F} = q(\mathbf{E} + \mathbf{v}\times \mathbf{B})
\]
The first aspect is governed by the \emph{Maxwell equations}.

\begin{align*}
  \nabla \cdot \mathbf{E} &= \frac{\rho}{\varepsilon_0}\\
  \nabla \cdot \mathbf{B} &= 0\\
  \nabla \times \mathbf{E} +\frac{\partial \mathbf{E}}{\partial t} &= 0\\
  \nabla \times \mathbf{B} - \mu_0\varepsilon_0 \frac{\partial \mathbf{E}}{\partial t} = \mu_0 \mathbf{j}
\end{align*}
where
\begin{itemize}
  \item $\varepsilon_0 = \SI{8.85e-12}{\per\metre\cubed\per\kilogram\s\squared\coulomb\squared}$ is the electric constant
  \item $\mu_0 = \SI{4\pi e-6}{\metre\kilogram\per\coulomb\squared}$ is the magnetic constant
\end{itemize}
These equations are special in a mathematical way. In fact, using quantum mechanics and relativity, it can be shown that these are the only equations that can possibly describe electromagnetism
\end{document}
