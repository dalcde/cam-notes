\documentclass[a4paper]{article}

\usepackage[pdftex,
  hidelinks,
  pdfauthor={Dexter Chua},
  pdfsubject={Cambridge Maths Notes: Part II - Logic and Set Theory},
  pdftitle={Part IB - Logic and Set Theory},
pdfkeywords={Cambridge Mathematics Maths Math II Lent Logic and Set Theory}]{hyperref}

% Imports
\ifx \nextra \undefined
  \usepackage[pdftex,
    hidelinks,
    pdfauthor={Dexter Chua},
    pdfsubject={Cambridge Maths Notes: Part \npart\ - \ncourse},
    pdftitle={Part \npart\ - \ncourse},
  pdfkeywords={Cambridge Mathematics Maths Math \npart\ \nterm\ \nyear\ \ncourse}]{hyperref}
  \title{Part \npart\ - \ncourse}
\else
  \usepackage[pdftex,
    hidelinks,
    pdfauthor={Dexter Chua},
    pdfsubject={Cambridge Maths Notes: Part \npart\ - \ncourse\ (\nextra)},
    pdftitle={Part \npart\ - \ncourse\ (\nextra)},
  pdfkeywords={Cambridge Mathematics Maths Math \npart\ \nterm\ \nyear\ \ncourse\ \nextra}]{hyperref}

  \title{Part \npart\ - \ncourse \\ {\Large \nextra}}
\fi

\author{Lectured by \nlecturer \\\small Notes taken by Dexter Chua}
\date{\nterm\ \nyear}

\usepackage{alltt}
\usepackage{amsfonts}
\usepackage{amsmath}
\usepackage{amssymb}
\usepackage{amsthm}
\usepackage{booktabs}
\usepackage{caption}
\usepackage{enumitem}
\usepackage{fancyhdr}
\usepackage{graphicx}
\usepackage{mathtools}
\usepackage{microtype}
\usepackage{multirow}
\usepackage{pdflscape}
\usepackage{pgfplots}
\usepackage{siunitx}
\usepackage{tabularx}
\usepackage{tikz}
\usepackage{tkz-euclide}
\usepackage[normalem]{ulem}
\usepackage[all]{xy}

\pgfplotsset{compat=1.12}

\pagestyle{fancyplain}
\lhead{\emph{\nouppercase{\leftmark}}}
\ifx \nextra \undefined
  \rhead{
    \ifnum\thepage=1
    \else
      \npart\ \ncourse
    \fi}
\else
  \rhead{
    \ifnum\thepage=1
    \else
      \npart\ \ncourse\ (\nextra)
    \fi}
\fi
\usetikzlibrary{arrows}
\usetikzlibrary{decorations.markings}
\usetikzlibrary{decorations.pathmorphing}
\usetikzlibrary{positioning}
\usetikzlibrary{fadings}
\usetikzlibrary{intersections}
\usetikzlibrary{cd}

\newcommand*{\Cdot}{\raisebox{-0.25ex}{\scalebox{1.5}{$\cdot$}}}
\newcommand {\pd}[2][ ]{
  \ifx #1 { }
    \frac{\partial}{\partial #2}
  \else
    \frac{\partial^{#1}}{\partial #2^{#1}}
  \fi
}

% Theorems
\theoremstyle{definition}
\newtheorem*{aim}{Aim}
\newtheorem*{axiom}{Axiom}
\newtheorem*{claim}{Claim}
\newtheorem*{cor}{Corollary}
\newtheorem*{defi}{Definition}
\newtheorem*{eg}{Example}
\newtheorem*{fact}{Fact}
\newtheorem*{law}{Law}
\newtheorem*{lemma}{Lemma}
\newtheorem*{notation}{Notation}
\newtheorem*{prop}{Proposition}
\newtheorem*{thm}{Theorem}

\renewcommand{\labelitemi}{--}
\renewcommand{\labelitemii}{$\circ$}
\renewcommand{\labelenumi}{(\roman{*})}

\let\stdsection\section
\renewcommand\section{\newpage\stdsection}

% Strike through
\def\st{\bgroup \ULdepth=-.55ex \ULset}

% Maths symbols
\newcommand{\bra}{\langle}
\newcommand{\ket}{\rangle}

\newcommand{\N}{\mathbb{N}}
\newcommand{\Z}{\mathbb{Z}}
\newcommand{\Q}{\mathbb{Q}}
\renewcommand{\H}{\mathbb{H}}
\newcommand{\R}{\mathbb{R}}
\newcommand{\C}{\mathbb{C}}
\newcommand{\Prob}{\mathbb{P}}
\renewcommand{\P}{\mathbb{P}}
\newcommand{\E}{\mathbb{E}}
\newcommand{\F}{\mathbb{F}}
\newcommand{\cU}{\mathcal{U}}
\newcommand{\RP}{\mathbb{RP}}
\newcommand{\CP}{\mathbb{CP}}

\newcommand{\ph}{\,\cdot\,}

\DeclareMathOperator{\sech}{sech}
\DeclareMathOperator{\cosech}{cosech}
\DeclareMathOperator{\cosec}{cosec}

\DeclareMathOperator{\covol}{covol}
\DeclareMathOperator{\vol}{vol}

\let\Im\relax
\let\Re\relax
\DeclareMathOperator{\Im}{Im}
\DeclareMathOperator{\Re}{Re}
\DeclareMathOperator{\im}{im}
\DeclareMathOperator{\image}{image}
\DeclareMathOperator{\Ann}{Ann}

\DeclareMathOperator*{\res}{res}
\DeclareMathOperator{\Res}{Res}
\DeclareMathOperator{\Ind}{Ind}

\DeclareMathOperator{\tr}{tr}
\DeclareMathOperator{\diag}{diag}
\DeclareMathOperator{\rank}{rank}
\DeclareMathOperator{\card}{card}
\DeclareMathOperator{\spn}{span}
\DeclareMathOperator{\adj}{adj}

\DeclareMathOperator{\erf}{erf}
\DeclareMathOperator{\erfc}{erfc}

\DeclareMathOperator{\ord}{ord}
\DeclareMathOperator{\Sym}{Sym}

\DeclareMathOperator{\sgn}{sgn}
\DeclareMathOperator{\orb}{orb}
\DeclareMathOperator{\stab}{stab}
\DeclareMathOperator{\ccl}{ccl}

\DeclareMathOperator{\lcm}{lcm}
\DeclareMathOperator{\hcf}{hcf}

\DeclareMathOperator{\Int}{Int}
\DeclareMathOperator{\id}{id}

\DeclareMathOperator{\betaD}{beta}
\DeclareMathOperator{\gammaD}{gamma}
\DeclareMathOperator{\Poisson}{Poisson}
\DeclareMathOperator{\binomial}{binomial}
\DeclareMathOperator{\multinomial}{multinomial}
\DeclareMathOperator{\Bernoulli}{Bernoulli}
\DeclareMathOperator{\like}{like}

\DeclareMathOperator{\var}{var}
\DeclareMathOperator{\cov}{cov}
\DeclareMathOperator{\bias}{bias}
\DeclareMathOperator{\mse}{mse}
\DeclareMathOperator{\corr}{corr}

\DeclareMathOperator{\otp}{otp}
\DeclareMathOperator{\dom}{dom}

\DeclareMathOperator{\Root}{Root}
\DeclareMathOperator{\supp}{supp}
\DeclareMathOperator{\rel}{rel}
\DeclareMathOperator{\Hom}{Hom}
\DeclareMathOperator{\Aut}{Aut}
\DeclareMathOperator{\Gal}{Gal}
\DeclareMathOperator{\Mat}{Mat}
\DeclareMathOperator{\End}{End}
\DeclareMathOperator{\Char}{char}
\DeclareMathOperator{\ev}{ev}
\DeclareMathOperator{\St}{St}
\DeclareMathOperator{\Lk}{Lk}
\DeclareMathOperator{\disc}{disc}
\DeclareMathOperator{\Isom}{Isom}
\DeclareMathOperator{\length}{length}
\DeclareMathOperator{\energy}{energy}
\DeclareMathOperator{\area}{area}
\DeclareMathOperator{\Syl}{Syl}
\DeclareMathOperator{\cl}{cl}
\DeclareMathOperator{\fix}{fix}

\newcommand{\GL}{\mathrm{GL}}
\newcommand{\SL}{\mathrm{SL}}
\newcommand{\PGL}{\mathrm{PGL}}
\newcommand{\PSL}{\mathrm{PSL}}
\newcommand{\PSU}{\mathrm{PSU}}
\newcommand{\Or}{\mathrm{O}}
\newcommand{\SO}{\mathrm{SO}}
\newcommand{\U}{\mathrm{U}}
\newcommand{\SU}{\mathrm{SU}}

\renewcommand{\d}{\mathrm{d}}
\newcommand{\D}{\mathrm{D}}

\tikzset{->/.style = {decoration={markings,
                                  mark=at position 1 with {\arrow[scale=2]{latex'}}},
                      postaction={decorate}}}
\tikzset{<-/.style = {decoration={markings,
                                  mark=at position 0 with {\arrowreversed[scale=2]{latex'}}},
                      postaction={decorate}}}
\tikzset{<->/.style = {decoration={markings,
                                   mark=at position 0 with {\arrowreversed[scale=2]{latex'}},
                                   mark=at position 1 with {\arrow[scale=2]{latex'}}},
                       postaction={decorate}}}
\tikzset{->-/.style = {decoration={markings,
                                   mark=at position #1 with {\arrow[scale=2]{latex'}}},
                       postaction={decorate}}}
\tikzset{-<-/.style = {decoration={markings,
                                   mark=at position #1 with {\arrowreversed[scale=2]{latex'}}},
                       postaction={decorate}}}

\tikzset{circ/.style = {fill, circle, inner sep = 0, minimum size = 3}}
\tikzset{mstate/.style={circle, draw, blue, text=black, minimum width=0.7cm}}

\definecolor{mblue}{rgb}{0.2, 0.3, 0.8}
\definecolor{morange}{rgb}{1, 0.5, 0}
\definecolor{mgreen}{rgb}{0.1, 0.4, 0.2}
\definecolor{mred}{rgb}{0.5, 0, 0}

\def\drawcirculararc(#1,#2)(#3,#4)(#5,#6){%
    \pgfmathsetmacro\cA{(#1*#1+#2*#2-#3*#3-#4*#4)/2}%
    \pgfmathsetmacro\cB{(#1*#1+#2*#2-#5*#5-#6*#6)/2}%
    \pgfmathsetmacro\cy{(\cB*(#1-#3)-\cA*(#1-#5))/%
                        ((#2-#6)*(#1-#3)-(#2-#4)*(#1-#5))}%
    \pgfmathsetmacro\cx{(\cA-\cy*(#2-#4))/(#1-#3)}%
    \pgfmathsetmacro\cr{sqrt((#1-\cx)*(#1-\cx)+(#2-\cy)*(#2-\cy))}%
    \pgfmathsetmacro\cA{atan2(#2-\cy,#1-\cx)}%
    \pgfmathsetmacro\cB{atan2(#6-\cy,#5-\cx)}%
    \pgfmathparse{\cB<\cA}%
    \ifnum\pgfmathresult=1
        \pgfmathsetmacro\cB{\cB+360}%
    \fi
    \draw (#1,#2) arc (\cA:\cB:\cr);%
}
\newcommand\getCoord[3]{\newdimen{#1}\newdimen{#2}\pgfextractx{#1}{\pgfpointanchor{#3}{center}}\pgfextracty{#2}{\pgfpointanchor{#3}{center}}}

\def\Xint#1{\mathchoice
   {\XXint\displaystyle\textstyle{#1}}%
   {\XXint\textstyle\scriptstyle{#1}}%
   {\XXint\scriptstyle\scriptscriptstyle{#1}}%
   {\XXint\scriptscriptstyle\scriptscriptstyle{#1}}%
   \!\int}
\def\XXint#1#2#3{{\setbox0=\hbox{$#1{#2#3}{\int}$}
     \vcenter{\hbox{$#2#3$}}\kern-.5\wd0}}
\def\ddashint{\Xint=}
\def\dashint{\Xint-}


\title{Part II - Logic and Set Theory}
\author{Lectured by I. B. Leader}
\date{Lent 2015}

\begin{document}
\maketitle
{\small
  \noindent\textbf{Ordinals and cardinals}\\
  Well-orderings and order-types. Examples of countable ordinals. Uncountable ordinals and Hartogs' lemma. Induction and recursion for ordinals. Ordinal arithmetic. Cardinals; the hierarchy of alephs. Cardinal arithmetic.\hspace*{\fill} [5]

  \vspace{10pt}
  \noindent\textbf{Posets and Zorn's lemma}\\
  Partially ordered sets; Hasse diagrams, chains, maximal elements. Lattices and Boolean algebras. Complete and chain-complete posets; fixed-point theorems. The axiom of choice and Zorn's lemma. Applications of Zorn's lemma in mathematics. The well-ordering principle.\hspace*{\fill} [5]

  \vspace{10pt}
  \noindent\textbf{Propositional logic}\\
  The propositional calculus. Semantic and syntactic entailment. The deduction and completeness theorems. Applications: compactness and decidability.\hspace*{\fill} [3]

  \vspace{10pt}
  \noindent\textbf{Predicate logic}\\
  The predicate calculus with equality. Examples of first-order languages and theories. Statement of the completeness theorem; *sketch of proof*. The compactness theorem and the Lowenheim-Skolem theorems. Limitations of first-order logic. Model theory.\hspace*{\fill} [5]

  \vspace{10pt}
  \noindent\textbf{Set theory}\\
  Set theory as a first-order theory; the axioms of ZF set theory. Transitive closures, epsilon-induction and epsilon-recursion. Well-founded relations. Mostowski's collapsing theorem. The rank function and the von Neumann hierarchy.\hspace*{\fill} [5]
  
  \vspace{10pt}
  \noindent\textbf{Consistency}\\
  *Problems of consistency and independence*\hspace*{\fill} [1]}

\tableofcontents

\section{Propositional calculus}
\begin{defi}[Propositions]
  Let $P$ be a set of \emph{primitive propositions}. These are a bunch of (meaningless) symbols, that are usually interpreted to take a truth value. Usually, any symbol (composed of alphabets and subscripts) is in the set of primitive propositions.

  The set of \emph{propositions}, written as $L$ or $L(P)$, is defined inductively by
  \begin{enumerate}
    \item If $p\in P$, then $p\in L$.
    \item $\bot\in L$, where $\bot$ is ``false'' (also a meaningless symbol).
    \item If $p, q\in L$, then $p\Rightarrow q)\in L$.
  \end{enumerate}
\end{defi}

\begin{eg}
  $p\Rightarrow q$, $p\Rightarrow \bot$, $((p\Rightarrow q)\Rightarrow (p\Rightarrow r))$ are propositions.
\end{eg}

To define $L$ formally, we let $L_1 = \{\bot\}\cup P$, and for $n\geq 1$, $L_{n + 1} = L_n\cup \{(p\Rightarrow q): p, q\in L_n\}$. Then set $L = L_1\cup L_2\cup \cdots$.

In formal language terms, $L$ is the set of finite strings of symbols from the alphabet $\bot$, $\bot$, $\Rightarrow $, $($, $)$, $p_1$, $p_2, \cdots$ that satisfies some formal grammar rule (e.g. brackets have to match).

We define the following abbreviations:
\begin{defi}[Logical symbols]\leavevmode
  \begin{center}
    \begin{tabular}[]{cccc}
      $\neg p$ & (``not $p$'') & is an abbreviation for & $(p\Rightarrow \bot)$\\
      $p\wedge q$ & (``$p$ and $q$'') & is an abbreviation for & $\neg(p\Rightarrow (\neg q))$\\
      $p\vee q$ & (``$p$ or $q$'') & is an abbreviation for & $(\neg p)\Rightarrow q$
    \end{tabular}
  \end{center}
\end{defi}
\subsection{Semantic implication}
\begin{defi}[Valuation]
  A \emph{valuation} on $L$ is a function $v: L\to \{0, 1\}$ such that:
  \begin{itemize}
  \item $v(\bot) = 0$,
  \item $v(p\Rightarrow q) = \begin{cases} 0 & \text{if }v(p) = 1, v(q) = 0,\\1 & \text{otherwise}\end{cases}$
  \end{itemize}
  We interpret $v(p)$ to be the truth value of $p$, with 0 denoting ``false'' and 1 denoting ``true''.

  Note that we do not impose any restriction of $v(p)$ when $p$ is a primitive proposition (that is not $\bot$).
\end{defi}

We can also give $\{0, 1\}$ a binary operation $\Rightarrow$ by
\[
  a\Rightarrow b = \begin{cases}
    0 & \text{if }a = 1, b = 0\\
    1 & \text{otherwise}
  \end{cases}
\]
with a constant $\bot = 0$. Then a valuation $v: K\to \{0, 1\}$ is simply a homomorphism between the two structures that preserve $\bot$ and $\Rightarrow $.

\begin{prop}\leavevmode
  \begin{enumerate}
    \item If $v$ and $v'$ are valuations with $v(p) = v'(p)$ for all $p\in P$, then $v = v'$.
    \item For any function $W: P \to \{0, 1\}$, there is a valuation $v$ such that $v(p) = w(p)$ for all $p\in L$, i.e. we can extend $w$ to a full valuation.
  \end{enumerate}
  This means ``A valuation is determined by its values on $P$, and any values will do''.
\end{prop}

\begin{proof}
  \begin{enumerate}
    \item Recall that $L$ is defined inductively. We are given that $v(p) = v'(p)$ on $L_1$. Then for all $p\in L_2$, $p$ must be in the form $q\Rightarrow r$ for $q, r\in L_1$. Then $v(q\Rightarrow r) = v(p\Rightarrow q)$ since the value of $v$ is uniquely determined by the definition. So for all $p\in L_2$, $v(p) = v'(p)$.

      Continue inductively to show that $v(p) = v'(p)$ for all $p\in L_n$ for any $n$.
    
    \item Set $v$ to agree with $w$ for all $p\in P$, and set $v(\bot) = 0$. Then define $v$ on $L_n$ inductively according to the definition.
  \end{enumerate}
\end{proof}

\begin{eg}
  Suppose $v$ is a valuation with $v(p) = v(q) = 1$, $v(r) = 0$. Then
  \[
    v((p\Rightarrow q)\Rightarrow r) = 0.
  \]
\end{eg}

\begin{defi}[Tautology]
  $t$ is a \emph{tautology}, written as $\models t$, if $v(t) = 1$. 
\end{defi}

\begin{eg}
  \begin{enumerate}
    \item $p\Rightarrow (q\Rightarrow p)$ ``A true statement is implied by anything''.\vspace{5pt}\\
      \begin{tabular}[]{cccc}
        $v(p)$ & $v(q)$ & $v(q\Rightarrow p)$ & $v(p\Rightarrow (q\Rightarrow p))$\\
        1 & 1 & 1 & 1\\
        1 & 0 & 1 & 1\\
        0 & 1 & 0 & 1\\
        0 & 0 & 1 & 1
      \end{tabular}
    \item $(\neg \neg p)\Rightarrow p$. Recall that $\neg\neg p$ is defined as $((p\Rightarrow \bot)\Rightarrow \bot)$.\vspace{5pt}\\
    \begin{tabular}{cccc}
      $v(p)$ & $v(p\Rightarrow \bot)$ & $v((p\Rightarrow \bot)\Rightarrow \bot)$ & $v(((p\Rightarrow \bot)\Rightarrow \bot)\Rightarrow p)$\\
      1 & 0 & 1 & 1\\
      0 & 1 & 0 & 1
    \end{tabular}
  \item $(p\Rightarrow (q\Rightarrow r))\Rightarrow ((p\Rightarrow q)\Rightarrow (p\Rightarrow r))$.

    Instead of creating a truth table, which would be horribly long, we show this by reasoning: Suppose it is not a tautology. So there is a $v$ such that $v(p\Rightarrow q\Rightarrow r)) = 1$ and $v((p\Rightarrow q)\Rightarrow (p\Rightarrow r)) =0 $. For the second equality to hold, we must have $v(p\Rightarrow q) = 1$ and $v(p\Rightarrow r) = 0$. So $v(p) = 1, v(r) = 0, v(q) = 1$. But then $v(p\Rightarrow q) = 0$. 
  \end{enumerate}
\end{eg}

\begin{defi}[Semantic entailment]
  For $S\subseteq L$, $t\in L$, we say $S$ \emph{entails} $t$, $S$ \emph{semantically implies t} or $S\models t$ if, for any $v$ such that $v(s) = 1$ for all $s\in S$, $v(t) = 1$.

  ``Whenever all of $S$ is true, $t$ is true as well.''
\end{defi}

\begin{eg}
  $\{p\Rightarrow q, q\Rightarrow r\}\models (p\Rightarrow r).$

  We want to show that for any valuation $v$ with $v(p\Rightarrow q) = v(q\Rightarrow r) = 1$, we have $v(p\Rightarrow r) = 1$. We prove the contrapositive.

  If $v(p\Rightarrow r) = 0$, then $v(p) = 1$ and $v(r) = 0$. If $v(q) = 0$, then $v(p\Rightarrow q) = 0$. If $v(q) = 1$, then $v(q\Rightarrow r) = 0$. So $v(p\Rightarrow r) = 0$ only if one of $v(p\Rightarrow q)$ or $v(q\Rightarrow r)$ is zero. 
\end{eg}
\begin{defi}[Truth and model]
  If $v(t) = 1$, then we say that $t$ is \emph{true} in $v$, or $v$ is a \emph{model} of $t$. For $S\subseteq L$, a valuation $v$ is a \emph{model} of $S$ if $v(s) = 1$. Then $\models t$ means $\emptyset \models t$. 
\end{defi}
\subsection{Syntactic implication}
While semantic implication captures the idea of truthfulness, syntactic implication captures the idea of proofs. To do so, we need to have axioms and deduction rules.

Our system of deduction composes of the following axioms:
\begin{enumerate}
  \item $p\Rightarrow(q\Rightarrow p)$
  \item $[p\Rightarrow(q\Rightarrow r)]\Rightarrow[(p\Rightarrow q)\Rightarrow(p \Rightarrow r)]$
  \item $(\neg\neg p)\Rightarrow p$
\end{enumerate}
and the deduction rule of \emph{modus ponens}: from $p$ and $p\Rightarrow q$, we can deduce $q$.

Note that every axiom is a tautology. 

\begin{defi}[Proof and syntactic entailment]
  For any $S\subseteq L$, a \emph{proof} of $t$ from $S$ is a finite sequence $t_1, t_2, \cdots t_n$ of propositions, with $t_n = t$, such that each $t_i$ is one of the following:
\begin{enumerate}
  \item An axiom
  \item A member of $S$
  \item A proposition $t_i$ such that there exist $j, k < i$ with $t_j = (t_k\Rightarrow t_i)$.
\end{enumerate}
If there is a proof of $t$ from $S$, we say that $S$ \emph{proves} or \emph{syntactically entails} $t$, written $S\vdash t$.

If $\emptyset \vdash t$, say $t$ is a \emph{theorem} and write $\vdash t$.

In a proof of $t$ from $S$, $t$ is the \emph{conclusion} and $S$ is the set of \emph{hypothesis} or \emph{premises}.


\end{defi}
\begin{eg}
  $\{p\Rightarrow q, q\Rightarrow r\} \vdash p\Rightarrow r$

  ``Go for $(p\Rightarrow  q)\Rightarrow  (p\Rightarrow  r)$ via Axiom 2.
  \begin{enumerate}[label=\arabic{*}.]
    \item $[p\Rightarrow (q\Rightarrow r)]\Rightarrow [(p\Rightarrow q) \Rightarrow (p\Rightarrow r)]$ \hfill Axiom 2
    \item $q\Rightarrow r$ \hfill Hypothesis
    \item $(q\Rightarrow r)\Rightarrow [q\Rightarrow (q\Rightarrow r)]$\hfill Axiom 1
    \item $p\Rightarrow (q\Rightarrow r)$ \hfill MP on 2,3
    \item $(p\Rightarrow q)\Rightarrow (p\Rightarrow r)$\hfill MP on 1, 4
    \item $p\Rightarrow q$\hfill Hypothesis
    \item $p\Rightarrow r$\hfill MP on 5, 6
  \end{enumerate}
\end{eg}

\begin{eg}
  $\vdash (p\Rightarrow p)$

  ``Go for $[p\Rightarrow (p\Rightarrow p)]\Rightarrow (p\Rightarrow p)$''
  \begin{enumerate}[label=\arabic{*}.]
    \item $[p\Rightarrow ((p\Rightarrow p)\Rightarrow p)]\Rightarrow [(p\Rightarrow (p\Rightarrow p))\Rightarrow (p\Rightarrow p)]$ \hfill Axiom 2
    \item $p\Rightarrow ( (p\Rightarrow p)\Rightarrow p)$\hfill Axiom 1
    \item $[p\Rightarrow (p\Rightarrow p)]\Rightarrow (p\Rightarrow p)$ \hfill MP on 1, 2
    \item $p\Rightarrow (p\Rightarrow p)$\hfill Axiom 1
    \item $p\Rightarrow p$ \hfill MP on 3, 4
  \end{enumerate}
\end{eg}

This seems like a really tedious way to prove things. We now prove that the \emph{deduction theorem}, that can usually help to prove $S\vdash t$.

\begin{prop}[Deduction theorem]
  Let $S\subset L$ and $p, q\in L$. Then $S\vdash (p\Rightarrow q)$ if and only if $S\cup {p} \vdash q$.\\

  ``$\vdash$ behaves like the connective $\Rightarrow $ in the language''
\end{prop}

\begin{proof}
  ($\Rightarrow $) Given a proof of $p\Rightarrow q$ from $S$, append the lines
  \begin{itemize}
    \item $p$\hfill Hypothesis
    \item $q$\hfill $MP$
  \end{itemize}
  to obtain a proof of $q$ from $S\cup \{q\}$.

  ($\Leftarrow$) Let $t_1, t_2, \cdots, t_n = q$ be a proof of $q$ from $S\cup \{p\}$. We'll show that $S\vdash p\Rightarrow t_i$ for all $i$.

  We consider different possibilities of $t_i$:
  \begin{itemize}
    \item $t_i$ is an axiom: Write down
      \begin{itemize}
        \item $t_i\Rightarrow (p\Rightarrow t_i)$\hfill (Axiom 1)
        \item $t_i$\hfill Axiom
        \item $p\Rightarrow t_i$ \hfill MP
      \end{itemize}
    \item $t_i\in S$: Write down
      \begin{itemize}
        \item $t_i\Rightarrow (p\Rightarrow t_i)$\hfill (Axiom 1)
        \item $t_i$\hfill Hypothesis
        \item $p\Rightarrow t_i$ \hfill MP
      \end{itemize}
      To get $S\models (p\Rightarrow t_i)$
    \item $t_i = p$: Write down our proof of $p\Rightarrow p$ from our example above.
    \item $t_i$ is obtained by MP: we have some $j, k< i$ such that $t_k = (t_k\Rightarrow t_i)$. We can assume that $S\vdash (p\Rightarrow t_j)$ and $S\vdash (p\Rightarrow t_k)$ by induction on $i$. Now we can write down
      \begin{itemize}
        \item $[p\Rightarrow (t_j\Rightarrow t_i)]\Rightarrow [(p\Rightarrow t_j)\Rightarrow (p\Rightarrow t_i)]$\hfill Axiom 2
        \item $p\Rightarrow (t_j\Rightarrow t_i)$\hfill Known already
        \item $(p\Rightarrow t_j)\Rightarrow (p\Rightarrow t_i)$\hfill MP
        \item $p\Rightarrow t_j$\hfill Known already
        \item $p\Rightarrow t_i$\hfill MP
      \end{itemize}
      to get $S\models (p\Rightarrow t_i)$. 
  \end{itemize}
  This is why Axiom 2 is as it is - it enables us to prove the deduction theorem. 
\end{proof}

\begin{eg}
  We want to show $\{p\Rightarrow  q, q\Rightarrow r\} \vdash (p\Rightarrow r)$. By the deduction theorem, it is enough to show that $\{p\Rightarrow q, q\Rightarrow  r, p\}\vdash r$, which is trivial by applying MP twice. 
\end{eg}

Now we have two notions: $\models$ and $\vdash$. How are they related? We want to show that they are equal: if something is true, we can prove it; if we can prove something, it must be true.

\noindent\textbf{Aim.} Show that $S\vdash t$ if and only if $S\models t$.
This is the \emph{completeness theorem}, made up of two directions:
\begin{enumerate}
  \item Soundness: If $S\vdash t$, then $S\models t$. ``Our axioms aren't absurd''
  \item Adequacy: If $S\models t$, $S\vdash t$.  ``Our axioms are strong enough to be able to deduce, from $S$, \emph{all} semantic consequences of $S$.''
\end{enumerate}

\begin{prop}[Soundness theorem]
  If $S\vdash t$, then $S\models t$.
\end{prop}

\begin{proof}
  Given valuation $v$ with $v(s) = 1$ for all $s\in S$, we need to show that $v(t) = 1$. But $v(p) = 1$ for all axioms $p$, and $v(p) = 1$ for all $p\in S$, and if $v(p) = 1$ and $v(p\Rightarrow q) = 1$, then $v(q) = 1$. Hence each line $t_i$ in a proof $t_1, \cdots, t_n$ of from $S$ has $v(t_i) = 1$. 
\end{proof}
This works because we know that all axioms of tautologies.

We first prove a special case of adequacy:
\begin{defi}[Consistent]
  $S$ is \emph{inconsistent} if $S\vdash \bot$. $S$ is \emph{consistent} if it is not inconsistent.
\end{defi}

We prove the following statement:
\begin{thm}[Model existence theorem]
  If $S\models \bot$, then $S\vdash \bot$. i.e., if $S$ has no model, then $S$ is consistent. i.e. If $S$ is consistent, then $S$ has a model.

  \note Some books call this the ``completeness theorem'', because the rest of the completeness theorem follows trivially from this. 
\end{thm}

In fact, this would imply adequacy. Indeed if $S\models t$, then  $S\cup \{\neg t\}\models \bot$. Hence $S\cup \{\neg t\}\vdash \bot$ by our special case. So $S\vdash \neg \neg t$ by the deduction theorem. But $\vdash (\neg\neg t)\Rightarrow t$ by Axiom 3. So $S\vdash t$.

\begin{proof}
  The idea is that we'd like to define $v: L \to \{0, 1\}$ by
  \[
    p\mapsto
    \begin{cases}
      1 & \text{if } p\in S\\
      0 & \text{if } p\in S
    \end{cases}
  \]
  However, this is obviously going to fail, because we might have implications of $S$ that are not in $S$, i.e. $S$ is not \emph{deductively closed} (deductively closed means $S\vdash p$ implies $p\in S$). Yet this is not a serious problem - we take the deductive closure first.

  But there is a more serious problem. There might be a $p$ with $S\not\vdash p$ and $S\not\vdash \neg p$. This is the case if $p$ never appears in $S$. We'll try to extend $S$ to ``swallow up'' half of $L$. i.e. we give $p$ an arbitrary truth value, without making $S$ inconsistent. The meat of this proof is then to see that we can ``swallow up'' propositions consistently.

  We first try to swallow up one proposition only.
  
  \noindent\textbf{Claim.} For consistent $S\subset L$ and $p\in L$, we have $S\cup \{p\}$ or $S\cup \{\neg p\}$ consistent.

  Suppose instead that $S\cup \{p\} \vdash \bot$ and $S\cup \{\neg p\}\vdash \bot$.Then by the deduction theorem, $S\vdash p$ and $S\vdash \neg p$. So $S\vdash \bot$.

  As a general rule of mathematics, after doing it for one thing, we do it for infinitely many things.

  Now we suppose $L$ is countable. So we can list $L$ as $\{t_1, t_2, \cdots\}$.

  Let $S_0 = S$. Then let $S_1 = S\cup \{t_1\}$ or $S\cup \{\neg t_1\}$ such that $S_1$ is consistent, and let $S_2 = S_1 \cup \{t_2\}$ or $S_1\cup \{\neg t_2\}$ such that $S_2$ is consistent. Continue inductively.

  Set $\bar{S} = S_0\cup S_1\cup S_2\cdot$. Then $p\in \bar{S}$ or $\neg p\in \bar{S}$ for each $p\in l$ by construction. Also $\bar S$ is consistent (If $\bar S\vdash \bot$, then some $S_n\vdash \bot$ since proofs are finite, but all $S_n$ are consistent). Finally, we check that $\bar S$ is deductively closed, since if $\bar S\models p$, and $p\not\in \bar S$, then $\neg p\not\in \bar S$. Then $\bar S$ would be inconsistent.

  Define $v: L\to \{0, 1\}$ by
  \[
    p \mapsto
    \begin{cases}
      1 & \text{if }p\in \bar S\\
      0 & \text{if not}
    \end{cases}.
  \]
  We claim this is a valuation:

  $v(\bot) = 0$ as $\bot \not\in \bar S$ (since $\bar S$ is consistent).

  For $p\Rightarrow q$,
  \begin{enumerate}
    \item If $v(p) = 1, v(q) = 0$, we have $p\in \bar S$, $q\not\in \bar S$. We want to show $p\Rightarrow q\not\in \bar S$. But if $p\Rightarrow q\in \bar S$, then $\bar S \vdash q$ by MP. Hence $q\in \bar S$ since $\bar S$ is deductively closed.
    \item If $v(q) = 1$, then $q\in \bar S$. We want $p\Rightarrow q\in \bar S$. But $\vdash p\Rightarrow (p\Rightarrow q)$ (Axiom 1). So $p\Rightarrow q \in \bar S$ by deductive closure.
    \item If $v(p) = 0$, then $p\not\in \bar S$. So $\neg p\in \bar S$, and we want $p\Rightarrow q\in \bar S$. So we want to show $\neg p\vdash p\Rightarrow q$. By the deduction theorem, this is equivalent to $\{p, \neg p\} \vdash q$. But $\{p, \neg p\}\vdash \bot$. So it is enough to show $\bot \vdash q$. But $\bot \models \neg \neg q$ by Axiom 1 (since $\bot \Rightarrow (\neg q\Rightarrow \bot)$). It is a theorem that $\neg \neg q\Rightarrow q$. So $\bot \vdash q$. 
  \end{enumerate}
\end{proof}
Note that this is the place where we really use Axiom 3 for the first time.

However, what if the primitives $P$ are not countable? (e.g. $P = \{p_r: r\in \R\}$) This proof would not work, and we will need Zorn's Lemma (c.f. later lectures).

By remark before our theorem, we have
\begin{cor}[Adequacy theorem]
  Let $S\subset L$, $t\in L$. Then $S\models t$ implies $S\vdash t$. 
\end{cor}

\begin{thm}[Completeness theorem]
  Le $S\subset L$ and $t\in L$. Then $S\models t$ if and only if $S\vdash t$.
\end{thm}

\begin{proof}
  ($\Leftarrow$): Soundness. ($\Rightarrow $) Adequacy.
\end{proof}

This has two nice consequences.
\begin{cor}[Compactness theorem]
  Let $S\subset L$ and $t\in L$ with $S\models t$. The some finite $S'\subset S$ has $S'\models t$.
\end{cor}

\begin{proof}
  Trivial with $\models$ replaced by $\vdash$, because proofs are finite.
\end{proof}
The special case $t = \bot$ says, if every finite subset of $S$ has a models, then $S$ has a model. This is also called the compactness theorem.

\begin{cor}[Decidability theorem]
  Let finite $S\subset L$, $t\in L$. Then there exists an algorithm that determines, in finite time, whether or not $S\vdash t$.
\end{cor}
This is \emph{very} surprising because proofs are HARD TO FIND!
\begin{proof}
  Trivial with $\vdash$ replaced by $\models$, by making a truth table.
\end{proof}
\end{document}
